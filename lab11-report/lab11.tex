\documentclass[11pt]{article}

    \usepackage[breakable]{tcolorbox}
    \usepackage{parskip} % Stop auto-indenting (to mimic markdown behaviour)
    

    % Basic figure setup, for now with no caption control since it's done
    % automatically by Pandoc (which extracts ![](path) syntax from Markdown).
    \usepackage{graphicx}
    % Maintain compatibility with old templates. Remove in nbconvert 6.0
    \let\Oldincludegraphics\includegraphics
    % Ensure that by default, figures have no caption (until we provide a
    % proper Figure object with a Caption API and a way to capture that
    % in the conversion process - todo).
    \usepackage{caption}
    \DeclareCaptionFormat{nocaption}{}
    \captionsetup{format=nocaption,aboveskip=0pt,belowskip=0pt}

    \usepackage{float}
    \floatplacement{figure}{H} % forces figures to be placed at the correct location
    \usepackage{xcolor} % Allow colors to be defined
    \usepackage{enumerate} % Needed for markdown enumerations to work
    \usepackage{geometry} % Used to adjust the document margins
    \usepackage{amsmath} % Equations
    \usepackage{amssymb} % Equations
    \usepackage{textcomp} % defines textquotesingle
    % Hack from http://tex.stackexchange.com/a/47451/13684:
    \AtBeginDocument{%
        \def\PYZsq{\textquotesingle}% Upright quotes in Pygmentized code
    }
    \usepackage{upquote} % Upright quotes for verbatim code
    \usepackage{eurosym} % defines \euro

    \usepackage{iftex}
    \ifPDFTeX
        \usepackage[T1]{fontenc}
        \IfFileExists{alphabeta.sty}{
              \usepackage{alphabeta}
          }{
              \usepackage[mathletters]{ucs}
              \usepackage[utf8x]{inputenc}
          }
    \else
        \usepackage{fontspec}
        \usepackage{unicode-math}
    \fi

    \usepackage{fancyvrb} % verbatim replacement that allows latex
    \usepackage{grffile} % extends the file name processing of package graphics
                         % to support a larger range
    \makeatletter % fix for old versions of grffile with XeLaTeX
    \@ifpackagelater{grffile}{2019/11/01}
    {
      % Do nothing on new versions
    }
    {
      \def\Gread@@xetex#1{%
        \IfFileExists{"\Gin@base".bb}%
        {\Gread@eps{\Gin@base.bb}}%
        {\Gread@@xetex@aux#1}%
      }
    }
    \makeatother
    \usepackage[Export]{adjustbox} % Used to constrain images to a maximum size
    \adjustboxset{max size={0.9\linewidth}{0.9\paperheight}}

    % The hyperref package gives us a pdf with properly built
    % internal navigation ('pdf bookmarks' for the table of contents,
    % internal cross-reference links, web links for URLs, etc.)
    \usepackage{hyperref}
    % The default LaTeX title has an obnoxious amount of whitespace. By default,
    % titling removes some of it. It also provides customization options.
    \usepackage{titling}
    \usepackage{longtable} % longtable support required by pandoc >1.10
    \usepackage{booktabs}  % table support for pandoc > 1.12.2
    \usepackage{array}     % table support for pandoc >= 2.11.3
    \usepackage{calc}      % table minipage width calculation for pandoc >= 2.11.1
    \usepackage[inline]{enumitem} % IRkernel/repr support (it uses the enumerate* environment)
    \usepackage[normalem]{ulem} % ulem is needed to support strikethroughs (\sout)
                                % normalem makes italics be italics, not underlines
    \usepackage{soul}      % strikethrough (\st) support for pandoc >= 3.0.0
    \usepackage{mathrsfs}
    

    
    % Colors for the hyperref package
    \definecolor{urlcolor}{rgb}{0,.145,.698}
    \definecolor{linkcolor}{rgb}{.71,0.21,0.01}
    \definecolor{citecolor}{rgb}{.12,.54,.11}

    % ANSI colors
    \definecolor{ansi-black}{HTML}{3E424D}
    \definecolor{ansi-black-intense}{HTML}{282C36}
    \definecolor{ansi-red}{HTML}{E75C58}
    \definecolor{ansi-red-intense}{HTML}{B22B31}
    \definecolor{ansi-green}{HTML}{00A250}
    \definecolor{ansi-green-intense}{HTML}{007427}
    \definecolor{ansi-yellow}{HTML}{DDB62B}
    \definecolor{ansi-yellow-intense}{HTML}{B27D12}
    \definecolor{ansi-blue}{HTML}{208FFB}
    \definecolor{ansi-blue-intense}{HTML}{0065CA}
    \definecolor{ansi-magenta}{HTML}{D160C4}
    \definecolor{ansi-magenta-intense}{HTML}{A03196}
    \definecolor{ansi-cyan}{HTML}{60C6C8}
    \definecolor{ansi-cyan-intense}{HTML}{258F8F}
    \definecolor{ansi-white}{HTML}{C5C1B4}
    \definecolor{ansi-white-intense}{HTML}{A1A6B2}
    \definecolor{ansi-default-inverse-fg}{HTML}{FFFFFF}
    \definecolor{ansi-default-inverse-bg}{HTML}{000000}

    % common color for the border for error outputs.
    \definecolor{outerrorbackground}{HTML}{FFDFDF}

    % commands and environments needed by pandoc snippets
    % extracted from the output of `pandoc -s`
    \providecommand{\tightlist}{%
      \setlength{\itemsep}{0pt}\setlength{\parskip}{0pt}}
    \DefineVerbatimEnvironment{Highlighting}{Verbatim}{commandchars=\\\{\}}
    % Add ',fontsize=\small' for more characters per line
    \newenvironment{Shaded}{}{}
    \newcommand{\KeywordTok}[1]{\textcolor[rgb]{0.00,0.44,0.13}{\textbf{{#1}}}}
    \newcommand{\DataTypeTok}[1]{\textcolor[rgb]{0.56,0.13,0.00}{{#1}}}
    \newcommand{\DecValTok}[1]{\textcolor[rgb]{0.25,0.63,0.44}{{#1}}}
    \newcommand{\BaseNTok}[1]{\textcolor[rgb]{0.25,0.63,0.44}{{#1}}}
    \newcommand{\FloatTok}[1]{\textcolor[rgb]{0.25,0.63,0.44}{{#1}}}
    \newcommand{\CharTok}[1]{\textcolor[rgb]{0.25,0.44,0.63}{{#1}}}
    \newcommand{\StringTok}[1]{\textcolor[rgb]{0.25,0.44,0.63}{{#1}}}
    \newcommand{\CommentTok}[1]{\textcolor[rgb]{0.38,0.63,0.69}{\textit{{#1}}}}
    \newcommand{\OtherTok}[1]{\textcolor[rgb]{0.00,0.44,0.13}{{#1}}}
    \newcommand{\AlertTok}[1]{\textcolor[rgb]{1.00,0.00,0.00}{\textbf{{#1}}}}
    \newcommand{\FunctionTok}[1]{\textcolor[rgb]{0.02,0.16,0.49}{{#1}}}
    \newcommand{\RegionMarkerTok}[1]{{#1}}
    \newcommand{\ErrorTok}[1]{\textcolor[rgb]{1.00,0.00,0.00}{\textbf{{#1}}}}
    \newcommand{\NormalTok}[1]{{#1}}

    % Additional commands for more recent versions of Pandoc
    \newcommand{\ConstantTok}[1]{\textcolor[rgb]{0.53,0.00,0.00}{{#1}}}
    \newcommand{\SpecialCharTok}[1]{\textcolor[rgb]{0.25,0.44,0.63}{{#1}}}
    \newcommand{\VerbatimStringTok}[1]{\textcolor[rgb]{0.25,0.44,0.63}{{#1}}}
    \newcommand{\SpecialStringTok}[1]{\textcolor[rgb]{0.73,0.40,0.53}{{#1}}}
    \newcommand{\ImportTok}[1]{{#1}}
    \newcommand{\DocumentationTok}[1]{\textcolor[rgb]{0.73,0.13,0.13}{\textit{{#1}}}}
    \newcommand{\AnnotationTok}[1]{\textcolor[rgb]{0.38,0.63,0.69}{\textbf{\textit{{#1}}}}}
    \newcommand{\CommentVarTok}[1]{\textcolor[rgb]{0.38,0.63,0.69}{\textbf{\textit{{#1}}}}}
    \newcommand{\VariableTok}[1]{\textcolor[rgb]{0.10,0.09,0.49}{{#1}}}
    \newcommand{\ControlFlowTok}[1]{\textcolor[rgb]{0.00,0.44,0.13}{\textbf{{#1}}}}
    \newcommand{\OperatorTok}[1]{\textcolor[rgb]{0.40,0.40,0.40}{{#1}}}
    \newcommand{\BuiltInTok}[1]{{#1}}
    \newcommand{\ExtensionTok}[1]{{#1}}
    \newcommand{\PreprocessorTok}[1]{\textcolor[rgb]{0.74,0.48,0.00}{{#1}}}
    \newcommand{\AttributeTok}[1]{\textcolor[rgb]{0.49,0.56,0.16}{{#1}}}
    \newcommand{\InformationTok}[1]{\textcolor[rgb]{0.38,0.63,0.69}{\textbf{\textit{{#1}}}}}
    \newcommand{\WarningTok}[1]{\textcolor[rgb]{0.38,0.63,0.69}{\textbf{\textit{{#1}}}}}


    % Define a nice break command that doesn't care if a line doesn't already
    % exist.
    \def\br{\hspace*{\fill} \\* }
    % Math Jax compatibility definitions
    \def\gt{>}
    \def\lt{<}
    \let\Oldtex\TeX
    \let\Oldlatex\LaTeX
    \renewcommand{\TeX}{\textrm{\Oldtex}}
    \renewcommand{\LaTeX}{\textrm{\Oldlatex}}
    % Document parameters
    % Document title
    \title{lab11}
    
    
    
    
    
    
    
% Pygments definitions
\makeatletter
\def\PY@reset{\let\PY@it=\relax \let\PY@bf=\relax%
    \let\PY@ul=\relax \let\PY@tc=\relax%
    \let\PY@bc=\relax \let\PY@ff=\relax}
\def\PY@tok#1{\csname PY@tok@#1\endcsname}
\def\PY@toks#1+{\ifx\relax#1\empty\else%
    \PY@tok{#1}\expandafter\PY@toks\fi}
\def\PY@do#1{\PY@bc{\PY@tc{\PY@ul{%
    \PY@it{\PY@bf{\PY@ff{#1}}}}}}}
\def\PY#1#2{\PY@reset\PY@toks#1+\relax+\PY@do{#2}}

\@namedef{PY@tok@w}{\def\PY@tc##1{\textcolor[rgb]{0.73,0.73,0.73}{##1}}}
\@namedef{PY@tok@c}{\let\PY@it=\textit\def\PY@tc##1{\textcolor[rgb]{0.24,0.48,0.48}{##1}}}
\@namedef{PY@tok@cp}{\def\PY@tc##1{\textcolor[rgb]{0.61,0.40,0.00}{##1}}}
\@namedef{PY@tok@k}{\let\PY@bf=\textbf\def\PY@tc##1{\textcolor[rgb]{0.00,0.50,0.00}{##1}}}
\@namedef{PY@tok@kp}{\def\PY@tc##1{\textcolor[rgb]{0.00,0.50,0.00}{##1}}}
\@namedef{PY@tok@kt}{\def\PY@tc##1{\textcolor[rgb]{0.69,0.00,0.25}{##1}}}
\@namedef{PY@tok@o}{\def\PY@tc##1{\textcolor[rgb]{0.40,0.40,0.40}{##1}}}
\@namedef{PY@tok@ow}{\let\PY@bf=\textbf\def\PY@tc##1{\textcolor[rgb]{0.67,0.13,1.00}{##1}}}
\@namedef{PY@tok@nb}{\def\PY@tc##1{\textcolor[rgb]{0.00,0.50,0.00}{##1}}}
\@namedef{PY@tok@nf}{\def\PY@tc##1{\textcolor[rgb]{0.00,0.00,1.00}{##1}}}
\@namedef{PY@tok@nc}{\let\PY@bf=\textbf\def\PY@tc##1{\textcolor[rgb]{0.00,0.00,1.00}{##1}}}
\@namedef{PY@tok@nn}{\let\PY@bf=\textbf\def\PY@tc##1{\textcolor[rgb]{0.00,0.00,1.00}{##1}}}
\@namedef{PY@tok@ne}{\let\PY@bf=\textbf\def\PY@tc##1{\textcolor[rgb]{0.80,0.25,0.22}{##1}}}
\@namedef{PY@tok@nv}{\def\PY@tc##1{\textcolor[rgb]{0.10,0.09,0.49}{##1}}}
\@namedef{PY@tok@no}{\def\PY@tc##1{\textcolor[rgb]{0.53,0.00,0.00}{##1}}}
\@namedef{PY@tok@nl}{\def\PY@tc##1{\textcolor[rgb]{0.46,0.46,0.00}{##1}}}
\@namedef{PY@tok@ni}{\let\PY@bf=\textbf\def\PY@tc##1{\textcolor[rgb]{0.44,0.44,0.44}{##1}}}
\@namedef{PY@tok@na}{\def\PY@tc##1{\textcolor[rgb]{0.41,0.47,0.13}{##1}}}
\@namedef{PY@tok@nt}{\let\PY@bf=\textbf\def\PY@tc##1{\textcolor[rgb]{0.00,0.50,0.00}{##1}}}
\@namedef{PY@tok@nd}{\def\PY@tc##1{\textcolor[rgb]{0.67,0.13,1.00}{##1}}}
\@namedef{PY@tok@s}{\def\PY@tc##1{\textcolor[rgb]{0.73,0.13,0.13}{##1}}}
\@namedef{PY@tok@sd}{\let\PY@it=\textit\def\PY@tc##1{\textcolor[rgb]{0.73,0.13,0.13}{##1}}}
\@namedef{PY@tok@si}{\let\PY@bf=\textbf\def\PY@tc##1{\textcolor[rgb]{0.64,0.35,0.47}{##1}}}
\@namedef{PY@tok@se}{\let\PY@bf=\textbf\def\PY@tc##1{\textcolor[rgb]{0.67,0.36,0.12}{##1}}}
\@namedef{PY@tok@sr}{\def\PY@tc##1{\textcolor[rgb]{0.64,0.35,0.47}{##1}}}
\@namedef{PY@tok@ss}{\def\PY@tc##1{\textcolor[rgb]{0.10,0.09,0.49}{##1}}}
\@namedef{PY@tok@sx}{\def\PY@tc##1{\textcolor[rgb]{0.00,0.50,0.00}{##1}}}
\@namedef{PY@tok@m}{\def\PY@tc##1{\textcolor[rgb]{0.40,0.40,0.40}{##1}}}
\@namedef{PY@tok@gh}{\let\PY@bf=\textbf\def\PY@tc##1{\textcolor[rgb]{0.00,0.00,0.50}{##1}}}
\@namedef{PY@tok@gu}{\let\PY@bf=\textbf\def\PY@tc##1{\textcolor[rgb]{0.50,0.00,0.50}{##1}}}
\@namedef{PY@tok@gd}{\def\PY@tc##1{\textcolor[rgb]{0.63,0.00,0.00}{##1}}}
\@namedef{PY@tok@gi}{\def\PY@tc##1{\textcolor[rgb]{0.00,0.52,0.00}{##1}}}
\@namedef{PY@tok@gr}{\def\PY@tc##1{\textcolor[rgb]{0.89,0.00,0.00}{##1}}}
\@namedef{PY@tok@ge}{\let\PY@it=\textit}
\@namedef{PY@tok@gs}{\let\PY@bf=\textbf}
\@namedef{PY@tok@ges}{\let\PY@bf=\textbf\let\PY@it=\textit}
\@namedef{PY@tok@gp}{\let\PY@bf=\textbf\def\PY@tc##1{\textcolor[rgb]{0.00,0.00,0.50}{##1}}}
\@namedef{PY@tok@go}{\def\PY@tc##1{\textcolor[rgb]{0.44,0.44,0.44}{##1}}}
\@namedef{PY@tok@gt}{\def\PY@tc##1{\textcolor[rgb]{0.00,0.27,0.87}{##1}}}
\@namedef{PY@tok@err}{\def\PY@bc##1{{\setlength{\fboxsep}{\string -\fboxrule}\fcolorbox[rgb]{1.00,0.00,0.00}{1,1,1}{\strut ##1}}}}
\@namedef{PY@tok@kc}{\let\PY@bf=\textbf\def\PY@tc##1{\textcolor[rgb]{0.00,0.50,0.00}{##1}}}
\@namedef{PY@tok@kd}{\let\PY@bf=\textbf\def\PY@tc##1{\textcolor[rgb]{0.00,0.50,0.00}{##1}}}
\@namedef{PY@tok@kn}{\let\PY@bf=\textbf\def\PY@tc##1{\textcolor[rgb]{0.00,0.50,0.00}{##1}}}
\@namedef{PY@tok@kr}{\let\PY@bf=\textbf\def\PY@tc##1{\textcolor[rgb]{0.00,0.50,0.00}{##1}}}
\@namedef{PY@tok@bp}{\def\PY@tc##1{\textcolor[rgb]{0.00,0.50,0.00}{##1}}}
\@namedef{PY@tok@fm}{\def\PY@tc##1{\textcolor[rgb]{0.00,0.00,1.00}{##1}}}
\@namedef{PY@tok@vc}{\def\PY@tc##1{\textcolor[rgb]{0.10,0.09,0.49}{##1}}}
\@namedef{PY@tok@vg}{\def\PY@tc##1{\textcolor[rgb]{0.10,0.09,0.49}{##1}}}
\@namedef{PY@tok@vi}{\def\PY@tc##1{\textcolor[rgb]{0.10,0.09,0.49}{##1}}}
\@namedef{PY@tok@vm}{\def\PY@tc##1{\textcolor[rgb]{0.10,0.09,0.49}{##1}}}
\@namedef{PY@tok@sa}{\def\PY@tc##1{\textcolor[rgb]{0.73,0.13,0.13}{##1}}}
\@namedef{PY@tok@sb}{\def\PY@tc##1{\textcolor[rgb]{0.73,0.13,0.13}{##1}}}
\@namedef{PY@tok@sc}{\def\PY@tc##1{\textcolor[rgb]{0.73,0.13,0.13}{##1}}}
\@namedef{PY@tok@dl}{\def\PY@tc##1{\textcolor[rgb]{0.73,0.13,0.13}{##1}}}
\@namedef{PY@tok@s2}{\def\PY@tc##1{\textcolor[rgb]{0.73,0.13,0.13}{##1}}}
\@namedef{PY@tok@sh}{\def\PY@tc##1{\textcolor[rgb]{0.73,0.13,0.13}{##1}}}
\@namedef{PY@tok@s1}{\def\PY@tc##1{\textcolor[rgb]{0.73,0.13,0.13}{##1}}}
\@namedef{PY@tok@mb}{\def\PY@tc##1{\textcolor[rgb]{0.40,0.40,0.40}{##1}}}
\@namedef{PY@tok@mf}{\def\PY@tc##1{\textcolor[rgb]{0.40,0.40,0.40}{##1}}}
\@namedef{PY@tok@mh}{\def\PY@tc##1{\textcolor[rgb]{0.40,0.40,0.40}{##1}}}
\@namedef{PY@tok@mi}{\def\PY@tc##1{\textcolor[rgb]{0.40,0.40,0.40}{##1}}}
\@namedef{PY@tok@il}{\def\PY@tc##1{\textcolor[rgb]{0.40,0.40,0.40}{##1}}}
\@namedef{PY@tok@mo}{\def\PY@tc##1{\textcolor[rgb]{0.40,0.40,0.40}{##1}}}
\@namedef{PY@tok@ch}{\let\PY@it=\textit\def\PY@tc##1{\textcolor[rgb]{0.24,0.48,0.48}{##1}}}
\@namedef{PY@tok@cm}{\let\PY@it=\textit\def\PY@tc##1{\textcolor[rgb]{0.24,0.48,0.48}{##1}}}
\@namedef{PY@tok@cpf}{\let\PY@it=\textit\def\PY@tc##1{\textcolor[rgb]{0.24,0.48,0.48}{##1}}}
\@namedef{PY@tok@c1}{\let\PY@it=\textit\def\PY@tc##1{\textcolor[rgb]{0.24,0.48,0.48}{##1}}}
\@namedef{PY@tok@cs}{\let\PY@it=\textit\def\PY@tc##1{\textcolor[rgb]{0.24,0.48,0.48}{##1}}}

\def\PYZbs{\char`\\}
\def\PYZus{\char`\_}
\def\PYZob{\char`\{}
\def\PYZcb{\char`\}}
\def\PYZca{\char`\^}
\def\PYZam{\char`\&}
\def\PYZlt{\char`\<}
\def\PYZgt{\char`\>}
\def\PYZsh{\char`\#}
\def\PYZpc{\char`\%}
\def\PYZdl{\char`\$}
\def\PYZhy{\char`\-}
\def\PYZsq{\char`\'}
\def\PYZdq{\char`\"}
\def\PYZti{\char`\~}
% for compatibility with earlier versions
\def\PYZat{@}
\def\PYZlb{[}
\def\PYZrb{]}
\makeatother


    % For linebreaks inside Verbatim environment from package fancyvrb.
    \makeatletter
        \newbox\Wrappedcontinuationbox
        \newbox\Wrappedvisiblespacebox
        \newcommand*\Wrappedvisiblespace {\textcolor{red}{\textvisiblespace}}
        \newcommand*\Wrappedcontinuationsymbol {\textcolor{red}{\llap{\tiny$\m@th\hookrightarrow$}}}
        \newcommand*\Wrappedcontinuationindent {3ex }
        \newcommand*\Wrappedafterbreak {\kern\Wrappedcontinuationindent\copy\Wrappedcontinuationbox}
        % Take advantage of the already applied Pygments mark-up to insert
        % potential linebreaks for TeX processing.
        %        {, <, #, %, $, ' and ": go to next line.
        %        _, }, ^, &, >, - and ~: stay at end of broken line.
        % Use of \textquotesingle for straight quote.
        \newcommand*\Wrappedbreaksatspecials {%
            \def\PYGZus{\discretionary{\char`\_}{\Wrappedafterbreak}{\char`\_}}%
            \def\PYGZob{\discretionary{}{\Wrappedafterbreak\char`\{}{\char`\{}}%
            \def\PYGZcb{\discretionary{\char`\}}{\Wrappedafterbreak}{\char`\}}}%
            \def\PYGZca{\discretionary{\char`\^}{\Wrappedafterbreak}{\char`\^}}%
            \def\PYGZam{\discretionary{\char`\&}{\Wrappedafterbreak}{\char`\&}}%
            \def\PYGZlt{\discretionary{}{\Wrappedafterbreak\char`\<}{\char`\<}}%
            \def\PYGZgt{\discretionary{\char`\>}{\Wrappedafterbreak}{\char`\>}}%
            \def\PYGZsh{\discretionary{}{\Wrappedafterbreak\char`\#}{\char`\#}}%
            \def\PYGZpc{\discretionary{}{\Wrappedafterbreak\char`\%}{\char`\%}}%
            \def\PYGZdl{\discretionary{}{\Wrappedafterbreak\char`\$}{\char`\$}}%
            \def\PYGZhy{\discretionary{\char`\-}{\Wrappedafterbreak}{\char`\-}}%
            \def\PYGZsq{\discretionary{}{\Wrappedafterbreak\textquotesingle}{\textquotesingle}}%
            \def\PYGZdq{\discretionary{}{\Wrappedafterbreak\char`\"}{\char`\"}}%
            \def\PYGZti{\discretionary{\char`\~}{\Wrappedafterbreak}{\char`\~}}%
        }
        % Some characters . , ; ? ! / are not pygmentized.
        % This macro makes them "active" and they will insert potential linebreaks
        \newcommand*\Wrappedbreaksatpunct {%
            \lccode`\~`\.\lowercase{\def~}{\discretionary{\hbox{\char`\.}}{\Wrappedafterbreak}{\hbox{\char`\.}}}%
            \lccode`\~`\,\lowercase{\def~}{\discretionary{\hbox{\char`\,}}{\Wrappedafterbreak}{\hbox{\char`\,}}}%
            \lccode`\~`\;\lowercase{\def~}{\discretionary{\hbox{\char`\;}}{\Wrappedafterbreak}{\hbox{\char`\;}}}%
            \lccode`\~`\:\lowercase{\def~}{\discretionary{\hbox{\char`\:}}{\Wrappedafterbreak}{\hbox{\char`\:}}}%
            \lccode`\~`\?\lowercase{\def~}{\discretionary{\hbox{\char`\?}}{\Wrappedafterbreak}{\hbox{\char`\?}}}%
            \lccode`\~`\!\lowercase{\def~}{\discretionary{\hbox{\char`\!}}{\Wrappedafterbreak}{\hbox{\char`\!}}}%
            \lccode`\~`\/\lowercase{\def~}{\discretionary{\hbox{\char`\/}}{\Wrappedafterbreak}{\hbox{\char`\/}}}%
            \catcode`\.\active
            \catcode`\,\active
            \catcode`\;\active
            \catcode`\:\active
            \catcode`\?\active
            \catcode`\!\active
            \catcode`\/\active
            \lccode`\~`\~
        }
    \makeatother

    \let\OriginalVerbatim=\Verbatim
    \makeatletter
    \renewcommand{\Verbatim}[1][1]{%
        %\parskip\z@skip
        \sbox\Wrappedcontinuationbox {\Wrappedcontinuationsymbol}%
        \sbox\Wrappedvisiblespacebox {\FV@SetupFont\Wrappedvisiblespace}%
        \def\FancyVerbFormatLine ##1{\hsize\linewidth
            \vtop{\raggedright\hyphenpenalty\z@\exhyphenpenalty\z@
                \doublehyphendemerits\z@\finalhyphendemerits\z@
                \strut ##1\strut}%
        }%
        % If the linebreak is at a space, the latter will be displayed as visible
        % space at end of first line, and a continuation symbol starts next line.
        % Stretch/shrink are however usually zero for typewriter font.
        \def\FV@Space {%
            \nobreak\hskip\z@ plus\fontdimen3\font minus\fontdimen4\font
            \discretionary{\copy\Wrappedvisiblespacebox}{\Wrappedafterbreak}
            {\kern\fontdimen2\font}%
        }%

        % Allow breaks at special characters using \PYG... macros.
        \Wrappedbreaksatspecials
        % Breaks at punctuation characters . , ; ? ! and / need catcode=\active
        \OriginalVerbatim[#1,codes*=\Wrappedbreaksatpunct]%
    }
    \makeatother

    % Exact colors from NB
    \definecolor{incolor}{HTML}{303F9F}
    \definecolor{outcolor}{HTML}{D84315}
    \definecolor{cellborder}{HTML}{CFCFCF}
    \definecolor{cellbackground}{HTML}{F7F7F7}

    % prompt
    \makeatletter
    \newcommand{\boxspacing}{\kern\kvtcb@left@rule\kern\kvtcb@boxsep}
    \makeatother
    \newcommand{\prompt}[4]{
        {\ttfamily\llap{{\color{#2}[#3]:\hspace{3pt}#4}}\vspace{-\baselineskip}}
    }
    

    
    % Prevent overflowing lines due to hard-to-break entities
    \sloppy
    % Setup hyperref package
    \hypersetup{
      breaklinks=true,  % so long urls are correctly broken across lines
      colorlinks=true,
      urlcolor=urlcolor,
      linkcolor=linkcolor,
      citecolor=citecolor,
      }
    % Slightly bigger margins than the latex defaults
    
    \geometry{verbose,tmargin=1in,bmargin=1in,lmargin=1in,rmargin=1in}
    
    

\begin{document}
    
    \begin{titlepage}
        \begin{center}
            \vspace*{1cm}
    
            \textbf{Laboratorium 11}
    
            \vspace{0.5cm}
            Teoria śladów\\
            Część II
                
            \vspace{1.5cm}
    
            \textbf{Danylo Knapp}

            \vfill

            \includegraphics[width=0.4\textwidth]{../report-templates/agh-logo.png}
    
            \vfill
                
            Teoria Współbieżności
                
            \vspace{0.8cm}

            Wydział Informatyki\\
            Akademia Górniczo-Hutnicza\\
            im. Stanisława Staszica w Krakowie\\
            17.12.23
                
        \end{center}
    \end{titlepage}
    
    \hypertarget{treux15bux107-zadania}{%
\section{Treść zadania}\label{treux15bux107-zadania}}

Dane są:

\begin{enumerate}
\def\labelenumi{\arabic{enumi}.}
\tightlist
\item
  Alfabet \(A\), w którym każda litera oznacza akcję.
\item
  Relacja niezależności \(I\), oznaczająca które akcje są niezależne
  (przemienne, tzn. można je wykonać w dowolnej kolejności i nie zmienia
  to wyniku końcowego).
\item
  Słowo \(w\) oznaczające przykładowe wykonanie sekwencji akcji.
\end{enumerate}

\hypertarget{zadanie}{%
\subsection{Zadanie}\label{zadanie}}

Napisz program w dowolnym języku, który:

\begin{enumerate}
\def\labelenumi{\arabic{enumi}.}
\tightlist
\item
  Wyznacza relację zależności \(D\)
\item
  Wyznacza ślad \([w]\) względem relacji \(I\)
\item
  Wyznacza postać normalną Foaty \(FNF([w])\) śladu \([w]\)
\item
  Wyznacza graf zależności dla słowa \(w\)
\end{enumerate}

    \hypertarget{wstux119p}{%
\section{Wstęp}\label{wstux119p}}

Wstęp teoretyczny, m.in opisy wykorzystanych algorytmów, można znaleźć w
sprawozdaniu z laboratorium 10.

    \hypertarget{rozwiux105zanie}{%
\section{Rozwiązanie}\label{rozwiux105zanie}}

W celu rozwiązania tego zadania zostanie użyty język Python 3.11.6.
Również została wykorzystana biblioteka \texttt{pandas\ 2.1.3} w celu
wypisywania niektórych wyników w postaci tabeli. Opis poszczególnych
funkcji tego rozwiązania znajduje się poniżej.

    \begin{tcolorbox}[breakable, size=fbox, boxrule=1pt, pad at break*=1mm,colback=cellbackground, colframe=cellborder]
\prompt{In}{incolor}{1}{\boxspacing}
\begin{Verbatim}[commandchars=\\\{\}]
\PY{k+kn}{from} \PY{n+nn}{typing} \PY{k+kn}{import} \PY{n}{Iterable}\PY{p}{,} \PY{n}{Sequence}
\PY{k+kn}{import} \PY{n+nn}{pandas} \PY{k}{as} \PY{n+nn}{pd}

\PY{n}{MARKER} \PY{o}{=} \PY{l+s+s1}{\PYZsq{}}\PY{l+s+s1}{*}\PY{l+s+s1}{\PYZsq{}}

\PY{n}{RelationIt} \PY{o}{=} \PY{n}{Iterable}\PY{p}{[}\PY{n+nb}{tuple}\PY{p}{[}\PY{n+nb}{str}\PY{p}{,} \PY{n+nb}{str}\PY{p}{]}\PY{p}{]}

\PY{n}{Stack} \PY{o}{=} \PY{n+nb}{list}\PY{p}{[}\PY{n+nb}{str}\PY{p}{]}
\PY{n}{MultiStack} \PY{o}{=} \PY{n+nb}{dict}\PY{p}{[}\PY{n+nb}{str}\PY{p}{,} \PY{n}{Stack}\PY{p}{]}
\PY{n}{Graph} \PY{o}{=} \PY{n+nb}{list}\PY{p}{[}\PY{n+nb}{list}\PY{p}{[}\PY{n+nb}{int}\PY{p}{]}\PY{p}{]}


\PY{k}{def} \PY{n+nf}{stack\PYZus{}empty}\PY{p}{(}\PY{n}{A}\PY{p}{:} \PY{n}{Iterable}\PY{p}{[}\PY{n+nb}{str}\PY{p}{]}\PY{p}{)} \PY{o}{\PYZhy{}}\PY{o}{\PYZgt{}} \PY{n}{MultiStack}\PY{p}{:}
    \PY{k}{return} \PY{n+nb}{dict}\PY{p}{(}\PY{p}{[}\PY{p}{(}\PY{n}{c}\PY{p}{,} \PY{p}{[}\PY{p}{]}\PY{p}{)} \PY{k}{for} \PY{n}{c} \PY{o+ow}{in} \PY{n}{A}\PY{p}{]}\PY{p}{)}


\PY{k}{def} \PY{n+nf}{stack\PYZus{}print}\PY{p}{(}\PY{n}{s}\PY{p}{:} \PY{n}{MultiStack}\PY{p}{)}\PY{p}{:}
    \PY{n}{df} \PY{o}{=} \PY{n}{pd}\PY{o}{.}\PY{n}{DataFrame}\PY{o}{.}\PY{n}{from\PYZus{}dict}\PY{p}{(}\PY{n}{s}\PY{p}{,} \PY{n}{orient}\PY{o}{=}\PY{l+s+s2}{\PYZdq{}}\PY{l+s+s2}{index}\PY{l+s+s2}{\PYZdq{}}\PY{p}{)}\PY{o}{.}\PY{n}{transpose}\PY{p}{(}\PY{p}{)}
    \PY{n}{df}\PY{o}{.}\PY{n}{replace}\PY{p}{(}\PY{n}{to\PYZus{}replace}\PY{o}{=}\PY{p}{[}\PY{k+kc}{None}\PY{p}{]}\PY{p}{,} \PY{n}{value}\PY{o}{=}\PY{l+s+s1}{\PYZsq{}}\PY{l+s+s1}{ }\PY{l+s+s1}{\PYZsq{}}\PY{p}{,} \PY{n}{inplace}\PY{o}{=}\PY{k+kc}{True}\PY{p}{)}
    \PY{n+nb}{print}\PY{p}{(}\PY{n}{df}\PY{p}{)}


\PY{k}{def} \PY{n+nf}{stack\PYZus{}init}\PY{p}{(}
        \PY{n}{word}\PY{p}{:} \PY{n}{Iterable}\PY{p}{[}\PY{n+nb}{str}\PY{p}{]} \PY{o}{|} \PY{n+nb}{str}\PY{p}{,}
        \PY{n}{A}\PY{p}{:} \PY{n}{Iterable}\PY{p}{[}\PY{n+nb}{str}\PY{p}{]}\PY{p}{,}
        \PY{n}{D}\PY{p}{:} \PY{n}{RelationIt}\PY{p}{,}
        \PY{n}{verbose}\PY{p}{:} \PY{n+nb}{bool} \PY{o}{=} \PY{k+kc}{False}\PY{p}{)} \PY{o}{\PYZhy{}}\PY{o}{\PYZgt{}} \PY{n}{MultiStack}\PY{p}{:}

    \PY{c+c1}{\PYZsh{} construct stack}
    \PY{n}{stack} \PY{o}{=} \PY{n}{stack\PYZus{}empty}\PY{p}{(}\PY{n}{A}\PY{p}{)}

    \PY{c+c1}{\PYZsh{} fill stack}
    \PY{k}{for} \PY{n}{curr\PYZus{}action} \PY{o+ow}{in} \PY{n+nb}{reversed}\PY{p}{(}\PY{n}{word}\PY{p}{)}\PY{p}{:}
        \PY{n}{stack}\PY{p}{[}\PY{n}{curr\PYZus{}action}\PY{p}{]}\PY{o}{.}\PY{n}{append}\PY{p}{(}\PY{n}{curr\PYZus{}action}\PY{p}{)}

        \PY{k}{for} \PY{n}{action} \PY{o+ow}{in} \PY{n}{A}\PY{p}{:}
            \PY{k}{if} \PY{n}{action} \PY{o}{==} \PY{n}{curr\PYZus{}action}\PY{p}{:}
                \PY{k}{continue}

            \PY{k}{if} \PY{p}{(}\PY{n}{curr\PYZus{}action}\PY{p}{,} \PY{n}{action}\PY{p}{)} \PY{o+ow}{in} \PY{n}{D}\PY{p}{:}
                \PY{c+c1}{\PYZsh{} dependent, non commutative}
                \PY{n}{stack}\PY{p}{[}\PY{n}{action}\PY{p}{]}\PY{o}{.}\PY{n}{append}\PY{p}{(}\PY{n}{MARKER}\PY{p}{)}
            
            \PY{k}{if} \PY{n}{verbose}\PY{p}{:}
                \PY{n}{stack\PYZus{}print}\PY{p}{(}\PY{n}{stack}\PY{p}{)}
                \PY{n+nb}{print}\PY{p}{(}\PY{p}{)}

    \PY{k}{return} \PY{n}{stack}


\PY{k}{def} \PY{n+nf}{stack\PYZus{}is\PYZus{}empty}\PY{p}{(}\PY{n}{s}\PY{p}{:} \PY{n}{MultiStack}\PY{p}{)} \PY{o}{\PYZhy{}}\PY{o}{\PYZgt{}} \PY{n+nb}{bool}\PY{p}{:}
    \PY{k}{for} \PY{n}{k} \PY{o+ow}{in} \PY{n}{s}\PY{p}{:}
        \PY{k}{if} \PY{n+nb}{len}\PY{p}{(}\PY{n}{s}\PY{p}{[}\PY{n}{k}\PY{p}{]}\PY{p}{)} \PY{o}{\PYZgt{}} \PY{l+m+mi}{0}\PY{p}{:}
            \PY{k}{return} \PY{k+kc}{False}
    \PY{k}{return} \PY{k+kc}{True}


\PY{k}{def} \PY{n+nf}{stack\PYZus{}top}\PY{p}{(}\PY{n}{stack}\PY{p}{:} \PY{n}{MultiStack}\PY{p}{,} \PY{n}{pop}\PY{p}{:} \PY{n+nb}{bool} \PY{o}{=} \PY{k+kc}{False}\PY{p}{)} \PY{o}{\PYZhy{}}\PY{o}{\PYZgt{}} \PY{n+nb}{list}\PY{p}{[}\PY{n+nb}{str}\PY{p}{]}\PY{p}{:}
    \PY{n}{top} \PY{o}{=} \PY{p}{[}\PY{p}{]}

    \PY{k}{for} \PY{n}{action} \PY{o+ow}{in} \PY{n}{stack}\PY{p}{:}
        \PY{n}{s} \PY{o}{=} \PY{n}{stack}\PY{p}{[}\PY{n}{action}\PY{p}{]}

        \PY{k}{if} \PY{n+nb}{len}\PY{p}{(}\PY{n}{s}\PY{p}{)} \PY{o}{\PYZgt{}} \PY{l+m+mi}{0} \PY{o+ow}{and} \PY{n}{s}\PY{p}{[}\PY{o}{\PYZhy{}}\PY{l+m+mi}{1}\PY{p}{]} \PY{o}{!=} \PY{n}{MARKER}\PY{p}{:}
            \PY{n}{top}\PY{o}{.}\PY{n}{append}\PY{p}{(}\PY{n}{s}\PY{o}{.}\PY{n}{pop}\PY{p}{(}\PY{p}{)} \PY{k}{if} \PY{n}{pop} \PY{k}{else} \PY{n}{s}\PY{p}{[}\PY{o}{\PYZhy{}}\PY{l+m+mi}{1}\PY{p}{]}\PY{p}{)}

    \PY{k}{return} \PY{n}{top}


\PY{k}{def} \PY{n+nf}{stack\PYZus{}remove\PYZus{}markers}\PY{p}{(}
        \PY{n}{stack}\PY{p}{:} \PY{n}{MultiStack}\PY{p}{,}
        \PY{n}{D}\PY{p}{:} \PY{n}{RelationIt}\PY{p}{,}
        \PY{n}{top}\PY{p}{:} \PY{n+nb}{list}\PY{p}{[}\PY{n+nb}{str}\PY{p}{]}\PY{p}{)} \PY{o}{\PYZhy{}}\PY{o}{\PYZgt{}} \PY{n+nb}{list}\PY{p}{[}\PY{n+nb}{str}\PY{p}{]}\PY{p}{:}
    
    \PY{n}{popped} \PY{o}{=} \PY{p}{[}\PY{p}{]}
    
    \PY{k}{for} \PY{n}{action} \PY{o+ow}{in} \PY{n}{stack}\PY{p}{:}
        \PY{k}{for} \PY{n}{top\PYZus{}action} \PY{o+ow}{in} \PY{n}{top}\PY{p}{:}
            \PY{k}{if} \PY{n}{action} \PY{o}{==} \PY{n}{top\PYZus{}action}\PY{p}{:}
                \PY{k}{continue}

            \PY{k}{if} \PY{p}{(}\PY{n}{top\PYZus{}action}\PY{p}{,} \PY{n}{action}\PY{p}{)} \PY{o+ow}{in} \PY{n}{D}\PY{p}{:}
                \PY{n}{s} \PY{o}{=} \PY{n}{stack}\PY{p}{[}\PY{n}{action}\PY{p}{]}

                \PY{k}{if} \PY{n+nb}{len}\PY{p}{(}\PY{n}{s}\PY{p}{)} \PY{o}{\PYZgt{}} \PY{l+m+mi}{0} \PY{o+ow}{and} \PY{n}{s}\PY{p}{[}\PY{o}{\PYZhy{}}\PY{l+m+mi}{1}\PY{p}{]} \PY{o}{==} \PY{n}{MARKER}\PY{p}{:}
                    \PY{n}{s}\PY{o}{.}\PY{n}{pop}\PY{p}{(}\PY{p}{)}  \PY{c+c1}{\PYZsh{} pop marker}
                    \PY{n}{popped}\PY{o}{.}\PY{n}{append}\PY{p}{(}\PY{n}{action}\PY{p}{)}

    \PY{k}{return} \PY{n}{popped}


\PY{k}{def} \PY{n+nf}{stack\PYZus{}copy}\PY{p}{(}\PY{n}{stack}\PY{p}{:} \PY{n}{MultiStack}\PY{p}{)} \PY{o}{\PYZhy{}}\PY{o}{\PYZgt{}} \PY{n}{MultiStack}\PY{p}{:}
    \PY{k}{return} \PY{n}{MultiStack}\PY{p}{(}\PY{p}{[}\PY{p}{(}\PY{n}{k}\PY{p}{,} \PY{n}{stack}\PY{p}{[}\PY{n}{k}\PY{p}{]}\PY{o}{.}\PY{n}{copy}\PY{p}{(}\PY{p}{)}\PY{p}{)} \PY{k}{for} \PY{n}{k} \PY{o+ow}{in} \PY{n}{stack}\PY{p}{]}\PY{p}{)}


\PY{k}{def} \PY{n+nf}{graph\PYZus{}path\PYZus{}exists}\PY{p}{(}\PY{n}{graph}\PY{p}{:} \PY{n}{Graph}\PY{p}{,} \PY{n}{src}\PY{p}{:} \PY{n+nb}{int}\PY{p}{,} \PY{n}{dst}\PY{p}{:} \PY{n+nb}{int}\PY{p}{)} \PY{o}{\PYZhy{}}\PY{o}{\PYZgt{}} \PY{n+nb}{bool}\PY{p}{:}
    \PY{k}{class} \PY{n+nc}{VertexInfo}\PY{p}{:}
        \PY{k}{def} \PY{n+nf+fm}{\PYZus{}\PYZus{}init\PYZus{}\PYZus{}}\PY{p}{(}\PY{n+nb+bp}{self}\PY{p}{,} \PY{n}{vtime}\PY{p}{:} \PY{n+nb}{int} \PY{o}{=} \PY{o}{\PYZhy{}}\PY{l+m+mi}{1}\PY{p}{,} \PY{n}{ptime}\PY{p}{:} \PY{n+nb}{int} \PY{o}{=} \PY{o}{\PYZhy{}}\PY{l+m+mi}{1}\PY{p}{,} \PY{n}{parent}\PY{p}{:} \PY{n+nb}{int} \PY{o}{=} \PY{k+kc}{None}\PY{p}{)}\PY{p}{:}
            \PY{n+nb+bp}{self}\PY{o}{.}\PY{n}{vtime} \PY{o}{=} \PY{n}{vtime}  \PY{c+c1}{\PYZsh{} visited time}
            \PY{n+nb+bp}{self}\PY{o}{.}\PY{n}{ptime} \PY{o}{=} \PY{n}{ptime}  \PY{c+c1}{\PYZsh{} processed time}
            \PY{n+nb+bp}{self}\PY{o}{.}\PY{n}{parent} \PY{o}{=} \PY{n}{parent}

        \PY{k}{def} \PY{n+nf+fm}{\PYZus{}\PYZus{}str\PYZus{}\PYZus{}}\PY{p}{(}\PY{n+nb+bp}{self}\PY{p}{)}\PY{p}{:}
            \PY{k}{return} \PY{l+s+s2}{\PYZdq{}}\PY{l+s+s2}{[vtime=}\PY{l+s+si}{\PYZob{}\PYZcb{}}\PY{l+s+s2}{, ptime=}\PY{l+s+si}{\PYZob{}\PYZcb{}}\PY{l+s+s2}{, parent=}\PY{l+s+si}{\PYZob{}\PYZcb{}}\PY{l+s+s2}{]}\PY{l+s+s2}{\PYZdq{}}\PY{o}{.}\PY{n}{format}\PY{p}{(}\PY{n+nb+bp}{self}\PY{o}{.}\PY{n}{vtime}\PY{p}{,} \PY{n+nb+bp}{self}\PY{o}{.}\PY{n}{ptime}\PY{p}{,} \PY{n+nb+bp}{self}\PY{o}{.}\PY{n}{parent}\PY{p}{)}

        \PY{k}{def} \PY{n+nf+fm}{\PYZus{}\PYZus{}repr\PYZus{}\PYZus{}}\PY{p}{(}\PY{n+nb+bp}{self}\PY{p}{)}\PY{p}{:}
            \PY{k}{return} \PY{n+nb+bp}{self}\PY{o}{.}\PY{n+nf+fm}{\PYZus{}\PYZus{}str\PYZus{}\PYZus{}}\PY{p}{(}\PY{p}{)}

    \PY{k}{def} \PY{n+nf}{dfs}\PY{p}{(}\PY{n}{g}\PY{p}{:} \PY{n}{Graph}\PY{p}{,} \PY{n}{start}\PY{p}{:} \PY{n+nb}{int}\PY{p}{)}\PY{p}{:}
        \PY{n}{n} \PY{o}{=} \PY{n+nb}{len}\PY{p}{(}\PY{n}{g}\PY{p}{)}
        \PY{n}{info} \PY{o}{=} \PY{p}{[}\PY{n}{VertexInfo}\PY{p}{(}\PY{p}{)} \PY{k}{for} \PY{n}{\PYZus{}} \PY{o+ow}{in} \PY{n+nb}{range}\PY{p}{(}\PY{n}{n}\PY{p}{)}\PY{p}{]}
        \PY{n}{time} \PY{o}{=} \PY{l+m+mi}{0}

        \PY{k}{def} \PY{n+nf}{visit}\PY{p}{(}\PY{n}{u}\PY{p}{:} \PY{n+nb}{int}\PY{p}{)}\PY{p}{:}
            \PY{k}{nonlocal} \PY{n}{time}
            \PY{n}{time} \PY{o}{+}\PY{o}{=} \PY{l+m+mi}{1}

            \PY{n}{info}\PY{p}{[}\PY{n}{u}\PY{p}{]}\PY{o}{.}\PY{n}{vtime} \PY{o}{=} \PY{n}{time}  \PY{c+c1}{\PYZsh{} wierzchołek został odwiedzony / czas odwiedzenia}

            \PY{k}{for} \PY{n}{v} \PY{o+ow}{in} \PY{n}{g}\PY{p}{[}\PY{n}{u}\PY{p}{]}\PY{p}{:}
                \PY{k}{if} \PY{n}{info}\PY{p}{[}\PY{n}{v}\PY{p}{]}\PY{o}{.}\PY{n}{vtime} \PY{o}{==} \PY{o}{\PYZhy{}}\PY{l+m+mi}{1}\PY{p}{:}
                    \PY{n}{info}\PY{p}{[}\PY{n}{v}\PY{p}{]}\PY{o}{.}\PY{n}{parent} \PY{o}{=} \PY{n}{u}
                    \PY{n}{visit}\PY{p}{(}\PY{n}{v}\PY{p}{)}

            \PY{n}{time} \PY{o}{+}\PY{o}{=} \PY{l+m+mi}{1}

            \PY{n}{info}\PY{p}{[}\PY{n}{u}\PY{p}{]}\PY{o}{.}\PY{n}{ptime} \PY{o}{=} \PY{n}{time}  \PY{c+c1}{\PYZsh{} wierzchołek został przetworzony / czas przetworzenia}

        \PY{n}{visit}\PY{p}{(}\PY{n}{start}\PY{p}{)}

        \PY{k}{return} \PY{n}{info}

    \PY{n}{inf} \PY{o}{=} \PY{n}{dfs}\PY{p}{(}\PY{n}{graph}\PY{p}{,} \PY{n}{src}\PY{p}{)}

    \PY{k}{return} \PY{n}{inf}\PY{p}{[}\PY{n}{dst}\PY{p}{]}\PY{o}{.}\PY{n}{vtime} \PY{o}{!=} \PY{o}{\PYZhy{}}\PY{l+m+mi}{1}


\PY{k}{def} \PY{n+nf}{graph\PYZus{}add\PYZus{}edge}\PY{p}{(}\PY{n}{graph}\PY{p}{:} \PY{n}{Graph}\PY{p}{,} \PY{n}{src}\PY{p}{:} \PY{n+nb}{int}\PY{p}{,} \PY{n}{dst}\PY{p}{:} \PY{n+nb}{int}\PY{p}{)}\PY{p}{:}
    \PY{n}{graph}\PY{p}{[}\PY{n}{src}\PY{p}{]}\PY{o}{.}\PY{n}{append}\PY{p}{(}\PY{n}{dst}\PY{p}{)}


\PY{c+c1}{\PYZsh{} Foata Normal Form}
\PY{k}{def} \PY{n+nf}{fnf}\PY{p}{(}
        \PY{n}{stack}\PY{p}{:} \PY{n}{MultiStack}\PY{p}{,}
        \PY{n}{D}\PY{p}{:} \PY{n}{RelationIt}\PY{p}{,}
        \PY{n}{verbose}\PY{p}{:} \PY{n+nb}{bool} \PY{o}{=} \PY{k+kc}{False}\PY{p}{)} \PY{o}{\PYZhy{}}\PY{o}{\PYZgt{}} \PY{n+nb}{list}\PY{p}{[}\PY{n+nb}{list}\PY{p}{[}\PY{n+nb}{str}\PY{p}{]}\PY{p}{]}\PY{p}{:}
    
    \PY{k}{if} \PY{n}{verbose}\PY{p}{:}
        \PY{n+nb}{print}\PY{p}{(}\PY{l+s+s2}{\PYZdq{}}\PY{l+s+s2}{=== FNF ===}\PY{l+s+s2}{\PYZdq{}}\PY{p}{)}
    
    \PY{c+c1}{\PYZsh{} To get the Foata normal form we take within a loop the set formed by}
    \PY{c+c1}{\PYZsh{} letters being on the top of stacks; arranging the letters in the lexicographic}
    \PY{c+c1}{\PYZsh{} order yields a step. As previously we pop the corresponding markers. Again}
    \PY{c+c1}{\PYZsh{} this loop is repeated until all stacks are empty.}
    
    \PY{n}{stack\PYZus{}fnf} \PY{o}{=} \PY{n}{stack\PYZus{}copy}\PY{p}{(}\PY{n}{stack}\PY{p}{)}

    \PY{n}{fnf}\PY{p}{:} \PY{n+nb}{list}\PY{p}{[}\PY{n+nb}{list}\PY{p}{[}\PY{n+nb}{str}\PY{p}{]}\PY{p}{]} \PY{o}{=} \PY{p}{[}\PY{p}{]}

    \PY{k}{while} \PY{o+ow}{not} \PY{n}{stack\PYZus{}is\PYZus{}empty}\PY{p}{(}\PY{n}{stack\PYZus{}fnf}\PY{p}{)}\PY{p}{:}
        \PY{k}{if} \PY{n}{verbose}\PY{p}{:}
            \PY{n}{stack\PYZus{}print}\PY{p}{(}\PY{n}{stack\PYZus{}fnf}\PY{p}{)}

        \PY{c+c1}{\PYZsh{} step 1: construct top}
        \PY{n}{top} \PY{o}{=} \PY{n}{stack\PYZus{}top}\PY{p}{(}\PY{n}{stack\PYZus{}fnf}\PY{p}{,} \PY{n}{pop}\PY{o}{=}\PY{k+kc}{True}\PY{p}{)}

        \PY{c+c1}{\PYZsh{} step 2: remove markers for non\PYZhy{}commutative actions}
        \PY{n}{removed} \PY{o}{=} \PY{n}{stack\PYZus{}remove\PYZus{}markers}\PY{p}{(}\PY{n}{stack\PYZus{}fnf}\PY{p}{,} \PY{n}{D}\PY{p}{,} \PY{n}{top}\PY{p}{)}

        \PY{n}{fnf}\PY{o}{.}\PY{n}{append}\PY{p}{(}\PY{n+nb}{sorted}\PY{p}{(}\PY{n}{top}\PY{p}{)}\PY{p}{)}

        \PY{k}{if} \PY{n}{verbose}\PY{p}{:}
            \PY{n+nb}{print}\PY{p}{(}\PY{l+s+s2}{\PYZdq{}}\PY{l+s+s2}{Popped top:}\PY{l+s+s2}{\PYZdq{}}\PY{p}{,} \PY{n}{top}\PY{p}{)}
            \PY{n+nb}{print}\PY{p}{(}\PY{l+s+s2}{\PYZdq{}}\PY{l+s+s2}{Removed markers:}\PY{l+s+s2}{\PYZdq{}}\PY{p}{,} \PY{n}{removed}\PY{p}{)}
            \PY{n+nb}{print}\PY{p}{(}\PY{p}{)}

    \PY{k}{return} \PY{n}{fnf}


\PY{c+c1}{\PYZsh{} Lexicographic Normal Form}
\PY{k}{def} \PY{n+nf}{lnf}\PY{p}{(}
        \PY{n}{stack}\PY{p}{:} \PY{n}{MultiStack}\PY{p}{,}
        \PY{n}{D}\PY{p}{:} \PY{n}{RelationIt}\PY{p}{,}
        \PY{n}{verbose}\PY{p}{:} \PY{n+nb}{bool} \PY{o}{=} \PY{k+kc}{False}\PY{p}{)} \PY{o}{\PYZhy{}}\PY{o}{\PYZgt{}} \PY{n+nb}{list}\PY{p}{[}\PY{n+nb}{str}\PY{p}{]}\PY{p}{:}
    
    \PY{k}{if} \PY{n}{verbose}\PY{p}{:}
        \PY{n+nb}{print}\PY{p}{(}\PY{l+s+s2}{\PYZdq{}}\PY{l+s+s2}{=== LNF ===}\PY{l+s+s2}{\PYZdq{}}\PY{p}{)}
    
    \PY{c+c1}{\PYZsh{} To get the lexicographic normal form: it suffices to take among the letters}
    \PY{c+c1}{\PYZsh{} being on the top of some stack that letter a being minimal with respect}
    \PY{c+c1}{\PYZsh{} to the given lexicographic ordering. We pop a marker on each stack corre\PYZhy{}}
    \PY{c+c1}{\PYZsh{} sponding to a letter b (b != a) which does not commute with a. We repeat}
    \PY{c+c1}{\PYZsh{} this loop until all stacks are empty.}

    \PY{n}{stack\PYZus{}norm} \PY{o}{=} \PY{n}{stack\PYZus{}copy}\PY{p}{(}\PY{n}{stack}\PY{p}{)}

    \PY{n}{norm}\PY{p}{:} \PY{n+nb}{list}\PY{p}{[}\PY{n+nb}{str}\PY{p}{]} \PY{o}{=} \PY{p}{[}\PY{p}{]}

    \PY{k}{while} \PY{o+ow}{not} \PY{n}{stack\PYZus{}is\PYZus{}empty}\PY{p}{(}\PY{n}{stack\PYZus{}norm}\PY{p}{)}\PY{p}{:}
        \PY{k}{if} \PY{n}{verbose}\PY{p}{:}
            \PY{n}{stack\PYZus{}print}\PY{p}{(}\PY{n}{stack\PYZus{}norm}\PY{p}{)}

        \PY{c+c1}{\PYZsh{} step 1: construct top}
        \PY{n}{top} \PY{o}{=} \PY{n}{stack\PYZus{}top}\PY{p}{(}\PY{n}{stack\PYZus{}norm}\PY{p}{,} \PY{n}{pop}\PY{o}{=}\PY{k+kc}{False}\PY{p}{)}

        \PY{c+c1}{\PYZsh{} step 2: select minimal action and pop it}
        \PY{n}{min\PYZus{}action} \PY{o}{=} \PY{n+nb}{min}\PY{p}{(}\PY{n}{top}\PY{p}{)}
        \PY{n}{stack\PYZus{}norm}\PY{p}{[}\PY{n}{min\PYZus{}action}\PY{p}{]}\PY{o}{.}\PY{n}{pop}\PY{p}{(}\PY{p}{)}

        \PY{c+c1}{\PYZsh{} step 3: remove markers for non\PYZhy{}commutative actions}
        \PY{n}{removed} \PY{o}{=} \PY{n}{stack\PYZus{}remove\PYZus{}markers}\PY{p}{(}\PY{n}{stack\PYZus{}norm}\PY{p}{,} \PY{n}{D}\PY{p}{,} \PY{p}{[}\PY{n}{min\PYZus{}action}\PY{p}{]}\PY{p}{)}

        \PY{n}{norm}\PY{o}{.}\PY{n}{append}\PY{p}{(}\PY{n}{min\PYZus{}action}\PY{p}{)}

        \PY{k}{if} \PY{n}{verbose}\PY{p}{:}
            \PY{n+nb}{print}\PY{p}{(}\PY{l+s+s2}{\PYZdq{}}\PY{l+s+s2}{Popped action:}\PY{l+s+s2}{\PYZdq{}}\PY{p}{,} \PY{n}{min\PYZus{}action}\PY{p}{)}
            \PY{n+nb}{print}\PY{p}{(}\PY{l+s+s2}{\PYZdq{}}\PY{l+s+s2}{Removed markers:}\PY{l+s+s2}{\PYZdq{}}\PY{p}{,} \PY{n}{removed}\PY{p}{)}
            \PY{n+nb}{print}\PY{p}{(}\PY{p}{)}

    \PY{k}{return} \PY{n}{norm}


\PY{k}{def} \PY{n+nf}{build\PYZus{}dot\PYZus{}graph}\PY{p}{(}
        \PY{n}{word}\PY{p}{:} \PY{n}{Sequence}\PY{p}{[}\PY{n+nb}{str}\PY{p}{]} \PY{o}{|} \PY{n+nb}{str}\PY{p}{,}
        \PY{n}{D}\PY{p}{:} \PY{n}{RelationIt}\PY{p}{)} \PY{o}{\PYZhy{}}\PY{o}{\PYZgt{}} \PY{n+nb}{str}\PY{p}{:}
    
    \PY{n}{n} \PY{o}{=} \PY{n+nb}{len}\PY{p}{(}\PY{n}{word}\PY{p}{)}
    \PY{n}{graph}\PY{p}{:} \PY{n}{Graph} \PY{o}{=} \PY{p}{[}\PY{p}{[}\PY{p}{]} \PY{k}{for} \PY{n}{\PYZus{}} \PY{o+ow}{in} \PY{n+nb}{range}\PY{p}{(}\PY{n}{n}\PY{p}{)}\PY{p}{]}
    
    \PY{k}{for} \PY{n}{dst} \PY{o+ow}{in} \PY{n+nb}{range}\PY{p}{(}\PY{l+m+mi}{1}\PY{p}{,} \PY{n}{n}\PY{p}{)}\PY{p}{:}
        \PY{n}{d\PYZus{}label} \PY{o}{=} \PY{n}{word}\PY{p}{[}\PY{n}{dst}\PY{p}{]}

        \PY{k}{for} \PY{n}{src} \PY{o+ow}{in} \PY{n+nb}{reversed}\PY{p}{(}\PY{n+nb}{range}\PY{p}{(}\PY{n}{dst}\PY{p}{)}\PY{p}{)}\PY{p}{:}
            \PY{n}{s\PYZus{}label} \PY{o}{=} \PY{n}{word}\PY{p}{[}\PY{n}{src}\PY{p}{]}

            \PY{k}{if} \PY{p}{(}\PY{n}{d\PYZus{}label}\PY{p}{,} \PY{n}{s\PYZus{}label}\PY{p}{)} \PY{o+ow}{in} \PY{n}{D} \PY{o+ow}{and} \PY{o+ow}{not} \PY{n}{graph\PYZus{}path\PYZus{}exists}\PY{p}{(}\PY{n}{graph}\PY{p}{,} \PY{n}{src}\PY{p}{,} \PY{n}{dst}\PY{p}{)}\PY{p}{:}
                \PY{n}{graph\PYZus{}add\PYZus{}edge}\PY{p}{(}\PY{n}{graph}\PY{p}{,} \PY{n}{src}\PY{p}{,} \PY{n}{dst}\PY{p}{)}
    
    \PY{c+c1}{\PYZsh{} generate dot graph}
    \PY{n}{dot} \PY{o}{=} \PY{l+s+s2}{\PYZdq{}}\PY{l+s+s2}{\PYZdq{}}

    \PY{k}{for} \PY{n}{parent}\PY{p}{,} \PY{n}{children} \PY{o+ow}{in} \PY{n+nb}{enumerate}\PY{p}{(}\PY{n}{graph}\PY{p}{)}\PY{p}{:}
        \PY{k}{for} \PY{n}{child} \PY{o+ow}{in} \PY{n}{children}\PY{p}{:}
            \PY{n}{dot} \PY{o}{+}\PY{o}{=} \PY{l+s+sa}{f}\PY{l+s+s2}{\PYZdq{}}\PY{l+s+se}{\PYZbs{}t}\PY{l+s+si}{\PYZob{}}\PY{n}{parent}\PY{+w}{ }\PY{o}{+}\PY{+w}{ }\PY{l+m+mi}{1}\PY{l+s+si}{\PYZcb{}}\PY{l+s+s2}{ \PYZhy{}\PYZgt{} }\PY{l+s+si}{\PYZob{}}\PY{n}{child}\PY{+w}{ }\PY{o}{+}\PY{+w}{ }\PY{l+m+mi}{1}\PY{l+s+si}{\PYZcb{}}\PY{l+s+se}{\PYZbs{}n}\PY{l+s+s2}{\PYZdq{}}

    \PY{k}{for} \PY{n}{i} \PY{o+ow}{in} \PY{n+nb}{range}\PY{p}{(}\PY{n}{n}\PY{p}{)}\PY{p}{:}
        \PY{n}{dot} \PY{o}{+}\PY{o}{=} \PY{l+s+sa}{f}\PY{l+s+s2}{\PYZdq{}}\PY{l+s+se}{\PYZbs{}t}\PY{l+s+si}{\PYZob{}}\PY{n}{i}\PY{+w}{ }\PY{o}{+}\PY{+w}{ }\PY{l+m+mi}{1}\PY{l+s+si}{\PYZcb{}}\PY{l+s+s2}{[label=}\PY{l+s+si}{\PYZob{}}\PY{n}{word}\PY{p}{[}\PY{n}{i}\PY{p}{]}\PY{l+s+si}{\PYZcb{}}\PY{l+s+s2}{]}\PY{l+s+se}{\PYZbs{}n}\PY{l+s+s2}{\PYZdq{}}

    \PY{k}{return} \PY{l+s+sa}{f}\PY{l+s+s2}{\PYZdq{}}\PY{l+s+s2}{digraph g }\PY{l+s+se}{\PYZob{}\PYZob{}}\PY{l+s+se}{\PYZbs{}n}\PY{l+s+si}{\PYZob{}}\PY{n}{dot}\PY{l+s+si}{\PYZcb{}}\PY{l+s+se}{\PYZcb{}\PYZcb{}}\PY{l+s+s2}{\PYZdq{}}


\PY{k}{def} \PY{n+nf}{print\PYZus{}summary}\PY{p}{(}
        \PY{n}{A}\PY{p}{:} \PY{n}{Iterable}\PY{p}{[}\PY{n+nb}{str}\PY{p}{]}\PY{p}{,}
        \PY{n}{I}\PY{p}{:} \PY{n}{RelationIt}\PY{p}{,} 
        \PY{n}{w}\PY{p}{:} \PY{n+nb}{str}\PY{p}{,}
        \PY{n}{verbose}\PY{p}{:} \PY{n+nb}{bool} \PY{o}{=} \PY{k+kc}{False}\PY{p}{)} \PY{o}{\PYZhy{}}\PY{o}{\PYZgt{}} \PY{k+kc}{None}\PY{p}{:}
    
    \PY{c+c1}{\PYZsh{} construct D set}
    \PY{n}{D}\PY{p}{:} \PY{n+nb}{set}\PY{p}{[}\PY{n+nb}{tuple}\PY{p}{[}\PY{n+nb}{str}\PY{p}{,} \PY{n+nb}{str}\PY{p}{]}\PY{p}{]} \PY{o}{=} \PY{n+nb}{set}\PY{p}{(}\PY{p}{)}

    \PY{k}{for} \PY{n}{p1} \PY{o+ow}{in} \PY{n}{A}\PY{p}{:}
        \PY{k}{for} \PY{n}{p2} \PY{o+ow}{in} \PY{n}{A}\PY{p}{:}
            \PY{n}{pair} \PY{o}{=} \PY{p}{(}\PY{n}{p1}\PY{p}{,} \PY{n}{p2}\PY{p}{)}

            \PY{k}{if} \PY{n}{pair} \PY{o+ow}{not} \PY{o+ow}{in} \PY{n}{I}\PY{p}{:}
                \PY{n}{D}\PY{o}{.}\PY{n}{add}\PY{p}{(}\PY{n}{pair}\PY{p}{)}

    \PY{c+c1}{\PYZsh{} print D}
    \PY{n+nb}{print}\PY{p}{(}\PY{l+s+sa}{f}\PY{l+s+s2}{\PYZdq{}}\PY{l+s+s2}{D = }\PY{l+s+si}{\PYZob{}}\PY{n+nb}{sorted}\PY{p}{(}\PY{n}{D}\PY{p}{)}\PY{l+s+si}{\PYZcb{}}\PY{l+s+s2}{\PYZdq{}}\PY{p}{)}

    \PY{n}{stack} \PY{o}{=} \PY{n}{stack\PYZus{}init}\PY{p}{(}\PY{n}{w}\PY{p}{,} \PY{n}{A}\PY{p}{,} \PY{n}{D}\PY{p}{,} \PY{n}{verbose}\PY{p}{)}

    \PY{c+c1}{\PYZsh{} calc FNF}
    \PY{n+nb}{print}\PY{p}{(}\PY{l+s+sa}{f}\PY{l+s+s2}{\PYZdq{}}\PY{l+s+s2}{FNF = }\PY{l+s+si}{\PYZob{}}\PY{n}{fnf}\PY{p}{(}\PY{n}{stack}\PY{p}{,}\PY{+w}{ }\PY{n}{D}\PY{p}{,}\PY{+w}{ }\PY{n}{verbose}\PY{p}{)}\PY{l+s+si}{\PYZcb{}}\PY{l+s+s2}{\PYZdq{}}\PY{p}{)}

    \PY{c+c1}{\PYZsh{} calc lexicographic normal form}
    \PY{n+nb}{print}\PY{p}{(}\PY{l+s+sa}{f}\PY{l+s+s2}{\PYZdq{}}\PY{l+s+s2}{LNF = }\PY{l+s+si}{\PYZob{}}\PY{n}{lnf}\PY{p}{(}\PY{n}{stack}\PY{p}{,}\PY{+w}{ }\PY{n}{D}\PY{p}{,}\PY{+w}{ }\PY{n}{verbose}\PY{p}{)}\PY{l+s+si}{\PYZcb{}}\PY{l+s+s2}{\PYZdq{}}\PY{p}{)}

    \PY{n+nb}{print}\PY{p}{(}\PY{n}{build\PYZus{}dot\PYZus{}graph}\PY{p}{(}\PY{n}{w}\PY{p}{,} \PY{n}{D}\PY{p}{)}\PY{p}{)}
\end{Verbatim}
\end{tcolorbox}

    Testowanie dla przykładowych danych testowych:

    \begin{tcolorbox}[breakable, size=fbox, boxrule=1pt, pad at break*=1mm,colback=cellbackground, colframe=cellborder]
\prompt{In}{incolor}{2}{\boxspacing}
\begin{Verbatim}[commandchars=\\\{\}]
\PY{n}{print\PYZus{}summary}\PY{p}{(}
    \PY{n}{A}\PY{o}{=}\PY{p}{[}\PY{l+s+s1}{\PYZsq{}}\PY{l+s+s1}{a}\PY{l+s+s1}{\PYZsq{}}\PY{p}{,} \PY{l+s+s1}{\PYZsq{}}\PY{l+s+s1}{b}\PY{l+s+s1}{\PYZsq{}}\PY{p}{,} \PY{l+s+s1}{\PYZsq{}}\PY{l+s+s1}{c}\PY{l+s+s1}{\PYZsq{}}\PY{p}{,} \PY{l+s+s1}{\PYZsq{}}\PY{l+s+s1}{d}\PY{l+s+s1}{\PYZsq{}}\PY{p}{]}\PY{p}{,}
    \PY{n}{I}\PY{o}{=}\PY{p}{\PYZob{}}\PY{p}{(}\PY{l+s+s1}{\PYZsq{}}\PY{l+s+s1}{a}\PY{l+s+s1}{\PYZsq{}}\PY{p}{,} \PY{l+s+s1}{\PYZsq{}}\PY{l+s+s1}{d}\PY{l+s+s1}{\PYZsq{}}\PY{p}{)}\PY{p}{,} \PY{p}{(}\PY{l+s+s1}{\PYZsq{}}\PY{l+s+s1}{d}\PY{l+s+s1}{\PYZsq{}}\PY{p}{,} \PY{l+s+s1}{\PYZsq{}}\PY{l+s+s1}{a}\PY{l+s+s1}{\PYZsq{}}\PY{p}{)}\PY{p}{,} \PY{p}{(}\PY{l+s+s1}{\PYZsq{}}\PY{l+s+s1}{b}\PY{l+s+s1}{\PYZsq{}}\PY{p}{,} \PY{l+s+s1}{\PYZsq{}}\PY{l+s+s1}{c}\PY{l+s+s1}{\PYZsq{}}\PY{p}{)}\PY{p}{,} \PY{p}{(}\PY{l+s+s1}{\PYZsq{}}\PY{l+s+s1}{c}\PY{l+s+s1}{\PYZsq{}}\PY{p}{,} \PY{l+s+s1}{\PYZsq{}}\PY{l+s+s1}{b}\PY{l+s+s1}{\PYZsq{}}\PY{p}{)}\PY{p}{\PYZcb{}}\PY{p}{,}
    \PY{n}{w}\PY{o}{=}\PY{l+s+s2}{\PYZdq{}}\PY{l+s+s2}{badacb}\PY{l+s+s2}{\PYZdq{}}\PY{p}{)}
\end{Verbatim}
\end{tcolorbox}

    \begin{Verbatim}[commandchars=\\\{\}]
D = [('a', 'a'), ('a', 'b'), ('a', 'c'), ('b', 'a'), ('b', 'b'), ('b', 'd'),
('c', 'a'), ('c', 'c'), ('c', 'd'), ('d', 'b'), ('d', 'c'), ('d', 'd')]
FNF = [['b'], ['a', 'd'], ['a'], ['b', 'c']]
LNF = ['b', 'a', 'a', 'd', 'b', 'c']
digraph g \{
        1 -> 2
        1 -> 3
        2 -> 4
        3 -> 5
        3 -> 6
        4 -> 5
        4 -> 6
        1[label=b]
        2[label=a]
        3[label=d]
        4[label=a]
        5[label=c]
        6[label=b]
\}
    \end{Verbatim}

    \begin{tcolorbox}[breakable, size=fbox, boxrule=1pt, pad at break*=1mm,colback=cellbackground, colframe=cellborder]
\prompt{In}{incolor}{3}{\boxspacing}
\begin{Verbatim}[commandchars=\\\{\}]
\PY{n}{print\PYZus{}summary}\PY{p}{(}
    \PY{n}{A}\PY{o}{=}\PY{p}{[}\PY{l+s+s1}{\PYZsq{}}\PY{l+s+s1}{a}\PY{l+s+s1}{\PYZsq{}}\PY{p}{,} \PY{l+s+s1}{\PYZsq{}}\PY{l+s+s1}{b}\PY{l+s+s1}{\PYZsq{}}\PY{p}{,} \PY{l+s+s1}{\PYZsq{}}\PY{l+s+s1}{c}\PY{l+s+s1}{\PYZsq{}}\PY{p}{,} \PY{l+s+s1}{\PYZsq{}}\PY{l+s+s1}{d}\PY{l+s+s1}{\PYZsq{}}\PY{p}{,} \PY{l+s+s1}{\PYZsq{}}\PY{l+s+s1}{e}\PY{l+s+s1}{\PYZsq{}}\PY{p}{,} \PY{l+s+s1}{\PYZsq{}}\PY{l+s+s1}{f}\PY{l+s+s1}{\PYZsq{}}\PY{p}{]}\PY{p}{,}
    \PY{n}{I}\PY{o}{=}\PY{p}{\PYZob{}}\PY{p}{(}\PY{l+s+s1}{\PYZsq{}}\PY{l+s+s1}{a}\PY{l+s+s1}{\PYZsq{}}\PY{p}{,} \PY{l+s+s1}{\PYZsq{}}\PY{l+s+s1}{d}\PY{l+s+s1}{\PYZsq{}}\PY{p}{)}\PY{p}{,} \PY{p}{(}\PY{l+s+s1}{\PYZsq{}}\PY{l+s+s1}{d}\PY{l+s+s1}{\PYZsq{}}\PY{p}{,} \PY{l+s+s1}{\PYZsq{}}\PY{l+s+s1}{a}\PY{l+s+s1}{\PYZsq{}}\PY{p}{)}\PY{p}{,} \PY{p}{(}\PY{l+s+s1}{\PYZsq{}}\PY{l+s+s1}{b}\PY{l+s+s1}{\PYZsq{}}\PY{p}{,} \PY{l+s+s1}{\PYZsq{}}\PY{l+s+s1}{e}\PY{l+s+s1}{\PYZsq{}}\PY{p}{)}\PY{p}{,} \PY{p}{(}\PY{l+s+s1}{\PYZsq{}}\PY{l+s+s1}{e}\PY{l+s+s1}{\PYZsq{}}\PY{p}{,} \PY{l+s+s1}{\PYZsq{}}\PY{l+s+s1}{b}\PY{l+s+s1}{\PYZsq{}}\PY{p}{)}\PY{p}{,} \PY{p}{(}\PY{l+s+s1}{\PYZsq{}}\PY{l+s+s1}{c}\PY{l+s+s1}{\PYZsq{}}\PY{p}{,} \PY{l+s+s1}{\PYZsq{}}\PY{l+s+s1}{d}\PY{l+s+s1}{\PYZsq{}}\PY{p}{)}\PY{p}{,} \PY{p}{(}\PY{l+s+s1}{\PYZsq{}}\PY{l+s+s1}{d}\PY{l+s+s1}{\PYZsq{}}\PY{p}{,} \PY{l+s+s1}{\PYZsq{}}\PY{l+s+s1}{c}\PY{l+s+s1}{\PYZsq{}}\PY{p}{)}\PY{p}{,} \PY{p}{(}\PY{l+s+s1}{\PYZsq{}}\PY{l+s+s1}{c}\PY{l+s+s1}{\PYZsq{}}\PY{p}{,} \PY{l+s+s1}{\PYZsq{}}\PY{l+s+s1}{f}\PY{l+s+s1}{\PYZsq{}}\PY{p}{)}\PY{p}{,} \PY{p}{(}\PY{l+s+s1}{\PYZsq{}}\PY{l+s+s1}{f}\PY{l+s+s1}{\PYZsq{}}\PY{p}{,} \PY{l+s+s1}{\PYZsq{}}\PY{l+s+s1}{c}\PY{l+s+s1}{\PYZsq{}}\PY{p}{)}\PY{p}{\PYZcb{}}\PY{p}{,}
    \PY{n}{w}\PY{o}{=}\PY{l+s+s2}{\PYZdq{}}\PY{l+s+s2}{acdcfbbe}\PY{l+s+s2}{\PYZdq{}}\PY{p}{)}
\end{Verbatim}
\end{tcolorbox}

    \begin{Verbatim}[commandchars=\\\{\}]
D = [('a', 'a'), ('a', 'b'), ('a', 'c'), ('a', 'e'), ('a', 'f'), ('b', 'a'),
('b', 'b'), ('b', 'c'), ('b', 'd'), ('b', 'f'), ('c', 'a'), ('c', 'b'), ('c',
'c'), ('c', 'e'), ('d', 'b'), ('d', 'd'), ('d', 'e'), ('d', 'f'), ('e', 'a'),
('e', 'c'), ('e', 'd'), ('e', 'e'), ('e', 'f'), ('f', 'a'), ('f', 'b'), ('f',
'd'), ('f', 'e'), ('f', 'f')]
FNF = [['a', 'd'], ['c', 'f'], ['c'], ['b', 'e'], ['b']]
LNF = ['a', 'c', 'c', 'd', 'f', 'b', 'b', 'e']
digraph g \{
        1 -> 2
        1 -> 5
        2 -> 4
        3 -> 5
        4 -> 6
        4 -> 8
        5 -> 6
        5 -> 8
        6 -> 7
        1[label=a]
        2[label=c]
        3[label=d]
        4[label=c]
        5[label=f]
        6[label=b]
        7[label=b]
        8[label=e]
\}
    \end{Verbatim}

    \begin{tcolorbox}[breakable, size=fbox, boxrule=1pt, pad at break*=1mm,colback=cellbackground, colframe=cellborder]
\prompt{In}{incolor}{4}{\boxspacing}
\begin{Verbatim}[commandchars=\\\{\}]
\PY{n}{print\PYZus{}summary}\PY{p}{(}
    \PY{n}{A}\PY{o}{=}\PY{p}{[}\PY{l+s+s1}{\PYZsq{}}\PY{l+s+s1}{a}\PY{l+s+s1}{\PYZsq{}}\PY{p}{,} \PY{l+s+s1}{\PYZsq{}}\PY{l+s+s1}{b}\PY{l+s+s1}{\PYZsq{}}\PY{p}{,} \PY{l+s+s1}{\PYZsq{}}\PY{l+s+s1}{c}\PY{l+s+s1}{\PYZsq{}}\PY{p}{]}\PY{p}{,}
    \PY{n}{I}\PY{o}{=}\PY{p}{\PYZob{}}\PY{p}{(}\PY{l+s+s1}{\PYZsq{}}\PY{l+s+s1}{b}\PY{l+s+s1}{\PYZsq{}}\PY{p}{,} \PY{l+s+s1}{\PYZsq{}}\PY{l+s+s1}{c}\PY{l+s+s1}{\PYZsq{}}\PY{p}{)}\PY{p}{,} \PY{p}{(}\PY{l+s+s1}{\PYZsq{}}\PY{l+s+s1}{c}\PY{l+s+s1}{\PYZsq{}}\PY{p}{,} \PY{l+s+s1}{\PYZsq{}}\PY{l+s+s1}{b}\PY{l+s+s1}{\PYZsq{}}\PY{p}{)}\PY{p}{\PYZcb{}}\PY{p}{,}
    \PY{n}{w}\PY{o}{=}\PY{l+s+s2}{\PYZdq{}}\PY{l+s+s2}{abababbca}\PY{l+s+s2}{\PYZdq{}}\PY{p}{)}
\end{Verbatim}
\end{tcolorbox}

    \begin{Verbatim}[commandchars=\\\{\}]
D = [('a', 'a'), ('a', 'b'), ('a', 'c'), ('b', 'a'), ('b', 'b'), ('c', 'a'),
('c', 'c')]
FNF = [['a'], ['b'], ['a'], ['b'], ['a'], ['b', 'c'], ['b'], ['a']]
LNF = ['a', 'b', 'a', 'b', 'a', 'b', 'b', 'c', 'a']
digraph g \{
        1 -> 2
        2 -> 3
        3 -> 4
        4 -> 5
        5 -> 6
        5 -> 8
        6 -> 7
        7 -> 9
        8 -> 9
        1[label=a]
        2[label=b]
        3[label=a]
        4[label=b]
        5[label=a]
        6[label=b]
        7[label=b]
        8[label=c]
        9[label=a]
\}
    \end{Verbatim}

    \hypertarget{opis-rozwiux105zania}{%
\subsection{Opis rozwiązania}\label{opis-rozwiux105zania}}

\begin{itemize}
\item
  Funkcje służące do zarządzania stosem

  \begin{itemize}
  \tightlist
  \item
    \texttt{stack\_empty} tworzy pusty zbiór stosów dla podanego
    alfabetu \texttt{A}
  \item
    \texttt{stack\_print} wypisuje podany zbiór stosów \texttt{s}
  \item
    \texttt{stack\_init} tworzy zbiór stosów dla danego słowa
    \texttt{word}, alfabetu \texttt{A} oraz relacji zależności
    \texttt{D}
  \item
    \texttt{stack\_is\_empty} sprawdza, czy zbiór stosów jest pusty.
    Zbiór stosów jest pusty wtedy i tylko wtedy, gdy wszystkie stosy w
    nim zawarte są puste
  \item
    \texttt{stack\_top} zwraca elementy z wierzchu stosów podanego
    zbioru stosów \texttt{stack}. Jeżeli \texttt{pop} jest ustawione na
    \texttt{True}, usuwa te elementy ze stosów
  \item
    \texttt{stack\_remove\_markers} usuwa markery ze stosów, które są
    zależne od elementów listy \texttt{top} względem relacji zależności
    \texttt{D}
  \item
    \texttt{stack\_copy} tworzy tzw. deep copy podanego zbioru stosów
    \texttt{stack}
  \end{itemize}
\item
  Funkcje służące do zarządzania grafem

  \begin{itemize}
  \tightlist
  \item
    \texttt{graph\_path\_exists} zwraca \texttt{True}, jeżeli w grafie
    \texttt{graph} istnieje ścieżka między wierzchołkiem o indeksie
    \texttt{src} a \texttt{dst}
    (\(src \rightarrow ... \rightarrow dst\))
  \item
    \texttt{graph\_add\_edge} dodaje krawędź skierowaną w grafie
    \texttt{graph} między wierzchołkiem o indeksie \texttt{src} a
    \texttt{dst} (\(src \rightarrow dst\))
  \end{itemize}
\item
  Funkcje służące do wyznaczania postaci normalnych

  \begin{itemize}
  \tightlist
  \item
    \texttt{fnf} służy do wyznaczania postaci normalnej Foaty (FNF)
  \item
    \texttt{lnf} służy do wyznaczania leksykograficznej postaci
    normalnej (LNF)
  \end{itemize}
\item
  Funkcja \texttt{build\_dot\_graph} służy do tworzenia grafu zależności
  Diekerta w formacie DOT. Do sprawdzenia czy istnieje ścieżka między
  danymi wierzchołkami, został uzyty algorytm DFS
\item
  Funkcja \texttt{print\_summary} służy do wypisywania wszystkich
  informacji dt. słowa \texttt{w} nad alfabetem \texttt{A} względem
  relacji niezależności \texttt{I}. Funkcja ta wypisze następujące
  informacje:

  \begin{itemize}
  \tightlist
  \item
    Relację zależności \(D\)
  \item
    Postać normalną Foaty \(FNF\)
  \item
    Leksykograficzną postać normalną \(LNF\)
  \item
    Graf w formacie DOT
  \end{itemize}
\end{itemize}

    \hypertarget{wnioski}{%
\section{Wnioski}\label{wnioski}}

Oprócz wniosków z laboratorium 10, można dodać następujące dodatkowe
wnioski:

\begin{itemize}
\item
  \textbf{Relacja zależności} \(D\) jest dopełnieniem relacji
  niezależności \(I\), tzn. \(D = A × A ∖ I\), \(D = \bar{I}\). Oznacza
  to, że dwie akcje są zależne, jeśli nie są niezależne.
\item
  \textbf{Graf zależności dla słowa} \(w\) jest grafem skierowanym, w
  którym wierzchołki są akcjami z \(w\), a krawędzie są relacją
  zależności \(D\). Oznacza to, że istnieje krawędź z \(a\) do \(b\),
  jeśli \(a\) i \(b\) są zależne i \(a\) występuje przed \(b\) w słowie
  \(w\).
\item
  \textbf{Algorytm DFS} (ang. Depth First Search) służy do przechodzenia
  lub przeszukiwania drzewa lub grafu. Został użyty w celu sprawdzenia,
  czy istnieje ścieżka między poszczególnymi wierzchołkami.
\item
  \textbf{DOT} jest językiem opisu grafów, opracowanym w ramach projektu
  Graphviz. Służy do definiowania wierzchołków, krawędzi, grafów,
  podgrafów i klastrów za pomocą prostych reguł składniowych.
\end{itemize}

    \hypertarget{bibliografia}{%
\section{Bibliografia}\label{bibliografia}}

\begin{enumerate}
\def\labelenumi{\arabic{enumi}.}
\item
  Materiały do laboratorium 10, dr inż. Włodzimierz Funika:\\
  \url{https://home.agh.edu.pl/~funika/tw/lab-trace/}
\item
  Materiały do laboratorium 11, dr inż. Włodzimierz Funika:\\
  \url{https://home.agh.edu.pl/~funika/tw/lab-trace2/}
\item
  Trace Theory, Volker Diekert, Anca Muscholl:\\
  \url{http://www2.informatik.uni-stuttgart.de/fmi/ti/veroeffentlichungen/pdffiles/DiekertMuscholl2011.pdf}
\item
  Partial Commutation and Traces, Volker Diekert, Yves Métivier:\\
  \url{https://www.researchgate.net/publication/280851316_Partial_Commutation_and_Traces}
\item
  A Foata Normal Form And Its Application For The Purpose Of
  Accel­erating Computations By A Multi-GPU, Ahmet A. Husainov:\\
  \url{https://www.researchgate.net/publication/283389280_A_FOATA_NORMAL_FORM_AND_ITS_APPLICATION_FOR_THE_PURPOSE_OF_ACCEL-ERATING_COMPUTATIONS_BY_A_MULTI-GPU}
\item
  Depth-first search, Wikipedia:\\
  \url{https://en.wikipedia.org/wiki/Depth-first_search}
\item
  DOT (graph description language), Wikipedia:\\
  \url{https://en.wikipedia.org/wiki/DOT_(graph_description_language)}
\end{enumerate}


    % Add a bibliography block to the postdoc
    
    
    
\end{document}
