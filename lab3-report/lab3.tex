\documentclass[11pt]{article}

    \usepackage[breakable]{tcolorbox}
    \usepackage{parskip} % Stop auto-indenting (to mimic markdown behaviour)
    

    % Basic figure setup, for now with no caption control since it's done
    % automatically by Pandoc (which extracts ![](path) syntax from Markdown).
    \usepackage{graphicx}
    % Maintain compatibility with old templates. Remove in nbconvert 6.0
    \let\Oldincludegraphics\includegraphics
    % Ensure that by default, figures have no caption (until we provide a
    % proper Figure object with a Caption API and a way to capture that
    % in the conversion process - todo).
    \usepackage{caption}
    \DeclareCaptionFormat{nocaption}{}
    \captionsetup{format=nocaption,aboveskip=0pt,belowskip=0pt}

    \usepackage{float}
    \floatplacement{figure}{H} % forces figures to be placed at the correct location
    \usepackage{xcolor} % Allow colors to be defined
    \usepackage{enumerate} % Needed for markdown enumerations to work
    \usepackage{geometry} % Used to adjust the document margins
    \usepackage{amsmath} % Equations
    \usepackage{amssymb} % Equations
    \usepackage{textcomp} % defines textquotesingle
    % Hack from http://tex.stackexchange.com/a/47451/13684:
    \AtBeginDocument{%
        \def\PYZsq{\textquotesingle}% Upright quotes in Pygmentized code
    }
    \usepackage{upquote} % Upright quotes for verbatim code
    \usepackage{eurosym} % defines \euro

    \usepackage{iftex}
    \ifPDFTeX
        \usepackage[T1]{fontenc}
        \IfFileExists{alphabeta.sty}{
              \usepackage{alphabeta}
          }{
              \usepackage[mathletters]{ucs}
              \usepackage[utf8x]{inputenc}
          }
    \else
        \usepackage{fontspec}
        \usepackage{unicode-math}
    \fi

    \usepackage{fancyvrb} % verbatim replacement that allows latex
    \usepackage{grffile} % extends the file name processing of package graphics
                         % to support a larger range
    \makeatletter % fix for old versions of grffile with XeLaTeX
    \@ifpackagelater{grffile}{2019/11/01}
    {
      % Do nothing on new versions
    }
    {
      \def\Gread@@xetex#1{%
        \IfFileExists{"\Gin@base".bb}%
        {\Gread@eps{\Gin@base.bb}}%
        {\Gread@@xetex@aux#1}%
      }
    }
    \makeatother
    \usepackage[Export]{adjustbox} % Used to constrain images to a maximum size
    \adjustboxset{max size={0.9\linewidth}{0.9\paperheight}}

    % The hyperref package gives us a pdf with properly built
    % internal navigation ('pdf bookmarks' for the table of contents,
    % internal cross-reference links, web links for URLs, etc.)
    \usepackage{hyperref}
    % The default LaTeX title has an obnoxious amount of whitespace. By default,
    % titling removes some of it. It also provides customization options.
    \usepackage{titling}
    \usepackage{longtable} % longtable support required by pandoc >1.10
    \usepackage{booktabs}  % table support for pandoc > 1.12.2
    \usepackage{array}     % table support for pandoc >= 2.11.3
    \usepackage{calc}      % table minipage width calculation for pandoc >= 2.11.1
    \usepackage[inline]{enumitem} % IRkernel/repr support (it uses the enumerate* environment)
    \usepackage[normalem]{ulem} % ulem is needed to support strikethroughs (\sout)
                                % normalem makes italics be italics, not underlines
    \usepackage{soul}      % strikethrough (\st) support for pandoc >= 3.0.0
    \usepackage{mathrsfs}
    

    
    % Colors for the hyperref package
    \definecolor{urlcolor}{rgb}{0,.145,.698}
    \definecolor{linkcolor}{rgb}{.71,0.21,0.01}
    \definecolor{citecolor}{rgb}{.12,.54,.11}

    % ANSI colors
    \definecolor{ansi-black}{HTML}{3E424D}
    \definecolor{ansi-black-intense}{HTML}{282C36}
    \definecolor{ansi-red}{HTML}{E75C58}
    \definecolor{ansi-red-intense}{HTML}{B22B31}
    \definecolor{ansi-green}{HTML}{00A250}
    \definecolor{ansi-green-intense}{HTML}{007427}
    \definecolor{ansi-yellow}{HTML}{DDB62B}
    \definecolor{ansi-yellow-intense}{HTML}{B27D12}
    \definecolor{ansi-blue}{HTML}{208FFB}
    \definecolor{ansi-blue-intense}{HTML}{0065CA}
    \definecolor{ansi-magenta}{HTML}{D160C4}
    \definecolor{ansi-magenta-intense}{HTML}{A03196}
    \definecolor{ansi-cyan}{HTML}{60C6C8}
    \definecolor{ansi-cyan-intense}{HTML}{258F8F}
    \definecolor{ansi-white}{HTML}{C5C1B4}
    \definecolor{ansi-white-intense}{HTML}{A1A6B2}
    \definecolor{ansi-default-inverse-fg}{HTML}{FFFFFF}
    \definecolor{ansi-default-inverse-bg}{HTML}{000000}

    % common color for the border for error outputs.
    \definecolor{outerrorbackground}{HTML}{FFDFDF}

    % commands and environments needed by pandoc snippets
    % extracted from the output of `pandoc -s`
    \providecommand{\tightlist}{%
      \setlength{\itemsep}{0pt}\setlength{\parskip}{0pt}}
    \DefineVerbatimEnvironment{Highlighting}{Verbatim}{commandchars=\\\{\}}
    % Add ',fontsize=\small' for more characters per line
    \newenvironment{Shaded}{}{}
    \newcommand{\KeywordTok}[1]{\textcolor[rgb]{0.00,0.44,0.13}{\textbf{{#1}}}}
    \newcommand{\DataTypeTok}[1]{\textcolor[rgb]{0.56,0.13,0.00}{{#1}}}
    \newcommand{\DecValTok}[1]{\textcolor[rgb]{0.25,0.63,0.44}{{#1}}}
    \newcommand{\BaseNTok}[1]{\textcolor[rgb]{0.25,0.63,0.44}{{#1}}}
    \newcommand{\FloatTok}[1]{\textcolor[rgb]{0.25,0.63,0.44}{{#1}}}
    \newcommand{\CharTok}[1]{\textcolor[rgb]{0.25,0.44,0.63}{{#1}}}
    \newcommand{\StringTok}[1]{\textcolor[rgb]{0.25,0.44,0.63}{{#1}}}
    \newcommand{\CommentTok}[1]{\textcolor[rgb]{0.38,0.63,0.69}{\textit{{#1}}}}
    \newcommand{\OtherTok}[1]{\textcolor[rgb]{0.00,0.44,0.13}{{#1}}}
    \newcommand{\AlertTok}[1]{\textcolor[rgb]{1.00,0.00,0.00}{\textbf{{#1}}}}
    \newcommand{\FunctionTok}[1]{\textcolor[rgb]{0.02,0.16,0.49}{{#1}}}
    \newcommand{\RegionMarkerTok}[1]{{#1}}
    \newcommand{\ErrorTok}[1]{\textcolor[rgb]{1.00,0.00,0.00}{\textbf{{#1}}}}
    \newcommand{\NormalTok}[1]{{#1}}

    % Additional commands for more recent versions of Pandoc
    \newcommand{\ConstantTok}[1]{\textcolor[rgb]{0.53,0.00,0.00}{{#1}}}
    \newcommand{\SpecialCharTok}[1]{\textcolor[rgb]{0.25,0.44,0.63}{{#1}}}
    \newcommand{\VerbatimStringTok}[1]{\textcolor[rgb]{0.25,0.44,0.63}{{#1}}}
    \newcommand{\SpecialStringTok}[1]{\textcolor[rgb]{0.73,0.40,0.53}{{#1}}}
    \newcommand{\ImportTok}[1]{{#1}}
    \newcommand{\DocumentationTok}[1]{\textcolor[rgb]{0.73,0.13,0.13}{\textit{{#1}}}}
    \newcommand{\AnnotationTok}[1]{\textcolor[rgb]{0.38,0.63,0.69}{\textbf{\textit{{#1}}}}}
    \newcommand{\CommentVarTok}[1]{\textcolor[rgb]{0.38,0.63,0.69}{\textbf{\textit{{#1}}}}}
    \newcommand{\VariableTok}[1]{\textcolor[rgb]{0.10,0.09,0.49}{{#1}}}
    \newcommand{\ControlFlowTok}[1]{\textcolor[rgb]{0.00,0.44,0.13}{\textbf{{#1}}}}
    \newcommand{\OperatorTok}[1]{\textcolor[rgb]{0.40,0.40,0.40}{{#1}}}
    \newcommand{\BuiltInTok}[1]{{#1}}
    \newcommand{\ExtensionTok}[1]{{#1}}
    \newcommand{\PreprocessorTok}[1]{\textcolor[rgb]{0.74,0.48,0.00}{{#1}}}
    \newcommand{\AttributeTok}[1]{\textcolor[rgb]{0.49,0.56,0.16}{{#1}}}
    \newcommand{\InformationTok}[1]{\textcolor[rgb]{0.38,0.63,0.69}{\textbf{\textit{{#1}}}}}
    \newcommand{\WarningTok}[1]{\textcolor[rgb]{0.38,0.63,0.69}{\textbf{\textit{{#1}}}}}


    % Define a nice break command that doesn't care if a line doesn't already
    % exist.
    \def\br{\hspace*{\fill} \\* }
    % Math Jax compatibility definitions
    \def\gt{>}
    \def\lt{<}
    \let\Oldtex\TeX
    \let\Oldlatex\LaTeX
    \renewcommand{\TeX}{\textrm{\Oldtex}}
    \renewcommand{\LaTeX}{\textrm{\Oldlatex}}
    % Document parameters
    % Document title
    \title{lab3}
    
    
    
    
    
    
    
% Pygments definitions
\makeatletter
\def\PY@reset{\let\PY@it=\relax \let\PY@bf=\relax%
    \let\PY@ul=\relax \let\PY@tc=\relax%
    \let\PY@bc=\relax \let\PY@ff=\relax}
\def\PY@tok#1{\csname PY@tok@#1\endcsname}
\def\PY@toks#1+{\ifx\relax#1\empty\else%
    \PY@tok{#1}\expandafter\PY@toks\fi}
\def\PY@do#1{\PY@bc{\PY@tc{\PY@ul{%
    \PY@it{\PY@bf{\PY@ff{#1}}}}}}}
\def\PY#1#2{\PY@reset\PY@toks#1+\relax+\PY@do{#2}}

\@namedef{PY@tok@w}{\def\PY@tc##1{\textcolor[rgb]{0.73,0.73,0.73}{##1}}}
\@namedef{PY@tok@c}{\let\PY@it=\textit\def\PY@tc##1{\textcolor[rgb]{0.24,0.48,0.48}{##1}}}
\@namedef{PY@tok@cp}{\def\PY@tc##1{\textcolor[rgb]{0.61,0.40,0.00}{##1}}}
\@namedef{PY@tok@k}{\let\PY@bf=\textbf\def\PY@tc##1{\textcolor[rgb]{0.00,0.50,0.00}{##1}}}
\@namedef{PY@tok@kp}{\def\PY@tc##1{\textcolor[rgb]{0.00,0.50,0.00}{##1}}}
\@namedef{PY@tok@kt}{\def\PY@tc##1{\textcolor[rgb]{0.69,0.00,0.25}{##1}}}
\@namedef{PY@tok@o}{\def\PY@tc##1{\textcolor[rgb]{0.40,0.40,0.40}{##1}}}
\@namedef{PY@tok@ow}{\let\PY@bf=\textbf\def\PY@tc##1{\textcolor[rgb]{0.67,0.13,1.00}{##1}}}
\@namedef{PY@tok@nb}{\def\PY@tc##1{\textcolor[rgb]{0.00,0.50,0.00}{##1}}}
\@namedef{PY@tok@nf}{\def\PY@tc##1{\textcolor[rgb]{0.00,0.00,1.00}{##1}}}
\@namedef{PY@tok@nc}{\let\PY@bf=\textbf\def\PY@tc##1{\textcolor[rgb]{0.00,0.00,1.00}{##1}}}
\@namedef{PY@tok@nn}{\let\PY@bf=\textbf\def\PY@tc##1{\textcolor[rgb]{0.00,0.00,1.00}{##1}}}
\@namedef{PY@tok@ne}{\let\PY@bf=\textbf\def\PY@tc##1{\textcolor[rgb]{0.80,0.25,0.22}{##1}}}
\@namedef{PY@tok@nv}{\def\PY@tc##1{\textcolor[rgb]{0.10,0.09,0.49}{##1}}}
\@namedef{PY@tok@no}{\def\PY@tc##1{\textcolor[rgb]{0.53,0.00,0.00}{##1}}}
\@namedef{PY@tok@nl}{\def\PY@tc##1{\textcolor[rgb]{0.46,0.46,0.00}{##1}}}
\@namedef{PY@tok@ni}{\let\PY@bf=\textbf\def\PY@tc##1{\textcolor[rgb]{0.44,0.44,0.44}{##1}}}
\@namedef{PY@tok@na}{\def\PY@tc##1{\textcolor[rgb]{0.41,0.47,0.13}{##1}}}
\@namedef{PY@tok@nt}{\let\PY@bf=\textbf\def\PY@tc##1{\textcolor[rgb]{0.00,0.50,0.00}{##1}}}
\@namedef{PY@tok@nd}{\def\PY@tc##1{\textcolor[rgb]{0.67,0.13,1.00}{##1}}}
\@namedef{PY@tok@s}{\def\PY@tc##1{\textcolor[rgb]{0.73,0.13,0.13}{##1}}}
\@namedef{PY@tok@sd}{\let\PY@it=\textit\def\PY@tc##1{\textcolor[rgb]{0.73,0.13,0.13}{##1}}}
\@namedef{PY@tok@si}{\let\PY@bf=\textbf\def\PY@tc##1{\textcolor[rgb]{0.64,0.35,0.47}{##1}}}
\@namedef{PY@tok@se}{\let\PY@bf=\textbf\def\PY@tc##1{\textcolor[rgb]{0.67,0.36,0.12}{##1}}}
\@namedef{PY@tok@sr}{\def\PY@tc##1{\textcolor[rgb]{0.64,0.35,0.47}{##1}}}
\@namedef{PY@tok@ss}{\def\PY@tc##1{\textcolor[rgb]{0.10,0.09,0.49}{##1}}}
\@namedef{PY@tok@sx}{\def\PY@tc##1{\textcolor[rgb]{0.00,0.50,0.00}{##1}}}
\@namedef{PY@tok@m}{\def\PY@tc##1{\textcolor[rgb]{0.40,0.40,0.40}{##1}}}
\@namedef{PY@tok@gh}{\let\PY@bf=\textbf\def\PY@tc##1{\textcolor[rgb]{0.00,0.00,0.50}{##1}}}
\@namedef{PY@tok@gu}{\let\PY@bf=\textbf\def\PY@tc##1{\textcolor[rgb]{0.50,0.00,0.50}{##1}}}
\@namedef{PY@tok@gd}{\def\PY@tc##1{\textcolor[rgb]{0.63,0.00,0.00}{##1}}}
\@namedef{PY@tok@gi}{\def\PY@tc##1{\textcolor[rgb]{0.00,0.52,0.00}{##1}}}
\@namedef{PY@tok@gr}{\def\PY@tc##1{\textcolor[rgb]{0.89,0.00,0.00}{##1}}}
\@namedef{PY@tok@ge}{\let\PY@it=\textit}
\@namedef{PY@tok@gs}{\let\PY@bf=\textbf}
\@namedef{PY@tok@ges}{\let\PY@bf=\textbf\let\PY@it=\textit}
\@namedef{PY@tok@gp}{\let\PY@bf=\textbf\def\PY@tc##1{\textcolor[rgb]{0.00,0.00,0.50}{##1}}}
\@namedef{PY@tok@go}{\def\PY@tc##1{\textcolor[rgb]{0.44,0.44,0.44}{##1}}}
\@namedef{PY@tok@gt}{\def\PY@tc##1{\textcolor[rgb]{0.00,0.27,0.87}{##1}}}
\@namedef{PY@tok@err}{\def\PY@bc##1{{\setlength{\fboxsep}{\string -\fboxrule}\fcolorbox[rgb]{1.00,0.00,0.00}{1,1,1}{\strut ##1}}}}
\@namedef{PY@tok@kc}{\let\PY@bf=\textbf\def\PY@tc##1{\textcolor[rgb]{0.00,0.50,0.00}{##1}}}
\@namedef{PY@tok@kd}{\let\PY@bf=\textbf\def\PY@tc##1{\textcolor[rgb]{0.00,0.50,0.00}{##1}}}
\@namedef{PY@tok@kn}{\let\PY@bf=\textbf\def\PY@tc##1{\textcolor[rgb]{0.00,0.50,0.00}{##1}}}
\@namedef{PY@tok@kr}{\let\PY@bf=\textbf\def\PY@tc##1{\textcolor[rgb]{0.00,0.50,0.00}{##1}}}
\@namedef{PY@tok@bp}{\def\PY@tc##1{\textcolor[rgb]{0.00,0.50,0.00}{##1}}}
\@namedef{PY@tok@fm}{\def\PY@tc##1{\textcolor[rgb]{0.00,0.00,1.00}{##1}}}
\@namedef{PY@tok@vc}{\def\PY@tc##1{\textcolor[rgb]{0.10,0.09,0.49}{##1}}}
\@namedef{PY@tok@vg}{\def\PY@tc##1{\textcolor[rgb]{0.10,0.09,0.49}{##1}}}
\@namedef{PY@tok@vi}{\def\PY@tc##1{\textcolor[rgb]{0.10,0.09,0.49}{##1}}}
\@namedef{PY@tok@vm}{\def\PY@tc##1{\textcolor[rgb]{0.10,0.09,0.49}{##1}}}
\@namedef{PY@tok@sa}{\def\PY@tc##1{\textcolor[rgb]{0.73,0.13,0.13}{##1}}}
\@namedef{PY@tok@sb}{\def\PY@tc##1{\textcolor[rgb]{0.73,0.13,0.13}{##1}}}
\@namedef{PY@tok@sc}{\def\PY@tc##1{\textcolor[rgb]{0.73,0.13,0.13}{##1}}}
\@namedef{PY@tok@dl}{\def\PY@tc##1{\textcolor[rgb]{0.73,0.13,0.13}{##1}}}
\@namedef{PY@tok@s2}{\def\PY@tc##1{\textcolor[rgb]{0.73,0.13,0.13}{##1}}}
\@namedef{PY@tok@sh}{\def\PY@tc##1{\textcolor[rgb]{0.73,0.13,0.13}{##1}}}
\@namedef{PY@tok@s1}{\def\PY@tc##1{\textcolor[rgb]{0.73,0.13,0.13}{##1}}}
\@namedef{PY@tok@mb}{\def\PY@tc##1{\textcolor[rgb]{0.40,0.40,0.40}{##1}}}
\@namedef{PY@tok@mf}{\def\PY@tc##1{\textcolor[rgb]{0.40,0.40,0.40}{##1}}}
\@namedef{PY@tok@mh}{\def\PY@tc##1{\textcolor[rgb]{0.40,0.40,0.40}{##1}}}
\@namedef{PY@tok@mi}{\def\PY@tc##1{\textcolor[rgb]{0.40,0.40,0.40}{##1}}}
\@namedef{PY@tok@il}{\def\PY@tc##1{\textcolor[rgb]{0.40,0.40,0.40}{##1}}}
\@namedef{PY@tok@mo}{\def\PY@tc##1{\textcolor[rgb]{0.40,0.40,0.40}{##1}}}
\@namedef{PY@tok@ch}{\let\PY@it=\textit\def\PY@tc##1{\textcolor[rgb]{0.24,0.48,0.48}{##1}}}
\@namedef{PY@tok@cm}{\let\PY@it=\textit\def\PY@tc##1{\textcolor[rgb]{0.24,0.48,0.48}{##1}}}
\@namedef{PY@tok@cpf}{\let\PY@it=\textit\def\PY@tc##1{\textcolor[rgb]{0.24,0.48,0.48}{##1}}}
\@namedef{PY@tok@c1}{\let\PY@it=\textit\def\PY@tc##1{\textcolor[rgb]{0.24,0.48,0.48}{##1}}}
\@namedef{PY@tok@cs}{\let\PY@it=\textit\def\PY@tc##1{\textcolor[rgb]{0.24,0.48,0.48}{##1}}}

\def\PYZbs{\char`\\}
\def\PYZus{\char`\_}
\def\PYZob{\char`\{}
\def\PYZcb{\char`\}}
\def\PYZca{\char`\^}
\def\PYZam{\char`\&}
\def\PYZlt{\char`\<}
\def\PYZgt{\char`\>}
\def\PYZsh{\char`\#}
\def\PYZpc{\char`\%}
\def\PYZdl{\char`\$}
\def\PYZhy{\char`\-}
\def\PYZsq{\char`\'}
\def\PYZdq{\char`\"}
\def\PYZti{\char`\~}
% for compatibility with earlier versions
\def\PYZat{@}
\def\PYZlb{[}
\def\PYZrb{]}
\makeatother


    % For linebreaks inside Verbatim environment from package fancyvrb.
    \makeatletter
        \newbox\Wrappedcontinuationbox
        \newbox\Wrappedvisiblespacebox
        \newcommand*\Wrappedvisiblespace {\textcolor{red}{\textvisiblespace}}
        \newcommand*\Wrappedcontinuationsymbol {\textcolor{red}{\llap{\tiny$\m@th\hookrightarrow$}}}
        \newcommand*\Wrappedcontinuationindent {3ex }
        \newcommand*\Wrappedafterbreak {\kern\Wrappedcontinuationindent\copy\Wrappedcontinuationbox}
        % Take advantage of the already applied Pygments mark-up to insert
        % potential linebreaks for TeX processing.
        %        {, <, #, %, $, ' and ": go to next line.
        %        _, }, ^, &, >, - and ~: stay at end of broken line.
        % Use of \textquotesingle for straight quote.
        \newcommand*\Wrappedbreaksatspecials {%
            \def\PYGZus{\discretionary{\char`\_}{\Wrappedafterbreak}{\char`\_}}%
            \def\PYGZob{\discretionary{}{\Wrappedafterbreak\char`\{}{\char`\{}}%
            \def\PYGZcb{\discretionary{\char`\}}{\Wrappedafterbreak}{\char`\}}}%
            \def\PYGZca{\discretionary{\char`\^}{\Wrappedafterbreak}{\char`\^}}%
            \def\PYGZam{\discretionary{\char`\&}{\Wrappedafterbreak}{\char`\&}}%
            \def\PYGZlt{\discretionary{}{\Wrappedafterbreak\char`\<}{\char`\<}}%
            \def\PYGZgt{\discretionary{\char`\>}{\Wrappedafterbreak}{\char`\>}}%
            \def\PYGZsh{\discretionary{}{\Wrappedafterbreak\char`\#}{\char`\#}}%
            \def\PYGZpc{\discretionary{}{\Wrappedafterbreak\char`\%}{\char`\%}}%
            \def\PYGZdl{\discretionary{}{\Wrappedafterbreak\char`\$}{\char`\$}}%
            \def\PYGZhy{\discretionary{\char`\-}{\Wrappedafterbreak}{\char`\-}}%
            \def\PYGZsq{\discretionary{}{\Wrappedafterbreak\textquotesingle}{\textquotesingle}}%
            \def\PYGZdq{\discretionary{}{\Wrappedafterbreak\char`\"}{\char`\"}}%
            \def\PYGZti{\discretionary{\char`\~}{\Wrappedafterbreak}{\char`\~}}%
        }
        % Some characters . , ; ? ! / are not pygmentized.
        % This macro makes them "active" and they will insert potential linebreaks
        \newcommand*\Wrappedbreaksatpunct {%
            \lccode`\~`\.\lowercase{\def~}{\discretionary{\hbox{\char`\.}}{\Wrappedafterbreak}{\hbox{\char`\.}}}%
            \lccode`\~`\,\lowercase{\def~}{\discretionary{\hbox{\char`\,}}{\Wrappedafterbreak}{\hbox{\char`\,}}}%
            \lccode`\~`\;\lowercase{\def~}{\discretionary{\hbox{\char`\;}}{\Wrappedafterbreak}{\hbox{\char`\;}}}%
            \lccode`\~`\:\lowercase{\def~}{\discretionary{\hbox{\char`\:}}{\Wrappedafterbreak}{\hbox{\char`\:}}}%
            \lccode`\~`\?\lowercase{\def~}{\discretionary{\hbox{\char`\?}}{\Wrappedafterbreak}{\hbox{\char`\?}}}%
            \lccode`\~`\!\lowercase{\def~}{\discretionary{\hbox{\char`\!}}{\Wrappedafterbreak}{\hbox{\char`\!}}}%
            \lccode`\~`\/\lowercase{\def~}{\discretionary{\hbox{\char`\/}}{\Wrappedafterbreak}{\hbox{\char`\/}}}%
            \catcode`\.\active
            \catcode`\,\active
            \catcode`\;\active
            \catcode`\:\active
            \catcode`\?\active
            \catcode`\!\active
            \catcode`\/\active
            \lccode`\~`\~
        }
    \makeatother

    \let\OriginalVerbatim=\Verbatim
    \makeatletter
    \renewcommand{\Verbatim}[1][1]{%
        %\parskip\z@skip
        \sbox\Wrappedcontinuationbox {\Wrappedcontinuationsymbol}%
        \sbox\Wrappedvisiblespacebox {\FV@SetupFont\Wrappedvisiblespace}%
        \def\FancyVerbFormatLine ##1{\hsize\linewidth
            \vtop{\raggedright\hyphenpenalty\z@\exhyphenpenalty\z@
                \doublehyphendemerits\z@\finalhyphendemerits\z@
                \strut ##1\strut}%
        }%
        % If the linebreak is at a space, the latter will be displayed as visible
        % space at end of first line, and a continuation symbol starts next line.
        % Stretch/shrink are however usually zero for typewriter font.
        \def\FV@Space {%
            \nobreak\hskip\z@ plus\fontdimen3\font minus\fontdimen4\font
            \discretionary{\copy\Wrappedvisiblespacebox}{\Wrappedafterbreak}
            {\kern\fontdimen2\font}%
        }%

        % Allow breaks at special characters using \PYG... macros.
        \Wrappedbreaksatspecials
        % Breaks at punctuation characters . , ; ? ! and / need catcode=\active
        \OriginalVerbatim[#1,codes*=\Wrappedbreaksatpunct]%
    }
    \makeatother

    % Exact colors from NB
    \definecolor{incolor}{HTML}{303F9F}
    \definecolor{outcolor}{HTML}{D84315}
    \definecolor{cellborder}{HTML}{CFCFCF}
    \definecolor{cellbackground}{HTML}{F7F7F7}

    % prompt
    \makeatletter
    \newcommand{\boxspacing}{\kern\kvtcb@left@rule\kern\kvtcb@boxsep}
    \makeatother
    \newcommand{\prompt}[4]{
        {\ttfamily\llap{{\color{#2}[#3]:\hspace{3pt}#4}}\vspace{-\baselineskip}}
    }
    

    
    % Prevent overflowing lines due to hard-to-break entities
    \sloppy
    % Setup hyperref package
    \hypersetup{
      breaklinks=true,  % so long urls are correctly broken across lines
      colorlinks=true,
      urlcolor=urlcolor,
      linkcolor=linkcolor,
      citecolor=citecolor,
      }
    % Slightly bigger margins than the latex defaults
    
    \geometry{verbose,tmargin=1in,bmargin=1in,lmargin=1in,rmargin=1in}
    
    

\begin{document}
    
\begin{titlepage}
    \begin{center}
        \vspace*{1cm}
 
        \textbf{Laboratorium 3}
 
        \vspace{0.5cm}
        Problem producenta-konsumenta
             
        \vspace{1.5cm}
 
        \textbf{Danylo Knapp}

        \vfill

        \includegraphics[width=0.4\textwidth]{../report-templates/agh-logo.png}
 
        \vfill
             
        Teoria Współbieżności
             
        \vspace{0.8cm}

        Wydział Informatyki\\
        Akademia Górniczo-Hutnicza\\
        im. Stanisława Staszica w Krakowie\\
        20.10.23
             
    \end{center}
\end{titlepage}
    
    

    
    \hypertarget{treux15bux107-zadania}{%
\section{Treść zadania}\label{treux15bux107-zadania}}

\hypertarget{problem-ograniczonego-bufora-producentuxf3w-konsumentuxf3w}{%
\subsection{Problem ograniczonego bufora
(producentów-konsumentów)}\label{problem-ograniczonego-bufora-producentuxf3w-konsumentuxf3w}}

Dany jest bufor, do którego producent może wkładać dane, a konsument
pobierać. Napisać program, który zorganizuje takie działanie producenta
i konsumenta, w którym zapewniona będzie własność bezpieczeństwa i
żywotności.

Zrealizować program:

\begin{enumerate}
\def\labelenumi{\arabic{enumi}.}
\tightlist
\item
  przy pomocy metod \texttt{wait()}/\texttt{notify()}

  \begin{itemize}
  \tightlist
  \item
    dla przypadku 1 producent/1 konsument
  \item
    dla przypadku \texttt{n1} producentów/\texttt{n2} konsumentów
    (\texttt{n1\ \textgreater{}\ n2}, \texttt{n1\ =\ n2},
    \texttt{n1\ \textless{}\ n2})
  \item
    wprowadzić wywołanie metody \texttt{sleep()} i wykonać pomiary,
    obserwując zachowanie producentów/konsumentów
  \end{itemize}
\item
  przy pomocy operacji \texttt{P()}/\texttt{V()} dla semafora:

  \begin{itemize}
  \tightlist
  \item
    \texttt{n1\ =\ n2\ =\ 1}
  \item
    \texttt{n1\ \textgreater{}\ 1}, \texttt{n2\ \textgreater{}\ 1}
  \end{itemize}
\end{enumerate}

\textbf{Uwaga}: W implementacji nie jest dozwolone
korzystanie/implementowanie własnych kolejek FIFO, należy używać tylko
mechanizmu monitorów lub semaforów!

\hypertarget{przetwarzanie-potokowe-z-buforem}{%
\subsection{Przetwarzanie potokowe z
buforem}\label{przetwarzanie-potokowe-z-buforem}}

\begin{itemize}
\tightlist
\item
  Bufor o rozmiarze \texttt{N} - wspólny dla wszystkich procesów!
\item
  Proces \emph{A} będący producentem.
\item
  Proces \emph{Z} będący konsumentem.
\item
  Procesy \emph{B}, \emph{C}, \ldots, \emph{Y} będące procesami
  przetwarzającymi. Każdy proces otrzymuje daną wejsciową od procesu
  poprzedniego, jego wyjście zaś jest konsumowane przez proces następny.
\item
  Procesy przetwarzają dane w miejscu, po czym przechodzą do kolejnej
  komórki bufora i znowu przetwarzają ją w miejscu.
\item
  Procesy działają z różnymi prędkościami.
\end{itemize}

\textbf{Uwaga}: Zaimplementować rozwiązanie przetwarzania potokowego
(Przykładowe założenia: bufor rozmiaru 100, 1 producent, 1 konsument, 5
uszeregowanych procesow przetwarzajacych). Od czego zalezy predkosc
obrobki w tym systemie? Rozwiązanie za pomocą semaforów lub monitorów
(dowolnie). \emph{Zrobić sprawozdanie z przetwarzania potokowego.}

    \hypertarget{rozwiux105zanie}{%
\section{Rozwiązanie}\label{rozwiux105zanie}}

Przed aktualnym rozpoczęciem rozwiązania warto najpierw przypomnieć, na
czym dokładnie polega problem producentów-konsumerów (Producer-consumer
problem).

\textbf{Problem producenta i konsumenta} - klasyczny informatyczny
problem synchronizacji. W problemie występują dwa rodzaje procesów:
producent i konsument, którzy dzielą wspólny zasób -- bufor -- dla
produkowanych (i konsumowanych) jednostek. Zadaniem producenta jest
wytworzenie produktu, umieszczenie go w buforze i rozpoczęcie pracy od
nowa. W tym samym czasie konsument ma pobrać produkt z bufora. Problemem
jest taka synchronizacja procesów, żeby producent nie dodawał nowych
jednostek gdy bufor jest pełny, a konsument nie pobierał gdy bufor jest
pusty.

Struktura rozwiązań poszczególnych zadań wygląda następująco:

\begin{verbatim}
src/main/java/pl/edu/agh/tw/knapp
    simplebuff
    pipeline
    Buffer.java
    RandomSleeper.java
    WorkerThread.java
\end{verbatim}

gdzie \texttt{simplebuff} - katalog zawierający implementację pierwszego
zadania, \texttt{pipeline} - katalog zawierający implementację drugiego
zadania (przetwarzanie potokowe).

Implementacja pozostałych klas:

\textbf{\texttt{Buffer}:}

Interfejs bufora.

\begin{Shaded}
\begin{Highlighting}[]
\CommentTok{// Buffer.java}

\KeywordTok{package}\ImportTok{ pl}\OperatorTok{.}\ImportTok{edu}\OperatorTok{.}\ImportTok{agh}\OperatorTok{.}\ImportTok{tw}\OperatorTok{.}\ImportTok{knapp}\OperatorTok{;}

\KeywordTok{import} \ImportTok{java}\OperatorTok{.}\ImportTok{util}\OperatorTok{.}\ImportTok{Optional}\OperatorTok{;}

\KeywordTok{public} \KeywordTok{interface} \BuiltInTok{Buffer}\OperatorTok{\textless{}}\NormalTok{T}\OperatorTok{\textgreater{}} \OperatorTok{\{}
    \DataTypeTok{boolean} \FunctionTok{put}\OperatorTok{(}\NormalTok{T val}\OperatorTok{);}
\NormalTok{    Optional}\OperatorTok{\textless{}}\NormalTok{T}\OperatorTok{\textgreater{}} \FunctionTok{get}\OperatorTok{();}
\OperatorTok{\}}
\end{Highlighting}
\end{Shaded}

\begin{itemize}
\tightlist
\item
  Metoda \texttt{put} zwraca \texttt{true}, jeżeli wartość została
  poprawnie dodana
\item
  Metoda \texttt{get} zwraca \texttt{Optional}, który jest pusty jeżeli
  wystąpił jakiś błąd
\end{itemize}

\textbf{\texttt{RandomSleeper}:}

Klasa służąca do uśpienia wątku na pewien czas, losowany z przedziału
\texttt{{[}delayMinMs,\ delayMaxMs)}.

\begin{Shaded}
\begin{Highlighting}[]
\CommentTok{// RandomSleeper.java}

\KeywordTok{package}\ImportTok{ pl}\OperatorTok{.}\ImportTok{edu}\OperatorTok{.}\ImportTok{agh}\OperatorTok{.}\ImportTok{tw}\OperatorTok{.}\ImportTok{knapp}\OperatorTok{;}

\KeywordTok{import} \ImportTok{java}\OperatorTok{.}\ImportTok{util}\OperatorTok{.}\ImportTok{Random}\OperatorTok{;}

\KeywordTok{public} \KeywordTok{class}\NormalTok{ RandomSleeper }\OperatorTok{\{}
    \KeywordTok{private} \DataTypeTok{final} \BuiltInTok{Random}\NormalTok{ delayRandom }\OperatorTok{=} \KeywordTok{new} \BuiltInTok{Random}\OperatorTok{();}
    \KeywordTok{private} \DataTypeTok{final} \DataTypeTok{long}\NormalTok{ delayMinMs}\OperatorTok{;}
    \KeywordTok{private} \DataTypeTok{final} \DataTypeTok{long}\NormalTok{ delayMaxMs}\OperatorTok{;}

    \KeywordTok{public} \FunctionTok{RandomSleeper}\OperatorTok{(}\DataTypeTok{long}\NormalTok{ delayMinMs}\OperatorTok{,} \DataTypeTok{long}\NormalTok{ delayMaxMs}\OperatorTok{)} \OperatorTok{\{}
        \KeywordTok{this}\OperatorTok{.}\FunctionTok{delayMinMs} \OperatorTok{=}\NormalTok{ delayMinMs}\OperatorTok{;}
        \KeywordTok{this}\OperatorTok{.}\FunctionTok{delayMaxMs} \OperatorTok{=}\NormalTok{ delayMaxMs}\OperatorTok{;}
    \OperatorTok{\}}

    \KeywordTok{public} \DataTypeTok{void} \FunctionTok{sleep}\OperatorTok{()} \KeywordTok{throws} \BuiltInTok{InterruptedException} \OperatorTok{\{}
        \ControlFlowTok{if} \OperatorTok{(}\NormalTok{delayMinMs }\OperatorTok{==} \DecValTok{0} \OperatorTok{\&\&}\NormalTok{ delayMaxMs }\OperatorTok{==} \DecValTok{0}\OperatorTok{)}
            \ControlFlowTok{return}\OperatorTok{;}
        \DataTypeTok{var}\NormalTok{ delay }\OperatorTok{=}\NormalTok{ delayRandom}\OperatorTok{.}\FunctionTok{nextLong}\OperatorTok{(}\NormalTok{delayMinMs}\OperatorTok{,}\NormalTok{ delayMaxMs}\OperatorTok{);}
        \BuiltInTok{Thread}\OperatorTok{.}\FunctionTok{sleep}\OperatorTok{(}\NormalTok{delay}\OperatorTok{);}
    \OperatorTok{\}}
\OperatorTok{\}}
\end{Highlighting}
\end{Shaded}

\textbf{\texttt{WorkerThread}:}

Klasa nadrzędna dla producentów i konsumentów. Zawiera referencję na
bufor, udostępnia funkcję umożliwiającą randomowe uśpienie wątku oraz
funkcję służącą do logowania.

\begin{Shaded}
\begin{Highlighting}[]
\CommentTok{// WorkerThread.java}

\KeywordTok{package}\ImportTok{ pl}\OperatorTok{.}\ImportTok{edu}\OperatorTok{.}\ImportTok{agh}\OperatorTok{.}\ImportTok{tw}\OperatorTok{.}\ImportTok{knapp}\OperatorTok{;}

\KeywordTok{public} \KeywordTok{class}\NormalTok{ WorkerThread}\OperatorTok{\textless{}}\NormalTok{T}\OperatorTok{\textgreater{}} \KeywordTok{extends} \BuiltInTok{Thread} \OperatorTok{\{}
    \KeywordTok{protected} \DataTypeTok{final} \BuiltInTok{Buffer}\OperatorTok{\textless{}}\NormalTok{T}\OperatorTok{\textgreater{}}\NormalTok{ buff}\OperatorTok{;}
    \KeywordTok{private} \DataTypeTok{final}\NormalTok{ RandomSleeper randomSleeper}\OperatorTok{;}

    \KeywordTok{public} \FunctionTok{WorkerThread}\OperatorTok{(}\BuiltInTok{Buffer}\OperatorTok{\textless{}}\NormalTok{T}\OperatorTok{\textgreater{}}\NormalTok{ buff}\OperatorTok{,} \DataTypeTok{long}\NormalTok{ delayMinMs}\OperatorTok{,} \DataTypeTok{long}\NormalTok{ delayMaxMs}\OperatorTok{)} \OperatorTok{\{}
        \KeywordTok{this}\OperatorTok{.}\FunctionTok{buff} \OperatorTok{=}\NormalTok{ buff}\OperatorTok{;}
\NormalTok{        randomSleeper }\OperatorTok{=} \KeywordTok{new} \FunctionTok{RandomSleeper}\OperatorTok{(}\NormalTok{delayMinMs}\OperatorTok{,}\NormalTok{ delayMaxMs}\OperatorTok{);}
    \OperatorTok{\}}

    \KeywordTok{public} \FunctionTok{WorkerThread}\OperatorTok{(}\BuiltInTok{Buffer}\OperatorTok{\textless{}}\NormalTok{T}\OperatorTok{\textgreater{}}\NormalTok{ buff}\OperatorTok{)} \OperatorTok{\{}
        \KeywordTok{this}\OperatorTok{(}\NormalTok{buff}\OperatorTok{,} \DecValTok{0}\OperatorTok{,} \DecValTok{0}\OperatorTok{);}
    \OperatorTok{\}}
    
    \KeywordTok{protected} \DataTypeTok{void} \FunctionTok{randomDelay}\OperatorTok{()} \KeywordTok{throws} \BuiltInTok{InterruptedException} \OperatorTok{\{}
\NormalTok{        randomSleeper}\OperatorTok{.}\FunctionTok{sleep}\OperatorTok{();}
    \OperatorTok{\}}

    \KeywordTok{protected} \DataTypeTok{void} \FunctionTok{log}\OperatorTok{(}\BuiltInTok{Object}\NormalTok{ o}\OperatorTok{)} \OperatorTok{\{}
        \BuiltInTok{System}\OperatorTok{.}\FunctionTok{out}\OperatorTok{.}\FunctionTok{printf}\OperatorTok{(}\StringTok{"[}\SpecialCharTok{\%s}\StringTok{ id }\SpecialCharTok{\%s}\StringTok{] }\SpecialCharTok{\%s\textbackslash{}n}\StringTok{"}\OperatorTok{,} \FunctionTok{getClass}\OperatorTok{().}\FunctionTok{getSimpleName}\OperatorTok{(),} \FunctionTok{getId}\OperatorTok{(),}\NormalTok{ o}\OperatorTok{);}
    \OperatorTok{\}}
\OperatorTok{\}}
\end{Highlighting}
\end{Shaded}

    \hypertarget{problem-ograniczonego-bufora}{%
\subsection{Problem ograniczonego
bufora}\label{problem-ograniczonego-bufora}}

Ze względu na to, że sprawozdanie z tej części nie jest wymagane,
poniżej ogólnie opiszę całą implementację. Analizy niektórych wyników
dokonam w rozdziale \emph{Wyniki}.

Od razu warto dodać, że uwagę

\begin{quote}
W implementacji nie jest dozwolone korzystanie/implementowanie własnych
kolejek FIFO, należy używać tylko mechanizmu monitorów lub semaforów!
\end{quote}

rozumiem w taki sposób, że zabronione jest korzystanie ze wszystkich
możliwych kolejek (w tym FIFO) i należy używać tylko i wyłącznie
mechanizmu semaforów lub monitorów, z czego można wywnioskować że:

\begin{itemize}
\tightlist
\item
  Nie wolno korzystać ani z żadnych list ani tablic w celu
  przechowywania danych wyprodukowanych przez producenta
\item
  Problem musi zostać zaimplementowany wyłącznie w oparciu o mechanizmy
  synchronizacji
\end{itemize}

z czego wynika, że bufor ma być o rozmiarze \texttt{1}. W tym przypadku
rzeczywiście nie potrzebujemy żadnych kolejek i całe rozwiązanie może
zostać zaimplementowane jedynie w oparciu o mechanizmy synchronizacji.

Struktura rozwiązania wygląda następująco:

\begin{verbatim}
src/main/java/pl/edu/agh/tw/knapp/simplebuff
    ConditionMonitor.java
    Consumer.java
    Main.java
    MonitorBuffer.java
    Producer.java
    SemaphoreBuffer.java
\end{verbatim}

Krótki opis poszczególnych klas:

\begin{itemize}
\tightlist
\item
  \texttt{ConditionMonitor}: klasa przypominająca
  \texttt{std::condition\_variable} z języka C++, lecz jest
  zaimplementowana w oparciu o Javowe monitory. Jest wykorzystywana
  przez \texttt{MonitorBuffer};
\item
  \texttt{Consumer}: implementacja konsumenta. \emph{Próbuje} pobrać 100
  razy dane z buforu. W przypadku powodzenia wypisuje pobraną wartość, w
  przypadku niepowodzenia wypisuje odpowiedni komunikat. Uwaga:
  niepowodzenie może wystąpić jeżeli żaden producent nie zapisał do
  bufora jakiejś wartości w ciągu określonego czasu. Wtedy konsument
  stwierdza, że ``transmisja'' jest zakończona i kończy swoje działanie.
  Ten mechanizm jest niezbędny dla niektórych przypadków, np. gdy liczba
  konsumentów jest większa od liczby producentów: w celu uniknięcia
  zawieszenia programu z powodu oczekiwania nowych danych nadanych przez
  producenta, musimy skorzystać z wyżej opisanego mechanizmu;
\item
  \texttt{Main}: klasa zawierająca niektóre ``testy'', bardziej
  szczegółowo jest opisana poniżej;
\item
  \texttt{MonitorBuffer}: implementacja bufora w oparciu o Javowe
  monitory;
\item
  \texttt{Producer}: producent. \emph{Próbuje} zapisać 100 różnych
  wartości do buforu. W przypadku niepowodzenia wypisuje odpowiedni
  komunikat. Niepowodzenie może zostać spowodowane tym, że już żaden
  konsument nie próbuje odczytać wartości z buforu. Ten mechanizm został
  zaimplementowany z przyczyn opisanych podczas omówienia konsumenta,
  tylko tym razem liczba producentów może być większa od liczby
  konsumentów;
\item
  \texttt{SemaphoreBuffer}: implementacja bufora w oparciu o semafory;
\end{itemize}

    \textbf{Klasa \texttt{Main}} zawiera następujące testy dla obu
implementacji bufora:

\begin{itemize}
\tightlist
\item
  1 producent 1 konsument
\item
  1 producent 10 konsumentów
\item
  10 producentów 1 konsument
\item
  100 producentów 100 konsumentów
\end{itemize}

Program również został przetestowany dla 5000 producentów i 5000
konsumentów, ale ten przypadek nie został zawarty w implementacji z tego
powodu, iż wymaga dość dużej ilości pamięci RAM.

\begin{quote}
Uwaga: omówienie wyników znajduje się w odpowiednim rozdziale.
\end{quote}

\begin{quote}
To zadanie może zostać uruchomione korzystając z polecenia

\begin{Shaded}
\begin{Highlighting}[]
\ExtensionTok{./gradlew}\NormalTok{ run }\AttributeTok{{-}Pmain}\OperatorTok{=}\NormalTok{pl.edu.agh.tw.knapp.simplebuff.Main}
\end{Highlighting}
\end{Shaded}
\end{quote}

    \hypertarget{przetwarzanie-potokowe-z-buforem}{%
\subsection{Przetwarzanie potokowe z
buforem}\label{przetwarzanie-potokowe-z-buforem}}

Przetwarzanie potokowe z buforem wymaga zaimplementowania następujących
mechanizmów:

\begin{enumerate}
\def\labelenumi{\arabic{enumi}.}
\tightlist
\item
  Bufor o określonym rozmiarze - jest wspólny dla wszystkich wątków, a
  więc musi być \emph{thread-safe}
\item
  Producent - produkuje dane i zamieszcza je w buforze
\item
  Konsument - pobiera dane z bufora i przekazuje je do przetwarzania
\item
  \emph{Pipe} (rura) - przetwarza otrzymane na wejściu dane i przekazuje
  je dalej: albo do następnej rury, albo do callbacku
\end{enumerate}

Schemat przetwarzania potokowego wygląda w następujący sposób:

\begin{verbatim}
Producer --> Consumer
                |
              Pipe 1
                |
              Pipe 2
                |
               ...
                |
              Pipe N --> Callback
\end{verbatim}

W trakcie wykonania zadania, struktura rozwiązania przybrała następującą
postać:

\begin{verbatim}
src/main/java/pl/edu/agh/tw/knapp/pipeline
    Box.java
    Consumer.java
    Main.java
    PipeAction.java
    Pipe.java
    Producer.java
    SemaphoreBuffer.java
    ThreadedPipe.java
\end{verbatim}

Już wiemy, na czym polega działanie bufora, producenta i konsumenta, a
więc skupmy się na tym, czym jest i jak działa \textbf{pipe}.

    \textbf{\texttt{Pipe}} jest klasą abstrakcyjną, zawierającą metody
umożliwiające sterowanie przetwarzaniem danych. Jest zaimplementowana
następująco:

\begin{Shaded}
\begin{Highlighting}[]
\CommentTok{// Pipe.java}

\KeywordTok{package}\ImportTok{ pl}\OperatorTok{.}\ImportTok{edu}\OperatorTok{.}\ImportTok{agh}\OperatorTok{.}\ImportTok{tw}\OperatorTok{.}\ImportTok{knapp}\OperatorTok{.}\ImportTok{pipeline}\OperatorTok{;}

\KeywordTok{public} \KeywordTok{abstract} \KeywordTok{class} \BuiltInTok{Pipe}\OperatorTok{\textless{}}\NormalTok{T}\OperatorTok{\textgreater{}} \OperatorTok{\{}
    \KeywordTok{private} \BuiltInTok{Pipe}\OperatorTok{\textless{}}\NormalTok{T}\OperatorTok{\textgreater{}}\NormalTok{ next}\OperatorTok{;}
    \KeywordTok{private}\NormalTok{ PipeAction}\OperatorTok{\textless{}}\NormalTok{T}\OperatorTok{\textgreater{}}\NormalTok{ action}\OperatorTok{;}

    \KeywordTok{public} \BuiltInTok{Pipe}\OperatorTok{\textless{}}\NormalTok{T}\OperatorTok{\textgreater{}} \FunctionTok{then}\OperatorTok{(}\BuiltInTok{Pipe}\OperatorTok{\textless{}}\NormalTok{T}\OperatorTok{\textgreater{}}\NormalTok{ next}\OperatorTok{)} \OperatorTok{\{}
        \KeywordTok{this}\OperatorTok{.}\FunctionTok{next} \OperatorTok{=}\NormalTok{ next}\OperatorTok{;}
\NormalTok{        action }\OperatorTok{=} \KeywordTok{null}\OperatorTok{;}
        \ControlFlowTok{return}\NormalTok{ next}\OperatorTok{;}
    \OperatorTok{\}}

    \KeywordTok{public} \DataTypeTok{void} \FunctionTok{action}\OperatorTok{(}\NormalTok{PipeAction}\OperatorTok{\textless{}}\NormalTok{T}\OperatorTok{\textgreater{}}\NormalTok{ action}\OperatorTok{)} \OperatorTok{\{}
        \KeywordTok{this}\OperatorTok{.}\FunctionTok{action} \OperatorTok{=}\NormalTok{ action}\OperatorTok{;}
\NormalTok{        next }\OperatorTok{=} \KeywordTok{null}\OperatorTok{;}
    \OperatorTok{\}}

    \KeywordTok{public} \KeywordTok{abstract} \DataTypeTok{void} \FunctionTok{close}\OperatorTok{();}

    \KeywordTok{public} \DataTypeTok{void} \FunctionTok{closeAll}\OperatorTok{()} \OperatorTok{\{}
        \FunctionTok{close}\OperatorTok{();}

        \ControlFlowTok{if} \OperatorTok{(}\NormalTok{next }\OperatorTok{!=} \KeywordTok{null}\OperatorTok{)} \OperatorTok{\{}
\NormalTok{            next}\OperatorTok{.}\FunctionTok{closeAll}\OperatorTok{();}
        \OperatorTok{\}}
    \OperatorTok{\}}

    \KeywordTok{public} \DataTypeTok{void} \FunctionTok{submitAsync}\OperatorTok{(}\NormalTok{T value}\OperatorTok{)} \OperatorTok{\{}
        \ControlFlowTok{throw} \KeywordTok{new} \BuiltInTok{RuntimeException}\OperatorTok{(}\StringTok{"Not implemented"}\OperatorTok{);}
    \OperatorTok{\}}

    \KeywordTok{public} \DataTypeTok{void} \FunctionTok{submit}\OperatorTok{(}\NormalTok{T value}\OperatorTok{)} \OperatorTok{\{}
        \FunctionTok{submitValue}\OperatorTok{(}\NormalTok{value}\OperatorTok{,} \BuiltInTok{Pipe}\OperatorTok{::}\NormalTok{submit}\OperatorTok{);}
    \OperatorTok{\}}

    \KeywordTok{protected} \DataTypeTok{void} \FunctionTok{submitValue}\OperatorTok{(}\NormalTok{T value}\OperatorTok{,}\NormalTok{ NextAction}\OperatorTok{\textless{}}\NormalTok{T}\OperatorTok{\textgreater{}}\NormalTok{ nextAction}\OperatorTok{)} \OperatorTok{\{}
        \DataTypeTok{var}\NormalTok{ result }\OperatorTok{=} \FunctionTok{onSubmit}\OperatorTok{(}\NormalTok{value}\OperatorTok{);}

        \ControlFlowTok{if} \OperatorTok{(}\NormalTok{next }\OperatorTok{!=} \KeywordTok{null}\OperatorTok{)} \OperatorTok{\{}
\NormalTok{            nextAction}\OperatorTok{.}\FunctionTok{onNext}\OperatorTok{(}\NormalTok{next}\OperatorTok{,}\NormalTok{ result}\OperatorTok{);}
        \OperatorTok{\}} \ControlFlowTok{else} \ControlFlowTok{if} \OperatorTok{(}\NormalTok{action }\OperatorTok{!=} \KeywordTok{null}\OperatorTok{)} \OperatorTok{\{}
\NormalTok{            action}\OperatorTok{.}\FunctionTok{onAction}\OperatorTok{(}\NormalTok{result}\OperatorTok{);}
        \OperatorTok{\}}
    \OperatorTok{\}}

    \KeywordTok{protected} \KeywordTok{abstract}\NormalTok{ T }\FunctionTok{onSubmit}\OperatorTok{(}\NormalTok{T value}\OperatorTok{);}

    \KeywordTok{protected} \KeywordTok{interface}\NormalTok{ NextAction}\OperatorTok{\textless{}}\NormalTok{T}\OperatorTok{\textgreater{}} \OperatorTok{\{}
        \DataTypeTok{void} \FunctionTok{onNext}\OperatorTok{(}\BuiltInTok{Pipe}\OperatorTok{\textless{}}\NormalTok{T}\OperatorTok{\textgreater{}}\NormalTok{ next}\OperatorTok{,}\NormalTok{ T value}\OperatorTok{);}
    \OperatorTok{\}}
\OperatorTok{\}}
\end{Highlighting}
\end{Shaded}

\begin{itemize}
\tightlist
\item
  Metoda \texttt{then} umożliwia tworzenie tzw. pipline'u: ``sklejania''
  kilku pipe'ów, tzn. na wejście podanego jako argument pipe'u zostanie
  przekazany wynik (wyjście) danego pipe'u (jak to dokładnie wygląda
  zostanie pokazane poniżej)
\item
  Metoda \texttt{action} jest metodą końcową (terminalną), tzn. jako
  argument przyjmuje callback, do którego zostanie przekazana wartość po
  przetworzeniu przez daną rurę (\texttt{Pipe\ N} na schemacie)
\item
  Metoda abstrakcyjna \texttt{close} zamyka daną rurę - rura nie będzie
  już w stanie przetwarzać nowych porcji danych
\item
  Metoda \texttt{closeAll} zamyka daną rurę oraz wszystkie inne rury
  niżej w hierarchii
\item
  Metoda \texttt{submit} służy do przekazania danych na wejście danej
  rury. Jest to funkcja blokująca: blokuje dany wątek dopóki dane nie
  zostaną przetworzone przez daną rurę oraz wszystkie pozostałe rury
  niżej w hierarchii
\item
  Metoda \texttt{submitAsync} działa podobnie do \texttt{submit}, lecz w
  sposób asynchroniczny, tzn. dostaje dane na wejściu i umieszcza je w
  kolejce ``do przetwarzania''. Po przetworzeniu, przekazuje je w sposób
  asynchroniczny do kolejnej rury lub do callbacku. Nie jest
  zaimplementowana
\end{itemize}

    \textbf{\texttt{ThreadedPipe}} jest klasą dziedziczącą po \texttt{Pipe}
i implementującą przetwarzanie asynchroniczne korzystając z
\texttt{java.util.concurrent.ExecutorService}, a mianowicie jest użyta
metoda \texttt{Executors.newSingleThreadExecutor()}.
\texttt{ExecutorService} jest użyty z powodu chęci podniesienia
czytelności kodu, bo choć implementowanie własnego \texttt{ThreadPool}
nie jest specjalnie skomplikowane, lecz w sposób oczywisty kompikuje
analizę rozwiązania ze względu na ilość kodu.

Oczywiście, możemy pozbyć się dodatkowej kolejki (tzn. zrezygnować z
\texttt{ExecutorService}) korzystając z mechanizmu podobnego do tego, co
został użyty do implementacji klasy \texttt{SemaphoreBuffer} /
\texttt{MonitorBuffer} z zadania 1, ale w takim razie nawet metoda
\texttt{submitAsync} może okazać się częściowo blokującą: szybkość
obróbki danych zostanie uzależniona również od zajętości
(\texttt{i\ +\ 1})-ej rury, która z kolei zostanie uzależniona od
(\texttt{i\ +\ 2})-iej rury itd. Z czego wynika, że szybkość obróbki
danych w rurze o indeksie \texttt{i} zostanie uzależniona od rur o
indeksach \texttt{i\ +\ 1}, \texttt{i\ +\ 2}, \ldots, \texttt{n}. Ze
względów wydajnościowych chcielibyśmy takiej sytuacji uniknąć, a więc
musimy dołożyć kolejną kolejkę, a w naszym przypadku - skorzystać z
\texttt{ExecutorService}. Przy takiej implementacji, szybkość obróbki
nowych danych w \texttt{i}-tej rurze zostanie uzależniona tylko od
szybkości nadchodzenia tych danych, od liczby danych już oczekujących na
obróbkę, oraz od własnej szybkości przetwarzania jednej porcji danych
przez daną rurę.

\begin{Shaded}
\begin{Highlighting}[]
\CommentTok{// ThreadedPipe.java}

\KeywordTok{package}\ImportTok{ pl}\OperatorTok{.}\ImportTok{edu}\OperatorTok{.}\ImportTok{agh}\OperatorTok{.}\ImportTok{tw}\OperatorTok{.}\ImportTok{knapp}\OperatorTok{.}\ImportTok{pipeline}\OperatorTok{;}

\KeywordTok{import} \ImportTok{java}\OperatorTok{.}\ImportTok{util}\OperatorTok{.}\ImportTok{concurrent}\OperatorTok{.}\ImportTok{ExecutionException}\OperatorTok{;}
\KeywordTok{import} \ImportTok{java}\OperatorTok{.}\ImportTok{util}\OperatorTok{.}\ImportTok{concurrent}\OperatorTok{.}\ImportTok{ExecutorService}\OperatorTok{;}
\KeywordTok{import} \ImportTok{java}\OperatorTok{.}\ImportTok{util}\OperatorTok{.}\ImportTok{concurrent}\OperatorTok{.}\ImportTok{Executors}\OperatorTok{;}

\KeywordTok{public} \KeywordTok{abstract} \KeywordTok{class}\NormalTok{ ThreadedPipe}\OperatorTok{\textless{}}\NormalTok{T}\OperatorTok{\textgreater{}} \KeywordTok{extends} \BuiltInTok{Pipe}\OperatorTok{\textless{}}\NormalTok{T}\OperatorTok{\textgreater{}} \OperatorTok{\{}
    \KeywordTok{private} \DataTypeTok{final} \BuiltInTok{ExecutorService}\NormalTok{ executorService }\OperatorTok{=} \BuiltInTok{Executors}\OperatorTok{.}\FunctionTok{newSingleThreadExecutor}\OperatorTok{();}

    \KeywordTok{public} \FunctionTok{ThreadedPipe}\OperatorTok{()} \OperatorTok{\{}
        \CommentTok{// empty}
    \OperatorTok{\}}

    \AttributeTok{@Override}
    \KeywordTok{public} \DataTypeTok{void} \FunctionTok{close}\OperatorTok{()} \OperatorTok{\{}
\NormalTok{        executorService}\OperatorTok{.}\FunctionTok{shutdown}\OperatorTok{();}
    \OperatorTok{\}}

    \AttributeTok{@Override}
    \KeywordTok{public} \DataTypeTok{void} \FunctionTok{submitAsync}\OperatorTok{(}\NormalTok{T value}\OperatorTok{)} \OperatorTok{\{}
\NormalTok{        executorService}\OperatorTok{.}\FunctionTok{submit}\OperatorTok{(()} \OperatorTok{{-}\textgreater{}} \FunctionTok{submitValue}\OperatorTok{(}\NormalTok{value}\OperatorTok{,} \BuiltInTok{Pipe}\OperatorTok{::}\NormalTok{submitAsync}\OperatorTok{));}
    \OperatorTok{\}}

    \AttributeTok{@Override}
    \KeywordTok{public} \DataTypeTok{void} \FunctionTok{submit}\OperatorTok{(}\NormalTok{T value}\OperatorTok{)} \OperatorTok{\{}
        \DataTypeTok{var}\NormalTok{ future }\OperatorTok{=}\NormalTok{ executorService}\OperatorTok{.}\FunctionTok{submit}\OperatorTok{(()} \OperatorTok{{-}\textgreater{}} \KeywordTok{super}\OperatorTok{.}\FunctionTok{submit}\OperatorTok{(}\NormalTok{value}\OperatorTok{));}

        \ControlFlowTok{try} \OperatorTok{\{}
\NormalTok{            future}\OperatorTok{.}\FunctionTok{get}\OperatorTok{();}
        \OperatorTok{\}} \ControlFlowTok{catch} \OperatorTok{(}\BuiltInTok{InterruptedException} \OperatorTok{|} \BuiltInTok{ExecutionException}\NormalTok{ e}\OperatorTok{)} \OperatorTok{\{}
            \ControlFlowTok{throw} \KeywordTok{new} \BuiltInTok{RuntimeException}\OperatorTok{(}\NormalTok{e}\OperatorTok{);}
        \OperatorTok{\}}
    \OperatorTok{\}}
\OperatorTok{\}}
\end{Highlighting}
\end{Shaded}

    \textbf{Klasa \texttt{Box}} została zaimplementowana ze względu na
następującą uwagę:

\begin{quote}
Procesy przetwarzają dane w miejscu, po czym przechodzą do kolejnej
komórki bufora i znowu przetwarzają ją w miejscu
\end{quote}

interpretuję to tak, że żaden proces (wątek, pipe) nie powinien tworzyć
kopii danych otrzymanych na wejściu. Zamiast tego, musi operować na już
istniejących danych (mówiąc w kontekście Javy: referencja na obiekt
wczytany przez konsumenta musi pozostać taka sama przez cały czas
przetwarzania tego obiektu przez poszczególne pipe'y).

Biorąc pod uwagę to, iż przetwarzać będziemy \texttt{Integer}, musimy
zadbać o zachowanie w/w warunku.

\begin{Shaded}
\begin{Highlighting}[]
\CommentTok{// Box.java}

\KeywordTok{package}\ImportTok{ pl}\OperatorTok{.}\ImportTok{edu}\OperatorTok{.}\ImportTok{agh}\OperatorTok{.}\ImportTok{tw}\OperatorTok{.}\ImportTok{knapp}\OperatorTok{.}\ImportTok{pipeline}\OperatorTok{;}

\KeywordTok{public} \KeywordTok{class} \BuiltInTok{Box}\OperatorTok{\textless{}}\NormalTok{T}\OperatorTok{\textgreater{}} \OperatorTok{\{}
    \KeywordTok{private}\NormalTok{ T value}\OperatorTok{;}

    \KeywordTok{public} \BuiltInTok{Box}\OperatorTok{()} \OperatorTok{\{}
        \CommentTok{// empty}
    \OperatorTok{\}}

    \KeywordTok{public} \BuiltInTok{Box}\OperatorTok{(}\NormalTok{T value}\OperatorTok{)} \OperatorTok{\{}
        \KeywordTok{this}\OperatorTok{.}\FunctionTok{value} \OperatorTok{=}\NormalTok{ value}\OperatorTok{;}
    \OperatorTok{\}}

    \KeywordTok{public}\NormalTok{ T }\FunctionTok{getValue}\OperatorTok{()} \OperatorTok{\{}
        \ControlFlowTok{return}\NormalTok{ value}\OperatorTok{;}
    \OperatorTok{\}}

    \KeywordTok{public} \DataTypeTok{void} \FunctionTok{setValue}\OperatorTok{(}\NormalTok{T value}\OperatorTok{)} \OperatorTok{\{}
        \KeywordTok{this}\OperatorTok{.}\FunctionTok{value} \OperatorTok{=}\NormalTok{ value}\OperatorTok{;}
    \OperatorTok{\}}

    \AttributeTok{@Override}
    \KeywordTok{public} \BuiltInTok{String} \FunctionTok{toString}\OperatorTok{()} \OperatorTok{\{}
        \ControlFlowTok{return} \StringTok{"Box \{"} \OperatorTok{+}\NormalTok{ value }\OperatorTok{+} \CharTok{\textquotesingle{}\}\textquotesingle{}}\OperatorTok{;}
    \OperatorTok{\}}
\OperatorTok{\}}
\end{Highlighting}
\end{Shaded}

\begin{itemize}
\tightlist
\item
  Producent tworzy instancję klasy \texttt{Box} i zapisuje ją do bufora
\item
  Konsument odczytuje i przekazuje do pipe'ów
\item
  Poszczególne pipe'y modyfikują wartość zawartą w tym obiekcie zamiast
  tworzenia nowego obiektu
\end{itemize}

    \textbf{Klasa \texttt{Main}} zawiera przykładowe korzystanie ze
stworzonego mechanizmu:

\begin{Shaded}
\begin{Highlighting}[]
\KeywordTok{package}\ImportTok{ pl}\OperatorTok{.}\ImportTok{edu}\OperatorTok{.}\ImportTok{agh}\OperatorTok{.}\ImportTok{tw}\OperatorTok{.}\ImportTok{knapp}\OperatorTok{.}\ImportTok{pipeline}\OperatorTok{;}

\KeywordTok{public} \KeywordTok{class}\NormalTok{ Main }\OperatorTok{\{}
    \KeywordTok{public} \DataTypeTok{static} \DataTypeTok{void} \FunctionTok{main}\OperatorTok{(}\BuiltInTok{String}\OperatorTok{[]}\NormalTok{ args}\OperatorTok{)} \KeywordTok{throws} \BuiltInTok{InterruptedException} \OperatorTok{\{}
        \DataTypeTok{var}\NormalTok{ buffer }\OperatorTok{=} \KeywordTok{new}\NormalTok{ SemaphoreBuffer}\OperatorTok{\textless{}}\BuiltInTok{Box}\OperatorTok{\textless{}}\BuiltInTok{Integer}\OperatorTok{\textgreater{}\textgreater{}(}\DecValTok{100}\OperatorTok{);}
        \DataTypeTok{var}\NormalTok{ producer }\OperatorTok{=} \KeywordTok{new} \FunctionTok{Producer}\OperatorTok{(}\NormalTok{buffer}\OperatorTok{,} \DecValTok{0}\OperatorTok{,} \DecValTok{50}\OperatorTok{);}
        \DataTypeTok{var}\NormalTok{ consumer }\OperatorTok{=} \KeywordTok{new} \FunctionTok{Consumer}\OperatorTok{(}\NormalTok{buffer}\OperatorTok{,} \DecValTok{0}\OperatorTok{,} \DecValTok{100}\OperatorTok{);}

\NormalTok{        PipeAction}\OperatorTok{\textless{}}\BuiltInTok{Box}\OperatorTok{\textless{}}\BuiltInTok{Integer}\OperatorTok{\textgreater{}\textgreater{}}\NormalTok{ resultAction }\OperatorTok{=}\NormalTok{ result }\OperatorTok{{-}\textgreater{}} \OperatorTok{\{}
            \BuiltInTok{System}\OperatorTok{.}\FunctionTok{out}\OperatorTok{.}\FunctionTok{println}\OperatorTok{(}\NormalTok{result}\OperatorTok{);}

            \ControlFlowTok{if} \OperatorTok{(}\NormalTok{result}\OperatorTok{.}\FunctionTok{getValue}\OperatorTok{()} \OperatorTok{==} \DecValTok{105}\OperatorTok{)} \OperatorTok{\{}
\NormalTok{                consumer}\OperatorTok{.}\FunctionTok{pipe}\OperatorTok{().}\FunctionTok{closeAll}\OperatorTok{();}
            \OperatorTok{\}}
        \OperatorTok{\};}

\NormalTok{        consumer}\OperatorTok{.}\FunctionTok{pipe}\OperatorTok{()}
                \OperatorTok{.}\FunctionTok{then}\OperatorTok{(}\FunctionTok{newAddPipe}\OperatorTok{(}\DecValTok{1}\OperatorTok{))}
                \OperatorTok{.}\FunctionTok{then}\OperatorTok{(}\FunctionTok{newAddPipe}\OperatorTok{(}\DecValTok{1}\OperatorTok{))}
                \OperatorTok{.}\FunctionTok{then}\OperatorTok{(}\FunctionTok{newAddPipe}\OperatorTok{(}\DecValTok{1}\OperatorTok{))}
                \OperatorTok{.}\FunctionTok{then}\OperatorTok{(}\FunctionTok{newAddPipe}\OperatorTok{(}\DecValTok{1}\OperatorTok{))}
                \OperatorTok{.}\FunctionTok{then}\OperatorTok{(}\FunctionTok{newAddPipe}\OperatorTok{(}\DecValTok{2}\OperatorTok{))}
                \OperatorTok{.}\FunctionTok{action}\OperatorTok{(}\NormalTok{resultAction}\OperatorTok{);}

\NormalTok{        producer}\OperatorTok{.}\FunctionTok{start}\OperatorTok{();}
\NormalTok{        consumer}\OperatorTok{.}\FunctionTok{start}\OperatorTok{();}

\NormalTok{        consumer}\OperatorTok{.}\FunctionTok{join}\OperatorTok{();}
\NormalTok{        producer}\OperatorTok{.}\FunctionTok{join}\OperatorTok{();}
    \OperatorTok{\}}

    \CommentTok{/**}
     \CommentTok{*}\NormalTok{ A simple wrapper for }\CommentTok{\textasciigrave{}}\NormalTok{Consumer}\CommentTok{\#}\NormalTok{newPipe}\CommentTok{\textasciigrave{}}\NormalTok{ that returns}
     \CommentTok{*}\NormalTok{ an additive pipe}
\CommentTok{     * @}\NormalTok{return A newly created pipe by the }\CommentTok{\textasciigrave{}}\NormalTok{Consumer}\CommentTok{\textasciigrave{}}\NormalTok{ class}\CommentTok{.}
     \CommentTok{*/}
    \KeywordTok{private} \DataTypeTok{static} \BuiltInTok{Pipe}\OperatorTok{\textless{}}\BuiltInTok{Box}\OperatorTok{\textless{}}\BuiltInTok{Integer}\OperatorTok{\textgreater{}\textgreater{}} \FunctionTok{newAddPipe}\OperatorTok{(}\DataTypeTok{int}\NormalTok{ value}\OperatorTok{)} \OperatorTok{\{}
        \ControlFlowTok{return}\NormalTok{ Consumer}\OperatorTok{.}\FunctionTok{newPipe}\OperatorTok{(}\NormalTok{v }\OperatorTok{{-}\textgreater{}} \OperatorTok{\{}
\NormalTok{            v}\OperatorTok{.}\FunctionTok{setValue}\OperatorTok{(}\NormalTok{v}\OperatorTok{.}\FunctionTok{getValue}\OperatorTok{()} \OperatorTok{+}\NormalTok{ value}\OperatorTok{);}
            \ControlFlowTok{return}\NormalTok{ v}\OperatorTok{;}
        \OperatorTok{\},} \DecValTok{0}\OperatorTok{,} \DecValTok{150}\OperatorTok{);}
    \OperatorTok{\}}

    \CommentTok{/**}
     \CommentTok{*}\NormalTok{ A simple wrapper for }\CommentTok{\textasciigrave{}}\NormalTok{Consumer}\CommentTok{\#}\NormalTok{newPipe}\CommentTok{\textasciigrave{}}\NormalTok{ that returns}
     \CommentTok{*}\NormalTok{ a multiplicative pipe}
\CommentTok{     * @}\NormalTok{return A newly created pipe by the }\CommentTok{\textasciigrave{}}\NormalTok{Consumer}\CommentTok{\textasciigrave{}}\NormalTok{ class}\CommentTok{.}
     \CommentTok{*/}
    \KeywordTok{private} \DataTypeTok{static} \BuiltInTok{Pipe}\OperatorTok{\textless{}}\BuiltInTok{Box}\OperatorTok{\textless{}}\BuiltInTok{Integer}\OperatorTok{\textgreater{}\textgreater{}} \FunctionTok{newMulPipe}\OperatorTok{(}\DataTypeTok{int}\NormalTok{ value}\OperatorTok{)} \OperatorTok{\{}
        \ControlFlowTok{return}\NormalTok{ Consumer}\OperatorTok{.}\FunctionTok{newPipe}\OperatorTok{(}\NormalTok{v }\OperatorTok{{-}\textgreater{}} \OperatorTok{\{}
\NormalTok{            v}\OperatorTok{.}\FunctionTok{setValue}\OperatorTok{(}\NormalTok{v}\OperatorTok{.}\FunctionTok{getValue}\OperatorTok{()} \OperatorTok{*}\NormalTok{ value}\OperatorTok{);}
            \ControlFlowTok{return}\NormalTok{ v}\OperatorTok{;}
        \OperatorTok{\},} \DecValTok{0}\OperatorTok{,} \DecValTok{150}\OperatorTok{);}
    \OperatorTok{\}}
\OperatorTok{\}}
\end{Highlighting}
\end{Shaded}

Jak można łatwo zauważyć, w tym przykładzie korzystając z \texttt{5}
pipe'ów zwiększamy wartość wyprodukowaną przez producenta o \texttt{6}.

Widać również, iż producent produkuje nowe dane co \emph{0-50ms},
konsument pobiera dane co \emph{0-100ms}, a szybkość obróbki danych
przez poszczególne pipe'y wynosi \emph{0-150ms}.

Warto dodać, że aby po wykonaniu wszystkich operacji aplikacja się nie
zawiesiła, musimy zadbać o zamknięcie wszystkich pipe'ów, co jest
pokazane wewnątrz callback'u \texttt{resultAction}.

Oczywiście, możemy tworzyć bardziej zaawansowane pipeline'y, np. możemy
skorzystać z rury mnożącej, która jest tworzona przez metodę
\texttt{newMulPipe}.

\begin{quote}
Uwaga: wyniki zostały omówione w rozdziale \emph{Wyniki}
\end{quote}

\begin{quote}
To zadanie może zostać uruchomione korzystając z polecenia

\begin{Shaded}
\begin{Highlighting}[]
\ExtensionTok{./gradlew}\NormalTok{ run }\AttributeTok{{-}Pmain}\OperatorTok{=}\NormalTok{pl.edu.agh.tw.knapp.pipeline.Main}
\end{Highlighting}
\end{Shaded}
\end{quote}

    \hypertarget{wyniki}{%
\section{Wyniki}\label{wyniki}}

W tym rozdziale są umieszczone wyniki działania poszczególnych zadań.

    \hypertarget{problem-ograniczonego-bufora}{%
\subsection{Problem ograniczonego
bufora}\label{problem-ograniczonego-bufora}}

Poniżej w skrócie zostaną omówione niektóre wyniki.

\hypertarget{producent-10-konsumentuxf3w}{%
\subsubsection{1 producent 10
konsumentów}\label{producent-10-konsumentuxf3w}}

Na wyjściu dostajemy następujący wynik:

\begin{verbatim}
******* producers = 1, consumers = 10, buffer: SemaphoreBuffer *******
[Consumer id 22] 0
[Consumer id 22] 1
[Consumer id 31] 2
[Consumer id 30] 3
[Consumer id 23] 4
[Consumer id 27] 5
[Consumer id 26] 6
[...]
[Consumer id 27] 94
[Consumer id 25] 95
[Consumer id 24] 96
[Consumer id 23] 97
[Consumer id 26] 98
[Consumer id 30] 99
[Consumer id 28] Buffer#get: end reached
[Consumer id 31] Buffer#get: end reached
[Consumer id 29] Buffer#get: end reached
[Consumer id 22] Buffer#get: end reached
[Consumer id 27] Buffer#get: end reached
[Consumer id 24] Buffer#get: end reached
[Consumer id 25] Buffer#get: end reached
[Consumer id 23] Buffer#get: end reached
[Consumer id 26] Buffer#get: end reached
[Consumer id 30] Buffer#get: end reached
\end{verbatim}

\textbf{Interpretacja wyniku:} ze względu na to, że w tym przypadku mamy
tylko 1 producenta, który wyprodukował 100 wartości od \texttt{0} do
\texttt{99}, i 10 konsumentów, każdy z których próbuje odczytać 100
wartości, w sposób oczywisty dochodzimy do pewnego problemu, a
mianowicie dla konsumentów zabrakło danych. Właśnie z tego powodu
widzimy 10 komunikatów od 10 różnych konsumentów, że bufor już jest
pusty (został przekroczony czas oczekiwania na kolejną wartość).

\hypertarget{producentuxf3w-1-konsument}{%
\subsubsection{10 producentów, 1
konsument}\label{producentuxf3w-1-konsument}}

\begin{verbatim}
******* producers = 10, consumers = 1, buffer: SemaphoreBuffer *******
[Consumer id 22] 0
[Consumer id 22] 0
[Consumer id 22] 0
[Consumer id 22] 1
[...]
[Consumer id 22] 10
[Consumer id 22] 9
[Consumer id 22] 9
[Consumer id 22] 10
[Producer id 26] Buffer#put error
[Producer id 29] Buffer#put error
[Producer id 23] Buffer#put error
[Producer id 24] Buffer#put error
[Producer id 28] Buffer#put error
[Producer id 25] Buffer#put error
[Producer id 30] Buffer#put error
[Producer id 31] Buffer#put error
[Producer id 32] Buffer#put error
[Producer id 27] Buffer#put error
\end{verbatim}

\textbf{Interpretacja wyniku:} otrzymane komunikaty są spowodowane tym,
iż producentów jest więcej od konsumentów, a więc wyprodukowanych
wartości jest więcej niż konsument jest w stanie przetworzyć.

    \hypertarget{przetwarzanie-potokowe-z-buforem}{%
\subsection{Przetwarzanie potokowe z
buforem}\label{przetwarzanie-potokowe-z-buforem}}

Po uruchomieniu zaprezentowanego przykładu na wyjściu otrzymujemy:

\begin{verbatim}
[Producer id 22] Started
[Consumer id 23] Started
Box {6}
Box {7}
Box {8}
[...]
Box {24}
Box {25}
Box {26}
[Producer id 22] Done
Box {27}
Box {28}
Box {29}
[...]
Box {73}
Box {74}
Box {75}
[Consumer id 23] Buffer#get: end reached
[Consumer id 23] Done
Box {76}
Box {77}
Box {78}
[...]
Box {103}
Box {104}
Box {105}
\end{verbatim}

Z czego można wyciągnąć następujące wnioski:

\begin{itemize}
\tightlist
\item
  Producent skończył swoje działanie jako pierwszy
\item
  Konsument skończył swoje działanie zanim wszystkie dane zostały
  przetworzone
\item
  Dane wyprodukowane przez producenta (\texttt{0..99}) zostały
  powiększone o \texttt{6} (\texttt{6..105})
\item
  Aplikacja zakończyła swoje działanie po przetworzeniu i wypisaniu
  ostatniej obrobionej liczby (\texttt{105})
\end{itemize}

    \hypertarget{wnioski}{%
\section{Wnioski}\label{wnioski}}

\begin{itemize}
\tightlist
\item
  Problem producentów-konsumentów (producer-consumer problem,
  bounded-buffer problem) da się rozwiązać korzystając z różnych
  mechanizmów synchronizacji, m.in. semaforów i monitorów
\item
  Przetwarzanie potokowe umożliwia niezależną wielowątkową obróbkę
  porcji danych, co ma pozytywny wpływ na wydajność
\item
  Pipeline (tzw. ``rurociąg'') składa się z łańcucha elementów
  przetwarzających (procesów, wątków, koprocedur, funkcji itp.),
  ułożonych tak, że wyjście każdego elementu jest wejściem następnego
\item
  Pipeline jest to łańcuch, składający się ze źródła, operacji
  pośrednich oraz operacji końcowej. W zaprezentowanej implementacji
  źródłem jest rura początkowa, tj. \texttt{Consumer\#pipe}, operacje
  pośrednie to są poszczególne pipe'y (inkrementujący, mnożący),
  operacją końcową jest callback, prezentowany przez interfejs
  \texttt{PipeAction}
\item
  Szybkość obróbki \emph{nowych} danych w \texttt{i}-tej rurze jest
  uzależniona tylko od szybkości nadchodzenia tych danych, od liczby
  danych już oczekujących na obróbkę, oraz od własnej szybkości
  przetwarzania jednej porcji danych przez daną rurę
\item
  Zaczynając od Java 8, w celu potokowego przetwarzania danych, możemy
  również korzystać z Java Stream API
\end{itemize}

    \hypertarget{bibliografia}{%
\section{Bibliografia}\label{bibliografia}}

\begin{enumerate}
\def\labelenumi{\arabic{enumi}.}
\tightlist
\item
  \href{https://home.agh.edu.pl/~funika/tw/lab3/}{Materiały do
  laboratorium}
\item
  \href{https://pl.wikipedia.org/wiki/Problem_producenta_i_konsumenta}{Wikipedia
  - Problem producenta i konsumenta}
\item
  \href{https://en.wikipedia.org/wiki/Producer\%E2\%80\%93consumer_problem}{Wikipedia
  - Producer-consumer problem}
\item
  \href{https://docs.oracle.com/en/java/javase/17/docs/api/java.base/java/util/concurrent/ExecutorService.html}{Java
  17 Docs - ExecutorService}
\item
  \href{https://www.baeldung.com/java-executor-service-tutorial}{Baeldung
  - A Guide to the Java ExecutorService}
\item
  \href{https://www.baeldung.com/java-8-streams}{Baeldung - The Java 8
  Stream API Tutorial}
\end{enumerate}


    % Add a bibliography block to the postdoc
    
    
    
\end{document}
