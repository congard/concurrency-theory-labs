\documentclass[11pt]{article}

    \usepackage[breakable]{tcolorbox}
    \usepackage{parskip} % Stop auto-indenting (to mimic markdown behaviour)
    

    % Basic figure setup, for now with no caption control since it's done
    % automatically by Pandoc (which extracts ![](path) syntax from Markdown).
    \usepackage{graphicx}
    % Maintain compatibility with old templates. Remove in nbconvert 6.0
    \let\Oldincludegraphics\includegraphics
    % Ensure that by default, figures have no caption (until we provide a
    % proper Figure object with a Caption API and a way to capture that
    % in the conversion process - todo).
    \usepackage{caption}
    \DeclareCaptionFormat{nocaption}{}
    \captionsetup{format=nocaption,aboveskip=0pt,belowskip=0pt}

    \usepackage{float}
    \floatplacement{figure}{H} % forces figures to be placed at the correct location
    \usepackage{xcolor} % Allow colors to be defined
    \usepackage{enumerate} % Needed for markdown enumerations to work
    \usepackage{geometry} % Used to adjust the document margins
    \usepackage{amsmath} % Equations
    \usepackage{amssymb} % Equations
    \usepackage{textcomp} % defines textquotesingle
    % Hack from http://tex.stackexchange.com/a/47451/13684:
    \AtBeginDocument{%
        \def\PYZsq{\textquotesingle}% Upright quotes in Pygmentized code
    }
    \usepackage{upquote} % Upright quotes for verbatim code
    \usepackage{eurosym} % defines \euro

    \usepackage{iftex}
    \ifPDFTeX
        \usepackage[T1]{fontenc}
        \IfFileExists{alphabeta.sty}{
              \usepackage{alphabeta}
          }{
              \usepackage[mathletters]{ucs}
              \usepackage[utf8x]{inputenc}
          }
    \else
        \usepackage{fontspec}
        \usepackage{unicode-math}
    \fi

    \usepackage{fancyvrb} % verbatim replacement that allows latex
    \usepackage{grffile} % extends the file name processing of package graphics
                         % to support a larger range
    \makeatletter % fix for old versions of grffile with XeLaTeX
    \@ifpackagelater{grffile}{2019/11/01}
    {
      % Do nothing on new versions
    }
    {
      \def\Gread@@xetex#1{%
        \IfFileExists{"\Gin@base".bb}%
        {\Gread@eps{\Gin@base.bb}}%
        {\Gread@@xetex@aux#1}%
      }
    }
    \makeatother
    \usepackage[Export]{adjustbox} % Used to constrain images to a maximum size
    \adjustboxset{max size={0.9\linewidth}{0.9\paperheight}}

    % The hyperref package gives us a pdf with properly built
    % internal navigation ('pdf bookmarks' for the table of contents,
    % internal cross-reference links, web links for URLs, etc.)
    \usepackage{hyperref}
    % The default LaTeX title has an obnoxious amount of whitespace. By default,
    % titling removes some of it. It also provides customization options.
    \usepackage{titling}
    \usepackage{longtable} % longtable support required by pandoc >1.10
    \usepackage{booktabs}  % table support for pandoc > 1.12.2
    \usepackage{array}     % table support for pandoc >= 2.11.3
    \usepackage{calc}      % table minipage width calculation for pandoc >= 2.11.1
    \usepackage[inline]{enumitem} % IRkernel/repr support (it uses the enumerate* environment)
    \usepackage[normalem]{ulem} % ulem is needed to support strikethroughs (\sout)
                                % normalem makes italics be italics, not underlines
    \usepackage{soul}      % strikethrough (\st) support for pandoc >= 3.0.0
    \usepackage{mathrsfs}
    

    
    % Colors for the hyperref package
    \definecolor{urlcolor}{rgb}{0,.145,.698}
    \definecolor{linkcolor}{rgb}{.71,0.21,0.01}
    \definecolor{citecolor}{rgb}{.12,.54,.11}

    % ANSI colors
    \definecolor{ansi-black}{HTML}{3E424D}
    \definecolor{ansi-black-intense}{HTML}{282C36}
    \definecolor{ansi-red}{HTML}{E75C58}
    \definecolor{ansi-red-intense}{HTML}{B22B31}
    \definecolor{ansi-green}{HTML}{00A250}
    \definecolor{ansi-green-intense}{HTML}{007427}
    \definecolor{ansi-yellow}{HTML}{DDB62B}
    \definecolor{ansi-yellow-intense}{HTML}{B27D12}
    \definecolor{ansi-blue}{HTML}{208FFB}
    \definecolor{ansi-blue-intense}{HTML}{0065CA}
    \definecolor{ansi-magenta}{HTML}{D160C4}
    \definecolor{ansi-magenta-intense}{HTML}{A03196}
    \definecolor{ansi-cyan}{HTML}{60C6C8}
    \definecolor{ansi-cyan-intense}{HTML}{258F8F}
    \definecolor{ansi-white}{HTML}{C5C1B4}
    \definecolor{ansi-white-intense}{HTML}{A1A6B2}
    \definecolor{ansi-default-inverse-fg}{HTML}{FFFFFF}
    \definecolor{ansi-default-inverse-bg}{HTML}{000000}

    % common color for the border for error outputs.
    \definecolor{outerrorbackground}{HTML}{FFDFDF}

    % commands and environments needed by pandoc snippets
    % extracted from the output of `pandoc -s`
    \providecommand{\tightlist}{%
      \setlength{\itemsep}{0pt}\setlength{\parskip}{0pt}}
    \DefineVerbatimEnvironment{Highlighting}{Verbatim}{commandchars=\\\{\}}
    % Add ',fontsize=\small' for more characters per line
    \newenvironment{Shaded}{}{}
    \newcommand{\KeywordTok}[1]{\textcolor[rgb]{0.00,0.44,0.13}{\textbf{{#1}}}}
    \newcommand{\DataTypeTok}[1]{\textcolor[rgb]{0.56,0.13,0.00}{{#1}}}
    \newcommand{\DecValTok}[1]{\textcolor[rgb]{0.25,0.63,0.44}{{#1}}}
    \newcommand{\BaseNTok}[1]{\textcolor[rgb]{0.25,0.63,0.44}{{#1}}}
    \newcommand{\FloatTok}[1]{\textcolor[rgb]{0.25,0.63,0.44}{{#1}}}
    \newcommand{\CharTok}[1]{\textcolor[rgb]{0.25,0.44,0.63}{{#1}}}
    \newcommand{\StringTok}[1]{\textcolor[rgb]{0.25,0.44,0.63}{{#1}}}
    \newcommand{\CommentTok}[1]{\textcolor[rgb]{0.38,0.63,0.69}{\textit{{#1}}}}
    \newcommand{\OtherTok}[1]{\textcolor[rgb]{0.00,0.44,0.13}{{#1}}}
    \newcommand{\AlertTok}[1]{\textcolor[rgb]{1.00,0.00,0.00}{\textbf{{#1}}}}
    \newcommand{\FunctionTok}[1]{\textcolor[rgb]{0.02,0.16,0.49}{{#1}}}
    \newcommand{\RegionMarkerTok}[1]{{#1}}
    \newcommand{\ErrorTok}[1]{\textcolor[rgb]{1.00,0.00,0.00}{\textbf{{#1}}}}
    \newcommand{\NormalTok}[1]{{#1}}

    % Additional commands for more recent versions of Pandoc
    \newcommand{\ConstantTok}[1]{\textcolor[rgb]{0.53,0.00,0.00}{{#1}}}
    \newcommand{\SpecialCharTok}[1]{\textcolor[rgb]{0.25,0.44,0.63}{{#1}}}
    \newcommand{\VerbatimStringTok}[1]{\textcolor[rgb]{0.25,0.44,0.63}{{#1}}}
    \newcommand{\SpecialStringTok}[1]{\textcolor[rgb]{0.73,0.40,0.53}{{#1}}}
    \newcommand{\ImportTok}[1]{{#1}}
    \newcommand{\DocumentationTok}[1]{\textcolor[rgb]{0.73,0.13,0.13}{\textit{{#1}}}}
    \newcommand{\AnnotationTok}[1]{\textcolor[rgb]{0.38,0.63,0.69}{\textbf{\textit{{#1}}}}}
    \newcommand{\CommentVarTok}[1]{\textcolor[rgb]{0.38,0.63,0.69}{\textbf{\textit{{#1}}}}}
    \newcommand{\VariableTok}[1]{\textcolor[rgb]{0.10,0.09,0.49}{{#1}}}
    \newcommand{\ControlFlowTok}[1]{\textcolor[rgb]{0.00,0.44,0.13}{\textbf{{#1}}}}
    \newcommand{\OperatorTok}[1]{\textcolor[rgb]{0.40,0.40,0.40}{{#1}}}
    \newcommand{\BuiltInTok}[1]{{#1}}
    \newcommand{\ExtensionTok}[1]{{#1}}
    \newcommand{\PreprocessorTok}[1]{\textcolor[rgb]{0.74,0.48,0.00}{{#1}}}
    \newcommand{\AttributeTok}[1]{\textcolor[rgb]{0.49,0.56,0.16}{{#1}}}
    \newcommand{\InformationTok}[1]{\textcolor[rgb]{0.38,0.63,0.69}{\textbf{\textit{{#1}}}}}
    \newcommand{\WarningTok}[1]{\textcolor[rgb]{0.38,0.63,0.69}{\textbf{\textit{{#1}}}}}


    % Define a nice break command that doesn't care if a line doesn't already
    % exist.
    \def\br{\hspace*{\fill} \\* }
    % Math Jax compatibility definitions
    \def\gt{>}
    \def\lt{<}
    \let\Oldtex\TeX
    \let\Oldlatex\LaTeX
    \renewcommand{\TeX}{\textrm{\Oldtex}}
    \renewcommand{\LaTeX}{\textrm{\Oldlatex}}
    % Document parameters
    % Document title
    \title{lab13}
    
    
    
    
    
    
    
% Pygments definitions
\makeatletter
\def\PY@reset{\let\PY@it=\relax \let\PY@bf=\relax%
    \let\PY@ul=\relax \let\PY@tc=\relax%
    \let\PY@bc=\relax \let\PY@ff=\relax}
\def\PY@tok#1{\csname PY@tok@#1\endcsname}
\def\PY@toks#1+{\ifx\relax#1\empty\else%
    \PY@tok{#1}\expandafter\PY@toks\fi}
\def\PY@do#1{\PY@bc{\PY@tc{\PY@ul{%
    \PY@it{\PY@bf{\PY@ff{#1}}}}}}}
\def\PY#1#2{\PY@reset\PY@toks#1+\relax+\PY@do{#2}}

\@namedef{PY@tok@w}{\def\PY@tc##1{\textcolor[rgb]{0.73,0.73,0.73}{##1}}}
\@namedef{PY@tok@c}{\let\PY@it=\textit\def\PY@tc##1{\textcolor[rgb]{0.24,0.48,0.48}{##1}}}
\@namedef{PY@tok@cp}{\def\PY@tc##1{\textcolor[rgb]{0.61,0.40,0.00}{##1}}}
\@namedef{PY@tok@k}{\let\PY@bf=\textbf\def\PY@tc##1{\textcolor[rgb]{0.00,0.50,0.00}{##1}}}
\@namedef{PY@tok@kp}{\def\PY@tc##1{\textcolor[rgb]{0.00,0.50,0.00}{##1}}}
\@namedef{PY@tok@kt}{\def\PY@tc##1{\textcolor[rgb]{0.69,0.00,0.25}{##1}}}
\@namedef{PY@tok@o}{\def\PY@tc##1{\textcolor[rgb]{0.40,0.40,0.40}{##1}}}
\@namedef{PY@tok@ow}{\let\PY@bf=\textbf\def\PY@tc##1{\textcolor[rgb]{0.67,0.13,1.00}{##1}}}
\@namedef{PY@tok@nb}{\def\PY@tc##1{\textcolor[rgb]{0.00,0.50,0.00}{##1}}}
\@namedef{PY@tok@nf}{\def\PY@tc##1{\textcolor[rgb]{0.00,0.00,1.00}{##1}}}
\@namedef{PY@tok@nc}{\let\PY@bf=\textbf\def\PY@tc##1{\textcolor[rgb]{0.00,0.00,1.00}{##1}}}
\@namedef{PY@tok@nn}{\let\PY@bf=\textbf\def\PY@tc##1{\textcolor[rgb]{0.00,0.00,1.00}{##1}}}
\@namedef{PY@tok@ne}{\let\PY@bf=\textbf\def\PY@tc##1{\textcolor[rgb]{0.80,0.25,0.22}{##1}}}
\@namedef{PY@tok@nv}{\def\PY@tc##1{\textcolor[rgb]{0.10,0.09,0.49}{##1}}}
\@namedef{PY@tok@no}{\def\PY@tc##1{\textcolor[rgb]{0.53,0.00,0.00}{##1}}}
\@namedef{PY@tok@nl}{\def\PY@tc##1{\textcolor[rgb]{0.46,0.46,0.00}{##1}}}
\@namedef{PY@tok@ni}{\let\PY@bf=\textbf\def\PY@tc##1{\textcolor[rgb]{0.44,0.44,0.44}{##1}}}
\@namedef{PY@tok@na}{\def\PY@tc##1{\textcolor[rgb]{0.41,0.47,0.13}{##1}}}
\@namedef{PY@tok@nt}{\let\PY@bf=\textbf\def\PY@tc##1{\textcolor[rgb]{0.00,0.50,0.00}{##1}}}
\@namedef{PY@tok@nd}{\def\PY@tc##1{\textcolor[rgb]{0.67,0.13,1.00}{##1}}}
\@namedef{PY@tok@s}{\def\PY@tc##1{\textcolor[rgb]{0.73,0.13,0.13}{##1}}}
\@namedef{PY@tok@sd}{\let\PY@it=\textit\def\PY@tc##1{\textcolor[rgb]{0.73,0.13,0.13}{##1}}}
\@namedef{PY@tok@si}{\let\PY@bf=\textbf\def\PY@tc##1{\textcolor[rgb]{0.64,0.35,0.47}{##1}}}
\@namedef{PY@tok@se}{\let\PY@bf=\textbf\def\PY@tc##1{\textcolor[rgb]{0.67,0.36,0.12}{##1}}}
\@namedef{PY@tok@sr}{\def\PY@tc##1{\textcolor[rgb]{0.64,0.35,0.47}{##1}}}
\@namedef{PY@tok@ss}{\def\PY@tc##1{\textcolor[rgb]{0.10,0.09,0.49}{##1}}}
\@namedef{PY@tok@sx}{\def\PY@tc##1{\textcolor[rgb]{0.00,0.50,0.00}{##1}}}
\@namedef{PY@tok@m}{\def\PY@tc##1{\textcolor[rgb]{0.40,0.40,0.40}{##1}}}
\@namedef{PY@tok@gh}{\let\PY@bf=\textbf\def\PY@tc##1{\textcolor[rgb]{0.00,0.00,0.50}{##1}}}
\@namedef{PY@tok@gu}{\let\PY@bf=\textbf\def\PY@tc##1{\textcolor[rgb]{0.50,0.00,0.50}{##1}}}
\@namedef{PY@tok@gd}{\def\PY@tc##1{\textcolor[rgb]{0.63,0.00,0.00}{##1}}}
\@namedef{PY@tok@gi}{\def\PY@tc##1{\textcolor[rgb]{0.00,0.52,0.00}{##1}}}
\@namedef{PY@tok@gr}{\def\PY@tc##1{\textcolor[rgb]{0.89,0.00,0.00}{##1}}}
\@namedef{PY@tok@ge}{\let\PY@it=\textit}
\@namedef{PY@tok@gs}{\let\PY@bf=\textbf}
\@namedef{PY@tok@ges}{\let\PY@bf=\textbf\let\PY@it=\textit}
\@namedef{PY@tok@gp}{\let\PY@bf=\textbf\def\PY@tc##1{\textcolor[rgb]{0.00,0.00,0.50}{##1}}}
\@namedef{PY@tok@go}{\def\PY@tc##1{\textcolor[rgb]{0.44,0.44,0.44}{##1}}}
\@namedef{PY@tok@gt}{\def\PY@tc##1{\textcolor[rgb]{0.00,0.27,0.87}{##1}}}
\@namedef{PY@tok@err}{\def\PY@bc##1{{\setlength{\fboxsep}{\string -\fboxrule}\fcolorbox[rgb]{1.00,0.00,0.00}{1,1,1}{\strut ##1}}}}
\@namedef{PY@tok@kc}{\let\PY@bf=\textbf\def\PY@tc##1{\textcolor[rgb]{0.00,0.50,0.00}{##1}}}
\@namedef{PY@tok@kd}{\let\PY@bf=\textbf\def\PY@tc##1{\textcolor[rgb]{0.00,0.50,0.00}{##1}}}
\@namedef{PY@tok@kn}{\let\PY@bf=\textbf\def\PY@tc##1{\textcolor[rgb]{0.00,0.50,0.00}{##1}}}
\@namedef{PY@tok@kr}{\let\PY@bf=\textbf\def\PY@tc##1{\textcolor[rgb]{0.00,0.50,0.00}{##1}}}
\@namedef{PY@tok@bp}{\def\PY@tc##1{\textcolor[rgb]{0.00,0.50,0.00}{##1}}}
\@namedef{PY@tok@fm}{\def\PY@tc##1{\textcolor[rgb]{0.00,0.00,1.00}{##1}}}
\@namedef{PY@tok@vc}{\def\PY@tc##1{\textcolor[rgb]{0.10,0.09,0.49}{##1}}}
\@namedef{PY@tok@vg}{\def\PY@tc##1{\textcolor[rgb]{0.10,0.09,0.49}{##1}}}
\@namedef{PY@tok@vi}{\def\PY@tc##1{\textcolor[rgb]{0.10,0.09,0.49}{##1}}}
\@namedef{PY@tok@vm}{\def\PY@tc##1{\textcolor[rgb]{0.10,0.09,0.49}{##1}}}
\@namedef{PY@tok@sa}{\def\PY@tc##1{\textcolor[rgb]{0.73,0.13,0.13}{##1}}}
\@namedef{PY@tok@sb}{\def\PY@tc##1{\textcolor[rgb]{0.73,0.13,0.13}{##1}}}
\@namedef{PY@tok@sc}{\def\PY@tc##1{\textcolor[rgb]{0.73,0.13,0.13}{##1}}}
\@namedef{PY@tok@dl}{\def\PY@tc##1{\textcolor[rgb]{0.73,0.13,0.13}{##1}}}
\@namedef{PY@tok@s2}{\def\PY@tc##1{\textcolor[rgb]{0.73,0.13,0.13}{##1}}}
\@namedef{PY@tok@sh}{\def\PY@tc##1{\textcolor[rgb]{0.73,0.13,0.13}{##1}}}
\@namedef{PY@tok@s1}{\def\PY@tc##1{\textcolor[rgb]{0.73,0.13,0.13}{##1}}}
\@namedef{PY@tok@mb}{\def\PY@tc##1{\textcolor[rgb]{0.40,0.40,0.40}{##1}}}
\@namedef{PY@tok@mf}{\def\PY@tc##1{\textcolor[rgb]{0.40,0.40,0.40}{##1}}}
\@namedef{PY@tok@mh}{\def\PY@tc##1{\textcolor[rgb]{0.40,0.40,0.40}{##1}}}
\@namedef{PY@tok@mi}{\def\PY@tc##1{\textcolor[rgb]{0.40,0.40,0.40}{##1}}}
\@namedef{PY@tok@il}{\def\PY@tc##1{\textcolor[rgb]{0.40,0.40,0.40}{##1}}}
\@namedef{PY@tok@mo}{\def\PY@tc##1{\textcolor[rgb]{0.40,0.40,0.40}{##1}}}
\@namedef{PY@tok@ch}{\let\PY@it=\textit\def\PY@tc##1{\textcolor[rgb]{0.24,0.48,0.48}{##1}}}
\@namedef{PY@tok@cm}{\let\PY@it=\textit\def\PY@tc##1{\textcolor[rgb]{0.24,0.48,0.48}{##1}}}
\@namedef{PY@tok@cpf}{\let\PY@it=\textit\def\PY@tc##1{\textcolor[rgb]{0.24,0.48,0.48}{##1}}}
\@namedef{PY@tok@c1}{\let\PY@it=\textit\def\PY@tc##1{\textcolor[rgb]{0.24,0.48,0.48}{##1}}}
\@namedef{PY@tok@cs}{\let\PY@it=\textit\def\PY@tc##1{\textcolor[rgb]{0.24,0.48,0.48}{##1}}}

\def\PYZbs{\char`\\}
\def\PYZus{\char`\_}
\def\PYZob{\char`\{}
\def\PYZcb{\char`\}}
\def\PYZca{\char`\^}
\def\PYZam{\char`\&}
\def\PYZlt{\char`\<}
\def\PYZgt{\char`\>}
\def\PYZsh{\char`\#}
\def\PYZpc{\char`\%}
\def\PYZdl{\char`\$}
\def\PYZhy{\char`\-}
\def\PYZsq{\char`\'}
\def\PYZdq{\char`\"}
\def\PYZti{\char`\~}
% for compatibility with earlier versions
\def\PYZat{@}
\def\PYZlb{[}
\def\PYZrb{]}
\makeatother


    % For linebreaks inside Verbatim environment from package fancyvrb.
    \makeatletter
        \newbox\Wrappedcontinuationbox
        \newbox\Wrappedvisiblespacebox
        \newcommand*\Wrappedvisiblespace {\textcolor{red}{\textvisiblespace}}
        \newcommand*\Wrappedcontinuationsymbol {\textcolor{red}{\llap{\tiny$\m@th\hookrightarrow$}}}
        \newcommand*\Wrappedcontinuationindent {3ex }
        \newcommand*\Wrappedafterbreak {\kern\Wrappedcontinuationindent\copy\Wrappedcontinuationbox}
        % Take advantage of the already applied Pygments mark-up to insert
        % potential linebreaks for TeX processing.
        %        {, <, #, %, $, ' and ": go to next line.
        %        _, }, ^, &, >, - and ~: stay at end of broken line.
        % Use of \textquotesingle for straight quote.
        \newcommand*\Wrappedbreaksatspecials {%
            \def\PYGZus{\discretionary{\char`\_}{\Wrappedafterbreak}{\char`\_}}%
            \def\PYGZob{\discretionary{}{\Wrappedafterbreak\char`\{}{\char`\{}}%
            \def\PYGZcb{\discretionary{\char`\}}{\Wrappedafterbreak}{\char`\}}}%
            \def\PYGZca{\discretionary{\char`\^}{\Wrappedafterbreak}{\char`\^}}%
            \def\PYGZam{\discretionary{\char`\&}{\Wrappedafterbreak}{\char`\&}}%
            \def\PYGZlt{\discretionary{}{\Wrappedafterbreak\char`\<}{\char`\<}}%
            \def\PYGZgt{\discretionary{\char`\>}{\Wrappedafterbreak}{\char`\>}}%
            \def\PYGZsh{\discretionary{}{\Wrappedafterbreak\char`\#}{\char`\#}}%
            \def\PYGZpc{\discretionary{}{\Wrappedafterbreak\char`\%}{\char`\%}}%
            \def\PYGZdl{\discretionary{}{\Wrappedafterbreak\char`\$}{\char`\$}}%
            \def\PYGZhy{\discretionary{\char`\-}{\Wrappedafterbreak}{\char`\-}}%
            \def\PYGZsq{\discretionary{}{\Wrappedafterbreak\textquotesingle}{\textquotesingle}}%
            \def\PYGZdq{\discretionary{}{\Wrappedafterbreak\char`\"}{\char`\"}}%
            \def\PYGZti{\discretionary{\char`\~}{\Wrappedafterbreak}{\char`\~}}%
        }
        % Some characters . , ; ? ! / are not pygmentized.
        % This macro makes them "active" and they will insert potential linebreaks
        \newcommand*\Wrappedbreaksatpunct {%
            \lccode`\~`\.\lowercase{\def~}{\discretionary{\hbox{\char`\.}}{\Wrappedafterbreak}{\hbox{\char`\.}}}%
            \lccode`\~`\,\lowercase{\def~}{\discretionary{\hbox{\char`\,}}{\Wrappedafterbreak}{\hbox{\char`\,}}}%
            \lccode`\~`\;\lowercase{\def~}{\discretionary{\hbox{\char`\;}}{\Wrappedafterbreak}{\hbox{\char`\;}}}%
            \lccode`\~`\:\lowercase{\def~}{\discretionary{\hbox{\char`\:}}{\Wrappedafterbreak}{\hbox{\char`\:}}}%
            \lccode`\~`\?\lowercase{\def~}{\discretionary{\hbox{\char`\?}}{\Wrappedafterbreak}{\hbox{\char`\?}}}%
            \lccode`\~`\!\lowercase{\def~}{\discretionary{\hbox{\char`\!}}{\Wrappedafterbreak}{\hbox{\char`\!}}}%
            \lccode`\~`\/\lowercase{\def~}{\discretionary{\hbox{\char`\/}}{\Wrappedafterbreak}{\hbox{\char`\/}}}%
            \catcode`\.\active
            \catcode`\,\active
            \catcode`\;\active
            \catcode`\:\active
            \catcode`\?\active
            \catcode`\!\active
            \catcode`\/\active
            \lccode`\~`\~
        }
    \makeatother

    \let\OriginalVerbatim=\Verbatim
    \makeatletter
    \renewcommand{\Verbatim}[1][1]{%
        %\parskip\z@skip
        \sbox\Wrappedcontinuationbox {\Wrappedcontinuationsymbol}%
        \sbox\Wrappedvisiblespacebox {\FV@SetupFont\Wrappedvisiblespace}%
        \def\FancyVerbFormatLine ##1{\hsize\linewidth
            \vtop{\raggedright\hyphenpenalty\z@\exhyphenpenalty\z@
                \doublehyphendemerits\z@\finalhyphendemerits\z@
                \strut ##1\strut}%
        }%
        % If the linebreak is at a space, the latter will be displayed as visible
        % space at end of first line, and a continuation symbol starts next line.
        % Stretch/shrink are however usually zero for typewriter font.
        \def\FV@Space {%
            \nobreak\hskip\z@ plus\fontdimen3\font minus\fontdimen4\font
            \discretionary{\copy\Wrappedvisiblespacebox}{\Wrappedafterbreak}
            {\kern\fontdimen2\font}%
        }%

        % Allow breaks at special characters using \PYG... macros.
        \Wrappedbreaksatspecials
        % Breaks at punctuation characters . , ; ? ! and / need catcode=\active
        \OriginalVerbatim[#1,codes*=\Wrappedbreaksatpunct]%
    }
    \makeatother

    % Exact colors from NB
    \definecolor{incolor}{HTML}{303F9F}
    \definecolor{outcolor}{HTML}{D84315}
    \definecolor{cellborder}{HTML}{CFCFCF}
    \definecolor{cellbackground}{HTML}{F7F7F7}

    % prompt
    \makeatletter
    \newcommand{\boxspacing}{\kern\kvtcb@left@rule\kern\kvtcb@boxsep}
    \makeatother
    \newcommand{\prompt}[4]{
        {\ttfamily\llap{{\color{#2}[#3]:\hspace{3pt}#4}}\vspace{-\baselineskip}}
    }
    

    
    % Prevent overflowing lines due to hard-to-break entities
    \sloppy
    % Setup hyperref package
    \hypersetup{
      breaklinks=true,  % so long urls are correctly broken across lines
      colorlinks=true,
      urlcolor=urlcolor,
      linkcolor=linkcolor,
      citecolor=citecolor,
      }
    % Slightly bigger margins than the latex defaults
    
    \geometry{verbose,tmargin=1in,bmargin=1in,lmargin=1in,rmargin=1in}
    
    

\begin{document}
    
    \begin{titlepage}
        \begin{center}
            \vspace*{1cm}
    
            \textbf{Laboratorium 13}
    
            \vspace{0.5cm}
            Communicating Sequential Processes (CSP)
                
            \vspace{1.5cm}
    
            \textbf{Danylo Knapp}

            \vfill

            \includegraphics[width=0.4\textwidth]{../report-templates/agh-logo.png}
    
            \vfill
                
            Teoria Współbieżności
                
            \vspace{0.8cm}

            Wydział Informatyki\\
            Akademia Górniczo-Hutnicza\\
            im. Stanisława Staszica w Krakowie\\
            14.01.24
                
        \end{center}
    \end{titlepage}
    
    

    
    \hypertarget{treux15bux107-zadania}{%
\section{Treść zadania}\label{treux15bux107-zadania}}

    \hypertarget{cel-ux107wiczenia}{%
\subsection{Cel ćwiczenia}\label{cel-ux107wiczenia}}

Celem niniejszego ćwiczenia jest zapoznanie się z koncepcją CSP oraz
wykonanie rozwiązania problemu ograniczonego bufora z użyciem
oprogramowania realizującego koncepcję CSP.

    \hypertarget{wprowadzenie-teoretyczne}{%
\subsection{Wprowadzenie teoretyczne}\label{wprowadzenie-teoretyczne}}

Teoria komunikujących się sekwencyjnych procesów (CSP) C.A.R. Hoare'a
dostarcza formalne podejście do opisu współbieżności i zbiór technik
projektowania współbieżnych programów. W założeniu procesy współbieżne
nie mają wspólnej przestrzeni adresowej. Proces CSP może być traktowany
jako szczególny rodzaj obiektu typu aktor, w którym:

\begin{itemize}
\item
  procesy nie mają interfejsu metod ani metod, które można wywołać z
  zewnątrz. Metod zatem nie można wywoływać z wątków. Tak więc nie ma
  potrzeby jawnego blokowania;
\item
  procesy komunikują tylko za pomocą czytania i zapisywania danych
  poprzez kanały;
\item
  procesy nie mają tożsamości, a więc do nich nie można jawnie się od-
  woływać. Jednakże kanały umożliwiają komunikację z dowolnym procesem
  na drugim końcu kanału;
\item
  procesy nie muszą pracować w pętli w nieskończoność odbierając
  komunikaty. Mogą pisać i czytać komunikaty na różnych kanałach, jeśli
  zachodzi taka potrzeba.
\end{itemize}

Kanał CSP może być rozumiany jako szczególny rodzaj kanału, przy czym:

\begin{itemize}
\item
  kanały są synchroniczne, a więc nie wspierają wewnętrznego
  buforowania. Można jednak zbudować procesy, które realizują
  buforowanie;
\item
  kanały obsługują tylko odczyt (''?'') i zapis (''!'') jako operacje
  przenoszące dane;
\item
  podstawowym typem kanałów jest \emph{one-to-one}. Mogą łączyć tylko
  jedną parę procesów, pisarza i czytelnika. Można również zdefiniować
  kanały do odczytu i do zapisu z/do wielu procesów.
\end{itemize}

    \hypertarget{plan-ux107wiczenia}{%
\subsection{Plan ćwiczenia}\label{plan-ux107wiczenia}}

Pakiet JCSP, opracowany na University of Kent, to platforma wykonawcza
dla programów współbieżnych w Javie, która wspiera konstrukcje CSP
reprezentowane przez interfejsy, klasy i metody, w tym:

\begin{itemize}
\item
  interfejsy \texttt{ChannelInput} (wsparcie dla odczytu),
  \texttt{ChannelOutput} (wsparcie dla zapisu) i \texttt{Channel}
  (obsługuje obydwie czynności) działają na argumentach typu
  \texttt{Object}, ale specjalne wersje przewidziane są też dla
  argumentów typu \texttt{int}. Główna klasa to \texttt{One2OneChannel},
  która wspiera obsługę jednego czytelnika i jednego pisarza.
\item
  interfejs \texttt{CSProcess} opisuje procesy wspierając tylko metodę
  \texttt{run}. Klasy \texttt{Parallel} and \texttt{Sequence} (i inne)
  mają konstruktory, które przyjmują tablice innych obiektów
  \texttt{CSProcess} i tworzą złożone obiekty (kompozyty).
\item
  operator wyboru \texttt{{[}{]}} jest obsługiwany za pośrednictwem
  klasy \texttt{Alternative}. Konstruktor przyjmuje tablice z elementami
  typu \texttt{Guard}. \texttt{Alternative} wspiera metodę
  \texttt{select}, zwraca ona indeks wskazujący, który z nich może (i
  powinien) być wybrany. Metoda \texttt{fairSelect} działa w ten sam
  sposób, ale zapewnia dodatkowe gwarancje sprawiedliwości - wybiera
  sprawiedliwie spośród wszystkich gotowych alternatyw.
\item
  dodatkowe środki programistyczne w JCSP to \emph{timer} (który
  wykonuje odłożone zapisy i może być również używany do określenia
  time-out'u w \texttt{Alternative}), \texttt{Generate} (generuje
  sekwencje liczb), \texttt{Skip} (która nic nie robi - jedna z prymityw
  CSP), i klasy, które umożliwiają interakcję poprzez GUI.
\end{itemize}

    \hypertarget{zadania}{%
\subsection{Zadania}\label{zadania}}

\hypertarget{zadanie-1}{%
\subsubsection{Zadanie 1}\label{zadanie-1}}

Proszę przeanalizować przykładowe rozwiązanie klasycznej postaci
problemu producentów i konsumentów, zapisane z użyciem JCSP.

\hypertarget{zadanie-2}{%
\subsubsection{Zadanie 2}\label{zadanie-2}}

Zaimplementuj w Javie z użyciem JCSP rozwiązanie problemu producenta i
konsumenta z buforem N-elementowym tak, aby każdy element bufora był
reprezentowany przez odrębny proces (taki wariant ma praktyczne
uzasadnienie w sytuacji, gdy pamięć lokalna procesora wykonującego
proces bufora jest na tyle mała, że mieści tylko jedną porcję).

Uwzględnij dwie możliwości:

\begin{enumerate}
\def\labelenumi{\arabic{enumi}.}
\tightlist
\item
  kolejność umieszczania wyprodukowanych elementów w buforze oraz
  kolejność pobierania nie mają znaczenia
\item
  pobieranie elementów powinno odbywać się w takiej kolejności, w jakiej
  były umieszczane w buforze
\end{enumerate}

Proszę wykonać pomiary wydajności kodu dla obu przypadków.

    \hypertarget{rozwiux105zania}{%
\section{Rozwiązania}\label{rozwiux105zania}}

Struktura projektu wygląda następująco:

    \begin{tcolorbox}[breakable, size=fbox, boxrule=1pt, pad at break*=1mm,colback=cellbackground, colframe=cellborder]
\prompt{In}{incolor}{1}{\boxspacing}
\begin{Verbatim}[commandchars=\\\{\}]
\PY{o}{!}tree\PY{+w}{ }/mnt/data/Studia/5s/TW/labs/tw\PYZhy{}lab13/src/main/java/pl/edu/agh/tw/knapp/lab13\PY{+w}{ }\PYZhy{}L\PY{+w}{ }\PY{l+m}{2}\PY{+w}{ }\PYZhy{}n
\end{Verbatim}
\end{tcolorbox}

    \begin{Verbatim}[commandchars=\\\{\}]
/mnt/data/Studia/5s/TW/labs/tw-lab13/src/main/java/pl/edu/agh/tw/knapp/lab13
    task1
        Consumer.java
        Producer.java
        Task1Main.java
    task1f
        Buffer.java
        Consumer.java
        Producer.java
        Task1fMain.java
    task2
        Buffer.java
        CSPConsumer.java
        CSPProducer.java
        Logger.java
        Portion.java
        Task2aMain.java
        Task2bMain.java
        Timer.java

4 directories, 15 files
    \end{Verbatim}

    Biblioteka JCSP została dołączona do projektu korzystając z narzędzia
Gradle w sposób następujący:

\begin{Shaded}
\begin{Highlighting}[]
\CommentTok{// build.gradle.kts}

\NormalTok{dependencies }\OperatorTok{\{}
    \CommentTok{// [...]}

    \CommentTok{// https://mvnrepository.com/artifact/org.codehaus.jcsp/jcsp}
\NormalTok{    implementation}\OperatorTok{(}\StringTok{"org.codehaus.jcsp:jcsp:1.1{-}rc5"}\OperatorTok{)}
\OperatorTok{\}}
\end{Highlighting}
\end{Shaded}

    \hypertarget{zadanie-1}{%
\subsection{Zadanie 1}\label{zadanie-1}}

    \hypertarget{wersja-podstawowa}{%
\subsubsection{Wersja podstawowa}\label{wersja-podstawowa}}

Wersja podstawowa wygląda w sposób następujący:

    \begin{Shaded}
\begin{Highlighting}[]
\CommentTok{// Consumer.java}

\KeywordTok{package}\ImportTok{ pl}\OperatorTok{.}\ImportTok{edu}\OperatorTok{.}\ImportTok{agh}\OperatorTok{.}\ImportTok{tw}\OperatorTok{.}\ImportTok{knapp}\OperatorTok{.}\ImportTok{lab13}\OperatorTok{.}\ImportTok{task1}\OperatorTok{;}

\KeywordTok{import} \ImportTok{org}\OperatorTok{.}\ImportTok{jcsp}\OperatorTok{.}\ImportTok{lang}\OperatorTok{.}\ImportTok{CSProcess}\OperatorTok{;}
\KeywordTok{import} \ImportTok{org}\OperatorTok{.}\ImportTok{jcsp}\OperatorTok{.}\ImportTok{lang}\OperatorTok{.}\ImportTok{ChannelInputInt}\OperatorTok{;}

\KeywordTok{public} \KeywordTok{class}\NormalTok{ Consumer }\KeywordTok{implements}\NormalTok{ CSProcess }\OperatorTok{\{}
    \KeywordTok{private} \DataTypeTok{final}\NormalTok{ ChannelInputInt channel}\OperatorTok{;}

    \KeywordTok{public} \FunctionTok{Consumer}\OperatorTok{(}\NormalTok{ChannelInputInt in}\OperatorTok{)} \OperatorTok{\{}
\NormalTok{        channel }\OperatorTok{=}\NormalTok{ in}\OperatorTok{;}
    \OperatorTok{\}}

    \AttributeTok{@Override}
    \KeywordTok{public} \DataTypeTok{void} \FunctionTok{run}\OperatorTok{()} \OperatorTok{\{}
        \DataTypeTok{int}\NormalTok{ item }\OperatorTok{=}\NormalTok{ channel}\OperatorTok{.}\FunctionTok{read}\OperatorTok{();}
        \BuiltInTok{System}\OperatorTok{.}\FunctionTok{out}\OperatorTok{.}\FunctionTok{println}\OperatorTok{(}\NormalTok{item}\OperatorTok{);}
    \OperatorTok{\}}
\OperatorTok{\}}
\end{Highlighting}
\end{Shaded}

    \begin{Shaded}
\begin{Highlighting}[]
\CommentTok{// Producer.java}

\KeywordTok{package}\ImportTok{ pl}\OperatorTok{.}\ImportTok{edu}\OperatorTok{.}\ImportTok{agh}\OperatorTok{.}\ImportTok{tw}\OperatorTok{.}\ImportTok{knapp}\OperatorTok{.}\ImportTok{lab13}\OperatorTok{.}\ImportTok{task1}\OperatorTok{;}

\KeywordTok{import} \ImportTok{org}\OperatorTok{.}\ImportTok{jcsp}\OperatorTok{.}\ImportTok{lang}\OperatorTok{.}\ImportTok{CSProcess}\OperatorTok{;}
\KeywordTok{import} \ImportTok{org}\OperatorTok{.}\ImportTok{jcsp}\OperatorTok{.}\ImportTok{lang}\OperatorTok{.}\ImportTok{ChannelOutputInt}\OperatorTok{;}

\KeywordTok{class}\NormalTok{ Producer }\KeywordTok{implements}\NormalTok{ CSProcess }\OperatorTok{\{}
    \KeywordTok{private} \DataTypeTok{final}\NormalTok{ ChannelOutputInt channel}\OperatorTok{;}

    \KeywordTok{public} \FunctionTok{Producer}\OperatorTok{(}\NormalTok{ChannelOutputInt out}\OperatorTok{)} \OperatorTok{\{}
\NormalTok{        channel }\OperatorTok{=}\NormalTok{ out}\OperatorTok{;}
    \OperatorTok{\}}

    \AttributeTok{@Override}
    \KeywordTok{public} \DataTypeTok{void} \FunctionTok{run}\OperatorTok{()} \OperatorTok{\{}
        \DataTypeTok{int}\NormalTok{ item }\OperatorTok{=} \OperatorTok{(}\DataTypeTok{int}\OperatorTok{)(}\BuiltInTok{Math}\OperatorTok{.}\FunctionTok{random}\OperatorTok{()} \OperatorTok{*} \DecValTok{100}\OperatorTok{)} \OperatorTok{+} \DecValTok{1}\OperatorTok{;}
\NormalTok{        channel}\OperatorTok{.}\FunctionTok{write}\OperatorTok{(}\NormalTok{item}\OperatorTok{);}
    \OperatorTok{\}}
\OperatorTok{\}}
\end{Highlighting}
\end{Shaded}

    \begin{Shaded}
\begin{Highlighting}[]
\CommentTok{// Task1Main.java}

\KeywordTok{package}\ImportTok{ pl}\OperatorTok{.}\ImportTok{edu}\OperatorTok{.}\ImportTok{agh}\OperatorTok{.}\ImportTok{tw}\OperatorTok{.}\ImportTok{knapp}\OperatorTok{.}\ImportTok{lab13}\OperatorTok{.}\ImportTok{task1}\OperatorTok{;}

\KeywordTok{import} \ImportTok{org}\OperatorTok{.}\ImportTok{jcsp}\OperatorTok{.}\ImportTok{lang}\OperatorTok{.}\ImportTok{CSProcess}\OperatorTok{;}
\KeywordTok{import} \ImportTok{org}\OperatorTok{.}\ImportTok{jcsp}\OperatorTok{.}\ImportTok{lang}\OperatorTok{.}\ImportTok{Channel}\OperatorTok{;}
\KeywordTok{import} \ImportTok{org}\OperatorTok{.}\ImportTok{jcsp}\OperatorTok{.}\ImportTok{lang}\OperatorTok{.}\ImportTok{One2OneChannelInt}\OperatorTok{;}
\KeywordTok{import} \ImportTok{org}\OperatorTok{.}\ImportTok{jcsp}\OperatorTok{.}\ImportTok{lang}\OperatorTok{.}\ImportTok{Parallel}\OperatorTok{;}

\KeywordTok{public} \KeywordTok{class}\NormalTok{ Task1Main }\OperatorTok{\{}
    \KeywordTok{public} \DataTypeTok{static} \DataTypeTok{void} \FunctionTok{main}\OperatorTok{(}\BuiltInTok{String}\OperatorTok{[]}\NormalTok{ args}\OperatorTok{)} \OperatorTok{\{}
        \CommentTok{// Create channel object}
\NormalTok{        One2OneChannelInt channel }\OperatorTok{=} \BuiltInTok{Channel}\OperatorTok{.}\FunctionTok{one2oneInt}\OperatorTok{();}

        \CommentTok{// Create and run with a list of parallel constructs}
\NormalTok{        CSProcess}\OperatorTok{[]}\NormalTok{ procList }\OperatorTok{=} \OperatorTok{\{}
            \KeywordTok{new} \FunctionTok{Producer}\OperatorTok{(}\NormalTok{channel}\OperatorTok{.}\FunctionTok{out}\OperatorTok{()),}
            \KeywordTok{new} \FunctionTok{Consumer}\OperatorTok{(}\NormalTok{channel}\OperatorTok{.}\FunctionTok{in}\OperatorTok{())}
        \OperatorTok{\};}

        \CommentTok{// Processes}
\NormalTok{        Parallel par }\OperatorTok{=} \KeywordTok{new} \FunctionTok{Parallel}\OperatorTok{(}\NormalTok{procList}\OperatorTok{);} \CommentTok{// PAR construct}
\NormalTok{        par}\OperatorTok{.}\FunctionTok{run}\OperatorTok{();} \CommentTok{// Execute processes in parallel}
    \OperatorTok{\}}
\OperatorTok{\}}
\end{Highlighting}
\end{Shaded}

    Jak widać, jest to dość prosta wersja problemu producenta-konsumenta, w
której zarówno producent jak i konsument mają wspólny kanał
\texttt{channel}, do którego producent zapisuje, a konsument z kolei
odczytuje.

    \hypertarget{wersja-peux142na}{%
\subsubsection{Wersja pełna}\label{wersja-peux142na}}

Wersja pełna została stworzona przez George Wells
\href{mailto:G.Wells@ru.ac.za}{\nolinkurl{G.Wells@ru.ac.za}} i oprócz
producenta i konsumenta zawiera również pośrednika - bufor o określonym
rozmiarze.

\textbf{Uwaga}: kod został zmodyfikowany, bo oryginalna implementacja
korzysta ze starszej wersji biblioteki JCSP.

Implementacja jest przedstawiona poniżej.

    \begin{Shaded}
\begin{Highlighting}[]
\CommentTok{// Consumer.java}

\KeywordTok{package}\ImportTok{ pl}\OperatorTok{.}\ImportTok{edu}\OperatorTok{.}\ImportTok{agh}\OperatorTok{.}\ImportTok{tw}\OperatorTok{.}\ImportTok{knapp}\OperatorTok{.}\ImportTok{lab13}\OperatorTok{.}\ImportTok{task1f}\OperatorTok{;}

\KeywordTok{import} \ImportTok{org}\OperatorTok{.}\ImportTok{jcsp}\OperatorTok{.}\ImportTok{lang}\OperatorTok{.}\ImportTok{CSProcess}\OperatorTok{;}
\KeywordTok{import} \ImportTok{org}\OperatorTok{.}\ImportTok{jcsp}\OperatorTok{.}\ImportTok{lang}\OperatorTok{.}\ImportTok{ChannelInputInt}\OperatorTok{;}
\KeywordTok{import} \ImportTok{org}\OperatorTok{.}\ImportTok{jcsp}\OperatorTok{.}\ImportTok{lang}\OperatorTok{.}\ImportTok{ChannelOutputInt}\OperatorTok{;}

\CommentTok{/**}
 \CommentTok{*}\NormalTok{ Consumer class}\CommentTok{:}\NormalTok{ reads integers from input channel}\CommentTok{,}\NormalTok{ displays them}\CommentTok{,}
 \CommentTok{*}\NormalTok{ then terminates when a negative value is read}\CommentTok{.}
 \CommentTok{*/}
\KeywordTok{public} \KeywordTok{class}\NormalTok{ Consumer }\KeywordTok{implements}\NormalTok{ CSProcess }\OperatorTok{\{}
    \KeywordTok{private} \DataTypeTok{final}\NormalTok{ ChannelInputInt in}\OperatorTok{;}
    \KeywordTok{private} \DataTypeTok{final}\NormalTok{ ChannelOutputInt req}\OperatorTok{;}

    \KeywordTok{public} \FunctionTok{Consumer}\OperatorTok{(}\NormalTok{ChannelOutputInt req}\OperatorTok{,}\NormalTok{ ChannelInputInt in}\OperatorTok{)} \OperatorTok{\{}
        \KeywordTok{this}\OperatorTok{.}\FunctionTok{req} \OperatorTok{=}\NormalTok{ req}\OperatorTok{;}
        \KeywordTok{this}\OperatorTok{.}\FunctionTok{in} \OperatorTok{=}\NormalTok{ in}\OperatorTok{;}
    \OperatorTok{\}}

    \AttributeTok{@Override}
    \KeywordTok{public} \DataTypeTok{void} \FunctionTok{run}\OperatorTok{()} \OperatorTok{\{}
        \ControlFlowTok{while} \OperatorTok{(}\KeywordTok{true}\OperatorTok{)} \OperatorTok{\{}
\NormalTok{            req}\OperatorTok{.}\FunctionTok{write}\OperatorTok{(}\DecValTok{0}\OperatorTok{);}

            \DataTypeTok{int}\NormalTok{ item }\OperatorTok{=}\NormalTok{ in}\OperatorTok{.}\FunctionTok{read}\OperatorTok{();}

            \ControlFlowTok{if} \OperatorTok{(}\NormalTok{item }\OperatorTok{\textless{}} \DecValTok{0}\OperatorTok{)}
                \ControlFlowTok{break}\OperatorTok{;}

            \BuiltInTok{System}\OperatorTok{.}\FunctionTok{out}\OperatorTok{.}\FunctionTok{println}\OperatorTok{(}\NormalTok{item}\OperatorTok{);}
        \OperatorTok{\}}

        \BuiltInTok{System}\OperatorTok{.}\FunctionTok{out}\OperatorTok{.}\FunctionTok{println}\OperatorTok{(}\StringTok{"Consumer ended."}\OperatorTok{);}
    \OperatorTok{\}}
\OperatorTok{\}}
\end{Highlighting}
\end{Shaded}

    \begin{Shaded}
\begin{Highlighting}[]
\CommentTok{// Producer.java}

\KeywordTok{package}\ImportTok{ pl}\OperatorTok{.}\ImportTok{edu}\OperatorTok{.}\ImportTok{agh}\OperatorTok{.}\ImportTok{tw}\OperatorTok{.}\ImportTok{knapp}\OperatorTok{.}\ImportTok{lab13}\OperatorTok{.}\ImportTok{task1f}\OperatorTok{;}

\KeywordTok{import} \ImportTok{org}\OperatorTok{.}\ImportTok{jcsp}\OperatorTok{.}\ImportTok{lang}\OperatorTok{.}\ImportTok{CSProcess}\OperatorTok{;}
\KeywordTok{import} \ImportTok{org}\OperatorTok{.}\ImportTok{jcsp}\OperatorTok{.}\ImportTok{lang}\OperatorTok{.}\ImportTok{ChannelOutputInt}\OperatorTok{;}

\CommentTok{/**}
 \CommentTok{*}\NormalTok{ Producer class}\CommentTok{:}\NormalTok{ produces }\CommentTok{100}\NormalTok{ random integers and sends them on}
 \CommentTok{*}\NormalTok{ output channel}\CommentTok{,}\NormalTok{ then sends }\CommentTok{{-}1}\NormalTok{ and terminates}\CommentTok{.}
 \CommentTok{*}\NormalTok{ The random integers are in a given range }\CommentTok{[}\NormalTok{start}\CommentTok{...}\NormalTok{start}\CommentTok{+100)}
 \CommentTok{*/}
\KeywordTok{public} \KeywordTok{class}\NormalTok{ Producer }\KeywordTok{implements}\NormalTok{ CSProcess }\OperatorTok{\{}
    \KeywordTok{private} \DataTypeTok{final}\NormalTok{ ChannelOutputInt channel}\OperatorTok{;}
    \KeywordTok{private} \DataTypeTok{final} \DataTypeTok{int}\NormalTok{ start}\OperatorTok{;}

    \KeywordTok{public} \FunctionTok{Producer}\OperatorTok{(}\NormalTok{ChannelOutputInt out}\OperatorTok{,} \DataTypeTok{int}\NormalTok{ start}\OperatorTok{)} \OperatorTok{\{}
\NormalTok{        channel }\OperatorTok{=}\NormalTok{ out}\OperatorTok{;}
        \KeywordTok{this}\OperatorTok{.}\FunctionTok{start} \OperatorTok{=}\NormalTok{ start}\OperatorTok{;}
    \OperatorTok{\}}

    \AttributeTok{@Override}
    \KeywordTok{public} \DataTypeTok{void} \FunctionTok{run}\OperatorTok{()} \OperatorTok{\{}
        \ControlFlowTok{for} \OperatorTok{(}\DataTypeTok{int}\NormalTok{ k }\OperatorTok{=} \DecValTok{0}\OperatorTok{;}\NormalTok{ k }\OperatorTok{\textless{}} \DecValTok{100}\OperatorTok{;}\NormalTok{ k}\OperatorTok{++)} \OperatorTok{\{}
            \DataTypeTok{int}\NormalTok{ item }\OperatorTok{=} \OperatorTok{(}\DataTypeTok{int}\OperatorTok{)} \OperatorTok{(}\BuiltInTok{Math}\OperatorTok{.}\FunctionTok{random}\OperatorTok{()} \OperatorTok{*} \DecValTok{100}\OperatorTok{)} \OperatorTok{+} \DecValTok{1} \OperatorTok{+}\NormalTok{ start}\OperatorTok{;}
\NormalTok{            channel}\OperatorTok{.}\FunctionTok{write}\OperatorTok{(}\NormalTok{item}\OperatorTok{);}
        \OperatorTok{\}}

\NormalTok{        channel}\OperatorTok{.}\FunctionTok{write}\OperatorTok{({-}}\DecValTok{1}\OperatorTok{);}

        \BuiltInTok{System}\OperatorTok{.}\FunctionTok{out}\OperatorTok{.}\FunctionTok{println}\OperatorTok{(}\StringTok{"Producer"} \OperatorTok{+}\NormalTok{ start }\OperatorTok{+} \StringTok{" ended."}\OperatorTok{);}
    \OperatorTok{\}}
\OperatorTok{\}}
\end{Highlighting}
\end{Shaded}

    \begin{Shaded}
\begin{Highlighting}[]
\CommentTok{// Buffer.java}

\KeywordTok{package}\ImportTok{ pl}\OperatorTok{.}\ImportTok{edu}\OperatorTok{.}\ImportTok{agh}\OperatorTok{.}\ImportTok{tw}\OperatorTok{.}\ImportTok{knapp}\OperatorTok{.}\ImportTok{lab13}\OperatorTok{.}\ImportTok{task1f}\OperatorTok{;}

\KeywordTok{import} \ImportTok{org}\OperatorTok{.}\ImportTok{jcsp}\OperatorTok{.}\ImportTok{lang}\OperatorTok{.}\ImportTok{Alternative}\OperatorTok{;}
\KeywordTok{import} \ImportTok{org}\OperatorTok{.}\ImportTok{jcsp}\OperatorTok{.}\ImportTok{lang}\OperatorTok{.}\ImportTok{CSProcess}\OperatorTok{;}
\KeywordTok{import} \ImportTok{org}\OperatorTok{.}\ImportTok{jcsp}\OperatorTok{.}\ImportTok{lang}\OperatorTok{.}\ImportTok{Guard}\OperatorTok{;}
\KeywordTok{import} \ImportTok{org}\OperatorTok{.}\ImportTok{jcsp}\OperatorTok{.}\ImportTok{lang}\OperatorTok{.}\ImportTok{One2OneChannelInt}\OperatorTok{;}

\CommentTok{/**}
 \CommentTok{*}\NormalTok{ Buffer class}\CommentTok{:}\NormalTok{ Manages communication between Producer}
 \CommentTok{*}\NormalTok{ and Consumer classes}\CommentTok{.}
 \CommentTok{*/}
\KeywordTok{public} \KeywordTok{class} \BuiltInTok{Buffer} \KeywordTok{implements}\NormalTok{ CSProcess }\OperatorTok{\{}
    \KeywordTok{private} \DataTypeTok{final}\NormalTok{ One2OneChannelInt}\OperatorTok{[]}\NormalTok{ in}\OperatorTok{;}
    \KeywordTok{private} \DataTypeTok{final}\NormalTok{ One2OneChannelInt}\OperatorTok{[]}\NormalTok{ req}\OperatorTok{;}
    \KeywordTok{private} \DataTypeTok{final}\NormalTok{ One2OneChannelInt}\OperatorTok{[]}\NormalTok{ out}\OperatorTok{;}
    \KeywordTok{private} \DataTypeTok{final} \DataTypeTok{int}\OperatorTok{[]}\NormalTok{ buffer }\OperatorTok{=} \KeywordTok{new} \DataTypeTok{int}\OperatorTok{[}\DecValTok{10}\OperatorTok{];}
    \KeywordTok{private} \DataTypeTok{int}\NormalTok{ hd }\OperatorTok{=} \OperatorTok{{-}}\DecValTok{1}\OperatorTok{;}
    \KeywordTok{private} \DataTypeTok{int}\NormalTok{ tl }\OperatorTok{=} \OperatorTok{{-}}\DecValTok{1}\OperatorTok{;}

    \KeywordTok{public} \BuiltInTok{Buffer}\OperatorTok{(}
\NormalTok{        One2OneChannelInt}\OperatorTok{[]}\NormalTok{ in}\OperatorTok{,}
\NormalTok{        One2OneChannelInt}\OperatorTok{[]}\NormalTok{ req}\OperatorTok{,}
\NormalTok{        One2OneChannelInt}\OperatorTok{[]}\NormalTok{ out}
    \OperatorTok{)} \OperatorTok{\{}
        \KeywordTok{this}\OperatorTok{.}\FunctionTok{in} \OperatorTok{=}\NormalTok{ in}\OperatorTok{;}
        \KeywordTok{this}\OperatorTok{.}\FunctionTok{req} \OperatorTok{=}\NormalTok{ req}\OperatorTok{;}
        \KeywordTok{this}\OperatorTok{.}\FunctionTok{out} \OperatorTok{=}\NormalTok{ out}\OperatorTok{;}
    \OperatorTok{\}}

    \AttributeTok{@Override}
    \KeywordTok{public} \DataTypeTok{void} \FunctionTok{run}\OperatorTok{()} \OperatorTok{\{}
        \BuiltInTok{Guard}\OperatorTok{[]}\NormalTok{ guards }\OperatorTok{=} \OperatorTok{\{}\NormalTok{in}\OperatorTok{[}\DecValTok{0}\OperatorTok{].}\FunctionTok{in}\OperatorTok{(),}\NormalTok{ in}\OperatorTok{[}\DecValTok{1}\OperatorTok{].}\FunctionTok{in}\OperatorTok{(),}\NormalTok{ req}\OperatorTok{[}\DecValTok{0}\OperatorTok{].}\FunctionTok{in}\OperatorTok{(),}\NormalTok{ req}\OperatorTok{[}\DecValTok{1}\OperatorTok{].}\FunctionTok{in}\OperatorTok{()\};}
\NormalTok{        Alternative alt }\OperatorTok{=} \KeywordTok{new} \FunctionTok{Alternative}\OperatorTok{(}\NormalTok{guards}\OperatorTok{);}

        \DataTypeTok{int}\NormalTok{ countdown }\OperatorTok{=} \DecValTok{4}\OperatorTok{;}

        \ControlFlowTok{while} \OperatorTok{(}\NormalTok{countdown }\OperatorTok{\textgreater{}} \DecValTok{0}\OperatorTok{)} \OperatorTok{\{}
            \DataTypeTok{int}\NormalTok{ index }\OperatorTok{=}\NormalTok{ alt}\OperatorTok{.}\FunctionTok{select}\OperatorTok{();}
            \ControlFlowTok{switch} \OperatorTok{(}\NormalTok{index}\OperatorTok{)} \OperatorTok{\{}
                \ControlFlowTok{case} \DecValTok{0}\OperatorTok{:}
                \ControlFlowTok{case} \DecValTok{1}\OperatorTok{:}
                    \ControlFlowTok{if} \OperatorTok{(}\NormalTok{hd }\OperatorTok{\textless{}}\NormalTok{ tl }\OperatorTok{+} \DecValTok{11}\OperatorTok{)} \OperatorTok{\{}
                        \DataTypeTok{int}\NormalTok{ item }\OperatorTok{=}\NormalTok{ in}\OperatorTok{[}\NormalTok{index}\OperatorTok{].}\FunctionTok{in}\OperatorTok{().}\FunctionTok{read}\OperatorTok{();}

                        \ControlFlowTok{if} \OperatorTok{(}\NormalTok{item }\OperatorTok{\textless{}} \DecValTok{0}\OperatorTok{)}
\NormalTok{                            countdown}\OperatorTok{{-}{-};}
                        \ControlFlowTok{else} \OperatorTok{\{}
\NormalTok{                            hd}\OperatorTok{++;}
\NormalTok{                            buffer}\OperatorTok{[}\NormalTok{hd }\OperatorTok{\%}\NormalTok{ buffer}\OperatorTok{.}\FunctionTok{length}\OperatorTok{]} \OperatorTok{=}\NormalTok{ item}\OperatorTok{;}
                        \OperatorTok{\}}
                    \OperatorTok{\}}
                    \ControlFlowTok{break}\OperatorTok{;}
                \ControlFlowTok{case} \DecValTok{2}\OperatorTok{:}
                \ControlFlowTok{case} \DecValTok{3}\OperatorTok{:}
                    \ControlFlowTok{if} \OperatorTok{(}\NormalTok{tl }\OperatorTok{\textless{}}\NormalTok{ hd}\OperatorTok{)} \OperatorTok{\{}
\NormalTok{                        req}\OperatorTok{[}\NormalTok{index }\OperatorTok{{-}} \DecValTok{2}\OperatorTok{].}\FunctionTok{in}\OperatorTok{().}\FunctionTok{read}\OperatorTok{();}
\NormalTok{                        tl}\OperatorTok{++;}
                        \DataTypeTok{int}\NormalTok{ item }\OperatorTok{=}\NormalTok{ buffer}\OperatorTok{[}\NormalTok{tl }\OperatorTok{\%}\NormalTok{ buffer}\OperatorTok{.}\FunctionTok{length}\OperatorTok{];}
\NormalTok{                        out}\OperatorTok{[}\NormalTok{index }\OperatorTok{{-}} \DecValTok{2}\OperatorTok{].}\FunctionTok{out}\OperatorTok{().}\FunctionTok{write}\OperatorTok{(}\NormalTok{item}\OperatorTok{);}
                    \OperatorTok{\}} \ControlFlowTok{else} \ControlFlowTok{if} \OperatorTok{(}\NormalTok{countdown }\OperatorTok{\textless{}=} \DecValTok{2}\OperatorTok{)} \OperatorTok{\{}
\NormalTok{                        req}\OperatorTok{[}\NormalTok{index }\OperatorTok{{-}} \DecValTok{2}\OperatorTok{].}\FunctionTok{in}\OperatorTok{().}\FunctionTok{read}\OperatorTok{();}
\NormalTok{                        out}\OperatorTok{[}\NormalTok{index }\OperatorTok{{-}} \DecValTok{2}\OperatorTok{].}\FunctionTok{out}\OperatorTok{().}\FunctionTok{write}\OperatorTok{({-}}\DecValTok{1}\OperatorTok{);}
\NormalTok{                        countdown}\OperatorTok{{-}{-};}
                    \OperatorTok{\}}
                    \ControlFlowTok{break}\OperatorTok{;}
            \OperatorTok{\}}
        \OperatorTok{\}}

        \BuiltInTok{System}\OperatorTok{.}\FunctionTok{out}\OperatorTok{.}\FunctionTok{println}\OperatorTok{(}\StringTok{"Buffer ended."}\OperatorTok{);}
    \OperatorTok{\}}
\OperatorTok{\}}
\end{Highlighting}
\end{Shaded}

    \begin{Shaded}
\begin{Highlighting}[]
\CommentTok{// Task1fMain.java}

\KeywordTok{package}\ImportTok{ pl}\OperatorTok{.}\ImportTok{edu}\OperatorTok{.}\ImportTok{agh}\OperatorTok{.}\ImportTok{tw}\OperatorTok{.}\ImportTok{knapp}\OperatorTok{.}\ImportTok{lab13}\OperatorTok{.}\ImportTok{task1f}\OperatorTok{;}

\KeywordTok{import} \ImportTok{org}\OperatorTok{.}\ImportTok{jcsp}\OperatorTok{.}\ImportTok{lang}\OperatorTok{.}\ImportTok{CSProcess}\OperatorTok{;}
\KeywordTok{import} \ImportTok{org}\OperatorTok{.}\ImportTok{jcsp}\OperatorTok{.}\ImportTok{lang}\OperatorTok{.}\ImportTok{Channel}\OperatorTok{;}
\KeywordTok{import} \ImportTok{org}\OperatorTok{.}\ImportTok{jcsp}\OperatorTok{.}\ImportTok{lang}\OperatorTok{.}\ImportTok{One2OneChannelInt}\OperatorTok{;}
\KeywordTok{import} \ImportTok{org}\OperatorTok{.}\ImportTok{jcsp}\OperatorTok{.}\ImportTok{lang}\OperatorTok{.}\ImportTok{Parallel}\OperatorTok{;}

\CommentTok{/**}
 \CommentTok{*}\NormalTok{ Main program class for Producer}\CommentTok{/}\NormalTok{Consumer example}\CommentTok{.}
 \CommentTok{*}\NormalTok{ Sets up channels}\CommentTok{,}\NormalTok{ creates processes then}
 \CommentTok{*}\NormalTok{ executes them in parallel}\CommentTok{,}\NormalTok{ using JCSP}\CommentTok{.}
 \CommentTok{*/}
\KeywordTok{public} \DataTypeTok{final} \KeywordTok{class}\NormalTok{ Task1fMain }\OperatorTok{\{}
    \KeywordTok{public} \DataTypeTok{static} \DataTypeTok{void} \FunctionTok{main}\OperatorTok{(}\BuiltInTok{String}\OperatorTok{[]}\NormalTok{ args}\OperatorTok{)} \OperatorTok{\{}
        \CommentTok{// Create channel objects}
\NormalTok{        One2OneChannelInt}\OperatorTok{[]}\NormalTok{ prodChan }\OperatorTok{=} \OperatorTok{\{}\BuiltInTok{Channel}\OperatorTok{.}\FunctionTok{one2oneInt}\OperatorTok{(),} \BuiltInTok{Channel}\OperatorTok{.}\FunctionTok{one2oneInt}\OperatorTok{()\};}
\NormalTok{        One2OneChannelInt}\OperatorTok{[]}\NormalTok{ consReq }\OperatorTok{=} \OperatorTok{\{}\BuiltInTok{Channel}\OperatorTok{.}\FunctionTok{one2oneInt}\OperatorTok{(),} \BuiltInTok{Channel}\OperatorTok{.}\FunctionTok{one2oneInt}\OperatorTok{()\};}
\NormalTok{        One2OneChannelInt}\OperatorTok{[]}\NormalTok{ consChan }\OperatorTok{=} \OperatorTok{\{}\BuiltInTok{Channel}\OperatorTok{.}\FunctionTok{one2oneInt}\OperatorTok{(),} \BuiltInTok{Channel}\OperatorTok{.}\FunctionTok{one2oneInt}\OperatorTok{()\};}

        \CommentTok{// Create parallel construct}
\NormalTok{        CSProcess}\OperatorTok{[]}\NormalTok{ procList }\OperatorTok{=} \OperatorTok{\{}
            \KeywordTok{new} \FunctionTok{Producer}\OperatorTok{(}\NormalTok{prodChan}\OperatorTok{[}\DecValTok{0}\OperatorTok{].}\FunctionTok{out}\OperatorTok{(),} \DecValTok{0}\OperatorTok{),}
            \KeywordTok{new} \FunctionTok{Producer}\OperatorTok{(}\NormalTok{prodChan}\OperatorTok{[}\DecValTok{1}\OperatorTok{].}\FunctionTok{out}\OperatorTok{(),} \DecValTok{100}\OperatorTok{),}
            \KeywordTok{new} \BuiltInTok{Buffer}\OperatorTok{(}\NormalTok{prodChan}\OperatorTok{,}\NormalTok{ consReq}\OperatorTok{,}\NormalTok{ consChan}\OperatorTok{),}
            \KeywordTok{new} \FunctionTok{Consumer}\OperatorTok{(}\NormalTok{consReq}\OperatorTok{[}\DecValTok{0}\OperatorTok{].}\FunctionTok{out}\OperatorTok{(),}\NormalTok{ consChan}\OperatorTok{[}\DecValTok{0}\OperatorTok{].}\FunctionTok{in}\OperatorTok{()),}
            \KeywordTok{new} \FunctionTok{Consumer}\OperatorTok{(}\NormalTok{consReq}\OperatorTok{[}\DecValTok{1}\OperatorTok{].}\FunctionTok{out}\OperatorTok{(),}\NormalTok{ consChan}\OperatorTok{[}\DecValTok{1}\OperatorTok{].}\FunctionTok{in}\OperatorTok{())}
        \OperatorTok{\};}

\NormalTok{        Parallel par }\OperatorTok{=} \KeywordTok{new} \FunctionTok{Parallel}\OperatorTok{(}\NormalTok{procList}\OperatorTok{);}
\NormalTok{        par}\OperatorTok{.}\FunctionTok{run}\OperatorTok{();}
    \OperatorTok{\}}
\OperatorTok{\}}
\end{Highlighting}
\end{Shaded}

    Opis poszczególnych klas:

\begin{enumerate}
\def\labelenumi{\arabic{enumi}.}
\item
  \textbf{Producer}: jest to najprostsza klasa, która generuje 100
  losowych liczb, a następnie wysyła -1, aby zasygnalizować, że
  producent zakończył pracę.
\item
  \textbf{Consumer}: jest to nieco bardziej skomplikowana klasa. Musi
  sygnalizować procesowi bufora, że konsument jest gotowy do odczytania
  elementu. Wynika to z faktu, że mechanizm ``alternatywy''
  (\texttt{Alternative}) JCSP działa tylko z kanałami ``wejściowymi''
  (tzn. \texttt{ChannelInput}) - nie ma bezpośredniego sposobu na
  sprawdzenie, czy kanał wyjściowy jest gotowy do odczytu. Konsument
  ``sygnalizuje'' buforowi, że jest gotowy do odczytu, pisząc do
  drugiego kanału \texttt{req}. Proces bufora następnie używa tego
  kanału ``żądania'' z alternatywą strzeżoną, aby powiedzieć, kiedy
  konsumenci są gotowi do odczytu/konsumpcji kolejnych danych.
\item
  \textbf{Buffer}: jest najbardziej skomplikowaną częścią systemu.
  Wykorzystuje konstrukcję \texttt{Alternative} z kanałami wejściowymi
  (od dwóch producentów i od dwóch konsumentów). Dzięki temu może
  reagować na gotowość dowolnego z czterech procesów, pod warunkiem, że
  dostępne są miejsce i/lub dane. Co ciekawe, ``czyste zakończenie''
  jest obsługiwane poprzez wysyłanie wartości ujemnej przez producenta.
  To pozwala na eleganckie zakończenie pracy systemu.
\end{enumerate}

    \hypertarget{zadanie-2}{%
\subsection{Zadanie 2}\label{zadanie-2}}

Implementacja wspólnych dla obu rozwiązań klas (bufor, producent,
konsument itd.) jest przedstawiona poniżej.

    \begin{Shaded}
\begin{Highlighting}[]
\CommentTok{// CSPConsumer.java}

\KeywordTok{package}\ImportTok{ pl}\OperatorTok{.}\ImportTok{edu}\OperatorTok{.}\ImportTok{agh}\OperatorTok{.}\ImportTok{tw}\OperatorTok{.}\ImportTok{knapp}\OperatorTok{.}\ImportTok{lab13}\OperatorTok{.}\ImportTok{task2}\OperatorTok{;}

\KeywordTok{import} \ImportTok{org}\OperatorTok{.}\ImportTok{jcsp}\OperatorTok{.}\ImportTok{lang}\OperatorTok{.}\ImportTok{CSProcess}\OperatorTok{;}

\KeywordTok{import} \ImportTok{java}\OperatorTok{.}\ImportTok{util}\OperatorTok{.}\ImportTok{function}\OperatorTok{.}\ImportTok{Supplier}\OperatorTok{;}

\KeywordTok{public} \KeywordTok{class}\NormalTok{ CSPConsumer }\KeywordTok{implements}\NormalTok{ CSProcess }\OperatorTok{\{}
    \KeywordTok{private} \DataTypeTok{static} \DataTypeTok{final} \BuiltInTok{Logger}\NormalTok{ logger }\OperatorTok{=} \BuiltInTok{Logger}\OperatorTok{.}\FunctionTok{getInstance}\OperatorTok{();}

    \KeywordTok{private} \DataTypeTok{final}\NormalTok{ Supplier}\OperatorTok{\textless{}}\NormalTok{Portion}\OperatorTok{\textgreater{}}\NormalTok{ supplier}\OperatorTok{;}
    \KeywordTok{private} \DataTypeTok{final} \BuiltInTok{Runnable}\NormalTok{ onFinishListener}\OperatorTok{;}
    \KeywordTok{private} \DataTypeTok{final} \DataTypeTok{int}\NormalTok{ maxIter}\OperatorTok{;}

    \KeywordTok{public} \FunctionTok{CSPConsumer}\OperatorTok{(}
\NormalTok{        Supplier}\OperatorTok{\textless{}}\NormalTok{Portion}\OperatorTok{\textgreater{}}\NormalTok{ supplier}\OperatorTok{,}
        \BuiltInTok{Runnable}\NormalTok{ onFinishListener}\OperatorTok{,}
        \DataTypeTok{int}\NormalTok{ maxIter}
    \OperatorTok{)} \OperatorTok{\{}
        \KeywordTok{this}\OperatorTok{.}\FunctionTok{supplier} \OperatorTok{=}\NormalTok{ supplier}\OperatorTok{;}
        \KeywordTok{this}\OperatorTok{.}\FunctionTok{onFinishListener} \OperatorTok{=}\NormalTok{ onFinishListener}\OperatorTok{;}
        \KeywordTok{this}\OperatorTok{.}\FunctionTok{maxIter} \OperatorTok{=}\NormalTok{ maxIter}\OperatorTok{;}
    \OperatorTok{\}}

    \AttributeTok{@Override}
    \KeywordTok{public} \DataTypeTok{void} \FunctionTok{run}\OperatorTok{()} \OperatorTok{\{}
        \ControlFlowTok{for} \OperatorTok{(}\DataTypeTok{int}\NormalTok{ i }\OperatorTok{=} \DecValTok{0}\OperatorTok{;}\NormalTok{ i }\OperatorTok{\textless{}}\NormalTok{ maxIter}\OperatorTok{;} \OperatorTok{++}\NormalTok{i}\OperatorTok{)} \OperatorTok{\{}
            \DataTypeTok{var}\NormalTok{ p }\OperatorTok{=}\NormalTok{ supplier}\OperatorTok{.}\FunctionTok{get}\OperatorTok{();}
\NormalTok{            logger}\OperatorTok{.}\FunctionTok{log}\OperatorTok{(}\KeywordTok{this}\OperatorTok{,} \StringTok{"Consumed: "} \OperatorTok{+}\NormalTok{ p}\OperatorTok{);}
        \OperatorTok{\}}

\NormalTok{        onFinishListener}\OperatorTok{.}\FunctionTok{run}\OperatorTok{();}
    \OperatorTok{\}}
\OperatorTok{\}}
\end{Highlighting}
\end{Shaded}

    \begin{Shaded}
\begin{Highlighting}[]
\CommentTok{// CSPProducer.java}

\KeywordTok{package}\ImportTok{ pl}\OperatorTok{.}\ImportTok{edu}\OperatorTok{.}\ImportTok{agh}\OperatorTok{.}\ImportTok{tw}\OperatorTok{.}\ImportTok{knapp}\OperatorTok{.}\ImportTok{lab13}\OperatorTok{.}\ImportTok{task2}\OperatorTok{;}

\KeywordTok{import} \ImportTok{org}\OperatorTok{.}\ImportTok{jcsp}\OperatorTok{.}\ImportTok{lang}\OperatorTok{.}\ImportTok{CSProcess}\OperatorTok{;}

\KeywordTok{import} \ImportTok{java}\OperatorTok{.}\ImportTok{util}\OperatorTok{.}\ImportTok{function}\OperatorTok{.}\ImportTok{Consumer}\OperatorTok{;}

\KeywordTok{public} \KeywordTok{class}\NormalTok{ CSPProducer }\KeywordTok{implements}\NormalTok{ CSProcess }\OperatorTok{\{}
    \KeywordTok{private} \DataTypeTok{static} \DataTypeTok{final} \BuiltInTok{Logger}\NormalTok{ logger }\OperatorTok{=} \BuiltInTok{Logger}\OperatorTok{.}\FunctionTok{getInstance}\OperatorTok{();}

    \KeywordTok{private} \DataTypeTok{final}\NormalTok{ Consumer}\OperatorTok{\textless{}}\NormalTok{Portion}\OperatorTok{\textgreater{}}\NormalTok{ consumer}\OperatorTok{;}
    \KeywordTok{private} \DataTypeTok{final} \DataTypeTok{int}\NormalTok{ maxIter}\OperatorTok{;}

    \KeywordTok{public} \FunctionTok{CSPProducer}\OperatorTok{(}\NormalTok{Consumer}\OperatorTok{\textless{}}\NormalTok{Portion}\OperatorTok{\textgreater{}}\NormalTok{ consumer}\OperatorTok{,} \DataTypeTok{int}\NormalTok{ maxIter}\OperatorTok{)} \OperatorTok{\{}
        \KeywordTok{this}\OperatorTok{.}\FunctionTok{consumer} \OperatorTok{=}\NormalTok{ consumer}\OperatorTok{;}
        \KeywordTok{this}\OperatorTok{.}\FunctionTok{maxIter} \OperatorTok{=}\NormalTok{ maxIter}\OperatorTok{;}
    \OperatorTok{\}}

    \AttributeTok{@Override}
    \KeywordTok{public} \DataTypeTok{void} \FunctionTok{run}\OperatorTok{()} \OperatorTok{\{}
        \ControlFlowTok{for} \OperatorTok{(}\DataTypeTok{int}\NormalTok{ i }\OperatorTok{=} \DecValTok{0}\OperatorTok{;}\NormalTok{ i }\OperatorTok{\textless{}}\NormalTok{ maxIter}\OperatorTok{;} \OperatorTok{++}\NormalTok{i}\OperatorTok{)} \OperatorTok{\{}
            \DataTypeTok{var}\NormalTok{ p }\OperatorTok{=} \KeywordTok{new} \FunctionTok{Portion}\OperatorTok{(}\NormalTok{i}\OperatorTok{);}
\NormalTok{            consumer}\OperatorTok{.}\FunctionTok{accept}\OperatorTok{(}\NormalTok{p}\OperatorTok{);}
\NormalTok{            logger}\OperatorTok{.}\FunctionTok{log}\OperatorTok{(}\KeywordTok{this}\OperatorTok{,} \StringTok{"Produced a new portion of data"}\OperatorTok{);}
        \OperatorTok{\}}
    \OperatorTok{\}}
\OperatorTok{\}}
\end{Highlighting}
\end{Shaded}

    \begin{Shaded}
\begin{Highlighting}[]
\CommentTok{// Buffer.java}

\KeywordTok{package}\ImportTok{ pl}\OperatorTok{.}\ImportTok{edu}\OperatorTok{.}\ImportTok{agh}\OperatorTok{.}\ImportTok{tw}\OperatorTok{.}\ImportTok{knapp}\OperatorTok{.}\ImportTok{lab13}\OperatorTok{.}\ImportTok{task2}\OperatorTok{;}

\KeywordTok{import} \ImportTok{org}\OperatorTok{.}\ImportTok{jcsp}\OperatorTok{.}\ImportTok{lang}\OperatorTok{.}\ImportTok{CSProcess}\OperatorTok{;}

\KeywordTok{import} \ImportTok{java}\OperatorTok{.}\ImportTok{util}\OperatorTok{.}\ImportTok{function}\OperatorTok{.}\ImportTok{Consumer}\OperatorTok{;}
\KeywordTok{import} \ImportTok{java}\OperatorTok{.}\ImportTok{util}\OperatorTok{.}\ImportTok{function}\OperatorTok{.}\ImportTok{Supplier}\OperatorTok{;}

\KeywordTok{public} \KeywordTok{class} \BuiltInTok{Buffer} \KeywordTok{implements}\NormalTok{ CSProcess }\OperatorTok{\{}
    \KeywordTok{private} \DataTypeTok{final}\NormalTok{ Supplier}\OperatorTok{\textless{}}\NormalTok{Portion}\OperatorTok{\textgreater{}}\NormalTok{ supplier}\OperatorTok{;}
    \KeywordTok{private} \DataTypeTok{final}\NormalTok{ Consumer}\OperatorTok{\textless{}}\NormalTok{Portion}\OperatorTok{\textgreater{}}\NormalTok{ consumer}\OperatorTok{;}

    \KeywordTok{public} \BuiltInTok{Buffer}\OperatorTok{(}\NormalTok{Supplier}\OperatorTok{\textless{}}\NormalTok{Portion}\OperatorTok{\textgreater{}}\NormalTok{ supplier}\OperatorTok{,}\NormalTok{ Consumer}\OperatorTok{\textless{}}\NormalTok{Portion}\OperatorTok{\textgreater{}}\NormalTok{ consumer}\OperatorTok{)} \OperatorTok{\{}
        \KeywordTok{this}\OperatorTok{.}\FunctionTok{supplier} \OperatorTok{=}\NormalTok{ supplier}\OperatorTok{;}
        \KeywordTok{this}\OperatorTok{.}\FunctionTok{consumer} \OperatorTok{=}\NormalTok{ consumer}\OperatorTok{;}
    \OperatorTok{\}}

    \AttributeTok{@Override}
    \KeywordTok{public} \DataTypeTok{void} \FunctionTok{run}\OperatorTok{()} \OperatorTok{\{}
        \ControlFlowTok{while} \OperatorTok{(}\KeywordTok{true}\OperatorTok{)} \OperatorTok{\{}
\NormalTok{            consumer}\OperatorTok{.}\FunctionTok{accept}\OperatorTok{(}\NormalTok{supplier}\OperatorTok{.}\FunctionTok{get}\OperatorTok{());}
        \OperatorTok{\}}
    \OperatorTok{\}}
\OperatorTok{\}}
\end{Highlighting}
\end{Shaded}

    \begin{Shaded}
\begin{Highlighting}[]
\CommentTok{// Portion.java}

\KeywordTok{package}\ImportTok{ pl}\OperatorTok{.}\ImportTok{edu}\OperatorTok{.}\ImportTok{agh}\OperatorTok{.}\ImportTok{tw}\OperatorTok{.}\ImportTok{knapp}\OperatorTok{.}\ImportTok{lab13}\OperatorTok{.}\ImportTok{task2}\OperatorTok{;}

\KeywordTok{public} \KeywordTok{record} \FunctionTok{Portion}\OperatorTok{(}
    \DataTypeTok{int}\NormalTok{ val}
\OperatorTok{)} \OperatorTok{\{\}}
\end{Highlighting}
\end{Shaded}

    \begin{Shaded}
\begin{Highlighting}[]
\CommentTok{// Logger.java}

\KeywordTok{package}\ImportTok{ pl}\OperatorTok{.}\ImportTok{edu}\OperatorTok{.}\ImportTok{agh}\OperatorTok{.}\ImportTok{tw}\OperatorTok{.}\ImportTok{knapp}\OperatorTok{.}\ImportTok{lab13}\OperatorTok{.}\ImportTok{task2}\OperatorTok{;}

\KeywordTok{import} \ImportTok{java}\OperatorTok{.}\ImportTok{util}\OperatorTok{.}\ImportTok{function}\OperatorTok{.}\ImportTok{Consumer}\OperatorTok{;}

\KeywordTok{public} \KeywordTok{class} \BuiltInTok{Logger} \OperatorTok{\{}
    \KeywordTok{private} \DataTypeTok{static} \DataTypeTok{final} \BuiltInTok{Logger}\NormalTok{ logger }\OperatorTok{=} \KeywordTok{new} \BuiltInTok{Logger}\OperatorTok{();}

    \KeywordTok{private}\NormalTok{ Consumer}\OperatorTok{\textless{}}\BuiltInTok{String}\OperatorTok{\textgreater{}}\NormalTok{ consumer }\OperatorTok{=} \FunctionTok{defaultConsumer}\OperatorTok{();}

    \KeywordTok{private} \DataTypeTok{static}\NormalTok{ Consumer}\OperatorTok{\textless{}}\BuiltInTok{String}\OperatorTok{\textgreater{}} \FunctionTok{defaultConsumer}\OperatorTok{()} \OperatorTok{\{}
        \ControlFlowTok{return} \BuiltInTok{System}\OperatorTok{.}\FunctionTok{out}\OperatorTok{::}\NormalTok{println}\OperatorTok{;}
    \OperatorTok{\}}

    \KeywordTok{private} \BuiltInTok{Logger}\OperatorTok{()} \OperatorTok{\{}
        \CommentTok{// empty}
    \OperatorTok{\}}

    \KeywordTok{public} \DataTypeTok{void} \FunctionTok{log}\OperatorTok{(}\BuiltInTok{String}\NormalTok{ tag}\OperatorTok{,} \BuiltInTok{Object}\NormalTok{ o}\OperatorTok{)} \OperatorTok{\{}
\NormalTok{        consumer}\OperatorTok{.}\FunctionTok{accept}\OperatorTok{(}\BuiltInTok{String}\OperatorTok{.}\FunctionTok{format}\OperatorTok{(}\StringTok{"[}\SpecialCharTok{\%s}\StringTok{] }\SpecialCharTok{\%s}\StringTok{"}\OperatorTok{,}\NormalTok{ tag}\OperatorTok{,}\NormalTok{ o}\OperatorTok{));}
    \OperatorTok{\}}

    \KeywordTok{public} \DataTypeTok{void} \FunctionTok{log}\OperatorTok{(}\BuiltInTok{Object}\NormalTok{ context}\OperatorTok{,} \BuiltInTok{Object}\NormalTok{ o}\OperatorTok{)} \OperatorTok{\{}
        \FunctionTok{log}\OperatorTok{(}\NormalTok{context}\OperatorTok{.}\FunctionTok{getClass}\OperatorTok{().}\FunctionTok{getSimpleName}\OperatorTok{(),}\NormalTok{ o}\OperatorTok{);}
    \OperatorTok{\}}

    \KeywordTok{public} \DataTypeTok{void} \FunctionTok{log}\OperatorTok{(}\BuiltInTok{Object}\NormalTok{ o}\OperatorTok{)} \OperatorTok{\{}
\NormalTok{        consumer}\OperatorTok{.}\FunctionTok{accept}\OperatorTok{(}\BuiltInTok{String}\OperatorTok{.}\FunctionTok{valueOf}\OperatorTok{(}\NormalTok{o}\OperatorTok{));}
    \OperatorTok{\}}

    \KeywordTok{public} \DataTypeTok{void} \FunctionTok{setConsumer}\OperatorTok{(}\NormalTok{Consumer}\OperatorTok{\textless{}}\BuiltInTok{String}\OperatorTok{\textgreater{}}\NormalTok{ consumer}\OperatorTok{)} \OperatorTok{\{}
        \KeywordTok{this}\OperatorTok{.}\FunctionTok{consumer} \OperatorTok{=}\NormalTok{ consumer}\OperatorTok{;}
    \OperatorTok{\}}

    \KeywordTok{public} \DataTypeTok{void} \FunctionTok{mute}\OperatorTok{()} \OperatorTok{\{}
        \FunctionTok{setConsumer}\OperatorTok{(}\NormalTok{s }\OperatorTok{{-}\textgreater{}} \OperatorTok{\{\});}
    \OperatorTok{\}}

    \KeywordTok{public} \DataTypeTok{void} \FunctionTok{unmute}\OperatorTok{()} \OperatorTok{\{}
        \FunctionTok{setConsumer}\OperatorTok{(}\FunctionTok{defaultConsumer}\OperatorTok{());}
    \OperatorTok{\}}

    \KeywordTok{public} \DataTypeTok{static} \BuiltInTok{Logger} \FunctionTok{getInstance}\OperatorTok{()} \OperatorTok{\{}
        \ControlFlowTok{return}\NormalTok{ logger}\OperatorTok{;}
    \OperatorTok{\}}
\OperatorTok{\}}
\end{Highlighting}
\end{Shaded}

    \begin{Shaded}
\begin{Highlighting}[]
\CommentTok{// Timer.java}

\KeywordTok{package}\ImportTok{ pl}\OperatorTok{.}\ImportTok{edu}\OperatorTok{.}\ImportTok{agh}\OperatorTok{.}\ImportTok{tw}\OperatorTok{.}\ImportTok{knapp}\OperatorTok{.}\ImportTok{lab13}\OperatorTok{.}\ImportTok{task2}\OperatorTok{;}

\KeywordTok{public} \KeywordTok{class} \BuiltInTok{Timer} \OperatorTok{\{}
    \KeywordTok{private} \DataTypeTok{long}\NormalTok{ startTime }\OperatorTok{=} \DecValTok{0}\OperatorTok{;}
    \KeywordTok{private} \DataTypeTok{long}\NormalTok{ endTime }\OperatorTok{=} \DecValTok{0}\OperatorTok{;}

    \KeywordTok{public} \DataTypeTok{void} \FunctionTok{start}\OperatorTok{()} \OperatorTok{\{}
\NormalTok{        startTime }\OperatorTok{=} \BuiltInTok{System}\OperatorTok{.}\FunctionTok{currentTimeMillis}\OperatorTok{();}
    \OperatorTok{\}}

    \KeywordTok{public} \DataTypeTok{void} \FunctionTok{end}\OperatorTok{()} \OperatorTok{\{}
\NormalTok{        endTime }\OperatorTok{=} \BuiltInTok{System}\OperatorTok{.}\FunctionTok{currentTimeMillis}\OperatorTok{();}
    \OperatorTok{\}}

    \KeywordTok{public} \DataTypeTok{long} \FunctionTok{elapsedTime}\OperatorTok{()} \OperatorTok{\{}
        \ControlFlowTok{return}\NormalTok{ endTime }\OperatorTok{{-}}\NormalTok{ startTime}\OperatorTok{;}
    \OperatorTok{\}}

    \AttributeTok{@Override}
    \KeywordTok{public} \BuiltInTok{String} \FunctionTok{toString}\OperatorTok{()} \OperatorTok{\{}
        \ControlFlowTok{return}\NormalTok{ endTime }\OperatorTok{\textgreater{}} \DecValTok{0} \OperatorTok{?}
                \BuiltInTok{String}\OperatorTok{.}\FunctionTok{format}\OperatorTok{(}\StringTok{"Timer(elapsed=}\SpecialCharTok{\%s}\StringTok{ms)"}\OperatorTok{,} \FunctionTok{elapsedTime}\OperatorTok{())} \OperatorTok{:}
                \BuiltInTok{String}\OperatorTok{.}\FunctionTok{format}\OperatorTok{(}\StringTok{"Timer(startTime=}\SpecialCharTok{\%s}\StringTok{ms)"}\OperatorTok{,}\NormalTok{ startTime}\OperatorTok{);}
    \OperatorTok{\}}
\OperatorTok{\}}
\end{Highlighting}
\end{Shaded}

    \hypertarget{kolejnoux15bux107-pobierania-nie-ma-znaczenia}{%
\subsubsection{Kolejność pobierania nie ma
znaczenia}\label{kolejnoux15bux107-pobierania-nie-ma-znaczenia}}

W tej wersji mamy \(n\) producentów, \(m\) konsumentów oraz \(k\)
buforów. Każdy bufor jest o rozmiarze 1.

Komunikacja będzie się odbywała jak na schemacie poniżej:

\begin{verbatim}
ppp ---> bbb ---> ccc
ppp ---> bbb ---> ccc
ppp ---> bbb ---> ccc
\end{verbatim}

gdzie \texttt{p} - producent, \texttt{b} - bufor, \texttt{c} -
konsument.

Warto zauważyć, że:

\begin{itemize}
\tightlist
\item
  Każdy producent może zapisać dane do dowolnego bufora
\item
  Każdy konsument może odczytać dane z dowolnego bufora
\end{itemize}

W związku z powyższym, do komunikacji należy stworzyć 2 kanały typu
\texttt{Any2AnyChannel}: jeden do komunikacji
\texttt{producent\ -\textgreater{}\ bufor}, zaś drugi
\texttt{bufor\ -\textgreater{}\ konsument}. W tym celu zostanie użyta
statyczna metoda
\href{https://www.cs.kent.ac.uk/projects/ofa/jcsp/jcsp-1.1-rc4/jcsp-doc/org/jcsp/lang/Channel.html\#any2any()}{\texttt{Channel.any2any()}}:

\begin{quote}
\begin{Shaded}
\begin{Highlighting}[]
\KeywordTok{public} \DataTypeTok{static}\NormalTok{ Any2AnyChannel }\FunctionTok{any2any}\OperatorTok{()}
\end{Highlighting}
\end{Shaded}

This constructs an Object carrying channel that may be connected to any
number of writer processes and any number of reader processes. The
writers contend safely with each other to send the next message. The
readers contend safely with each other to take the next message. Each
message flows from just one of the writers to just one of the readers --
this is not a broadcasting-and-combining channel. The channel is
zero-buffered -- the writer and reader processes must synchronise.

Returns the channel.
\end{quote}

Implementacja jest przedstawiona poniżej.

    \begin{Shaded}
\begin{Highlighting}[]
\CommentTok{// Task2aMain.java}

\KeywordTok{package}\ImportTok{ pl}\OperatorTok{.}\ImportTok{edu}\OperatorTok{.}\ImportTok{agh}\OperatorTok{.}\ImportTok{tw}\OperatorTok{.}\ImportTok{knapp}\OperatorTok{.}\ImportTok{lab13}\OperatorTok{.}\ImportTok{task2}\OperatorTok{;}

\KeywordTok{import} \ImportTok{org}\OperatorTok{.}\ImportTok{jcsp}\OperatorTok{.}\ImportTok{lang}\OperatorTok{.*;}

\KeywordTok{import} \ImportTok{java}\OperatorTok{.}\ImportTok{util}\OperatorTok{.}\ImportTok{concurrent}\OperatorTok{.}\ImportTok{atomic}\OperatorTok{.}\ImportTok{AtomicInteger}\OperatorTok{;}

\KeywordTok{public} \KeywordTok{class}\NormalTok{ Task2aMain }\OperatorTok{\{}
    \KeywordTok{private} \DataTypeTok{static} \DataTypeTok{final} \DataTypeTok{int}\NormalTok{ BUFFERS }\OperatorTok{=} \DecValTok{100}\OperatorTok{;}
    \KeywordTok{private} \DataTypeTok{static} \DataTypeTok{final} \DataTypeTok{int}\NormalTok{ CONSUMERS }\OperatorTok{=} \DecValTok{1}\OperatorTok{;}
    \KeywordTok{private} \DataTypeTok{static} \DataTypeTok{final} \DataTypeTok{int}\NormalTok{ PRODUCERS }\OperatorTok{=} \DecValTok{1}\OperatorTok{;}

    \KeywordTok{private} \DataTypeTok{static} \DataTypeTok{final} \DataTypeTok{int}\NormalTok{ CONSUMER\_MAX\_ITER }\OperatorTok{=} \DecValTok{100}\OperatorTok{;}
    \KeywordTok{private} \DataTypeTok{static} \DataTypeTok{final} \DataTypeTok{int}\NormalTok{ PRODUCER\_MAX\_ITER }\OperatorTok{=} \DecValTok{100}\OperatorTok{;}

    \KeywordTok{public} \DataTypeTok{static} \DataTypeTok{void} \FunctionTok{main}\OperatorTok{(}\BuiltInTok{String}\OperatorTok{[]}\NormalTok{ args}\OperatorTok{)} \OperatorTok{\{}
        \DataTypeTok{var}\NormalTok{ timer }\OperatorTok{=} \KeywordTok{new} \BuiltInTok{Timer}\OperatorTok{();}

        \CommentTok{// Array of all processes}
        \DataTypeTok{var}\NormalTok{ processes }\OperatorTok{=} \KeywordTok{new}\NormalTok{ CSProcess}\OperatorTok{[}\NormalTok{BUFFERS }\OperatorTok{+}\NormalTok{ CONSUMERS }\OperatorTok{+}\NormalTok{ PRODUCERS}\OperatorTok{];}

        \CommentTok{// Channels for interprocess communication}
\NormalTok{        Any2AnyChannel}\OperatorTok{\textless{}}\NormalTok{Portion}\OperatorTok{\textgreater{}}\NormalTok{ consChannel }\OperatorTok{=} \BuiltInTok{Channel}\OperatorTok{.}\FunctionTok{any2any}\OperatorTok{();}
\NormalTok{        Any2AnyChannel}\OperatorTok{\textless{}}\NormalTok{Portion}\OperatorTok{\textgreater{}}\NormalTok{ prodChannel }\OperatorTok{=} \BuiltInTok{Channel}\OperatorTok{.}\FunctionTok{any2any}\OperatorTok{();}

        \CommentTok{// Producers}
        \ControlFlowTok{for} \OperatorTok{(}\DataTypeTok{int}\NormalTok{ i }\OperatorTok{=} \DecValTok{0}\OperatorTok{;}\NormalTok{ i }\OperatorTok{\textless{}}\NormalTok{ PRODUCERS}\OperatorTok{;} \OperatorTok{++}\NormalTok{i}\OperatorTok{)}
\NormalTok{            processes}\OperatorTok{[}\NormalTok{i}\OperatorTok{]} \OperatorTok{=} \KeywordTok{new} \FunctionTok{CSPProducer}\OperatorTok{(}
\NormalTok{                p }\OperatorTok{{-}\textgreater{}}\NormalTok{ prodChannel}\OperatorTok{.}\FunctionTok{out}\OperatorTok{().}\FunctionTok{write}\OperatorTok{(}\NormalTok{p}\OperatorTok{),}
\NormalTok{                PRODUCER\_MAX\_ITER}\OperatorTok{);}

        \CommentTok{// Consumers}
        \DataTypeTok{var}\NormalTok{ consumerOnFinishListener }\OperatorTok{=} \KeywordTok{new} \BuiltInTok{Runnable}\OperatorTok{()} \OperatorTok{\{}
            \KeywordTok{private} \DataTypeTok{final} \BuiltInTok{AtomicInteger}\NormalTok{ counter }\OperatorTok{=} \KeywordTok{new} \BuiltInTok{AtomicInteger}\OperatorTok{(}\DecValTok{0}\OperatorTok{);}

            \AttributeTok{@Override}
            \KeywordTok{public} \DataTypeTok{void} \FunctionTok{run}\OperatorTok{()} \OperatorTok{\{}
                \ControlFlowTok{if} \OperatorTok{(}\NormalTok{counter}\OperatorTok{.}\FunctionTok{incrementAndGet}\OperatorTok{()} \OperatorTok{==}\NormalTok{ CONSUMERS}\OperatorTok{)} \OperatorTok{\{}
\NormalTok{                    timer}\OperatorTok{.}\FunctionTok{end}\OperatorTok{();}
                    \BuiltInTok{System}\OperatorTok{.}\FunctionTok{out}\OperatorTok{.}\FunctionTok{println}\OperatorTok{(}\NormalTok{timer}\OperatorTok{);}
                    \BuiltInTok{System}\OperatorTok{.}\FunctionTok{exit}\OperatorTok{(}\DecValTok{0}\OperatorTok{);}
                \OperatorTok{\}}
            \OperatorTok{\}}
        \OperatorTok{\};}

        \ControlFlowTok{for} \OperatorTok{(}\DataTypeTok{int}\NormalTok{ i }\OperatorTok{=} \DecValTok{0}\OperatorTok{;}\NormalTok{ i }\OperatorTok{\textless{}}\NormalTok{ CONSUMERS}\OperatorTok{;} \OperatorTok{++}\NormalTok{i}\OperatorTok{)} \OperatorTok{\{}
\NormalTok{            processes}\OperatorTok{[}\NormalTok{i }\OperatorTok{+}\NormalTok{ PRODUCERS}\OperatorTok{]} \OperatorTok{=} \KeywordTok{new} \FunctionTok{CSPConsumer}\OperatorTok{(}
                    \OperatorTok{()} \OperatorTok{{-}\textgreater{}}\NormalTok{ consChannel}\OperatorTok{.}\FunctionTok{in}\OperatorTok{().}\FunctionTok{read}\OperatorTok{(),}
\NormalTok{                    consumerOnFinishListener}\OperatorTok{,}
\NormalTok{                    CONSUMER\_MAX\_ITER}\OperatorTok{);}
        \OperatorTok{\}}

        \CommentTok{// Buffers}
        \ControlFlowTok{for} \OperatorTok{(}\DataTypeTok{int}\NormalTok{ i }\OperatorTok{=} \DecValTok{0}\OperatorTok{;}\NormalTok{ i }\OperatorTok{\textless{}}\NormalTok{ BUFFERS}\OperatorTok{;}\NormalTok{ i}\OperatorTok{++)} \OperatorTok{\{}
\NormalTok{            processes}\OperatorTok{[}\NormalTok{i }\OperatorTok{+}\NormalTok{ PRODUCERS }\OperatorTok{+}\NormalTok{ CONSUMERS}\OperatorTok{]} \OperatorTok{=} \KeywordTok{new} \BuiltInTok{Buffer}\OperatorTok{(}
                    \OperatorTok{()} \OperatorTok{{-}\textgreater{}}\NormalTok{ prodChannel}\OperatorTok{.}\FunctionTok{in}\OperatorTok{().}\FunctionTok{read}\OperatorTok{(),}
\NormalTok{                    p }\OperatorTok{{-}\textgreater{}}\NormalTok{ consChannel}\OperatorTok{.}\FunctionTok{out}\OperatorTok{().}\FunctionTok{write}\OperatorTok{(}\NormalTok{p}\OperatorTok{));}
        \OperatorTok{\}}

        \CommentTok{// Run in parallel}
        \DataTypeTok{var}\NormalTok{ par }\OperatorTok{=} \KeywordTok{new} \FunctionTok{Parallel}\OperatorTok{(}\NormalTok{processes}\OperatorTok{);}

\NormalTok{        timer}\OperatorTok{.}\FunctionTok{start}\OperatorTok{();}
\NormalTok{        par}\OperatorTok{.}\FunctionTok{run}\OperatorTok{();}
    \OperatorTok{\}}
\OperatorTok{\}}
\end{Highlighting}
\end{Shaded}

    \hypertarget{kolejnoux15bux107-pobierania-ma-znaczenie}{%
\subsubsection{Kolejność pobierania ma
znaczenie}\label{kolejnoux15bux107-pobierania-ma-znaczenie}}

W tej wersji mamy również \(n\) producentów, \(m\) konsumentów oraz
\(k\) buforów. Każdy bufor jest o rozmiarze 1.

W tym przypadku producenci przekazują dane do pierwszego bufora, który
przekazuje swoją zawartość do nastepnego, aż dane nie trafią do
konsumenta. W ten sposób da się zachować kolejność. Schematycznie do
wygląda następująco:

\begin{verbatim}
ppp
ppp ----- b
ppp       |
          b
          |
          b
          |       ccc
          b ----- ccc
                  ccc
\end{verbatim}

gdzie \texttt{p} - producent, \texttt{b} - bufor, \texttt{c} -
konsument.

Warto zauważyć, że:

\begin{itemize}
\tightlist
\item
  Każdy producent może zapisać dane do pierwszego bufora
\item
  Każdy konsument może odczytać dane z ostatniego bufora
\end{itemize}

W związku z powyższym, do komunikacji należy stworzyć kanały o
następujących typach i liczbie:

\begin{itemize}
\tightlist
\item
  1 kanał typu
  \href{https://www.cs.kent.ac.uk/projects/ofa/jcsp/jcsp-1.1-rc4/jcsp-doc/org/jcsp/lang/Channel.html\#any2one()}{\texttt{Any2OneChannel}}
  służący do komunikacji producentów z pierwszym buforem
\item
  1 kanał typu
  \href{https://www.cs.kent.ac.uk/projects/ofa/jcsp/jcsp-1.1-rc4/jcsp-doc/org/jcsp/lang/Channel.html\#one2any()}{\texttt{One2AnyChannel}}
  służący do komunikacji ostatniego bufora z konsumentami
\item
  \((k - 1)\) kanałów typu
  \href{https://www.cs.kent.ac.uk/projects/ofa/jcsp/jcsp-1.1-rc4/jcsp-doc/org/jcsp/lang/Channel.html\#one2one()}{\texttt{One2OneChannel}}
  służących do komunikacji buforów między sobą
\end{itemize}

Implementacja jest przedstawiona poniżej.

    \begin{Shaded}
\begin{Highlighting}[]
\KeywordTok{package}\ImportTok{ pl}\OperatorTok{.}\ImportTok{edu}\OperatorTok{.}\ImportTok{agh}\OperatorTok{.}\ImportTok{tw}\OperatorTok{.}\ImportTok{knapp}\OperatorTok{.}\ImportTok{lab13}\OperatorTok{.}\ImportTok{task2}\OperatorTok{;}

\KeywordTok{import} \ImportTok{org}\OperatorTok{.}\ImportTok{jcsp}\OperatorTok{.}\ImportTok{lang}\OperatorTok{.*;}

\KeywordTok{import} \ImportTok{java}\OperatorTok{.}\ImportTok{util}\OperatorTok{.}\ImportTok{concurrent}\OperatorTok{.}\ImportTok{atomic}\OperatorTok{.}\ImportTok{AtomicInteger}\OperatorTok{;}
\KeywordTok{import} \ImportTok{java}\OperatorTok{.}\ImportTok{util}\OperatorTok{.}\ImportTok{function}\OperatorTok{.}\ImportTok{Consumer}\OperatorTok{;}
\KeywordTok{import} \ImportTok{java}\OperatorTok{.}\ImportTok{util}\OperatorTok{.}\ImportTok{function}\OperatorTok{.}\ImportTok{Supplier}\OperatorTok{;}

\KeywordTok{public} \KeywordTok{class}\NormalTok{ Task2bMain }\OperatorTok{\{}
    \KeywordTok{private} \DataTypeTok{static} \DataTypeTok{final} \DataTypeTok{int}\NormalTok{ BUFFERS }\OperatorTok{=} \DecValTok{100}\OperatorTok{;}
    \KeywordTok{private} \DataTypeTok{static} \DataTypeTok{final} \DataTypeTok{int}\NormalTok{ CONSUMERS }\OperatorTok{=} \DecValTok{1}\OperatorTok{;}
    \KeywordTok{private} \DataTypeTok{static} \DataTypeTok{final} \DataTypeTok{int}\NormalTok{ PRODUCERS }\OperatorTok{=} \DecValTok{1}\OperatorTok{;}

    \KeywordTok{private} \DataTypeTok{static} \DataTypeTok{final} \DataTypeTok{int}\NormalTok{ CONSUMER\_MAX\_ITER }\OperatorTok{=} \DecValTok{100}\OperatorTok{;}
    \KeywordTok{private} \DataTypeTok{static} \DataTypeTok{final} \DataTypeTok{int}\NormalTok{ PRODUCER\_MAX\_ITER }\OperatorTok{=} \DecValTok{100}\OperatorTok{;}

    \KeywordTok{public} \DataTypeTok{static} \DataTypeTok{void} \FunctionTok{main}\OperatorTok{(}\BuiltInTok{String}\OperatorTok{[]}\NormalTok{ args}\OperatorTok{)} \OperatorTok{\{}
        \DataTypeTok{var}\NormalTok{ timer }\OperatorTok{=} \KeywordTok{new} \BuiltInTok{Timer}\OperatorTok{();}

        \CommentTok{// Array of all processes}
        \DataTypeTok{var}\NormalTok{ processes }\OperatorTok{=} \KeywordTok{new}\NormalTok{ CSProcess}\OperatorTok{[}\NormalTok{BUFFERS }\OperatorTok{+}\NormalTok{ CONSUMERS }\OperatorTok{+}\NormalTok{ PRODUCERS}\OperatorTok{];}

        \CommentTok{// Channels for interprocess communication}
\NormalTok{        One2OneChannel}\OperatorTok{\textless{}}\NormalTok{Portion}\OperatorTok{\textgreater{}[]}\NormalTok{ buffChannels }\OperatorTok{=} \BuiltInTok{Channel}\OperatorTok{.}\FunctionTok{one2oneArray}\OperatorTok{(}\NormalTok{BUFFERS }\OperatorTok{{-}} \DecValTok{1}\OperatorTok{);}
\NormalTok{        Any2OneChannel}\OperatorTok{\textless{}}\NormalTok{Portion}\OperatorTok{\textgreater{}}\NormalTok{ prodChannel }\OperatorTok{=} \BuiltInTok{Channel}\OperatorTok{.}\FunctionTok{any2one}\OperatorTok{();}
\NormalTok{        One2AnyChannel}\OperatorTok{\textless{}}\NormalTok{Portion}\OperatorTok{\textgreater{}}\NormalTok{ consChannel }\OperatorTok{=} \BuiltInTok{Channel}\OperatorTok{.}\FunctionTok{one2any}\OperatorTok{();}

\NormalTok{        Consumer}\OperatorTok{\textless{}}\NormalTok{Portion}\OperatorTok{\textgreater{}}\NormalTok{ prodConsumer }\OperatorTok{=}\NormalTok{ p }\OperatorTok{{-}\textgreater{}}\NormalTok{ prodChannel}\OperatorTok{.}\FunctionTok{out}\OperatorTok{().}\FunctionTok{write}\OperatorTok{(}\NormalTok{p}\OperatorTok{);}
\NormalTok{        Supplier}\OperatorTok{\textless{}}\NormalTok{Portion}\OperatorTok{\textgreater{}}\NormalTok{ consSupplier }\OperatorTok{=} \OperatorTok{()} \OperatorTok{{-}\textgreater{}}\NormalTok{ consChannel}\OperatorTok{.}\FunctionTok{in}\OperatorTok{().}\FunctionTok{read}\OperatorTok{();}

        \CommentTok{// Producers}
        \ControlFlowTok{for} \OperatorTok{(}\DataTypeTok{int}\NormalTok{ i }\OperatorTok{=} \DecValTok{0}\OperatorTok{;}\NormalTok{ i }\OperatorTok{\textless{}}\NormalTok{ PRODUCERS}\OperatorTok{;} \OperatorTok{++}\NormalTok{i}\OperatorTok{)}
\NormalTok{            processes}\OperatorTok{[}\NormalTok{i}\OperatorTok{]} \OperatorTok{=} \KeywordTok{new} \FunctionTok{CSPProducer}\OperatorTok{(}\NormalTok{prodConsumer}\OperatorTok{,}\NormalTok{ PRODUCER\_MAX\_ITER}\OperatorTok{);}

        \CommentTok{// Buffers}
\NormalTok{        processes}\OperatorTok{[}\NormalTok{PRODUCERS}\OperatorTok{]} \OperatorTok{=} \KeywordTok{new} \BuiltInTok{Buffer}\OperatorTok{(}
                \OperatorTok{()} \OperatorTok{{-}\textgreater{}}\NormalTok{ prodChannel}\OperatorTok{.}\FunctionTok{in}\OperatorTok{().}\FunctionTok{read}\OperatorTok{(),}
\NormalTok{                p }\OperatorTok{{-}\textgreater{}}\NormalTok{ buffChannels}\OperatorTok{[}\DecValTok{0}\OperatorTok{].}\FunctionTok{out}\OperatorTok{().}\FunctionTok{write}\OperatorTok{(}\NormalTok{p}\OperatorTok{));}

        \ControlFlowTok{for} \OperatorTok{(}\DataTypeTok{int}\NormalTok{ i }\OperatorTok{=} \DecValTok{1}\OperatorTok{;}\NormalTok{ i }\OperatorTok{\textless{}}\NormalTok{ BUFFERS }\OperatorTok{{-}} \DecValTok{1}\OperatorTok{;}\NormalTok{ i}\OperatorTok{++)} \OperatorTok{\{}
            \DataTypeTok{int}\NormalTok{ j }\OperatorTok{=}\NormalTok{ i}\OperatorTok{;}

\NormalTok{            processes}\OperatorTok{[}\NormalTok{i }\OperatorTok{+}\NormalTok{ PRODUCERS}\OperatorTok{]} \OperatorTok{=} \KeywordTok{new} \BuiltInTok{Buffer}\OperatorTok{(}
                    \OperatorTok{()} \OperatorTok{{-}\textgreater{}}\NormalTok{ buffChannels}\OperatorTok{[}\NormalTok{j }\OperatorTok{{-}} \DecValTok{1}\OperatorTok{].}\FunctionTok{in}\OperatorTok{().}\FunctionTok{read}\OperatorTok{(),}
\NormalTok{                    p }\OperatorTok{{-}\textgreater{}}\NormalTok{ buffChannels}\OperatorTok{[}\NormalTok{j}\OperatorTok{].}\FunctionTok{out}\OperatorTok{().}\FunctionTok{write}\OperatorTok{(}\NormalTok{p}\OperatorTok{));}
        \OperatorTok{\}}

\NormalTok{        processes}\OperatorTok{[}\NormalTok{PRODUCERS }\OperatorTok{+}\NormalTok{ BUFFERS }\OperatorTok{{-}} \DecValTok{1}\OperatorTok{]} \OperatorTok{=} \KeywordTok{new} \BuiltInTok{Buffer}\OperatorTok{(}
                \OperatorTok{()} \OperatorTok{{-}\textgreater{}}\NormalTok{ buffChannels}\OperatorTok{[}\NormalTok{BUFFERS }\OperatorTok{{-}} \DecValTok{2}\OperatorTok{].}\FunctionTok{in}\OperatorTok{().}\FunctionTok{read}\OperatorTok{(),}
\NormalTok{                p }\OperatorTok{{-}\textgreater{}}\NormalTok{ consChannel}\OperatorTok{.}\FunctionTok{out}\OperatorTok{().}\FunctionTok{write}\OperatorTok{(}\NormalTok{p}\OperatorTok{));}

        \CommentTok{// Consumers}
        \DataTypeTok{var}\NormalTok{ consumerOnFinishListener }\OperatorTok{=} \KeywordTok{new} \BuiltInTok{Runnable}\OperatorTok{()} \OperatorTok{\{}
            \KeywordTok{private} \DataTypeTok{final} \BuiltInTok{AtomicInteger}\NormalTok{ counter }\OperatorTok{=} \KeywordTok{new} \BuiltInTok{AtomicInteger}\OperatorTok{(}\DecValTok{0}\OperatorTok{);}

            \AttributeTok{@Override}
            \KeywordTok{public} \DataTypeTok{void} \FunctionTok{run}\OperatorTok{()} \OperatorTok{\{}
                \ControlFlowTok{if} \OperatorTok{(}\NormalTok{counter}\OperatorTok{.}\FunctionTok{incrementAndGet}\OperatorTok{()} \OperatorTok{==}\NormalTok{ CONSUMERS}\OperatorTok{)} \OperatorTok{\{}
\NormalTok{                    timer}\OperatorTok{.}\FunctionTok{end}\OperatorTok{();}
                    \BuiltInTok{System}\OperatorTok{.}\FunctionTok{out}\OperatorTok{.}\FunctionTok{println}\OperatorTok{(}\NormalTok{timer}\OperatorTok{);}
                    \BuiltInTok{System}\OperatorTok{.}\FunctionTok{exit}\OperatorTok{(}\DecValTok{0}\OperatorTok{);}
                \OperatorTok{\}}
            \OperatorTok{\}}
        \OperatorTok{\};}

        \ControlFlowTok{for} \OperatorTok{(}\DataTypeTok{int}\NormalTok{ i }\OperatorTok{=} \DecValTok{0}\OperatorTok{;}\NormalTok{ i }\OperatorTok{\textless{}}\NormalTok{ CONSUMERS}\OperatorTok{;} \OperatorTok{++}\NormalTok{i}\OperatorTok{)}
\NormalTok{            processes}\OperatorTok{[}\NormalTok{i }\OperatorTok{+}\NormalTok{ PRODUCERS }\OperatorTok{+}\NormalTok{ BUFFERS}\OperatorTok{]} \OperatorTok{=} \KeywordTok{new} \FunctionTok{CSPConsumer}\OperatorTok{(}
\NormalTok{                consSupplier}\OperatorTok{,}\NormalTok{ consumerOnFinishListener}\OperatorTok{,}\NormalTok{ CONSUMER\_MAX\_ITER}\OperatorTok{);}

        \CommentTok{// Run in parallel}
        \DataTypeTok{var}\NormalTok{ par }\OperatorTok{=} \KeywordTok{new} \FunctionTok{Parallel}\OperatorTok{(}\NormalTok{processes}\OperatorTok{);}

\NormalTok{        timer}\OperatorTok{.}\FunctionTok{start}\OperatorTok{();}
\NormalTok{        par}\OperatorTok{.}\FunctionTok{run}\OperatorTok{();}
    \OperatorTok{\}}
\OperatorTok{\}}
\end{Highlighting}
\end{Shaded}

    \hypertarget{wyniki}{%
\section{Wyniki}\label{wyniki}}

W tym rozdziale zostaną zaprezentowane wyniki poszczególnych rozwiązań.

    \hypertarget{zadanie-1}{%
\subsection{Zadanie 1}\label{zadanie-1}}

    \hypertarget{wersja-podstawowa}{%
\subsubsection{Wersja podstawowa}\label{wersja-podstawowa}}

Wersja podstawowa demonstruje komunikację
\texttt{producent\ -\textgreater{}\ konsument} korzystając z 1 kanału.
Producent zapisuje do kanału liczbę losową. Wynik działania tego
rozwiązania może wyglądać w sposób następujący:

\begin{verbatim}
76
\end{verbatim}

    \hypertarget{wersja-peux142na}{%
\subsubsection{Wersja pełna}\label{wersja-peux142na}}

W wersji pełnej 2 producentów zapisuje do bufora 100 losowych liczb.
Następnie 2 konsumentów odczytuje te liczby z bufora. Możliwy wynik:

\begin{verbatim}
125
55
59
78
[...]
147
198
197
101
Consumer ended.
Consumer ended.
Buffer ended.
Producer0 ended.
Producer100 ended.
\end{verbatim}

    \hypertarget{zadanie-2}{%
\subsection{Zadanie 2}\label{zadanie-2}}

    \hypertarget{kolejnoux15bux107-pobierania-nie-ma-znaczenia}{%
\subsubsection{Kolejność pobierania nie ma
znaczenia}\label{kolejnoux15bux107-pobierania-nie-ma-znaczenia}}

Możliwy wynik (100 buforów, 1 producent i 1 konsument, 100 iteracji):

\begin{verbatim}
[CSPProducer] Produced a new portion of data
[CSPProducer] Produced a new portion of data
[CSPProducer] Produced a new portion of data
[...]
[CSPConsumer] Consumed: Portion[val=0]
[CSPConsumer] Consumed: Portion[val=1]
[CSPConsumer] Consumed: Portion[val=99]
[CSPConsumer] Consumed: Portion[val=98]
[CSPConsumer] Consumed: Portion[val=97]
[...]
[CSPConsumer] Consumed: Portion[val=4]
[CSPConsumer] Consumed: Portion[val=3]
[CSPConsumer] Consumed: Portion[val=2]
Timer(elapsed=31ms)
\end{verbatim}

    \hypertarget{kolejnoux15bux107-pobierania-ma-znaczenie}{%
\subsubsection{Kolejność pobierania ma
znaczenie}\label{kolejnoux15bux107-pobierania-ma-znaczenie}}

Wynik (100 buforów, 1 producent i 1 konsument, 100 iteracji):

\begin{verbatim}
[CSPProducer] Produced a new portion of data
[CSPProducer] Produced a new portion of data
[CSPProducer] Produced a new portion of data
[...]
[CSPConsumer] Consumed: Portion[val=0]
[CSPConsumer] Consumed: Portion[val=1]
[CSPConsumer] Consumed: Portion[val=2]
[CSPConsumer] Consumed: Portion[val=3]
[CSPConsumer] Consumed: Portion[val=4]
[CSPConsumer] Consumed: Portion[val=5]
[...]
[CSPConsumer] Consumed: Portion[val=95]
[CSPConsumer] Consumed: Portion[val=96]
[CSPConsumer] Consumed: Portion[val=97]
[CSPConsumer] Consumed: Portion[val=98]
[CSPConsumer] Consumed: Portion[val=99]
Timer(elapsed=38ms)
\end{verbatim}

    \hypertarget{poruxf3wnanie-wydajnoux15bci}{%
\subsubsection{Porównanie
wydajności}\label{poruxf3wnanie-wydajnoux15bci}}

W celu porównania wydajności powyższe 2 warianty zostaną uruchomione 10
razy z następującymi parametrami:

\begin{itemize}
\tightlist
\item
  Liczba buforów: 100
\item
  Liczba producentów: 25
\item
  Liczba konsumentów: 25
\item
  Liczba iteracji: 1000
\item
  Logowanie zostanie wyłączone
\end{itemize}

Podczas testowania został użyty następujący sprzęt i oprogramowanie:

\begin{itemize}
\tightlist
\item
  16 × AMD Ryzen 7 4800H with Radeon Graphics
\item
  Fedora 38, Linux 6.6.4-200.fc39.x86\_64
\item
  openjdk 17.0.9 2023-10-17
\end{itemize}

W celu przetwarzania i wyświetlania wyników został użyty język Python
oraz biblioteki \texttt{pandas} i \texttt{matplotlib}.

    \begin{tcolorbox}[breakable, size=fbox, boxrule=1pt, pad at break*=1mm,colback=cellbackground, colframe=cellborder]
\prompt{In}{incolor}{2}{\boxspacing}
\begin{Verbatim}[commandchars=\\\{\}]
\PY{o}{\PYZpc{}}\PY{k}{pip} \PYZhy{}\PYZhy{}version
\end{Verbatim}
\end{tcolorbox}

    \begin{Verbatim}[commandchars=\\\{\}]
pip 23.3.2 from
/home/congard/Development/Tools/python/csvenv/lib64/python3.11/site-packages/pip
(python 3.11)
Note: you may need to restart the kernel to use updated packages.
    \end{Verbatim}

    \begin{tcolorbox}[breakable, size=fbox, boxrule=1pt, pad at break*=1mm,colback=cellbackground, colframe=cellborder]
\prompt{In}{incolor}{3}{\boxspacing}
\begin{Verbatim}[commandchars=\\\{\}]
\PY{o}{\PYZpc{}}\PY{k}{pip} show pandas | grep Version
\end{Verbatim}
\end{tcolorbox}

    \begin{Verbatim}[commandchars=\\\{\}]
Version: 2.1.3
Note: you may need to restart the kernel to use updated packages.
    \end{Verbatim}

    \begin{tcolorbox}[breakable, size=fbox, boxrule=1pt, pad at break*=1mm,colback=cellbackground, colframe=cellborder]
\prompt{In}{incolor}{4}{\boxspacing}
\begin{Verbatim}[commandchars=\\\{\}]
\PY{o}{\PYZpc{}}\PY{k}{pip} show matplotlib | grep Version
\end{Verbatim}
\end{tcolorbox}

    \begin{Verbatim}[commandchars=\\\{\}]
Version: 3.8.1
Note: you may need to restart the kernel to use updated packages.
    \end{Verbatim}

    \begin{tcolorbox}[breakable, size=fbox, boxrule=1pt, pad at break*=1mm,colback=cellbackground, colframe=cellborder]
\prompt{In}{incolor}{5}{\boxspacing}
\begin{Verbatim}[commandchars=\\\{\}]
\PY{k+kn}{import} \PY{n+nn}{pandas} \PY{k}{as} \PY{n+nn}{pd}

\PY{n}{df} \PY{o}{=} \PY{n}{pd}\PY{o}{.}\PY{n}{DataFrame}\PY{p}{(}\PY{p}{\PYZob{}}
    \PY{l+s+s2}{\PYZdq{}}\PY{l+s+s2}{unordered}\PY{l+s+s2}{\PYZdq{}}\PY{p}{:} \PY{p}{[}\PY{l+m+mi}{390}\PY{p}{,} \PY{l+m+mi}{367}\PY{p}{,} \PY{l+m+mi}{377}\PY{p}{,} \PY{l+m+mi}{390}\PY{p}{,} \PY{l+m+mi}{382}\PY{p}{,} \PY{l+m+mi}{367}\PY{p}{,} \PY{l+m+mi}{383}\PY{p}{,} \PY{l+m+mi}{377}\PY{p}{,} \PY{l+m+mi}{371}\PY{p}{,} \PY{l+m+mi}{375}\PY{p}{]}\PY{p}{,}
    \PY{l+s+s2}{\PYZdq{}}\PY{l+s+s2}{ordered}\PY{l+s+s2}{\PYZdq{}}\PY{p}{:} \PY{p}{[}\PY{l+m+mi}{1738}\PY{p}{,} \PY{l+m+mi}{1628}\PY{p}{,} \PY{l+m+mi}{1666}\PY{p}{,} \PY{l+m+mi}{1677}\PY{p}{,} \PY{l+m+mi}{1673}\PY{p}{,} \PY{l+m+mi}{1717}\PY{p}{,} \PY{l+m+mi}{1780}\PY{p}{,} \PY{l+m+mi}{1685}\PY{p}{,} \PY{l+m+mi}{1683}\PY{p}{,} \PY{l+m+mi}{1653}\PY{p}{]}
\PY{p}{\PYZcb{}}\PY{p}{)}
\end{Verbatim}
\end{tcolorbox}

    \begin{tcolorbox}[breakable, size=fbox, boxrule=1pt, pad at break*=1mm,colback=cellbackground, colframe=cellborder]
\prompt{In}{incolor}{6}{\boxspacing}
\begin{Verbatim}[commandchars=\\\{\}]
\PY{n}{df}
\end{Verbatim}
\end{tcolorbox}

            \begin{tcolorbox}[breakable, size=fbox, boxrule=.5pt, pad at break*=1mm, opacityfill=0]
\prompt{Out}{outcolor}{6}{\boxspacing}
\begin{Verbatim}[commandchars=\\\{\}]
   unordered  ordered
0        390     1738
1        367     1628
2        377     1666
3        390     1677
4        382     1673
5        367     1717
6        383     1780
7        377     1685
8        371     1683
9        375     1653
\end{Verbatim}
\end{tcolorbox}
        
    \begin{tcolorbox}[breakable, size=fbox, boxrule=1pt, pad at break*=1mm,colback=cellbackground, colframe=cellborder]
\prompt{In}{incolor}{7}{\boxspacing}
\begin{Verbatim}[commandchars=\\\{\}]
\PY{n}{df}\PY{o}{.}\PY{n}{mean}\PY{p}{(}\PY{p}{)}
\end{Verbatim}
\end{tcolorbox}

            \begin{tcolorbox}[breakable, size=fbox, boxrule=.5pt, pad at break*=1mm, opacityfill=0]
\prompt{Out}{outcolor}{7}{\boxspacing}
\begin{Verbatim}[commandchars=\\\{\}]
unordered     377.9
ordered      1690.0
dtype: float64
\end{Verbatim}
\end{tcolorbox}
        
    \begin{tcolorbox}[breakable, size=fbox, boxrule=1pt, pad at break*=1mm,colback=cellbackground, colframe=cellborder]
\prompt{In}{incolor}{8}{\boxspacing}
\begin{Verbatim}[commandchars=\\\{\}]
\PY{o}{\PYZpc{}}\PY{k}{config} InlineBackend.figure\PYZus{}formats = [\PYZdq{}svg\PYZdq{}]
\end{Verbatim}
\end{tcolorbox}

    \begin{tcolorbox}[breakable, size=fbox, boxrule=1pt, pad at break*=1mm,colback=cellbackground, colframe=cellborder]
\prompt{In}{incolor}{9}{\boxspacing}
\begin{Verbatim}[commandchars=\\\{\}]
\PY{n}{df}\PY{o}{.}\PY{n}{mean}\PY{p}{(}\PY{p}{)}\PY{o}{.}\PY{n}{plot}\PY{o}{.}\PY{n}{bar}\PY{p}{(}
    \PY{n}{title}\PY{o}{=}\PY{l+s+s2}{\PYZdq{}}\PY{l+s+s2}{Performance comparison (less is better)}\PY{l+s+s2}{\PYZdq{}}\PY{p}{,}
    \PY{n}{ylabel}\PY{o}{=}\PY{l+s+s2}{\PYZdq{}}\PY{l+s+s2}{time, [\PYZdl{}ms\PYZdl{}]}\PY{l+s+s2}{\PYZdq{}}
\PY{p}{)}
\end{Verbatim}
\end{tcolorbox}

            \begin{tcolorbox}[breakable, size=fbox, boxrule=.5pt, pad at break*=1mm, opacityfill=0]
\prompt{Out}{outcolor}{9}{\boxspacing}
\begin{Verbatim}[commandchars=\\\{\}]
<Axes: title=\{'center': 'Performance comparison (less is better)'\},
ylabel='time, [\$ms\$]'>
\end{Verbatim}
\end{tcolorbox}
        
    \begin{center}
    \adjustimage{max size={0.9\linewidth}{0.9\paperheight}}{output_46_1.pdf}
    \end{center}
    { \hspace*{\fill} \\}
    
    \hypertarget{wnioski}{%
\section{Wnioski}\label{wnioski}}

\begin{itemize}
\item
  Teoria komunikujących się sekwencyjnych procesów (CSP) C.A.R. Hoare'a
  dostarcza formalne podejście do opisu współbieżności i zbiór technik
  projektowania współbieżnych programów.
\item
  Procesy CSP nie mają wspólnej przestrzeni adresowej, nie mają
  interfejsu metod ani tożsamości, a komunikują się tylko za pomocą
  kanałów.
\item
  Kanały CSP są synchroniczne, obsługują tylko odczyt i zapis, a
  podstawowym typem kanałów jest \emph{one-to-one}.
\item
  Pakiet JCSP, opracowany na University of Kent, to platforma wykonawcza
  dla programów współbieżnych w Javie, która wspiera konstrukcje CSP.
\item
  W JCSP, interfejsy \texttt{ChannelInput}, \texttt{ChannelOutput} i
  \texttt{Channel} działają na argumentach typu \texttt{Object} lub
  \texttt{int}.
\item
  Interfejs \texttt{CSProcess} opisuje procesy wspierając tylko metodę
  \texttt{run}.
\item
  Operator wyboru \texttt{{[}{]}} jest obsługiwany za pośrednictwem
  klasy \texttt{Alternative}, która wspiera metody \texttt{select} i
  \texttt{fairSelect}.
\item
  Dodatkowo, JCSP oferuje narzędzia takie jak \emph{timer},
  \texttt{Generate}, \texttt{Skip} i klasy umożliwiające interakcję
  poprzez GUI.
\item
  Jak widać z wykresu oraz tabeli, pobieranie elementów przez konsumenta
  bez zachowania kolejności (\texttt{unordered}) jest zdecydowanie
  szybsze niż z zachowaniem kolejności (\texttt{ordered}).
\end{itemize}

    \hypertarget{bibliografia}{%
\section{Bibliografia}\label{bibliografia}}

\begin{enumerate}
\def\labelenumi{\arabic{enumi}.}
\item
  Materiały do laboratorium 13, dr inż. Włodzimierz Funika:\\
  \url{https://home.agh.edu.pl/~funika/tw/lab-csp/}
\item
  \texttt{Channel}, University of Kent JCSP Reference:\\
  \url{https://www.cs.kent.ac.uk/projects/ofa/jcsp/jcsp-1.1-rc4/jcsp-doc/org/jcsp/lang/Channel.html}
\end{enumerate}


    % Add a bibliography block to the postdoc
    
    
    
\end{document}
