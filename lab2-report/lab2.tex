\documentclass[11pt]{article}

    \usepackage[breakable]{tcolorbox}
    \usepackage{parskip} % Stop auto-indenting (to mimic markdown behaviour)
    

    % Basic figure setup, for now with no caption control since it's done
    % automatically by Pandoc (which extracts ![](path) syntax from Markdown).
    \usepackage{graphicx}
    % Maintain compatibility with old templates. Remove in nbconvert 6.0
    \let\Oldincludegraphics\includegraphics
    % Ensure that by default, figures have no caption (until we provide a
    % proper Figure object with a Caption API and a way to capture that
    % in the conversion process - todo).
    \usepackage{caption}
    \DeclareCaptionFormat{nocaption}{}
    \captionsetup{format=nocaption,aboveskip=0pt,belowskip=0pt}

    \usepackage{float}
    \floatplacement{figure}{H} % forces figures to be placed at the correct location
    \usepackage{xcolor} % Allow colors to be defined
    \usepackage{enumerate} % Needed for markdown enumerations to work
    \usepackage{geometry} % Used to adjust the document margins
    \usepackage{amsmath} % Equations
    \usepackage{amssymb} % Equations
    \usepackage{textcomp} % defines textquotesingle
    % Hack from http://tex.stackexchange.com/a/47451/13684:
    \AtBeginDocument{%
        \def\PYZsq{\textquotesingle}% Upright quotes in Pygmentized code
    }
    \usepackage{upquote} % Upright quotes for verbatim code
    \usepackage{eurosym} % defines \euro

    \usepackage{iftex}
    \ifPDFTeX
        \usepackage[T1]{fontenc}
        \IfFileExists{alphabeta.sty}{
              \usepackage{alphabeta}
          }{
              \usepackage[mathletters]{ucs}
              \usepackage[utf8x]{inputenc}
          }
    \else
        \usepackage{fontspec}
        \usepackage{unicode-math}
    \fi

    \usepackage{fancyvrb} % verbatim replacement that allows latex
    \usepackage{grffile} % extends the file name processing of package graphics
                         % to support a larger range
    \makeatletter % fix for old versions of grffile with XeLaTeX
    \@ifpackagelater{grffile}{2019/11/01}
    {
      % Do nothing on new versions
    }
    {
      \def\Gread@@xetex#1{%
        \IfFileExists{"\Gin@base".bb}%
        {\Gread@eps{\Gin@base.bb}}%
        {\Gread@@xetex@aux#1}%
      }
    }
    \makeatother
    \usepackage[Export]{adjustbox} % Used to constrain images to a maximum size
    \adjustboxset{max size={0.9\linewidth}{0.9\paperheight}}

    % The hyperref package gives us a pdf with properly built
    % internal navigation ('pdf bookmarks' for the table of contents,
    % internal cross-reference links, web links for URLs, etc.)
    \usepackage{hyperref}
    % The default LaTeX title has an obnoxious amount of whitespace. By default,
    % titling removes some of it. It also provides customization options.
    \usepackage{titling}
    \usepackage{longtable} % longtable support required by pandoc >1.10
    \usepackage{booktabs}  % table support for pandoc > 1.12.2
    \usepackage{array}     % table support for pandoc >= 2.11.3
    \usepackage{calc}      % table minipage width calculation for pandoc >= 2.11.1
    \usepackage[inline]{enumitem} % IRkernel/repr support (it uses the enumerate* environment)
    \usepackage[normalem]{ulem} % ulem is needed to support strikethroughs (\sout)
                                % normalem makes italics be italics, not underlines
    \usepackage{soul}      % strikethrough (\st) support for pandoc >= 3.0.0
    \usepackage{mathrsfs}
    

    
    % Colors for the hyperref package
    \definecolor{urlcolor}{rgb}{0,.145,.698}
    \definecolor{linkcolor}{rgb}{.71,0.21,0.01}
    \definecolor{citecolor}{rgb}{.12,.54,.11}

    % ANSI colors
    \definecolor{ansi-black}{HTML}{3E424D}
    \definecolor{ansi-black-intense}{HTML}{282C36}
    \definecolor{ansi-red}{HTML}{E75C58}
    \definecolor{ansi-red-intense}{HTML}{B22B31}
    \definecolor{ansi-green}{HTML}{00A250}
    \definecolor{ansi-green-intense}{HTML}{007427}
    \definecolor{ansi-yellow}{HTML}{DDB62B}
    \definecolor{ansi-yellow-intense}{HTML}{B27D12}
    \definecolor{ansi-blue}{HTML}{208FFB}
    \definecolor{ansi-blue-intense}{HTML}{0065CA}
    \definecolor{ansi-magenta}{HTML}{D160C4}
    \definecolor{ansi-magenta-intense}{HTML}{A03196}
    \definecolor{ansi-cyan}{HTML}{60C6C8}
    \definecolor{ansi-cyan-intense}{HTML}{258F8F}
    \definecolor{ansi-white}{HTML}{C5C1B4}
    \definecolor{ansi-white-intense}{HTML}{A1A6B2}
    \definecolor{ansi-default-inverse-fg}{HTML}{FFFFFF}
    \definecolor{ansi-default-inverse-bg}{HTML}{000000}

    % common color for the border for error outputs.
    \definecolor{outerrorbackground}{HTML}{FFDFDF}

    % commands and environments needed by pandoc snippets
    % extracted from the output of `pandoc -s`
    \providecommand{\tightlist}{%
      \setlength{\itemsep}{0pt}\setlength{\parskip}{0pt}}
    \DefineVerbatimEnvironment{Highlighting}{Verbatim}{commandchars=\\\{\}}
    % Add ',fontsize=\small' for more characters per line
    \newenvironment{Shaded}{}{}
    \newcommand{\KeywordTok}[1]{\textcolor[rgb]{0.00,0.44,0.13}{\textbf{{#1}}}}
    \newcommand{\DataTypeTok}[1]{\textcolor[rgb]{0.56,0.13,0.00}{{#1}}}
    \newcommand{\DecValTok}[1]{\textcolor[rgb]{0.25,0.63,0.44}{{#1}}}
    \newcommand{\BaseNTok}[1]{\textcolor[rgb]{0.25,0.63,0.44}{{#1}}}
    \newcommand{\FloatTok}[1]{\textcolor[rgb]{0.25,0.63,0.44}{{#1}}}
    \newcommand{\CharTok}[1]{\textcolor[rgb]{0.25,0.44,0.63}{{#1}}}
    \newcommand{\StringTok}[1]{\textcolor[rgb]{0.25,0.44,0.63}{{#1}}}
    \newcommand{\CommentTok}[1]{\textcolor[rgb]{0.38,0.63,0.69}{\textit{{#1}}}}
    \newcommand{\OtherTok}[1]{\textcolor[rgb]{0.00,0.44,0.13}{{#1}}}
    \newcommand{\AlertTok}[1]{\textcolor[rgb]{1.00,0.00,0.00}{\textbf{{#1}}}}
    \newcommand{\FunctionTok}[1]{\textcolor[rgb]{0.02,0.16,0.49}{{#1}}}
    \newcommand{\RegionMarkerTok}[1]{{#1}}
    \newcommand{\ErrorTok}[1]{\textcolor[rgb]{1.00,0.00,0.00}{\textbf{{#1}}}}
    \newcommand{\NormalTok}[1]{{#1}}

    % Additional commands for more recent versions of Pandoc
    \newcommand{\ConstantTok}[1]{\textcolor[rgb]{0.53,0.00,0.00}{{#1}}}
    \newcommand{\SpecialCharTok}[1]{\textcolor[rgb]{0.25,0.44,0.63}{{#1}}}
    \newcommand{\VerbatimStringTok}[1]{\textcolor[rgb]{0.25,0.44,0.63}{{#1}}}
    \newcommand{\SpecialStringTok}[1]{\textcolor[rgb]{0.73,0.40,0.53}{{#1}}}
    \newcommand{\ImportTok}[1]{{#1}}
    \newcommand{\DocumentationTok}[1]{\textcolor[rgb]{0.73,0.13,0.13}{\textit{{#1}}}}
    \newcommand{\AnnotationTok}[1]{\textcolor[rgb]{0.38,0.63,0.69}{\textbf{\textit{{#1}}}}}
    \newcommand{\CommentVarTok}[1]{\textcolor[rgb]{0.38,0.63,0.69}{\textbf{\textit{{#1}}}}}
    \newcommand{\VariableTok}[1]{\textcolor[rgb]{0.10,0.09,0.49}{{#1}}}
    \newcommand{\ControlFlowTok}[1]{\textcolor[rgb]{0.00,0.44,0.13}{\textbf{{#1}}}}
    \newcommand{\OperatorTok}[1]{\textcolor[rgb]{0.40,0.40,0.40}{{#1}}}
    \newcommand{\BuiltInTok}[1]{{#1}}
    \newcommand{\ExtensionTok}[1]{{#1}}
    \newcommand{\PreprocessorTok}[1]{\textcolor[rgb]{0.74,0.48,0.00}{{#1}}}
    \newcommand{\AttributeTok}[1]{\textcolor[rgb]{0.49,0.56,0.16}{{#1}}}
    \newcommand{\InformationTok}[1]{\textcolor[rgb]{0.38,0.63,0.69}{\textbf{\textit{{#1}}}}}
    \newcommand{\WarningTok}[1]{\textcolor[rgb]{0.38,0.63,0.69}{\textbf{\textit{{#1}}}}}


    % Define a nice break command that doesn't care if a line doesn't already
    % exist.
    \def\br{\hspace*{\fill} \\* }
    % Math Jax compatibility definitions
    \def\gt{>}
    \def\lt{<}
    \let\Oldtex\TeX
    \let\Oldlatex\LaTeX
    \renewcommand{\TeX}{\textrm{\Oldtex}}
    \renewcommand{\LaTeX}{\textrm{\Oldlatex}}
    % Document parameters
    % Document title
    \title{lab2}
    
    
    
    
    
    
    
% Pygments definitions
\makeatletter
\def\PY@reset{\let\PY@it=\relax \let\PY@bf=\relax%
    \let\PY@ul=\relax \let\PY@tc=\relax%
    \let\PY@bc=\relax \let\PY@ff=\relax}
\def\PY@tok#1{\csname PY@tok@#1\endcsname}
\def\PY@toks#1+{\ifx\relax#1\empty\else%
    \PY@tok{#1}\expandafter\PY@toks\fi}
\def\PY@do#1{\PY@bc{\PY@tc{\PY@ul{%
    \PY@it{\PY@bf{\PY@ff{#1}}}}}}}
\def\PY#1#2{\PY@reset\PY@toks#1+\relax+\PY@do{#2}}

\@namedef{PY@tok@w}{\def\PY@tc##1{\textcolor[rgb]{0.73,0.73,0.73}{##1}}}
\@namedef{PY@tok@c}{\let\PY@it=\textit\def\PY@tc##1{\textcolor[rgb]{0.24,0.48,0.48}{##1}}}
\@namedef{PY@tok@cp}{\def\PY@tc##1{\textcolor[rgb]{0.61,0.40,0.00}{##1}}}
\@namedef{PY@tok@k}{\let\PY@bf=\textbf\def\PY@tc##1{\textcolor[rgb]{0.00,0.50,0.00}{##1}}}
\@namedef{PY@tok@kp}{\def\PY@tc##1{\textcolor[rgb]{0.00,0.50,0.00}{##1}}}
\@namedef{PY@tok@kt}{\def\PY@tc##1{\textcolor[rgb]{0.69,0.00,0.25}{##1}}}
\@namedef{PY@tok@o}{\def\PY@tc##1{\textcolor[rgb]{0.40,0.40,0.40}{##1}}}
\@namedef{PY@tok@ow}{\let\PY@bf=\textbf\def\PY@tc##1{\textcolor[rgb]{0.67,0.13,1.00}{##1}}}
\@namedef{PY@tok@nb}{\def\PY@tc##1{\textcolor[rgb]{0.00,0.50,0.00}{##1}}}
\@namedef{PY@tok@nf}{\def\PY@tc##1{\textcolor[rgb]{0.00,0.00,1.00}{##1}}}
\@namedef{PY@tok@nc}{\let\PY@bf=\textbf\def\PY@tc##1{\textcolor[rgb]{0.00,0.00,1.00}{##1}}}
\@namedef{PY@tok@nn}{\let\PY@bf=\textbf\def\PY@tc##1{\textcolor[rgb]{0.00,0.00,1.00}{##1}}}
\@namedef{PY@tok@ne}{\let\PY@bf=\textbf\def\PY@tc##1{\textcolor[rgb]{0.80,0.25,0.22}{##1}}}
\@namedef{PY@tok@nv}{\def\PY@tc##1{\textcolor[rgb]{0.10,0.09,0.49}{##1}}}
\@namedef{PY@tok@no}{\def\PY@tc##1{\textcolor[rgb]{0.53,0.00,0.00}{##1}}}
\@namedef{PY@tok@nl}{\def\PY@tc##1{\textcolor[rgb]{0.46,0.46,0.00}{##1}}}
\@namedef{PY@tok@ni}{\let\PY@bf=\textbf\def\PY@tc##1{\textcolor[rgb]{0.44,0.44,0.44}{##1}}}
\@namedef{PY@tok@na}{\def\PY@tc##1{\textcolor[rgb]{0.41,0.47,0.13}{##1}}}
\@namedef{PY@tok@nt}{\let\PY@bf=\textbf\def\PY@tc##1{\textcolor[rgb]{0.00,0.50,0.00}{##1}}}
\@namedef{PY@tok@nd}{\def\PY@tc##1{\textcolor[rgb]{0.67,0.13,1.00}{##1}}}
\@namedef{PY@tok@s}{\def\PY@tc##1{\textcolor[rgb]{0.73,0.13,0.13}{##1}}}
\@namedef{PY@tok@sd}{\let\PY@it=\textit\def\PY@tc##1{\textcolor[rgb]{0.73,0.13,0.13}{##1}}}
\@namedef{PY@tok@si}{\let\PY@bf=\textbf\def\PY@tc##1{\textcolor[rgb]{0.64,0.35,0.47}{##1}}}
\@namedef{PY@tok@se}{\let\PY@bf=\textbf\def\PY@tc##1{\textcolor[rgb]{0.67,0.36,0.12}{##1}}}
\@namedef{PY@tok@sr}{\def\PY@tc##1{\textcolor[rgb]{0.64,0.35,0.47}{##1}}}
\@namedef{PY@tok@ss}{\def\PY@tc##1{\textcolor[rgb]{0.10,0.09,0.49}{##1}}}
\@namedef{PY@tok@sx}{\def\PY@tc##1{\textcolor[rgb]{0.00,0.50,0.00}{##1}}}
\@namedef{PY@tok@m}{\def\PY@tc##1{\textcolor[rgb]{0.40,0.40,0.40}{##1}}}
\@namedef{PY@tok@gh}{\let\PY@bf=\textbf\def\PY@tc##1{\textcolor[rgb]{0.00,0.00,0.50}{##1}}}
\@namedef{PY@tok@gu}{\let\PY@bf=\textbf\def\PY@tc##1{\textcolor[rgb]{0.50,0.00,0.50}{##1}}}
\@namedef{PY@tok@gd}{\def\PY@tc##1{\textcolor[rgb]{0.63,0.00,0.00}{##1}}}
\@namedef{PY@tok@gi}{\def\PY@tc##1{\textcolor[rgb]{0.00,0.52,0.00}{##1}}}
\@namedef{PY@tok@gr}{\def\PY@tc##1{\textcolor[rgb]{0.89,0.00,0.00}{##1}}}
\@namedef{PY@tok@ge}{\let\PY@it=\textit}
\@namedef{PY@tok@gs}{\let\PY@bf=\textbf}
\@namedef{PY@tok@ges}{\let\PY@bf=\textbf\let\PY@it=\textit}
\@namedef{PY@tok@gp}{\let\PY@bf=\textbf\def\PY@tc##1{\textcolor[rgb]{0.00,0.00,0.50}{##1}}}
\@namedef{PY@tok@go}{\def\PY@tc##1{\textcolor[rgb]{0.44,0.44,0.44}{##1}}}
\@namedef{PY@tok@gt}{\def\PY@tc##1{\textcolor[rgb]{0.00,0.27,0.87}{##1}}}
\@namedef{PY@tok@err}{\def\PY@bc##1{{\setlength{\fboxsep}{\string -\fboxrule}\fcolorbox[rgb]{1.00,0.00,0.00}{1,1,1}{\strut ##1}}}}
\@namedef{PY@tok@kc}{\let\PY@bf=\textbf\def\PY@tc##1{\textcolor[rgb]{0.00,0.50,0.00}{##1}}}
\@namedef{PY@tok@kd}{\let\PY@bf=\textbf\def\PY@tc##1{\textcolor[rgb]{0.00,0.50,0.00}{##1}}}
\@namedef{PY@tok@kn}{\let\PY@bf=\textbf\def\PY@tc##1{\textcolor[rgb]{0.00,0.50,0.00}{##1}}}
\@namedef{PY@tok@kr}{\let\PY@bf=\textbf\def\PY@tc##1{\textcolor[rgb]{0.00,0.50,0.00}{##1}}}
\@namedef{PY@tok@bp}{\def\PY@tc##1{\textcolor[rgb]{0.00,0.50,0.00}{##1}}}
\@namedef{PY@tok@fm}{\def\PY@tc##1{\textcolor[rgb]{0.00,0.00,1.00}{##1}}}
\@namedef{PY@tok@vc}{\def\PY@tc##1{\textcolor[rgb]{0.10,0.09,0.49}{##1}}}
\@namedef{PY@tok@vg}{\def\PY@tc##1{\textcolor[rgb]{0.10,0.09,0.49}{##1}}}
\@namedef{PY@tok@vi}{\def\PY@tc##1{\textcolor[rgb]{0.10,0.09,0.49}{##1}}}
\@namedef{PY@tok@vm}{\def\PY@tc##1{\textcolor[rgb]{0.10,0.09,0.49}{##1}}}
\@namedef{PY@tok@sa}{\def\PY@tc##1{\textcolor[rgb]{0.73,0.13,0.13}{##1}}}
\@namedef{PY@tok@sb}{\def\PY@tc##1{\textcolor[rgb]{0.73,0.13,0.13}{##1}}}
\@namedef{PY@tok@sc}{\def\PY@tc##1{\textcolor[rgb]{0.73,0.13,0.13}{##1}}}
\@namedef{PY@tok@dl}{\def\PY@tc##1{\textcolor[rgb]{0.73,0.13,0.13}{##1}}}
\@namedef{PY@tok@s2}{\def\PY@tc##1{\textcolor[rgb]{0.73,0.13,0.13}{##1}}}
\@namedef{PY@tok@sh}{\def\PY@tc##1{\textcolor[rgb]{0.73,0.13,0.13}{##1}}}
\@namedef{PY@tok@s1}{\def\PY@tc##1{\textcolor[rgb]{0.73,0.13,0.13}{##1}}}
\@namedef{PY@tok@mb}{\def\PY@tc##1{\textcolor[rgb]{0.40,0.40,0.40}{##1}}}
\@namedef{PY@tok@mf}{\def\PY@tc##1{\textcolor[rgb]{0.40,0.40,0.40}{##1}}}
\@namedef{PY@tok@mh}{\def\PY@tc##1{\textcolor[rgb]{0.40,0.40,0.40}{##1}}}
\@namedef{PY@tok@mi}{\def\PY@tc##1{\textcolor[rgb]{0.40,0.40,0.40}{##1}}}
\@namedef{PY@tok@il}{\def\PY@tc##1{\textcolor[rgb]{0.40,0.40,0.40}{##1}}}
\@namedef{PY@tok@mo}{\def\PY@tc##1{\textcolor[rgb]{0.40,0.40,0.40}{##1}}}
\@namedef{PY@tok@ch}{\let\PY@it=\textit\def\PY@tc##1{\textcolor[rgb]{0.24,0.48,0.48}{##1}}}
\@namedef{PY@tok@cm}{\let\PY@it=\textit\def\PY@tc##1{\textcolor[rgb]{0.24,0.48,0.48}{##1}}}
\@namedef{PY@tok@cpf}{\let\PY@it=\textit\def\PY@tc##1{\textcolor[rgb]{0.24,0.48,0.48}{##1}}}
\@namedef{PY@tok@c1}{\let\PY@it=\textit\def\PY@tc##1{\textcolor[rgb]{0.24,0.48,0.48}{##1}}}
\@namedef{PY@tok@cs}{\let\PY@it=\textit\def\PY@tc##1{\textcolor[rgb]{0.24,0.48,0.48}{##1}}}

\def\PYZbs{\char`\\}
\def\PYZus{\char`\_}
\def\PYZob{\char`\{}
\def\PYZcb{\char`\}}
\def\PYZca{\char`\^}
\def\PYZam{\char`\&}
\def\PYZlt{\char`\<}
\def\PYZgt{\char`\>}
\def\PYZsh{\char`\#}
\def\PYZpc{\char`\%}
\def\PYZdl{\char`\$}
\def\PYZhy{\char`\-}
\def\PYZsq{\char`\'}
\def\PYZdq{\char`\"}
\def\PYZti{\char`\~}
% for compatibility with earlier versions
\def\PYZat{@}
\def\PYZlb{[}
\def\PYZrb{]}
\makeatother


    % For linebreaks inside Verbatim environment from package fancyvrb.
    \makeatletter
        \newbox\Wrappedcontinuationbox
        \newbox\Wrappedvisiblespacebox
        \newcommand*\Wrappedvisiblespace {\textcolor{red}{\textvisiblespace}}
        \newcommand*\Wrappedcontinuationsymbol {\textcolor{red}{\llap{\tiny$\m@th\hookrightarrow$}}}
        \newcommand*\Wrappedcontinuationindent {3ex }
        \newcommand*\Wrappedafterbreak {\kern\Wrappedcontinuationindent\copy\Wrappedcontinuationbox}
        % Take advantage of the already applied Pygments mark-up to insert
        % potential linebreaks for TeX processing.
        %        {, <, #, %, $, ' and ": go to next line.
        %        _, }, ^, &, >, - and ~: stay at end of broken line.
        % Use of \textquotesingle for straight quote.
        \newcommand*\Wrappedbreaksatspecials {%
            \def\PYGZus{\discretionary{\char`\_}{\Wrappedafterbreak}{\char`\_}}%
            \def\PYGZob{\discretionary{}{\Wrappedafterbreak\char`\{}{\char`\{}}%
            \def\PYGZcb{\discretionary{\char`\}}{\Wrappedafterbreak}{\char`\}}}%
            \def\PYGZca{\discretionary{\char`\^}{\Wrappedafterbreak}{\char`\^}}%
            \def\PYGZam{\discretionary{\char`\&}{\Wrappedafterbreak}{\char`\&}}%
            \def\PYGZlt{\discretionary{}{\Wrappedafterbreak\char`\<}{\char`\<}}%
            \def\PYGZgt{\discretionary{\char`\>}{\Wrappedafterbreak}{\char`\>}}%
            \def\PYGZsh{\discretionary{}{\Wrappedafterbreak\char`\#}{\char`\#}}%
            \def\PYGZpc{\discretionary{}{\Wrappedafterbreak\char`\%}{\char`\%}}%
            \def\PYGZdl{\discretionary{}{\Wrappedafterbreak\char`\$}{\char`\$}}%
            \def\PYGZhy{\discretionary{\char`\-}{\Wrappedafterbreak}{\char`\-}}%
            \def\PYGZsq{\discretionary{}{\Wrappedafterbreak\textquotesingle}{\textquotesingle}}%
            \def\PYGZdq{\discretionary{}{\Wrappedafterbreak\char`\"}{\char`\"}}%
            \def\PYGZti{\discretionary{\char`\~}{\Wrappedafterbreak}{\char`\~}}%
        }
        % Some characters . , ; ? ! / are not pygmentized.
        % This macro makes them "active" and they will insert potential linebreaks
        \newcommand*\Wrappedbreaksatpunct {%
            \lccode`\~`\.\lowercase{\def~}{\discretionary{\hbox{\char`\.}}{\Wrappedafterbreak}{\hbox{\char`\.}}}%
            \lccode`\~`\,\lowercase{\def~}{\discretionary{\hbox{\char`\,}}{\Wrappedafterbreak}{\hbox{\char`\,}}}%
            \lccode`\~`\;\lowercase{\def~}{\discretionary{\hbox{\char`\;}}{\Wrappedafterbreak}{\hbox{\char`\;}}}%
            \lccode`\~`\:\lowercase{\def~}{\discretionary{\hbox{\char`\:}}{\Wrappedafterbreak}{\hbox{\char`\:}}}%
            \lccode`\~`\?\lowercase{\def~}{\discretionary{\hbox{\char`\?}}{\Wrappedafterbreak}{\hbox{\char`\?}}}%
            \lccode`\~`\!\lowercase{\def~}{\discretionary{\hbox{\char`\!}}{\Wrappedafterbreak}{\hbox{\char`\!}}}%
            \lccode`\~`\/\lowercase{\def~}{\discretionary{\hbox{\char`\/}}{\Wrappedafterbreak}{\hbox{\char`\/}}}%
            \catcode`\.\active
            \catcode`\,\active
            \catcode`\;\active
            \catcode`\:\active
            \catcode`\?\active
            \catcode`\!\active
            \catcode`\/\active
            \lccode`\~`\~
        }
    \makeatother

    \let\OriginalVerbatim=\Verbatim
    \makeatletter
    \renewcommand{\Verbatim}[1][1]{%
        %\parskip\z@skip
        \sbox\Wrappedcontinuationbox {\Wrappedcontinuationsymbol}%
        \sbox\Wrappedvisiblespacebox {\FV@SetupFont\Wrappedvisiblespace}%
        \def\FancyVerbFormatLine ##1{\hsize\linewidth
            \vtop{\raggedright\hyphenpenalty\z@\exhyphenpenalty\z@
                \doublehyphendemerits\z@\finalhyphendemerits\z@
                \strut ##1\strut}%
        }%
        % If the linebreak is at a space, the latter will be displayed as visible
        % space at end of first line, and a continuation symbol starts next line.
        % Stretch/shrink are however usually zero for typewriter font.
        \def\FV@Space {%
            \nobreak\hskip\z@ plus\fontdimen3\font minus\fontdimen4\font
            \discretionary{\copy\Wrappedvisiblespacebox}{\Wrappedafterbreak}
            {\kern\fontdimen2\font}%
        }%

        % Allow breaks at special characters using \PYG... macros.
        \Wrappedbreaksatspecials
        % Breaks at punctuation characters . , ; ? ! and / need catcode=\active
        \OriginalVerbatim[#1,codes*=\Wrappedbreaksatpunct]%
    }
    \makeatother

    % Exact colors from NB
    \definecolor{incolor}{HTML}{303F9F}
    \definecolor{outcolor}{HTML}{D84315}
    \definecolor{cellborder}{HTML}{CFCFCF}
    \definecolor{cellbackground}{HTML}{F7F7F7}

    % prompt
    \makeatletter
    \newcommand{\boxspacing}{\kern\kvtcb@left@rule\kern\kvtcb@boxsep}
    \makeatother
    \newcommand{\prompt}[4]{
        {\ttfamily\llap{{\color{#2}[#3]:\hspace{3pt}#4}}\vspace{-\baselineskip}}
    }
    

    
    % Prevent overflowing lines due to hard-to-break entities
    \sloppy
    % Setup hyperref package
    \hypersetup{
      breaklinks=true,  % so long urls are correctly broken across lines
      colorlinks=true,
      urlcolor=urlcolor,
      linkcolor=linkcolor,
      citecolor=citecolor,
      }
    % Slightly bigger margins than the latex defaults
    
    \geometry{verbose,tmargin=1in,bmargin=1in,lmargin=1in,rmargin=1in}
    
    

\begin{document}
    
    \begin{titlepage}
        \begin{center}
            \vspace*{1cm}
    
            \textbf{Laboratorium 2}
    
            \vspace{0.5cm}
            Teoria współbieżności
                
            \vspace{1.5cm}
    
            \textbf{Danylo Knapp}

            \vfill

            \includegraphics[width=0.4\textwidth]{../report-templates/agh-logo.png}
    
            \vfill
                
            Teoria Współbieżności
                
            \vspace{0.8cm}

            Wydział Informatyki\\
            Akademia Górniczo-Hutnicza\\
            im. Stanisława Staszica w Krakowie\\
            14.10.23
                
        \end{center}
    \end{titlepage}
    
    

    
    \hypertarget{treux15bux107-zadania}{%
\section{Treść zadania}\label{treux15bux107-zadania}}

\begin{enumerate}
\def\labelenumi{\arabic{enumi}.}
\item
  Zaimplementować semafor binarny za pomocą metod \texttt{wait} i
  \texttt{notify}, użyć go do synchronizacji programu \emph{Wyścig}
\item
  Pokazać, że do implementacji semafora za pomocą metod \texttt{wait} i
  \texttt{notify} nie wystarczy instrukcja \texttt{if} tylko potrzeba
  użyć \texttt{while}. Wyjaśnić teoretycznie dlaczego i potwierdzić
  eksperymentem w praktyce. (wskazówka: rozważyć dwie kolejki: czekającą
  na wejście do monitora obiektu oraz kolejkę związaną z instrukcją
  \texttt{wait}, rozważyć kto kiedy jest budzony i kiedy następuje
  wyścig).
\item
  Zaimplementować semafor licznikowy (ogólny) za pomocą semaforów
  binarnych. Czy semafor binarny jest szczególnym przypadkiem semafora
  ogólnego?
\end{enumerate}

    \hypertarget{rozwiux105zanie}{%
\section{Rozwiązanie}\label{rozwiux105zanie}}

Najpierw warto przypomnieć, że program \emph{wyścig} polegał na tym, że
mamy licznik, który jest modyfikowany przez dwa wątki: pierwszy zwiększa
licznik o \texttt{1}, drugi zmniejsza o \texttt{1}. Race condition
(wyścig) wynika wtedy, gdy oba wątki próbują jednocześnie zmienić
wartość licznika. Jako wynik, dostajemy na wyjściu wartość, która różni
się od oczekiwanej, tzn. \texttt{0}.

W celach synchronizacji dostępu do licznika, tym razem skorzystamy z
semafora binarnego, czyli takiego, który ma tylko 2 stany: albo jest
zablokowany przez któryś wątek, albo odblokowany.

Chcąc zaimplementować taki mechanizm synchronizujący, użyjemy Javowych
monitorów, w tym celu będziemy korzystali z metod \texttt{Object\#wait}
i \texttt{Object\#notify}. Dodatkowo, zostanie wykorzystane słowo
kluczowe \texttt{synchronized}.

Odwołamy się do dokumentacji (Java 17):

\begin{quote}
\texttt{Object\#wait}: Causes the current thread to wait until it is
awakened, typically by being \emph{notified} or \emph{interrupted}.
{[}\ldots{]} \textbf{This method causes the current thread} (referred to
here as T) to place itself in the wait set for this object and then
\textbf{to relinquish any and all synchronization claims on this
object}. Note that only the locks on this object are relinquished; any
other objects on which the current thread may be synchronized remain
locked while the thread waits.
\end{quote}

\begin{quote}
\texttt{Object\#notify}: Wakes up a single thread that is waiting on
this object's monitor. {[}\ldots{]} This method should only be called by
a thread that is the owner of this object's monitor. A thread becomes
the owner of the object's monitor in one of three ways:

\begin{itemize}
\tightlist
\item
  By executing a synchronized instance method of that object.
\item
  By executing the body of a synchronized statement that synchronizes on
  the object.
\item
  For objects of type Class, by executing a synchronized static method
  of that class.
\end{itemize}

Only one thread at a time can own an object's monitor.
\end{quote}

Z tego wynika, że:

\begin{itemize}
\tightlist
\item
  \texttt{Object\#wait}

  \begin{itemize}
  \tightlist
  \item
    Może być wywołana tylko przez wątek będący w posiadaniu monitora
  \item
    Powoduje uśpienie wątku w kolejce związanej z monitorem
  \item
    \textbf{Zwalnia monitor}
  \item
    Wątek może być obudzony gdy inny wątek wywoła \texttt{notify}, albo
    gdy wątek zostanie przerwany (interrupted)
  \end{itemize}
\item
  \texttt{Object\#notify}

  \begin{itemize}
  \tightlist
  \item
    Budzi jeden wątek spośród oczekujących w kolejce `wait'
  \item
    Obudzony wątek oczekuje aż wątek wywołujący \texttt{notify} zwolni
    monitor
  \item
    Żeby poprawnie wywołać tę metodę, musi zostać spełniony co najmniej
    jeden z trzech warunków:

    \begin{enumerate}
    \def\labelenumi{\arabic{enumi}.}
    \tightlist
    \item
      Metoda została wywołana z zsynchronizowanej metody danego obiektu
      (synchronized method)
    \item
      Metoda została wywołana z zsynchronizowanego bloku danego obiektu
    \item
      Dla obiektów typu \texttt{Class}, poprzez wykonanie
      zsynchronizowanej metody statycznej tej klasy
    \end{enumerate}
  \end{itemize}
\end{itemize}

    \hypertarget{semafor-binarny-punkt-1}{%
\subsection{Semafor binarny (punkt 1)}\label{semafor-binarny-punkt-1}}

Poniżej została przedstawiona implementacja semafora binarnego.

\begin{Shaded}
\begin{Highlighting}[]
\CommentTok{// Semafor.java}

\KeywordTok{package}\ImportTok{ pl}\OperatorTok{.}\ImportTok{edu}\OperatorTok{.}\ImportTok{agh}\OperatorTok{.}\ImportTok{tw}\OperatorTok{.}\ImportTok{knapp}\OperatorTok{;}

\KeywordTok{class}\NormalTok{ Semafor }\KeywordTok{implements}\NormalTok{ ISemaphore }\OperatorTok{\{}
    \KeywordTok{private} \DataTypeTok{boolean}\NormalTok{ \_stan }\OperatorTok{=} \KeywordTok{true}\OperatorTok{;}
    \KeywordTok{private} \DataTypeTok{int}\NormalTok{ \_czeka }\OperatorTok{=} \DecValTok{0}\OperatorTok{;}

    \KeywordTok{public} \FunctionTok{Semafor}\OperatorTok{()} \OperatorTok{\{}
    \OperatorTok{\}}

    \AttributeTok{@Override}
    \KeywordTok{public} \KeywordTok{synchronized} \DataTypeTok{void} \FunctionTok{P}\OperatorTok{()} \OperatorTok{\{}
        \OperatorTok{++}\NormalTok{\_czeka}\OperatorTok{;}

        \ControlFlowTok{while} \OperatorTok{(!}\NormalTok{\_stan}\OperatorTok{)} \OperatorTok{\{}
            \ControlFlowTok{try} \OperatorTok{\{}
                \FunctionTok{wait}\OperatorTok{();}
            \OperatorTok{\}} \ControlFlowTok{catch} \OperatorTok{(}\BuiltInTok{InterruptedException}\NormalTok{ e}\OperatorTok{)} \OperatorTok{\{}
                \ControlFlowTok{throw} \KeywordTok{new} \BuiltInTok{RuntimeException}\OperatorTok{(}\NormalTok{e}\OperatorTok{);}
            \OperatorTok{\}}
        \OperatorTok{\}}

\NormalTok{        \_stan }\OperatorTok{=} \KeywordTok{false}\OperatorTok{;}
        \OperatorTok{{-}{-}}\NormalTok{\_czeka}\OperatorTok{;}
    \OperatorTok{\}}

    \AttributeTok{@Override}
    \KeywordTok{public} \KeywordTok{synchronized} \DataTypeTok{void} \FunctionTok{V}\OperatorTok{()} \OperatorTok{\{}
\NormalTok{        \_stan }\OperatorTok{=} \KeywordTok{true}\OperatorTok{;}

        \ControlFlowTok{if} \OperatorTok{(}\NormalTok{\_czeka }\OperatorTok{\textgreater{}} \DecValTok{0}\OperatorTok{)} \OperatorTok{\{}
            \FunctionTok{notify}\OperatorTok{();}
        \OperatorTok{\}}
    \OperatorTok{\}}
\OperatorTok{\}}
\end{Highlighting}
\end{Shaded}

Dostępne operacje na tym semaforze:

\begin{itemize}
\tightlist
\item
  Opuszczenie, \texttt{Semafor\#P}: Jeśli \texttt{S\ ==\ 1} to
  \texttt{S\ =\ 0}, w przeciwnym wypadku wstrzymaj działanie procesu
  wykonującego tę operację
\item
  Podniesienie, \texttt{Semafor\#V}: Jeśli są procesy wstrzymane w
  wyniku opuszczania semafora \texttt{S}, to wznów jeden z nich, w
  przeciwnym razie \texttt{S\ =\ 1}
\end{itemize}

    \hypertarget{if-a-while-punkt-2}{%
\subsection{\texorpdfstring{\texttt{if} a \texttt{while} (punkt
2)}{if a while (punkt 2)}}\label{if-a-while-punkt-2}}

Do implementacji semafora za pomocą metod \texttt{wait} i
\texttt{notify} nie wystarczy instrukcji \texttt{if}, trzeba użyć
\texttt{while}. To wynika z kilku powodów, a mianowicie:

\begin{enumerate}
\def\labelenumi{\arabic{enumi}.}
\item
  Spurious wakeup (Fałszywe obudzenie) - wątek może zostać obudzony nie
  przez inny wątek, lecz przez system. W takim przypadku warunek, na
  który czeka wątek, może nie zostać spełniony. W celu zapobiegania
  takiej sytuacji, po obudzeniu wątek musi sprawdzić, czy warunek został
  rzeczywiście spełniony;
\item
  Zgodnie z dokumentacją, metoda \texttt{wait} zwalnia monitor. To
  znaczy, że inny wątek \texttt{A} może spróbować zablokować dany
  semafor. Biorąc pod uwagę fakt, iż obudzenie wątku \texttt{B} nie
  następuje natychmiast, semafor może zostać zablokowany przez wątek
  \texttt{A}. Uwzględniając to, że została użyta instrukcja \texttt{if},
  wątek \texttt{B} również zablokuje semafor (tu już mamy do czynienia z
  undefined behavior - synchronizacja nie zadziałała poprawnie):

\begin{verbatim}
Wątek A      Wątek B
P
***          P
***          ...
V            ...
P            obudzenie się zaczęło
***          wątek został obudzony
***          ***
==================================
    BŁĄD: oba wątki mają dostęp
      do tego samego zasobu
          jednocześnie
==================================
\end{verbatim}

  Gdzie \texttt{***} - wykonanie jakiejś pracy przez wątek, \texttt{...}
  - czekanie na zwolnienie semafora.
\end{enumerate}

\begin{quote}
Uwaga: eksperymentalne potwierdzenie znajduje się w rozdziale
\emph{Wyniki}.
\end{quote}

    \hypertarget{semafor-licznikowy-punkt-3}{%
\subsection{Semafor licznikowy (punkt
3)}\label{semafor-licznikowy-punkt-3}}

Semafor licznikowy jest uogólnieniem semafora binarnego. Ten mechanizm
synchronizacji może zostać zaimplementowany na kilka sposobów, między
innymi:

\begin{itemize}
\tightlist
\item
  Z wykorzystaniem semaforów binarnych
\item
  Z wykorzystaniem metod \texttt{wait}, \texttt{notify} i słowa
  kluczowego \texttt{synchronized}
\end{itemize}

    \hypertarget{i-sposuxf3b}{%
\subsubsection{I sposób}\label{i-sposuxf3b}}

Ten sposób przedstawia implementację semafora licznikowego z
wykorzystaniem 2 semaforów binarnych: pierwszy jest potrzebny do
synchronizacji licznika, drugi do uśpienia czekającego wątku.

\begin{Shaded}
\begin{Highlighting}[]
\CommentTok{// CountingSemaphore.java}

\KeywordTok{package}\ImportTok{ pl}\OperatorTok{.}\ImportTok{edu}\OperatorTok{.}\ImportTok{agh}\OperatorTok{.}\ImportTok{tw}\OperatorTok{.}\ImportTok{knapp}\OperatorTok{;}

\KeywordTok{public} \KeywordTok{class}\NormalTok{ CountingSemaphore }\KeywordTok{implements}\NormalTok{ ISemaphore }\OperatorTok{\{}
    \KeywordTok{private} \DataTypeTok{final}\NormalTok{ Semafor threadLocker }\OperatorTok{=} \KeywordTok{new} \FunctionTok{Semafor}\OperatorTok{();}
    \KeywordTok{private} \DataTypeTok{final}\NormalTok{ Semafor counterLocker }\OperatorTok{=} \KeywordTok{new} \FunctionTok{Semafor}\OperatorTok{();}

    \KeywordTok{private} \DataTypeTok{int}\NormalTok{ counter}\OperatorTok{;}

    \KeywordTok{public} \FunctionTok{CountingSemaphore}\OperatorTok{(}\DataTypeTok{int}\NormalTok{ initialValue}\OperatorTok{)} \OperatorTok{\{}
\NormalTok{        counter }\OperatorTok{=}\NormalTok{ initialValue}\OperatorTok{;}
    \OperatorTok{\}}

    \AttributeTok{@Override}
    \KeywordTok{public} \DataTypeTok{void} \FunctionTok{P}\OperatorTok{()} \OperatorTok{\{}
\NormalTok{        counterLocker}\OperatorTok{.}\FunctionTok{P}\OperatorTok{();}

        \ControlFlowTok{if} \OperatorTok{(}\NormalTok{counter }\OperatorTok{\textgreater{}} \DecValTok{0}\OperatorTok{)} \OperatorTok{\{}
            \OperatorTok{{-}{-}}\NormalTok{counter}\OperatorTok{;}
\NormalTok{            counterLocker}\OperatorTok{.}\FunctionTok{V}\OperatorTok{();}
        \OperatorTok{\}} \ControlFlowTok{else} \OperatorTok{\{}
\NormalTok{            counterLocker}\OperatorTok{.}\FunctionTok{V}\OperatorTok{();}
\NormalTok{            threadLocker}\OperatorTok{.}\FunctionTok{P}\OperatorTok{();}
            \FunctionTok{P}\OperatorTok{();}
        \OperatorTok{\}}
    \OperatorTok{\}}

    \AttributeTok{@Override}
    \KeywordTok{public} \DataTypeTok{void} \FunctionTok{V}\OperatorTok{()} \OperatorTok{\{}
\NormalTok{        counterLocker}\OperatorTok{.}\FunctionTok{P}\OperatorTok{();}

        \ControlFlowTok{if} \OperatorTok{(++}\NormalTok{counter }\OperatorTok{==} \DecValTok{1}\OperatorTok{)} \OperatorTok{\{}
\NormalTok{            threadLocker}\OperatorTok{.}\FunctionTok{V}\OperatorTok{();}
        \OperatorTok{\}}

\NormalTok{        counterLocker}\OperatorTok{.}\FunctionTok{V}\OperatorTok{();}
    \OperatorTok{\}}
\OperatorTok{\}}
\end{Highlighting}
\end{Shaded}

Jak widać, implementacja jest prosta, lecz w przypadku Javy
małoefektywna: klasa \texttt{Semafor} ma swój własny mechanizm
synchronizacji który jest oparty na metodach zsynchronizowanych, a także
\texttt{wait} i \texttt{notify}.

    \hypertarget{ii-sposuxf3b}{%
\subsubsection{II sposób}\label{ii-sposuxf3b}}

Ten sposób przedstawia efektywną implementację semafora licznikowego
(ogólnego) nie korzystając z wcześniej zaimplementowanego semafora
binarnego. Poniższa implementacja w sposób oczywisty pokazuje, iż
semafor binarny jest tylko szczególnym przypadkiem semafora licznikowego
(jest to semafor licznikowy z wartością początkową równą \texttt{1}).

\begin{Shaded}
\begin{Highlighting}[]
\CommentTok{// Semaphore.java}

\KeywordTok{package}\ImportTok{ pl}\OperatorTok{.}\ImportTok{edu}\OperatorTok{.}\ImportTok{agh}\OperatorTok{.}\ImportTok{tw}\OperatorTok{.}\ImportTok{knapp}\OperatorTok{;}

\KeywordTok{public} \KeywordTok{class} \BuiltInTok{Semaphore} \KeywordTok{implements}\NormalTok{ ISemaphore }\OperatorTok{\{}
    \KeywordTok{private} \DataTypeTok{int}\NormalTok{ value}\OperatorTok{;}

    \KeywordTok{public} \BuiltInTok{Semaphore}\OperatorTok{(}\DataTypeTok{int}\NormalTok{ initialValue}\OperatorTok{)} \OperatorTok{\{}
\NormalTok{        value }\OperatorTok{=}\NormalTok{ initialValue}\OperatorTok{;}
    \OperatorTok{\}}

    \AttributeTok{@Override}
    \KeywordTok{public} \KeywordTok{synchronized} \DataTypeTok{void} \FunctionTok{P}\OperatorTok{()} \OperatorTok{\{}
        \ControlFlowTok{while} \OperatorTok{(}\NormalTok{value }\OperatorTok{==} \DecValTok{0}\OperatorTok{)} \OperatorTok{\{}
            \ControlFlowTok{try} \OperatorTok{\{}
                \FunctionTok{wait}\OperatorTok{();}
            \OperatorTok{\}} \ControlFlowTok{catch} \OperatorTok{(}\BuiltInTok{InterruptedException}\NormalTok{ e}\OperatorTok{)} \OperatorTok{\{}
                \ControlFlowTok{throw} \KeywordTok{new} \BuiltInTok{RuntimeException}\OperatorTok{(}\NormalTok{e}\OperatorTok{);}
            \OperatorTok{\}}
        \OperatorTok{\}}

        \OperatorTok{{-}{-}}\NormalTok{value}\OperatorTok{;}
    \OperatorTok{\}}

    \AttributeTok{@Override}
    \KeywordTok{public} \KeywordTok{synchronized} \DataTypeTok{void} \FunctionTok{V}\OperatorTok{()} \OperatorTok{\{}
        \OperatorTok{++}\NormalTok{value}\OperatorTok{;}

        \ControlFlowTok{if} \OperatorTok{(}\NormalTok{value }\OperatorTok{==} \DecValTok{1}\OperatorTok{)} \OperatorTok{\{}
            \FunctionTok{notify}\OperatorTok{();}
        \OperatorTok{\}}
    \OperatorTok{\}}
\OperatorTok{\}}
\end{Highlighting}
\end{Shaded}

W porównaniu z poprzednim sposobem, ta implementacja jest krótsza,
prostsza do zrozumienia i, w sposób oczywisty, wydajniejsza.

    \hypertarget{wyniki}{%
\section{Wyniki}\label{wyniki}}

Wyniki zostały uzyskane poprzez uruchomienie programu \texttt{100} razy,
następnie pobrane za pomocą poniższego \texttt{bash} skryptu i
przetworzone z wykorzystaniem języka \texttt{python}.

\begin{Shaded}
\begin{Highlighting}[]
\ControlFlowTok{for}\NormalTok{ \_ }\KeywordTok{in} \DataTypeTok{\{}\DecValTok{1}\DataTypeTok{..}\DecValTok{100}\DataTypeTok{\}}
\ControlFlowTok{do}
    \ExtensionTok{./gradlew}\NormalTok{ run }\KeywordTok{|} \FunctionTok{sed} \AttributeTok{{-}n} \StringTok{\textquotesingle{}s/\^{}.*stan=\textbackslash{}s*\textbackslash{}(\textbackslash{}S*\textbackslash{}).*$/\textbackslash{}1/p\textquotesingle{}}
\ControlFlowTok{done}
\end{Highlighting}
\end{Shaded}

    \hypertarget{semafor-binarny}{%
\subsection{Semafor binarny}\label{semafor-binarny}}

W tym rozdziale pokażę, że instrukcji \texttt{if} nie wystarczy do
implementacji semafora.

    \hypertarget{instrukcja-if}{%
\subsubsection{\texorpdfstring{Instrukcja
\texttt{if}}{Instrukcja if}}\label{instrukcja-if}}

Modyfikuję klasę \texttt{Semafor} w następujący sposób:

\begin{Shaded}
\begin{Highlighting}[]
\CommentTok{// Semafor.java}

\KeywordTok{package}\ImportTok{ pl}\OperatorTok{.}\ImportTok{edu}\OperatorTok{.}\ImportTok{agh}\OperatorTok{.}\ImportTok{tw}\OperatorTok{.}\ImportTok{knapp}\OperatorTok{;}

\KeywordTok{class}\NormalTok{ Semafor }\OperatorTok{\{}
    \CommentTok{// ...}

    \KeywordTok{public} \KeywordTok{synchronized} \DataTypeTok{void} \FunctionTok{P}\OperatorTok{()} \OperatorTok{\{}
        \OperatorTok{++}\NormalTok{\_czeka}\OperatorTok{;}

        \ControlFlowTok{if} \OperatorTok{(!}\NormalTok{\_stan}\OperatorTok{)} \OperatorTok{\{}
            \ControlFlowTok{try} \OperatorTok{\{}
                \FunctionTok{wait}\OperatorTok{();}
            \OperatorTok{\}} \ControlFlowTok{catch} \OperatorTok{(}\BuiltInTok{InterruptedException}\NormalTok{ e}\OperatorTok{)} \OperatorTok{\{}
                \ControlFlowTok{throw} \KeywordTok{new} \BuiltInTok{RuntimeException}\OperatorTok{(}\NormalTok{e}\OperatorTok{);}
            \OperatorTok{\}}
        \OperatorTok{\}}

\NormalTok{        \_stan }\OperatorTok{=} \KeywordTok{false}\OperatorTok{;}
        \OperatorTok{{-}{-}}\NormalTok{\_czeka}\OperatorTok{;}
    \OperatorTok{\}}

    \CommentTok{// ...}
\OperatorTok{\}}
\end{Highlighting}
\end{Shaded}

Po uruchomieniu uzyskuję wyniki i zapisuję je do pliku
\texttt{semafor\_if.txt}, następnie przetwarzam:

    \begin{tcolorbox}[breakable, size=fbox, boxrule=1pt, pad at break*=1mm,colback=cellbackground, colframe=cellborder]
\prompt{In}{incolor}{2}{\boxspacing}
\begin{Verbatim}[commandchars=\\\{\}]
\PY{k+kn}{import} \PY{n+nn}{matplotlib}\PY{n+nn}{.}\PY{n+nn}{pyplot} \PY{k}{as} \PY{n+nn}{plt}
\PY{k+kn}{import} \PY{n+nn}{pandas} \PY{k}{as} \PY{n+nn}{pd}

\PY{n}{df\PYZus{}if} \PY{o}{=} \PY{n}{pd}\PY{o}{.}\PY{n}{read\PYZus{}csv}\PY{p}{(}\PY{l+s+s2}{\PYZdq{}}\PY{l+s+s2}{semafor\PYZus{}if.txt}\PY{l+s+s2}{\PYZdq{}}\PY{p}{,} \PY{n}{header}\PY{o}{=}\PY{k+kc}{None}\PY{p}{)} 

\PY{n+nb}{print}\PY{p}{(}\PY{n}{df\PYZus{}if}\PY{p}{)}
\end{Verbatim}
\end{tcolorbox}

    \begin{Verbatim}[commandchars=\\\{\}]
    0
0  -3
1  -1
2   0
3   0
4   1
.. ..
95  0
96 -2
97  0
98 -5
99  1

[100 rows x 1 columns]
    \end{Verbatim}

    \begin{tcolorbox}[breakable, size=fbox, boxrule=1pt, pad at break*=1mm,colback=cellbackground, colframe=cellborder]
\prompt{In}{incolor}{3}{\boxspacing}
\begin{Verbatim}[commandchars=\\\{\}]
\PY{n}{df\PYZus{}if}\PY{o}{.}\PY{n}{hist}\PY{p}{(}\PY{n}{bins}\PY{o}{=}\PY{l+m+mi}{20}\PY{p}{)}
\end{Verbatim}
\end{tcolorbox}

            \begin{tcolorbox}[breakable, size=fbox, boxrule=.5pt, pad at break*=1mm, opacityfill=0]
\prompt{Out}{outcolor}{3}{\boxspacing}
\begin{Verbatim}[commandchars=\\\{\}]
array([[<Axes: title=\{'center': '0'\}>]], dtype=object)
\end{Verbatim}
\end{tcolorbox}
        
    \begin{center}
    \adjustimage{max size={0.9\linewidth}{0.9\paperheight}}{output_11_1.png}
    \end{center}
    { \hspace*{\fill} \\}
    
    Jak widać z tego histogramu, wartości nie są aż tak porozrzucane jak w
przypadku całkowitego braku synchronizacji (laboratorium 1), ale można
stwierdzić, że mechanizm synchronizacji nie działa poprawnie. W
przypadku poprawnego działania musielibyśmy zobaczyć same zera.

    \hypertarget{instrukcja-while}{%
\subsubsection{\texorpdfstring{Instrukcja
\texttt{while}}{Instrukcja while}}\label{instrukcja-while}}

Modyfikuję klasę \texttt{Semafor} w następujący sposób:

\begin{Shaded}
\begin{Highlighting}[]
\CommentTok{// Semafor.java}

\KeywordTok{package}\ImportTok{ pl}\OperatorTok{.}\ImportTok{edu}\OperatorTok{.}\ImportTok{agh}\OperatorTok{.}\ImportTok{tw}\OperatorTok{.}\ImportTok{knapp}\OperatorTok{;}

\KeywordTok{class}\NormalTok{ Semafor }\OperatorTok{\{}
    \CommentTok{// ...}

    \KeywordTok{public} \KeywordTok{synchronized} \DataTypeTok{void} \FunctionTok{P}\OperatorTok{()} \OperatorTok{\{}
        \OperatorTok{++}\NormalTok{\_czeka}\OperatorTok{;}

        \ControlFlowTok{while} \OperatorTok{(!}\NormalTok{\_stan}\OperatorTok{)} \OperatorTok{\{}
            \ControlFlowTok{try} \OperatorTok{\{}
                \FunctionTok{wait}\OperatorTok{();}
            \OperatorTok{\}} \ControlFlowTok{catch} \OperatorTok{(}\BuiltInTok{InterruptedException}\NormalTok{ e}\OperatorTok{)} \OperatorTok{\{}
                \ControlFlowTok{throw} \KeywordTok{new} \BuiltInTok{RuntimeException}\OperatorTok{(}\NormalTok{e}\OperatorTok{);}
            \OperatorTok{\}}
        \OperatorTok{\}}

\NormalTok{        \_stan }\OperatorTok{=} \KeywordTok{false}\OperatorTok{;}
        \OperatorTok{{-}{-}}\NormalTok{\_czeka}\OperatorTok{;}
    \OperatorTok{\}}

    \CommentTok{// ...}
\OperatorTok{\}}
\end{Highlighting}
\end{Shaded}

Po uruchomieniu uzyskuję wyniki i zapisuję je do pliku
\texttt{semafor\_while.txt}, następnie przetwarzam:

    \begin{tcolorbox}[breakable, size=fbox, boxrule=1pt, pad at break*=1mm,colback=cellbackground, colframe=cellborder]
\prompt{In}{incolor}{4}{\boxspacing}
\begin{Verbatim}[commandchars=\\\{\}]
\PY{n}{df\PYZus{}while} \PY{o}{=} \PY{n}{pd}\PY{o}{.}\PY{n}{read\PYZus{}csv}\PY{p}{(}\PY{l+s+s2}{\PYZdq{}}\PY{l+s+s2}{semafor\PYZus{}while.txt}\PY{l+s+s2}{\PYZdq{}}\PY{p}{,} \PY{n}{header}\PY{o}{=}\PY{k+kc}{None}\PY{p}{)} 
\PY{n+nb}{print}\PY{p}{(}\PY{n}{df\PYZus{}while}\PY{p}{)}
\end{Verbatim}
\end{tcolorbox}

    \begin{Verbatim}[commandchars=\\\{\}]
    0
0   0
1   0
2   0
3   0
4   0
.. ..
95  0
96  0
97  0
98  0
99  0

[100 rows x 1 columns]
    \end{Verbatim}

    \begin{tcolorbox}[breakable, size=fbox, boxrule=1pt, pad at break*=1mm,colback=cellbackground, colframe=cellborder]
\prompt{In}{incolor}{5}{\boxspacing}
\begin{Verbatim}[commandchars=\\\{\}]
\PY{n}{df\PYZus{}while}\PY{p}{[}\PY{n}{df\PYZus{}while} \PY{o}{==} \PY{l+m+mi}{0}\PY{p}{]}
\end{Verbatim}
\end{tcolorbox}

            \begin{tcolorbox}[breakable, size=fbox, boxrule=.5pt, pad at break*=1mm, opacityfill=0]
\prompt{Out}{outcolor}{5}{\boxspacing}
\begin{Verbatim}[commandchars=\\\{\}]
    0
0   0
1   0
2   0
3   0
4   0
.. ..
95  0
96  0
97  0
98  0
99  0

[100 rows x 1 columns]
\end{Verbatim}
\end{tcolorbox}
        
    Jak widać z powyższego wyniku, wszystkie \texttt{100} wartości są równe
\texttt{0}, a więc semafor działa poprawnie.

    \hypertarget{semafor-licznikowy}{%
\subsection{Semafor licznikowy}\label{semafor-licznikowy}}

Poniżej został przetestowany semafor licznikowy jako semafor binarny.

W celach lepszego zrozumienia, warto dodać, że klasa zsynchronizowanego
licznika wygląda następująco:

\begin{Shaded}
\begin{Highlighting}[]
\CommentTok{// SynchronizedCounter.java}

\KeywordTok{package}\ImportTok{ pl}\OperatorTok{.}\ImportTok{edu}\OperatorTok{.}\ImportTok{agh}\OperatorTok{.}\ImportTok{tw}\OperatorTok{.}\ImportTok{knapp}\OperatorTok{;}

\KeywordTok{public} \KeywordTok{class}\NormalTok{ SynchronizedCounter }\KeywordTok{extends}\NormalTok{ Counter }\OperatorTok{\{}
    \KeywordTok{private} \DataTypeTok{final}\NormalTok{ ISemaphore semaphore}\OperatorTok{;}

    \KeywordTok{public} \FunctionTok{SynchronizedCounter}\OperatorTok{(}\DataTypeTok{int}\NormalTok{ n}\OperatorTok{,}\NormalTok{ ISemaphore semaphore}\OperatorTok{)} \OperatorTok{\{}
        \KeywordTok{super}\OperatorTok{(}\NormalTok{n}\OperatorTok{);}
        \KeywordTok{this}\OperatorTok{.}\FunctionTok{semaphore} \OperatorTok{=}\NormalTok{ semaphore}\OperatorTok{;}
    \OperatorTok{\}}

    \AttributeTok{@Override}
    \KeywordTok{public} \DataTypeTok{void} \FunctionTok{inc}\OperatorTok{()} \OperatorTok{\{}
\NormalTok{        semaphore}\OperatorTok{.}\FunctionTok{P}\OperatorTok{();}
        \KeywordTok{super}\OperatorTok{.}\FunctionTok{inc}\OperatorTok{();}
\NormalTok{        semaphore}\OperatorTok{.}\FunctionTok{V}\OperatorTok{();}
    \OperatorTok{\}}

    \AttributeTok{@Override}
    \KeywordTok{public} \DataTypeTok{void} \FunctionTok{dec}\OperatorTok{()} \OperatorTok{\{}
\NormalTok{        semaphore}\OperatorTok{.}\FunctionTok{P}\OperatorTok{();}
        \KeywordTok{super}\OperatorTok{.}\FunctionTok{dec}\OperatorTok{();}
\NormalTok{        semaphore}\OperatorTok{.}\FunctionTok{V}\OperatorTok{();}
    \OperatorTok{\}}
\OperatorTok{\}}
\end{Highlighting}
\end{Shaded}

gdzie \texttt{ISemaphore} - interfejs, reprezentujący semafor:

\begin{Shaded}
\begin{Highlighting}[]
\CommentTok{// ISemaphore.java}

\KeywordTok{package}\ImportTok{ pl}\OperatorTok{.}\ImportTok{edu}\OperatorTok{.}\ImportTok{agh}\OperatorTok{.}\ImportTok{tw}\OperatorTok{.}\ImportTok{knapp}\OperatorTok{;}

\KeywordTok{public} \KeywordTok{interface}\NormalTok{ ISemaphore }\OperatorTok{\{}
    \DataTypeTok{void} \FunctionTok{P}\OperatorTok{();}
    \DataTypeTok{void} \FunctionTok{V}\OperatorTok{();}
\OperatorTok{\}}
\end{Highlighting}
\end{Shaded}

    \hypertarget{semafor-licznikowy-zaimplementowany-za-pomocux105-semaforuxf3w-binarnych}{%
\subsubsection{Semafor licznikowy zaimplementowany za pomocą semaforów
binarnych}\label{semafor-licznikowy-zaimplementowany-za-pomocux105-semaforuxf3w-binarnych}}

Zsynchronizowany licznik jest tworzony w następujący sposób:

\begin{Shaded}
\begin{Highlighting}[]
\DataTypeTok{var}\NormalTok{ counter }\OperatorTok{=} \KeywordTok{new} \FunctionTok{SynchronizedCounter}\OperatorTok{(}\DecValTok{0}\OperatorTok{,} \KeywordTok{new} \FunctionTok{CountingSemaphore}\OperatorTok{(}\DecValTok{1}\OperatorTok{));}
\end{Highlighting}
\end{Shaded}

Po uruchomieniu uzyskuję wyniki i zapisuję je do pliku
\texttt{counting\_semaphore.txt}, następnie przetwarzam:

    \begin{tcolorbox}[breakable, size=fbox, boxrule=1pt, pad at break*=1mm,colback=cellbackground, colframe=cellborder]
\prompt{In}{incolor}{6}{\boxspacing}
\begin{Verbatim}[commandchars=\\\{\}]
\PY{n}{df\PYZus{}counting} \PY{o}{=} \PY{n}{pd}\PY{o}{.}\PY{n}{read\PYZus{}csv}\PY{p}{(}\PY{l+s+s2}{\PYZdq{}}\PY{l+s+s2}{counting\PYZus{}semaphore.txt}\PY{l+s+s2}{\PYZdq{}}\PY{p}{,} \PY{n}{header}\PY{o}{=}\PY{k+kc}{None}\PY{p}{)} 
\PY{n+nb}{print}\PY{p}{(}\PY{n}{df\PYZus{}counting}\PY{p}{)}
\end{Verbatim}
\end{tcolorbox}

    \begin{Verbatim}[commandchars=\\\{\}]
    0
0   0
1   0
2   0
3   0
4   0
.. ..
95  0
96  0
97  0
98  0
99  0

[100 rows x 1 columns]
    \end{Verbatim}

    \begin{tcolorbox}[breakable, size=fbox, boxrule=1pt, pad at break*=1mm,colback=cellbackground, colframe=cellborder]
\prompt{In}{incolor}{7}{\boxspacing}
\begin{Verbatim}[commandchars=\\\{\}]
\PY{n}{df\PYZus{}counting}\PY{p}{[}\PY{n}{df\PYZus{}counting} \PY{o}{==} \PY{l+m+mi}{0}\PY{p}{]}
\end{Verbatim}
\end{tcolorbox}

            \begin{tcolorbox}[breakable, size=fbox, boxrule=.5pt, pad at break*=1mm, opacityfill=0]
\prompt{Out}{outcolor}{7}{\boxspacing}
\begin{Verbatim}[commandchars=\\\{\}]
    0
0   0
1   0
2   0
3   0
4   0
.. ..
95  0
96  0
97  0
98  0
99  0

[100 rows x 1 columns]
\end{Verbatim}
\end{tcolorbox}
        
    Mechanizm synchronizacji działa poprawnie - na wyjściu mamy same zera.

    \hypertarget{semafor-licznikowy-oguxf3lny}{%
\subsubsection{Semafor licznikowy
ogólny}\label{semafor-licznikowy-oguxf3lny}}

W tym rozdziale zostały zareprezentowane wyniki działania semafora
ogólnego, zaimplementowanego bez korzystania z semaforów binarnych.

Zsynchronizowany licznik jest tworzony w następujący sposób:

\begin{Shaded}
\begin{Highlighting}[]
\DataTypeTok{var}\NormalTok{ counter }\OperatorTok{=} \KeywordTok{new} \FunctionTok{SynchronizedCounter}\OperatorTok{(}\DecValTok{0}\OperatorTok{,} \KeywordTok{new} \BuiltInTok{Semaphore}\OperatorTok{(}\DecValTok{1}\OperatorTok{));}
\end{Highlighting}
\end{Shaded}

Po uruchomieniu uzyskuję wyniki i zapisuję je do pliku
\texttt{semaphore.txt}, następnie przetwarzam:

    \begin{tcolorbox}[breakable, size=fbox, boxrule=1pt, pad at break*=1mm,colback=cellbackground, colframe=cellborder]
\prompt{In}{incolor}{8}{\boxspacing}
\begin{Verbatim}[commandchars=\\\{\}]
\PY{n}{df\PYZus{}generalized} \PY{o}{=} \PY{n}{pd}\PY{o}{.}\PY{n}{read\PYZus{}csv}\PY{p}{(}\PY{l+s+s2}{\PYZdq{}}\PY{l+s+s2}{semaphore.txt}\PY{l+s+s2}{\PYZdq{}}\PY{p}{,} \PY{n}{header}\PY{o}{=}\PY{k+kc}{None}\PY{p}{)} 
\PY{n+nb}{print}\PY{p}{(}\PY{n}{df\PYZus{}generalized}\PY{p}{)}
\end{Verbatim}
\end{tcolorbox}

    \begin{Verbatim}[commandchars=\\\{\}]
    0
0   0
1   0
2   0
3   0
4   0
.. ..
95  0
96  0
97  0
98  0
99  0

[100 rows x 1 columns]
    \end{Verbatim}

    \begin{tcolorbox}[breakable, size=fbox, boxrule=1pt, pad at break*=1mm,colback=cellbackground, colframe=cellborder]
\prompt{In}{incolor}{9}{\boxspacing}
\begin{Verbatim}[commandchars=\\\{\}]
\PY{n}{df\PYZus{}generalized}\PY{p}{[}\PY{n}{df\PYZus{}generalized} \PY{o}{==} \PY{l+m+mi}{0}\PY{p}{]}
\end{Verbatim}
\end{tcolorbox}

            \begin{tcolorbox}[breakable, size=fbox, boxrule=.5pt, pad at break*=1mm, opacityfill=0]
\prompt{Out}{outcolor}{9}{\boxspacing}
\begin{Verbatim}[commandchars=\\\{\}]
    0
0   0
1   0
2   0
3   0
4   0
.. ..
95  0
96  0
97  0
98  0
99  0

[100 rows x 1 columns]
\end{Verbatim}
\end{tcolorbox}
        
    Mechanizm synchronizacji działa poprawnie - na wyjściu mamy same zera.

    \hypertarget{wnioski}{%
\section{Wnioski}\label{wnioski}}

\begin{itemize}
\tightlist
\item
  Monitory w Javie synchronizują dostęp do metod i bloków synchronized i
  są związane z każdym obiektem
\item
  Metod \texttt{wait}, \texttt{notify} / \texttt{notifyAll} można używać
  do synchronizacji działania wielu wątków
\item
  Semafor binarny da się zaimplementować za pomoca metod \texttt{wait} i
  \texttt{notify} i słowa kluczowego \texttt{synchronized}
\item
  Do implementacji semafora za pomocą metod \texttt{wait} i
  \texttt{notify} nie wystarczy instrukcji \texttt{if}, trzeba użyć
  instrukcji \texttt{while}
\item
  Semafor licznikowy jest semaforem ogólnym - semafor binarny jest
  szczególnym przypadkiem semafora licznikowego (jest to semafor
  licznikowy z wartością początkową \texttt{1})
\item
  Semafor licznikowy da się zaimplementować korzystając tylko z
  semaforów binarnych, choć nie w bardzo wydajny sposób
\end{itemize}

    \hypertarget{bibliografia}{%
\section{Bibliografia}\label{bibliografia}}

\begin{enumerate}
\def\labelenumi{\arabic{enumi}.}
\tightlist
\item
  \href{https://home.agh.edu.pl/~funika/tw/lab2/}{Materiały do
  laboratorium}
\item
  \href{https://docs.oracle.com/en/java/javase/17/docs/api/java.base/java/lang/Object.html}{Java
  17 Docs - Object}
\item
  \href{https://en.wikipedia.org/wiki/Spurious_wakeup}{Wikipedia -
  Spurious wakeup}
\end{enumerate}


    % Add a bibliography block to the postdoc
    
    
    
\end{document}
