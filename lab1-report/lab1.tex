\documentclass[11pt]{article}

    \usepackage[breakable]{tcolorbox}
    \usepackage{parskip} % Stop auto-indenting (to mimic markdown behaviour)
    

    % Basic figure setup, for now with no caption control since it's done
    % automatically by Pandoc (which extracts ![](path) syntax from Markdown).
    \usepackage{graphicx}
    % Maintain compatibility with old templates. Remove in nbconvert 6.0
    \let\Oldincludegraphics\includegraphics
    % Ensure that by default, figures have no caption (until we provide a
    % proper Figure object with a Caption API and a way to capture that
    % in the conversion process - todo).
    \usepackage{caption}
    \DeclareCaptionFormat{nocaption}{}
    \captionsetup{format=nocaption,aboveskip=0pt,belowskip=0pt}

    \usepackage{float}
    \floatplacement{figure}{H} % forces figures to be placed at the correct location
    \usepackage{xcolor} % Allow colors to be defined
    \usepackage{enumerate} % Needed for markdown enumerations to work
    \usepackage{geometry} % Used to adjust the document margins
    \usepackage{amsmath} % Equations
    \usepackage{amssymb} % Equations
    \usepackage{textcomp} % defines textquotesingle
    % Hack from http://tex.stackexchange.com/a/47451/13684:
    \AtBeginDocument{%
        \def\PYZsq{\textquotesingle}% Upright quotes in Pygmentized code
    }
    \usepackage{upquote} % Upright quotes for verbatim code
    \usepackage{eurosym} % defines \euro

    \usepackage{iftex}
    \ifPDFTeX
        \usepackage[T1]{fontenc}
        \IfFileExists{alphabeta.sty}{
              \usepackage{alphabeta}
          }{
              \usepackage[mathletters]{ucs}
              \usepackage[utf8x]{inputenc}
          }
    \else
        \usepackage{fontspec}
        \usepackage{unicode-math}
    \fi

    \usepackage{fancyvrb} % verbatim replacement that allows latex
    \usepackage{grffile} % extends the file name processing of package graphics
                         % to support a larger range
    \makeatletter % fix for old versions of grffile with XeLaTeX
    \@ifpackagelater{grffile}{2019/11/01}
    {
      % Do nothing on new versions
    }
    {
      \def\Gread@@xetex#1{%
        \IfFileExists{"\Gin@base".bb}%
        {\Gread@eps{\Gin@base.bb}}%
        {\Gread@@xetex@aux#1}%
      }
    }
    \makeatother
    \usepackage[Export]{adjustbox} % Used to constrain images to a maximum size
    \adjustboxset{max size={0.9\linewidth}{0.9\paperheight}}

    % The hyperref package gives us a pdf with properly built
    % internal navigation ('pdf bookmarks' for the table of contents,
    % internal cross-reference links, web links for URLs, etc.)
    \usepackage{hyperref}
    % The default LaTeX title has an obnoxious amount of whitespace. By default,
    % titling removes some of it. It also provides customization options.
    \usepackage{titling}
    \usepackage{longtable} % longtable support required by pandoc >1.10
    \usepackage{booktabs}  % table support for pandoc > 1.12.2
    \usepackage{array}     % table support for pandoc >= 2.11.3
    \usepackage{calc}      % table minipage width calculation for pandoc >= 2.11.1
    \usepackage[inline]{enumitem} % IRkernel/repr support (it uses the enumerate* environment)
    \usepackage[normalem]{ulem} % ulem is needed to support strikethroughs (\sout)
                                % normalem makes italics be italics, not underlines
    \usepackage{soul}      % strikethrough (\st) support for pandoc >= 3.0.0
    \usepackage{mathrsfs}
    

    
    % Colors for the hyperref package
    \definecolor{urlcolor}{rgb}{0,.145,.698}
    \definecolor{linkcolor}{rgb}{.71,0.21,0.01}
    \definecolor{citecolor}{rgb}{.12,.54,.11}

    % ANSI colors
    \definecolor{ansi-black}{HTML}{3E424D}
    \definecolor{ansi-black-intense}{HTML}{282C36}
    \definecolor{ansi-red}{HTML}{E75C58}
    \definecolor{ansi-red-intense}{HTML}{B22B31}
    \definecolor{ansi-green}{HTML}{00A250}
    \definecolor{ansi-green-intense}{HTML}{007427}
    \definecolor{ansi-yellow}{HTML}{DDB62B}
    \definecolor{ansi-yellow-intense}{HTML}{B27D12}
    \definecolor{ansi-blue}{HTML}{208FFB}
    \definecolor{ansi-blue-intense}{HTML}{0065CA}
    \definecolor{ansi-magenta}{HTML}{D160C4}
    \definecolor{ansi-magenta-intense}{HTML}{A03196}
    \definecolor{ansi-cyan}{HTML}{60C6C8}
    \definecolor{ansi-cyan-intense}{HTML}{258F8F}
    \definecolor{ansi-white}{HTML}{C5C1B4}
    \definecolor{ansi-white-intense}{HTML}{A1A6B2}
    \definecolor{ansi-default-inverse-fg}{HTML}{FFFFFF}
    \definecolor{ansi-default-inverse-bg}{HTML}{000000}

    % common color for the border for error outputs.
    \definecolor{outerrorbackground}{HTML}{FFDFDF}

    % commands and environments needed by pandoc snippets
    % extracted from the output of `pandoc -s`
    \providecommand{\tightlist}{%
      \setlength{\itemsep}{0pt}\setlength{\parskip}{0pt}}
    \DefineVerbatimEnvironment{Highlighting}{Verbatim}{commandchars=\\\{\}}
    % Add ',fontsize=\small' for more characters per line
    \newenvironment{Shaded}{}{}
    \newcommand{\KeywordTok}[1]{\textcolor[rgb]{0.00,0.44,0.13}{\textbf{{#1}}}}
    \newcommand{\DataTypeTok}[1]{\textcolor[rgb]{0.56,0.13,0.00}{{#1}}}
    \newcommand{\DecValTok}[1]{\textcolor[rgb]{0.25,0.63,0.44}{{#1}}}
    \newcommand{\BaseNTok}[1]{\textcolor[rgb]{0.25,0.63,0.44}{{#1}}}
    \newcommand{\FloatTok}[1]{\textcolor[rgb]{0.25,0.63,0.44}{{#1}}}
    \newcommand{\CharTok}[1]{\textcolor[rgb]{0.25,0.44,0.63}{{#1}}}
    \newcommand{\StringTok}[1]{\textcolor[rgb]{0.25,0.44,0.63}{{#1}}}
    \newcommand{\CommentTok}[1]{\textcolor[rgb]{0.38,0.63,0.69}{\textit{{#1}}}}
    \newcommand{\OtherTok}[1]{\textcolor[rgb]{0.00,0.44,0.13}{{#1}}}
    \newcommand{\AlertTok}[1]{\textcolor[rgb]{1.00,0.00,0.00}{\textbf{{#1}}}}
    \newcommand{\FunctionTok}[1]{\textcolor[rgb]{0.02,0.16,0.49}{{#1}}}
    \newcommand{\RegionMarkerTok}[1]{{#1}}
    \newcommand{\ErrorTok}[1]{\textcolor[rgb]{1.00,0.00,0.00}{\textbf{{#1}}}}
    \newcommand{\NormalTok}[1]{{#1}}

    % Additional commands for more recent versions of Pandoc
    \newcommand{\ConstantTok}[1]{\textcolor[rgb]{0.53,0.00,0.00}{{#1}}}
    \newcommand{\SpecialCharTok}[1]{\textcolor[rgb]{0.25,0.44,0.63}{{#1}}}
    \newcommand{\VerbatimStringTok}[1]{\textcolor[rgb]{0.25,0.44,0.63}{{#1}}}
    \newcommand{\SpecialStringTok}[1]{\textcolor[rgb]{0.73,0.40,0.53}{{#1}}}
    \newcommand{\ImportTok}[1]{{#1}}
    \newcommand{\DocumentationTok}[1]{\textcolor[rgb]{0.73,0.13,0.13}{\textit{{#1}}}}
    \newcommand{\AnnotationTok}[1]{\textcolor[rgb]{0.38,0.63,0.69}{\textbf{\textit{{#1}}}}}
    \newcommand{\CommentVarTok}[1]{\textcolor[rgb]{0.38,0.63,0.69}{\textbf{\textit{{#1}}}}}
    \newcommand{\VariableTok}[1]{\textcolor[rgb]{0.10,0.09,0.49}{{#1}}}
    \newcommand{\ControlFlowTok}[1]{\textcolor[rgb]{0.00,0.44,0.13}{\textbf{{#1}}}}
    \newcommand{\OperatorTok}[1]{\textcolor[rgb]{0.40,0.40,0.40}{{#1}}}
    \newcommand{\BuiltInTok}[1]{{#1}}
    \newcommand{\ExtensionTok}[1]{{#1}}
    \newcommand{\PreprocessorTok}[1]{\textcolor[rgb]{0.74,0.48,0.00}{{#1}}}
    \newcommand{\AttributeTok}[1]{\textcolor[rgb]{0.49,0.56,0.16}{{#1}}}
    \newcommand{\InformationTok}[1]{\textcolor[rgb]{0.38,0.63,0.69}{\textbf{\textit{{#1}}}}}
    \newcommand{\WarningTok}[1]{\textcolor[rgb]{0.38,0.63,0.69}{\textbf{\textit{{#1}}}}}


    % Define a nice break command that doesn't care if a line doesn't already
    % exist.
    \def\br{\hspace*{\fill} \\* }
    % Math Jax compatibility definitions
    \def\gt{>}
    \def\lt{<}
    \let\Oldtex\TeX
    \let\Oldlatex\LaTeX
    \renewcommand{\TeX}{\textrm{\Oldtex}}
    \renewcommand{\LaTeX}{\textrm{\Oldlatex}}
    % Document parameters
    % Document title
    \title{lab1}
    
    
    
    
    
    
    
% Pygments definitions
\makeatletter
\def\PY@reset{\let\PY@it=\relax \let\PY@bf=\relax%
    \let\PY@ul=\relax \let\PY@tc=\relax%
    \let\PY@bc=\relax \let\PY@ff=\relax}
\def\PY@tok#1{\csname PY@tok@#1\endcsname}
\def\PY@toks#1+{\ifx\relax#1\empty\else%
    \PY@tok{#1}\expandafter\PY@toks\fi}
\def\PY@do#1{\PY@bc{\PY@tc{\PY@ul{%
    \PY@it{\PY@bf{\PY@ff{#1}}}}}}}
\def\PY#1#2{\PY@reset\PY@toks#1+\relax+\PY@do{#2}}

\@namedef{PY@tok@w}{\def\PY@tc##1{\textcolor[rgb]{0.73,0.73,0.73}{##1}}}
\@namedef{PY@tok@c}{\let\PY@it=\textit\def\PY@tc##1{\textcolor[rgb]{0.24,0.48,0.48}{##1}}}
\@namedef{PY@tok@cp}{\def\PY@tc##1{\textcolor[rgb]{0.61,0.40,0.00}{##1}}}
\@namedef{PY@tok@k}{\let\PY@bf=\textbf\def\PY@tc##1{\textcolor[rgb]{0.00,0.50,0.00}{##1}}}
\@namedef{PY@tok@kp}{\def\PY@tc##1{\textcolor[rgb]{0.00,0.50,0.00}{##1}}}
\@namedef{PY@tok@kt}{\def\PY@tc##1{\textcolor[rgb]{0.69,0.00,0.25}{##1}}}
\@namedef{PY@tok@o}{\def\PY@tc##1{\textcolor[rgb]{0.40,0.40,0.40}{##1}}}
\@namedef{PY@tok@ow}{\let\PY@bf=\textbf\def\PY@tc##1{\textcolor[rgb]{0.67,0.13,1.00}{##1}}}
\@namedef{PY@tok@nb}{\def\PY@tc##1{\textcolor[rgb]{0.00,0.50,0.00}{##1}}}
\@namedef{PY@tok@nf}{\def\PY@tc##1{\textcolor[rgb]{0.00,0.00,1.00}{##1}}}
\@namedef{PY@tok@nc}{\let\PY@bf=\textbf\def\PY@tc##1{\textcolor[rgb]{0.00,0.00,1.00}{##1}}}
\@namedef{PY@tok@nn}{\let\PY@bf=\textbf\def\PY@tc##1{\textcolor[rgb]{0.00,0.00,1.00}{##1}}}
\@namedef{PY@tok@ne}{\let\PY@bf=\textbf\def\PY@tc##1{\textcolor[rgb]{0.80,0.25,0.22}{##1}}}
\@namedef{PY@tok@nv}{\def\PY@tc##1{\textcolor[rgb]{0.10,0.09,0.49}{##1}}}
\@namedef{PY@tok@no}{\def\PY@tc##1{\textcolor[rgb]{0.53,0.00,0.00}{##1}}}
\@namedef{PY@tok@nl}{\def\PY@tc##1{\textcolor[rgb]{0.46,0.46,0.00}{##1}}}
\@namedef{PY@tok@ni}{\let\PY@bf=\textbf\def\PY@tc##1{\textcolor[rgb]{0.44,0.44,0.44}{##1}}}
\@namedef{PY@tok@na}{\def\PY@tc##1{\textcolor[rgb]{0.41,0.47,0.13}{##1}}}
\@namedef{PY@tok@nt}{\let\PY@bf=\textbf\def\PY@tc##1{\textcolor[rgb]{0.00,0.50,0.00}{##1}}}
\@namedef{PY@tok@nd}{\def\PY@tc##1{\textcolor[rgb]{0.67,0.13,1.00}{##1}}}
\@namedef{PY@tok@s}{\def\PY@tc##1{\textcolor[rgb]{0.73,0.13,0.13}{##1}}}
\@namedef{PY@tok@sd}{\let\PY@it=\textit\def\PY@tc##1{\textcolor[rgb]{0.73,0.13,0.13}{##1}}}
\@namedef{PY@tok@si}{\let\PY@bf=\textbf\def\PY@tc##1{\textcolor[rgb]{0.64,0.35,0.47}{##1}}}
\@namedef{PY@tok@se}{\let\PY@bf=\textbf\def\PY@tc##1{\textcolor[rgb]{0.67,0.36,0.12}{##1}}}
\@namedef{PY@tok@sr}{\def\PY@tc##1{\textcolor[rgb]{0.64,0.35,0.47}{##1}}}
\@namedef{PY@tok@ss}{\def\PY@tc##1{\textcolor[rgb]{0.10,0.09,0.49}{##1}}}
\@namedef{PY@tok@sx}{\def\PY@tc##1{\textcolor[rgb]{0.00,0.50,0.00}{##1}}}
\@namedef{PY@tok@m}{\def\PY@tc##1{\textcolor[rgb]{0.40,0.40,0.40}{##1}}}
\@namedef{PY@tok@gh}{\let\PY@bf=\textbf\def\PY@tc##1{\textcolor[rgb]{0.00,0.00,0.50}{##1}}}
\@namedef{PY@tok@gu}{\let\PY@bf=\textbf\def\PY@tc##1{\textcolor[rgb]{0.50,0.00,0.50}{##1}}}
\@namedef{PY@tok@gd}{\def\PY@tc##1{\textcolor[rgb]{0.63,0.00,0.00}{##1}}}
\@namedef{PY@tok@gi}{\def\PY@tc##1{\textcolor[rgb]{0.00,0.52,0.00}{##1}}}
\@namedef{PY@tok@gr}{\def\PY@tc##1{\textcolor[rgb]{0.89,0.00,0.00}{##1}}}
\@namedef{PY@tok@ge}{\let\PY@it=\textit}
\@namedef{PY@tok@gs}{\let\PY@bf=\textbf}
\@namedef{PY@tok@ges}{\let\PY@bf=\textbf\let\PY@it=\textit}
\@namedef{PY@tok@gp}{\let\PY@bf=\textbf\def\PY@tc##1{\textcolor[rgb]{0.00,0.00,0.50}{##1}}}
\@namedef{PY@tok@go}{\def\PY@tc##1{\textcolor[rgb]{0.44,0.44,0.44}{##1}}}
\@namedef{PY@tok@gt}{\def\PY@tc##1{\textcolor[rgb]{0.00,0.27,0.87}{##1}}}
\@namedef{PY@tok@err}{\def\PY@bc##1{{\setlength{\fboxsep}{\string -\fboxrule}\fcolorbox[rgb]{1.00,0.00,0.00}{1,1,1}{\strut ##1}}}}
\@namedef{PY@tok@kc}{\let\PY@bf=\textbf\def\PY@tc##1{\textcolor[rgb]{0.00,0.50,0.00}{##1}}}
\@namedef{PY@tok@kd}{\let\PY@bf=\textbf\def\PY@tc##1{\textcolor[rgb]{0.00,0.50,0.00}{##1}}}
\@namedef{PY@tok@kn}{\let\PY@bf=\textbf\def\PY@tc##1{\textcolor[rgb]{0.00,0.50,0.00}{##1}}}
\@namedef{PY@tok@kr}{\let\PY@bf=\textbf\def\PY@tc##1{\textcolor[rgb]{0.00,0.50,0.00}{##1}}}
\@namedef{PY@tok@bp}{\def\PY@tc##1{\textcolor[rgb]{0.00,0.50,0.00}{##1}}}
\@namedef{PY@tok@fm}{\def\PY@tc##1{\textcolor[rgb]{0.00,0.00,1.00}{##1}}}
\@namedef{PY@tok@vc}{\def\PY@tc##1{\textcolor[rgb]{0.10,0.09,0.49}{##1}}}
\@namedef{PY@tok@vg}{\def\PY@tc##1{\textcolor[rgb]{0.10,0.09,0.49}{##1}}}
\@namedef{PY@tok@vi}{\def\PY@tc##1{\textcolor[rgb]{0.10,0.09,0.49}{##1}}}
\@namedef{PY@tok@vm}{\def\PY@tc##1{\textcolor[rgb]{0.10,0.09,0.49}{##1}}}
\@namedef{PY@tok@sa}{\def\PY@tc##1{\textcolor[rgb]{0.73,0.13,0.13}{##1}}}
\@namedef{PY@tok@sb}{\def\PY@tc##1{\textcolor[rgb]{0.73,0.13,0.13}{##1}}}
\@namedef{PY@tok@sc}{\def\PY@tc##1{\textcolor[rgb]{0.73,0.13,0.13}{##1}}}
\@namedef{PY@tok@dl}{\def\PY@tc##1{\textcolor[rgb]{0.73,0.13,0.13}{##1}}}
\@namedef{PY@tok@s2}{\def\PY@tc##1{\textcolor[rgb]{0.73,0.13,0.13}{##1}}}
\@namedef{PY@tok@sh}{\def\PY@tc##1{\textcolor[rgb]{0.73,0.13,0.13}{##1}}}
\@namedef{PY@tok@s1}{\def\PY@tc##1{\textcolor[rgb]{0.73,0.13,0.13}{##1}}}
\@namedef{PY@tok@mb}{\def\PY@tc##1{\textcolor[rgb]{0.40,0.40,0.40}{##1}}}
\@namedef{PY@tok@mf}{\def\PY@tc##1{\textcolor[rgb]{0.40,0.40,0.40}{##1}}}
\@namedef{PY@tok@mh}{\def\PY@tc##1{\textcolor[rgb]{0.40,0.40,0.40}{##1}}}
\@namedef{PY@tok@mi}{\def\PY@tc##1{\textcolor[rgb]{0.40,0.40,0.40}{##1}}}
\@namedef{PY@tok@il}{\def\PY@tc##1{\textcolor[rgb]{0.40,0.40,0.40}{##1}}}
\@namedef{PY@tok@mo}{\def\PY@tc##1{\textcolor[rgb]{0.40,0.40,0.40}{##1}}}
\@namedef{PY@tok@ch}{\let\PY@it=\textit\def\PY@tc##1{\textcolor[rgb]{0.24,0.48,0.48}{##1}}}
\@namedef{PY@tok@cm}{\let\PY@it=\textit\def\PY@tc##1{\textcolor[rgb]{0.24,0.48,0.48}{##1}}}
\@namedef{PY@tok@cpf}{\let\PY@it=\textit\def\PY@tc##1{\textcolor[rgb]{0.24,0.48,0.48}{##1}}}
\@namedef{PY@tok@c1}{\let\PY@it=\textit\def\PY@tc##1{\textcolor[rgb]{0.24,0.48,0.48}{##1}}}
\@namedef{PY@tok@cs}{\let\PY@it=\textit\def\PY@tc##1{\textcolor[rgb]{0.24,0.48,0.48}{##1}}}

\def\PYZbs{\char`\\}
\def\PYZus{\char`\_}
\def\PYZob{\char`\{}
\def\PYZcb{\char`\}}
\def\PYZca{\char`\^}
\def\PYZam{\char`\&}
\def\PYZlt{\char`\<}
\def\PYZgt{\char`\>}
\def\PYZsh{\char`\#}
\def\PYZpc{\char`\%}
\def\PYZdl{\char`\$}
\def\PYZhy{\char`\-}
\def\PYZsq{\char`\'}
\def\PYZdq{\char`\"}
\def\PYZti{\char`\~}
% for compatibility with earlier versions
\def\PYZat{@}
\def\PYZlb{[}
\def\PYZrb{]}
\makeatother


    % For linebreaks inside Verbatim environment from package fancyvrb.
    \makeatletter
        \newbox\Wrappedcontinuationbox
        \newbox\Wrappedvisiblespacebox
        \newcommand*\Wrappedvisiblespace {\textcolor{red}{\textvisiblespace}}
        \newcommand*\Wrappedcontinuationsymbol {\textcolor{red}{\llap{\tiny$\m@th\hookrightarrow$}}}
        \newcommand*\Wrappedcontinuationindent {3ex }
        \newcommand*\Wrappedafterbreak {\kern\Wrappedcontinuationindent\copy\Wrappedcontinuationbox}
        % Take advantage of the already applied Pygments mark-up to insert
        % potential linebreaks for TeX processing.
        %        {, <, #, %, $, ' and ": go to next line.
        %        _, }, ^, &, >, - and ~: stay at end of broken line.
        % Use of \textquotesingle for straight quote.
        \newcommand*\Wrappedbreaksatspecials {%
            \def\PYGZus{\discretionary{\char`\_}{\Wrappedafterbreak}{\char`\_}}%
            \def\PYGZob{\discretionary{}{\Wrappedafterbreak\char`\{}{\char`\{}}%
            \def\PYGZcb{\discretionary{\char`\}}{\Wrappedafterbreak}{\char`\}}}%
            \def\PYGZca{\discretionary{\char`\^}{\Wrappedafterbreak}{\char`\^}}%
            \def\PYGZam{\discretionary{\char`\&}{\Wrappedafterbreak}{\char`\&}}%
            \def\PYGZlt{\discretionary{}{\Wrappedafterbreak\char`\<}{\char`\<}}%
            \def\PYGZgt{\discretionary{\char`\>}{\Wrappedafterbreak}{\char`\>}}%
            \def\PYGZsh{\discretionary{}{\Wrappedafterbreak\char`\#}{\char`\#}}%
            \def\PYGZpc{\discretionary{}{\Wrappedafterbreak\char`\%}{\char`\%}}%
            \def\PYGZdl{\discretionary{}{\Wrappedafterbreak\char`\$}{\char`\$}}%
            \def\PYGZhy{\discretionary{\char`\-}{\Wrappedafterbreak}{\char`\-}}%
            \def\PYGZsq{\discretionary{}{\Wrappedafterbreak\textquotesingle}{\textquotesingle}}%
            \def\PYGZdq{\discretionary{}{\Wrappedafterbreak\char`\"}{\char`\"}}%
            \def\PYGZti{\discretionary{\char`\~}{\Wrappedafterbreak}{\char`\~}}%
        }
        % Some characters . , ; ? ! / are not pygmentized.
        % This macro makes them "active" and they will insert potential linebreaks
        \newcommand*\Wrappedbreaksatpunct {%
            \lccode`\~`\.\lowercase{\def~}{\discretionary{\hbox{\char`\.}}{\Wrappedafterbreak}{\hbox{\char`\.}}}%
            \lccode`\~`\,\lowercase{\def~}{\discretionary{\hbox{\char`\,}}{\Wrappedafterbreak}{\hbox{\char`\,}}}%
            \lccode`\~`\;\lowercase{\def~}{\discretionary{\hbox{\char`\;}}{\Wrappedafterbreak}{\hbox{\char`\;}}}%
            \lccode`\~`\:\lowercase{\def~}{\discretionary{\hbox{\char`\:}}{\Wrappedafterbreak}{\hbox{\char`\:}}}%
            \lccode`\~`\?\lowercase{\def~}{\discretionary{\hbox{\char`\?}}{\Wrappedafterbreak}{\hbox{\char`\?}}}%
            \lccode`\~`\!\lowercase{\def~}{\discretionary{\hbox{\char`\!}}{\Wrappedafterbreak}{\hbox{\char`\!}}}%
            \lccode`\~`\/\lowercase{\def~}{\discretionary{\hbox{\char`\/}}{\Wrappedafterbreak}{\hbox{\char`\/}}}%
            \catcode`\.\active
            \catcode`\,\active
            \catcode`\;\active
            \catcode`\:\active
            \catcode`\?\active
            \catcode`\!\active
            \catcode`\/\active
            \lccode`\~`\~
        }
    \makeatother

    \let\OriginalVerbatim=\Verbatim
    \makeatletter
    \renewcommand{\Verbatim}[1][1]{%
        %\parskip\z@skip
        \sbox\Wrappedcontinuationbox {\Wrappedcontinuationsymbol}%
        \sbox\Wrappedvisiblespacebox {\FV@SetupFont\Wrappedvisiblespace}%
        \def\FancyVerbFormatLine ##1{\hsize\linewidth
            \vtop{\raggedright\hyphenpenalty\z@\exhyphenpenalty\z@
                \doublehyphendemerits\z@\finalhyphendemerits\z@
                \strut ##1\strut}%
        }%
        % If the linebreak is at a space, the latter will be displayed as visible
        % space at end of first line, and a continuation symbol starts next line.
        % Stretch/shrink are however usually zero for typewriter font.
        \def\FV@Space {%
            \nobreak\hskip\z@ plus\fontdimen3\font minus\fontdimen4\font
            \discretionary{\copy\Wrappedvisiblespacebox}{\Wrappedafterbreak}
            {\kern\fontdimen2\font}%
        }%

        % Allow breaks at special characters using \PYG... macros.
        \Wrappedbreaksatspecials
        % Breaks at punctuation characters . , ; ? ! and / need catcode=\active
        \OriginalVerbatim[#1,codes*=\Wrappedbreaksatpunct]%
    }
    \makeatother

    % Exact colors from NB
    \definecolor{incolor}{HTML}{303F9F}
    \definecolor{outcolor}{HTML}{D84315}
    \definecolor{cellborder}{HTML}{CFCFCF}
    \definecolor{cellbackground}{HTML}{F7F7F7}

    % prompt
    \makeatletter
    \newcommand{\boxspacing}{\kern\kvtcb@left@rule\kern\kvtcb@boxsep}
    \makeatother
    \newcommand{\prompt}[4]{
        {\ttfamily\llap{{\color{#2}[#3]:\hspace{3pt}#4}}\vspace{-\baselineskip}}
    }
    

    
    % Prevent overflowing lines due to hard-to-break entities
    \sloppy
    % Setup hyperref package
    \hypersetup{
      breaklinks=true,  % so long urls are correctly broken across lines
      colorlinks=true,
      urlcolor=urlcolor,
      linkcolor=linkcolor,
      citecolor=citecolor,
      }
    % Slightly bigger margins than the latex defaults
    
    \geometry{verbose,tmargin=1in,bmargin=1in,lmargin=1in,rmargin=1in}
    
    

\begin{document}
    
    % \maketitle

    \begin{titlepage}
        \begin{center}
            \vspace*{1cm}
     
            \textbf{Laboratorium 1}
     
            \vspace{0.5cm}
            Współbieżność w Javie
                 
            \vspace{1.5cm}
     
            \textbf{Danylo Knapp}
    
            \vfill
    
            \includegraphics[width=0.4\textwidth]{agh-logo.png}
     
            \vfill
                 
            Teoria Współbieżności
                 
            \vspace{0.8cm}
    
            Wydział Informatyki\\
            Akademia Górniczo-Hutnicza\\
            im. Stanisława Staszica w Krakowie\\
            07.10.23
                 
        \end{center}
    \end{titlepage}
    
    

    
    \hypertarget{treux15bux107-zadania}{%
\section{Treść zadania}\label{treux15bux107-zadania}}

\begin{enumerate}
\def\labelenumi{\arabic{enumi}.}
\item
  Napisać program, który uruchamia 2 wątki, z których jeden zwiększa
  wartość zmiennej całkowitej o 1, drugi wątek zmniejsza wartość o 1.
  Zakładając, że na początku wartość zmiennej Counter była 0,
  chcielibyśmy wiedzieć jaka będzie wartość tej zmiennej po wykonaniu
  10000 operacji zwiększania i zmniejszania przez obydwa wątki.
\item
  Na podstawie 100 wykonań programu z p.1, stworzyć histogram końcowych
  wartości zmiennej Counter.
\item
  Spróbować wprowadzić mechanizm do programu z p.1, który
  zagwarantowałby przewidywalną końcową wartość zmiennej Counter. Nie
  używać żadnych systemowych mechanizmów, tylko swój autorski.
\item
  Napisać sprawozdanie z realizacji pp.~1-3, z argumentacją i
  interpretacją wyników.
\end{enumerate}

\hypertarget{rozwiux105zanie}{%
\section{Rozwiązanie}\label{rozwiux105zanie}}

Poniżej znajdują się 2 rozwiązania: pierwsze jest implementacją punktów
1 i 2, z kolei drugie jest realizacją punktu 3.

\hypertarget{race-condition-p.-1-2}{%
\subsection{Race condition (p.~1-2)}\label{race-condition-p.-1-2}}

Poniższe rozwiązanie demonstruje tzw. \emph{Race condition}, czyli
\emph{Wyścig}. Jest to zjawisko, które charakteryzuje się
niedeterministycznym zachowaniem programu, może powodować błędy trudne
do wykrycia i pojawia się wtedy, gdy więcej niż jeden wątek korzysta
jednocześnie z zasobu dzielonego, przy czym co najmniej jeden próbuje go
zmienić.

Żeby zademonstrować to zjawisko, wystarczy zaimplementować prosty
licznik, który będzie modyfikowany jednocześnie przez 2 wątki (np.
zwiększany przez jeden i zmniejszany przez drugi).

Przykładowa implementacja:

\textbf{Licznik:}

\begin{Shaded}
\begin{Highlighting}[]
\CommentTok{// Counter.java}

\KeywordTok{package}\ImportTok{ pl}\OperatorTok{.}\ImportTok{edu}\OperatorTok{.}\ImportTok{agh}\OperatorTok{.}\ImportTok{tw}\OperatorTok{.}\ImportTok{knapp}\OperatorTok{.}\ImportTok{counter}\OperatorTok{;}

\KeywordTok{public} \KeywordTok{class}\NormalTok{ Counter }\OperatorTok{\{}
    \KeywordTok{private} \DataTypeTok{int}\NormalTok{ \_val}\OperatorTok{;}

    \KeywordTok{public} \FunctionTok{Counter}\OperatorTok{(}\DataTypeTok{int}\NormalTok{ n}\OperatorTok{)} \OperatorTok{\{}
\NormalTok{        \_val }\OperatorTok{=}\NormalTok{ n}\OperatorTok{;}
    \OperatorTok{\}}

    \KeywordTok{public} \DataTypeTok{void} \FunctionTok{inc}\OperatorTok{()} \OperatorTok{\{}
\NormalTok{        \_val}\OperatorTok{++;}
    \OperatorTok{\}}

    \KeywordTok{public} \DataTypeTok{void} \FunctionTok{dec}\OperatorTok{()} \OperatorTok{\{}
\NormalTok{        \_val}\OperatorTok{{-}{-};}
    \OperatorTok{\}}

    \KeywordTok{public} \DataTypeTok{int} \FunctionTok{value}\OperatorTok{()} \OperatorTok{\{}
        \ControlFlowTok{return}\NormalTok{ \_val}\OperatorTok{;}
    \OperatorTok{\}}
\OperatorTok{\}}
\end{Highlighting}
\end{Shaded}

\textbf{Wątek zmniejszający licznik:}

\begin{Shaded}
\begin{Highlighting}[]
\CommentTok{// DThread.java}

\KeywordTok{package}\ImportTok{ pl}\OperatorTok{.}\ImportTok{edu}\OperatorTok{.}\ImportTok{agh}\OperatorTok{.}\ImportTok{tw}\OperatorTok{.}\ImportTok{knapp}\OperatorTok{;}

\KeywordTok{import} \ImportTok{pl}\OperatorTok{.}\ImportTok{edu}\OperatorTok{.}\ImportTok{agh}\OperatorTok{.}\ImportTok{tw}\OperatorTok{.}\ImportTok{knapp}\OperatorTok{.}\ImportTok{counter}\OperatorTok{.}\ImportTok{Counter}\OperatorTok{;}

\KeywordTok{public} \KeywordTok{class}\NormalTok{ DThread }\KeywordTok{extends} \BuiltInTok{Thread} \OperatorTok{\{}
    \KeywordTok{private} \DataTypeTok{final} \DataTypeTok{int}\NormalTok{ count}\OperatorTok{;}

    \KeywordTok{private}\NormalTok{ Counter counter}\OperatorTok{;}

    \KeywordTok{public} \FunctionTok{DThread}\OperatorTok{(}\DataTypeTok{int}\NormalTok{ count}\OperatorTok{)} \OperatorTok{\{}
        \KeywordTok{this}\OperatorTok{.}\FunctionTok{count} \OperatorTok{=}\NormalTok{ count}\OperatorTok{;}
    \OperatorTok{\}}

    \KeywordTok{public} \DataTypeTok{void} \FunctionTok{setCounter}\OperatorTok{(}\NormalTok{Counter counter}\OperatorTok{)} \OperatorTok{\{}
        \KeywordTok{this}\OperatorTok{.}\FunctionTok{counter} \OperatorTok{=}\NormalTok{ counter}\OperatorTok{;}
    \OperatorTok{\}}

    \AttributeTok{@Override}
    \KeywordTok{public} \DataTypeTok{void} \FunctionTok{run}\OperatorTok{()} \OperatorTok{\{}
        \ControlFlowTok{for} \OperatorTok{(}\DataTypeTok{int}\NormalTok{ i }\OperatorTok{=} \DecValTok{0}\OperatorTok{;}\NormalTok{ i }\OperatorTok{\textless{}}\NormalTok{ count}\OperatorTok{;}\NormalTok{ i}\OperatorTok{++)} \OperatorTok{\{}
\NormalTok{            counter}\OperatorTok{.}\FunctionTok{dec}\OperatorTok{();}
        \OperatorTok{\}}
    \OperatorTok{\}}
\OperatorTok{\}}
\end{Highlighting}
\end{Shaded}

\textbf{Wątek zwiększający licznik:}

\begin{Shaded}
\begin{Highlighting}[]
\CommentTok{// IThread.java}

\KeywordTok{package}\ImportTok{ pl}\OperatorTok{.}\ImportTok{edu}\OperatorTok{.}\ImportTok{agh}\OperatorTok{.}\ImportTok{tw}\OperatorTok{.}\ImportTok{knapp}\OperatorTok{;}

\KeywordTok{import} \ImportTok{pl}\OperatorTok{.}\ImportTok{edu}\OperatorTok{.}\ImportTok{agh}\OperatorTok{.}\ImportTok{tw}\OperatorTok{.}\ImportTok{knapp}\OperatorTok{.}\ImportTok{counter}\OperatorTok{.}\ImportTok{Counter}\OperatorTok{;}

\KeywordTok{public} \KeywordTok{class}\NormalTok{ IThread }\KeywordTok{extends} \BuiltInTok{Thread} \OperatorTok{\{}
    \KeywordTok{private} \DataTypeTok{final} \DataTypeTok{int}\NormalTok{ count}\OperatorTok{;}

    \KeywordTok{private}\NormalTok{ Counter counter}\OperatorTok{;}

    \KeywordTok{public} \FunctionTok{IThread}\OperatorTok{(}\DataTypeTok{int}\NormalTok{ count}\OperatorTok{)} \OperatorTok{\{}
        \KeywordTok{this}\OperatorTok{.}\FunctionTok{count} \OperatorTok{=}\NormalTok{ count}\OperatorTok{;}
    \OperatorTok{\}}

    \KeywordTok{public} \DataTypeTok{void} \FunctionTok{setCounter}\OperatorTok{(}\NormalTok{Counter counter}\OperatorTok{)} \OperatorTok{\{}
        \KeywordTok{this}\OperatorTok{.}\FunctionTok{counter} \OperatorTok{=}\NormalTok{ counter}\OperatorTok{;}
    \OperatorTok{\}}

    \AttributeTok{@Override}
    \KeywordTok{public} \DataTypeTok{void} \FunctionTok{run}\OperatorTok{()} \OperatorTok{\{}
        \ControlFlowTok{for} \OperatorTok{(}\DataTypeTok{int}\NormalTok{ i }\OperatorTok{=} \DecValTok{0}\OperatorTok{;}\NormalTok{ i }\OperatorTok{\textless{}}\NormalTok{ count}\OperatorTok{;}\NormalTok{ i}\OperatorTok{++)} \OperatorTok{\{}
\NormalTok{            counter}\OperatorTok{.}\FunctionTok{inc}\OperatorTok{();}
        \OperatorTok{\}}
    \OperatorTok{\}}
\OperatorTok{\}}
\end{Highlighting}
\end{Shaded}

\textbf{Głowna klasa:}

\begin{Shaded}
\begin{Highlighting}[]
\KeywordTok{package}\ImportTok{ pl}\OperatorTok{.}\ImportTok{edu}\OperatorTok{.}\ImportTok{agh}\OperatorTok{.}\ImportTok{tw}\OperatorTok{.}\ImportTok{knapp}\OperatorTok{;}

\KeywordTok{import} \ImportTok{pl}\OperatorTok{.}\ImportTok{edu}\OperatorTok{.}\ImportTok{agh}\OperatorTok{.}\ImportTok{tw}\OperatorTok{.}\ImportTok{knapp}\OperatorTok{.}\ImportTok{counter}\OperatorTok{.}\ImportTok{Counter}\OperatorTok{;}

\KeywordTok{public} \KeywordTok{class}\NormalTok{ Race }\OperatorTok{\{}
    \KeywordTok{public} \DataTypeTok{static} \DataTypeTok{void} \FunctionTok{main}\OperatorTok{(}\BuiltInTok{String}\OperatorTok{[]}\NormalTok{ args}\OperatorTok{)} \KeywordTok{throws} \BuiltInTok{InterruptedException} \OperatorTok{\{}
        \DataTypeTok{final} \DataTypeTok{int}\NormalTok{ operationCount }\OperatorTok{=} \DecValTok{10\_000}\OperatorTok{;}

        \DataTypeTok{var}\NormalTok{ incThread }\OperatorTok{=} \KeywordTok{new} \FunctionTok{IThread}\OperatorTok{(}\NormalTok{operationCount}\OperatorTok{);}
        \DataTypeTok{var}\NormalTok{ decThread }\OperatorTok{=} \KeywordTok{new} \FunctionTok{DThread}\OperatorTok{(}\NormalTok{operationCount}\OperatorTok{);}

        \DataTypeTok{var}\NormalTok{ counter }\OperatorTok{=} \KeywordTok{new} \FunctionTok{Counter}\OperatorTok{(}\DecValTok{0}\OperatorTok{);}

\NormalTok{        incThread}\OperatorTok{.}\FunctionTok{setCounter}\OperatorTok{(}\NormalTok{counter}\OperatorTok{);}
\NormalTok{        decThread}\OperatorTok{.}\FunctionTok{setCounter}\OperatorTok{(}\NormalTok{counter}\OperatorTok{);}

\NormalTok{        incThread}\OperatorTok{.}\FunctionTok{start}\OperatorTok{();}
\NormalTok{        decThread}\OperatorTok{.}\FunctionTok{start}\OperatorTok{();}

\NormalTok{        incThread}\OperatorTok{.}\FunctionTok{join}\OperatorTok{();}
\NormalTok{        decThread}\OperatorTok{.}\FunctionTok{join}\OperatorTok{();}

        \BuiltInTok{System}\OperatorTok{.}\FunctionTok{out}\OperatorTok{.}\FunctionTok{println}\OperatorTok{(}\StringTok{"stan="} \OperatorTok{+}\NormalTok{ counter}\OperatorTok{.}\FunctionTok{value}\OperatorTok{());}
    \OperatorTok{\}}
\OperatorTok{\}}
\end{Highlighting}
\end{Shaded}

\begin{quote}
Uwaga 1: klasy \texttt{DThread} i \texttt{IThread} można zgeneralizować
(tzn. stworzyć wspólną klasę nadrzędną, ale w celach czytelności to
zostało pominięte).
\end{quote}

\begin{quote}
Uwaga 2: wyniki znajdują się w rozdziale ``Wyniki''
\end{quote}

\hypertarget{synchronizacja-wux105tkuxf3w-p.-3}{%
\subsection{Synchronizacja wątków
(p.~3)}\label{synchronizacja-wux105tkuxf3w-p.-3}}

Synchronizacji wątków można dokonać na wiele sposobów, przede wszystkim
użyć mechanizmów wbudowanych, np. mutexów, atomiców, lockerów,
\texttt{synchronized} itd., ale to zadanie musi zostać wykonane bez
korzystania z w/w mechanizmów.

W przypadku prostego licznika, nie korzystając w mechanizmów
systemowych, możemy otrzymać przewidywalną wartość na kilka sposobów:

\begin{enumerate}
\def\labelenumi{\arabic{enumi}.}
\item
  Drugi wątek zostanie uruchomiony po zakończeniu pierwszego - w tym
  przypadku dostaniemy przewidywalną wartość (czyli 0), lecz to
  rozwiązanie jest równoważne inkrementacji i dekrementacji licznika w
  jednym wątku;
\item
  Każdy wątek dostaje na wejściu kopię (klona) licznika, i wykonuje na
  nim swoje operacje. Następnie główny wątek mergeuje wyniki (inaczej
  mówiąc - dodaje wartości liczników). Takiego rozwiązania można użyć,
  ale ono nie pokazuje jednoczesnego korzystania z zasobu dzielonego -
  licznika;
\item
  Własny mechanizm synchronizacji, który w \emph{jakimś} stopniu będzie
  działał podobnie do mutexa. To rozwiązanie zostanie zaimplementowane
  poniżej.
\end{enumerate}

W celu implementacji własnego mechanizmu synchronizacji, musimy najpierw
zrozumieć, jak wygląda dostęp do zmiennych w przypadku wielowątkowości.

\hypertarget{przykux142ad}{%
\subsubsection{Przykład}\label{przykux142ad}}

Rozważmy następujący przykład:

\begin{Shaded}
\begin{Highlighting}[]
\KeywordTok{public} \KeywordTok{class}\NormalTok{ TaskRunner }\OperatorTok{\{}

    \KeywordTok{private} \DataTypeTok{static} \DataTypeTok{int}\NormalTok{ number}\OperatorTok{;}
    \KeywordTok{private} \DataTypeTok{static} \DataTypeTok{boolean}\NormalTok{ ready}\OperatorTok{;}

    \KeywordTok{private} \DataTypeTok{static} \KeywordTok{class} \BuiltInTok{Reader} \KeywordTok{extends} \BuiltInTok{Thread} \OperatorTok{\{}

        \AttributeTok{@Override}
        \KeywordTok{public} \DataTypeTok{void} \FunctionTok{run}\OperatorTok{()} \OperatorTok{\{}
            \ControlFlowTok{while} \OperatorTok{(!}\NormalTok{ready}\OperatorTok{)} \OperatorTok{\{}
                \BuiltInTok{Thread}\OperatorTok{.}\FunctionTok{yield}\OperatorTok{();}
            \OperatorTok{\}}

            \BuiltInTok{System}\OperatorTok{.}\FunctionTok{out}\OperatorTok{.}\FunctionTok{println}\OperatorTok{(}\NormalTok{number}\OperatorTok{);}
        \OperatorTok{\}}
    \OperatorTok{\}}

    \KeywordTok{public} \DataTypeTok{static} \DataTypeTok{void} \FunctionTok{main}\OperatorTok{(}\BuiltInTok{String}\OperatorTok{[]}\NormalTok{ args}\OperatorTok{)} \OperatorTok{\{}
        \KeywordTok{new} \BuiltInTok{Reader}\OperatorTok{().}\FunctionTok{start}\OperatorTok{();}
\NormalTok{        number }\OperatorTok{=} \DecValTok{42}\OperatorTok{;}
\NormalTok{        ready }\OperatorTok{=} \KeywordTok{true}\OperatorTok{;}
    \OperatorTok{\}}
\OperatorTok{\}}
\end{Highlighting}
\end{Shaded}

Oczekujemy, że wątek \texttt{Reader} wypisze \texttt{42}, ale w
rzeczywistości niekoniecznie tak się stanie. Program może się zawiesić,
wypisać wynik po jakimś czasie, a nawet możemy dostać na wyjściu
\texttt{0}. To się dzieje z różnych powodów, między innymi:
optymalizacje (out-of-order execution, \texttt{ready\ =\ true} może
zostać wykonane przed \texttt{number\ =\ 42}), caching (na systemach
wielordzeniowych, każdy rdzeń posiada własną pamięć podręczną w celu
przyspieszenia działania).

\hypertarget{sux142owo-kluczowe-volatile}{%
\subsubsection{\texorpdfstring{Słowo kluczowe
\texttt{volatile}}{Słowo kluczowe volatile}}\label{sux142owo-kluczowe-volatile}}

Aby zapewnić, że aktualizacje zmiennych przekazywane są przewidywalnie
do innych wątków, powinniśmy zastosować modyfikator \texttt{volatile} do
tych zmiennych. Tzn., w przypadku w/w przykładu:

\begin{Shaded}
\begin{Highlighting}[]
\KeywordTok{public} \KeywordTok{class}\NormalTok{ TaskRunner }\OperatorTok{\{}

    \KeywordTok{private} \DataTypeTok{static} \KeywordTok{volatile} \DataTypeTok{int}\NormalTok{ number}\OperatorTok{;}
    \KeywordTok{private} \DataTypeTok{static} \KeywordTok{volatile} \DataTypeTok{boolean}\NormalTok{ ready}\OperatorTok{;}

    \CommentTok{// ...}

\OperatorTok{\}}
\end{Highlighting}
\end{Shaded}

Stosując dane podejście, wszelkie zmiany dokonane na tych zmiennych będą
przekazywane do innych wątków.

\hypertarget{implementacja}{%
\subsubsection{Implementacja}\label{implementacja}}

Moja własna implementacja posiada następujący interfejs:

\begin{Shaded}
\begin{Highlighting}[]
\CommentTok{// ILocker.java}

\KeywordTok{package}\ImportTok{ pl}\OperatorTok{.}\ImportTok{edu}\OperatorTok{.}\ImportTok{agh}\OperatorTok{.}\ImportTok{tw}\OperatorTok{.}\ImportTok{knapp}\OperatorTok{.}\ImportTok{locker}\OperatorTok{;}

\KeywordTok{public} \KeywordTok{interface}\NormalTok{ ILocker }\OperatorTok{\{}
    \DataTypeTok{void} \FunctionTok{lock}\OperatorTok{();}
    \DataTypeTok{void} \FunctionTok{unlock}\OperatorTok{();}
\OperatorTok{\}}
\end{Highlighting}
\end{Shaded}

Od razu musimy przyjąć pewne założenia:

\begin{itemize}
\tightlist
\item
  Kilka wątków może wywołać metodę \texttt{lock} jednocześnie;
\item
  Tylko jeden wątek może wywołać metodę \texttt{unlock} w tym samym
  czasie;
\end{itemize}

Dodatkowo:

\begin{itemize}
\tightlist
\item
  W celu uniknięcia zawieszenia (deadlock), po wywołaniu \texttt{lock}
  musi nastąpić \texttt{unlock};
\item
  Metoda \texttt{unlock} musi zostać wywołana z tego samego wątku co i
  \texttt{lock}, w przeciwnym przypadku program może się zachowywać w
  sposób nieprzywidywalny (undefined behavior);
\item
  Wątki będą mogły zablokować Locker w z góry zadanej kolejności;
\end{itemize}

Warto również przypomnieć, że każdy wątek ma swój własny unikalny
identyfikator. W Javie możemy go pobrać korzystając z
\texttt{Thread\#getId()}. Żeby otrzymać wątek, w którym wykonuje się
dana funkcja, możemy skorzystać z \texttt{Thread.currentThread()}.

Implementacja własnego lockera:

\begin{Shaded}
\begin{Highlighting}[]
\CommentTok{// ThreadLocker.java}

\KeywordTok{package}\ImportTok{ pl}\OperatorTok{.}\ImportTok{edu}\OperatorTok{.}\ImportTok{agh}\OperatorTok{.}\ImportTok{tw}\OperatorTok{.}\ImportTok{knapp}\OperatorTok{.}\ImportTok{locker}\OperatorTok{;}

\KeywordTok{import} \ImportTok{java}\OperatorTok{.}\ImportTok{util}\OperatorTok{.*;}
\KeywordTok{import} \ImportTok{java}\OperatorTok{.}\ImportTok{util}\OperatorTok{.}\ImportTok{stream}\OperatorTok{.}\ImportTok{Collectors}\OperatorTok{;}
\KeywordTok{import} \ImportTok{java}\OperatorTok{.}\ImportTok{util}\OperatorTok{.}\ImportTok{stream}\OperatorTok{.}\ImportTok{IntStream}\OperatorTok{;}

\KeywordTok{public} \KeywordTok{class}\NormalTok{ ThreadLocker }\KeywordTok{implements}\NormalTok{ ILocker }\OperatorTok{\{}
    \CommentTok{// \textless{}thread id, thread index\textgreater{}}
    \KeywordTok{private} \DataTypeTok{final} \BuiltInTok{Map}\OperatorTok{\textless{}}\BuiltInTok{Long}\OperatorTok{,} \BuiltInTok{Integer}\OperatorTok{\textgreater{}}\NormalTok{ threadIds}\OperatorTok{;}

    \KeywordTok{private} \KeywordTok{volatile} \DataTypeTok{int}\NormalTok{ activeThreadIndex }\OperatorTok{=} \DecValTok{0}\OperatorTok{;}

    \KeywordTok{public} \FunctionTok{ThreadLocker}\OperatorTok{(}\BuiltInTok{Thread}\KeywordTok{...}\NormalTok{ threads}\OperatorTok{)} \OperatorTok{\{}
\NormalTok{        threadIds }\OperatorTok{=}\NormalTok{ IntStream}\OperatorTok{.}\FunctionTok{range}\OperatorTok{(}\DecValTok{0}\OperatorTok{,}\NormalTok{ threads}\OperatorTok{.}\FunctionTok{length}\OperatorTok{)}
                \OperatorTok{.}\FunctionTok{mapToObj}\OperatorTok{(}\NormalTok{i }\OperatorTok{{-}\textgreater{}} \KeywordTok{new} \BuiltInTok{AbstractMap}\OperatorTok{.}\FunctionTok{SimpleEntry}\OperatorTok{\textless{}\textgreater{}(}\NormalTok{threads}\OperatorTok{[}\NormalTok{i}\OperatorTok{].}\FunctionTok{getId}\OperatorTok{(),}\NormalTok{ i}\OperatorTok{))}
                \OperatorTok{.}\FunctionTok{collect}\OperatorTok{(}\NormalTok{Collectors}\OperatorTok{.}\FunctionTok{toUnmodifiableMap}\OperatorTok{(}
                    \BuiltInTok{AbstractMap}\OperatorTok{.}\FunctionTok{SimpleEntry}\OperatorTok{::}\NormalTok{getKey}\OperatorTok{,} \BuiltInTok{AbstractMap}\OperatorTok{.}\FunctionTok{SimpleEntry}\OperatorTok{::}\NormalTok{getValue}\OperatorTok{));}
    \OperatorTok{\}}

    \AttributeTok{@Override}
    \KeywordTok{public} \DataTypeTok{void} \FunctionTok{lock}\OperatorTok{()} \OperatorTok{\{}
        \ControlFlowTok{while} \OperatorTok{(!}\FunctionTok{canBeLocked}\OperatorTok{())} \OperatorTok{\{}
            \BuiltInTok{Thread}\OperatorTok{.}\FunctionTok{yield}\OperatorTok{();}
        \OperatorTok{\}}
    \OperatorTok{\}}

    \AttributeTok{@Override}
    \KeywordTok{public} \DataTypeTok{void} \FunctionTok{unlock}\OperatorTok{()} \OperatorTok{\{}
\NormalTok{        activeThreadIndex }\OperatorTok{=} \OperatorTok{(}\NormalTok{activeThreadIndex }\OperatorTok{+} \DecValTok{1}\OperatorTok{)} \OperatorTok{\%}\NormalTok{ threadIds}\OperatorTok{.}\FunctionTok{size}\OperatorTok{();}
    \OperatorTok{\}}

    \KeywordTok{private} \DataTypeTok{boolean} \FunctionTok{canBeLocked}\OperatorTok{()} \OperatorTok{\{}
        \DataTypeTok{var}\NormalTok{ currentThreadIndex }\OperatorTok{=}\NormalTok{ threadIds}\OperatorTok{.}\FunctionTok{get}\OperatorTok{(}\BuiltInTok{Thread}\OperatorTok{.}\FunctionTok{currentThread}\OperatorTok{().}\FunctionTok{getId}\OperatorTok{());}
        \ControlFlowTok{return}\NormalTok{ currentThreadIndex }\OperatorTok{==}\NormalTok{ activeThreadIndex}\OperatorTok{;}
    \OperatorTok{\}}
\OperatorTok{\}}
\end{Highlighting}
\end{Shaded}

Ten locker działa w następujący sposób:

\begin{itemize}
\tightlist
\item
  Tworzymy go przekazując jako argumenty wątki, które będą z niego
  korzystały;
\item
  Każdy taki wątek dostaje indeks;
\item
  W celu mapowania id wątku na jego indeks korzystamy z mapy (warto
  zwrócić uwagę: jest ona \emph{niemodyfikowalna} i oznaczona jako
  \texttt{final}, a więc możemy bezpiecznie odczytywać z niej wartości z
  różnych wątków, tzn. nie potrzebujemy żadnej dodatkowej
  synchronizacji);
\item
  Zmienna \texttt{activeThreadIndex} wskazuje, dla którego wątku ten
  locker może zostać zablokowany;
\item
  Metoda \texttt{lock} sprawdza, czy locker może zostać zablokowany
  \textbf{dla danego wątku} (mapując \emph{thread id} na \emph{thread
  index}), i jeżeli tak, blokuje go przerywając pętlę (warto zauważyć,
  że ta pętla powoduje tzw. \emph{busy-waiting}, w celu zmniejszenia
  (raczej próby zmniejszenia) negatywnych efektów użyto
  \texttt{Thread.yield});
\item
  Metoda \texttt{unlock} zwiększa \texttt{activeThreadIndex} o 1 (modulo
  \texttt{threadIds.size()});
\end{itemize}

Dzięki temu, że \texttt{activeThreadIndex} jest użyty z modyfikatorem
\texttt{volatile}, ten mechanizm działa poprawnie, tzn. zgodnie z
oczekiwaniami.

\begin{quote}
Uwaga: wyniki znajdują się w rozdziale ``Wyniki''
\end{quote}

\hypertarget{zalety}{%
\paragraph{Zalety}\label{zalety}}

\begin{itemize}
\tightlist
\item
  Prosty w implementacji;
\item
  Pozwala tworzyć tzw. sekcje krytyczne, dzięki czemu mamy możliwość
  zsynchronizowania działania kilku wątków;
\item
  Łatwy w użyciu - ma interfejs podobny do interfejsu mutexa;
\end{itemize}

\hypertarget{wady}{%
\paragraph{Wady}\label{wady}}

\begin{itemize}
\tightlist
\item
  Powoduje tzw. \textbf{busy-waiting} - ciągłe sprawdzanie, czy Locker
  jest dostępny dla danego wątku - niepotrzebnie korzystamy z czasu
  procesora, który mógłby zostać wykorzystany na wykonanie ważniejszych
  rzeczy;
\item
  Wymaga, aby wątki, które będą korzystały z Lockera, były z góry znane
  i przekazane do konstruktora;
\item
  Zakłada, że wszystkie wątki będą mogły zablokować Lockera w
  kolejności, w jakiej zostały przekazane do konstruktora. Z tego
  wynika, że jeżeli np. wątek A wywołuje \texttt{lock} 1000 razy, a
  wątek B - 998 razy, to program się zawiesi;
\item
  Nie nadaje się do praktycznego użycia;
\end{itemize}

Zsynchronizowany (thread-safe) licznik:

\begin{Shaded}
\begin{Highlighting}[]
\CommentTok{// SynchronizedCounter.java}

\KeywordTok{package}\ImportTok{ pl}\OperatorTok{.}\ImportTok{edu}\OperatorTok{.}\ImportTok{agh}\OperatorTok{.}\ImportTok{tw}\OperatorTok{.}\ImportTok{knapp}\OperatorTok{.}\ImportTok{counter}\OperatorTok{;}

\KeywordTok{import} \ImportTok{pl}\OperatorTok{.}\ImportTok{edu}\OperatorTok{.}\ImportTok{agh}\OperatorTok{.}\ImportTok{tw}\OperatorTok{.}\ImportTok{knapp}\OperatorTok{.}\ImportTok{locker}\OperatorTok{.}\ImportTok{ThreadLocker}\OperatorTok{;}

\KeywordTok{public} \KeywordTok{class}\NormalTok{ SynchronizedCounter }\KeywordTok{extends}\NormalTok{ Counter }\OperatorTok{\{}
    \KeywordTok{private} \DataTypeTok{final}\NormalTok{ ThreadLocker locker}\OperatorTok{;}

    \KeywordTok{public} \FunctionTok{SynchronizedCounter}\OperatorTok{(}\DataTypeTok{int}\NormalTok{ n}\OperatorTok{,} \BuiltInTok{Thread}\KeywordTok{...}\NormalTok{ threads}\OperatorTok{)} \OperatorTok{\{}
        \KeywordTok{super}\OperatorTok{(}\NormalTok{n}\OperatorTok{);}
\NormalTok{        locker }\OperatorTok{=} \KeywordTok{new} \FunctionTok{ThreadLocker}\OperatorTok{(}\NormalTok{threads}\OperatorTok{);}
    \OperatorTok{\}}

    \AttributeTok{@Override}
    \KeywordTok{public} \DataTypeTok{void} \FunctionTok{inc}\OperatorTok{()} \OperatorTok{\{}
\NormalTok{        locker}\OperatorTok{.}\FunctionTok{lock}\OperatorTok{();}
        \KeywordTok{super}\OperatorTok{.}\FunctionTok{inc}\OperatorTok{();}
\NormalTok{        locker}\OperatorTok{.}\FunctionTok{unlock}\OperatorTok{();}
    \OperatorTok{\}}

    \AttributeTok{@Override}
    \KeywordTok{public} \DataTypeTok{void} \FunctionTok{dec}\OperatorTok{()} \OperatorTok{\{}
\NormalTok{        locker}\OperatorTok{.}\FunctionTok{lock}\OperatorTok{();}
        \KeywordTok{super}\OperatorTok{.}\FunctionTok{dec}\OperatorTok{();}
\NormalTok{        locker}\OperatorTok{.}\FunctionTok{unlock}\OperatorTok{();}
    \OperatorTok{\}}
\OperatorTok{\}}
\end{Highlighting}
\end{Shaded}

    \hypertarget{wyniki}{%
\section{Wyniki}\label{wyniki}}

\hypertarget{race-condition-p.-1-2}{%
\subsection{Race condition (p.~1-2)}\label{race-condition-p.-1-2}}

Zgodnie z poleceniem, \textbf{program} musi zostać uruchomiony
\texttt{100} razy. W celu ułatwienia, korzystam z narzędzia
\texttt{gradle} i prostego \texttt{bash} skryptu, który pobiera wyniki
wykonania programu:

\begin{Shaded}
\begin{Highlighting}[]
\ControlFlowTok{for}\NormalTok{ \_ }\KeywordTok{in} \DataTypeTok{\{}\DecValTok{1}\DataTypeTok{..}\DecValTok{100}\DataTypeTok{\}}
\ControlFlowTok{do}
    \ExtensionTok{./gradlew}\NormalTok{ run }\KeywordTok{|} \FunctionTok{sed} \AttributeTok{{-}n} \StringTok{\textquotesingle{}s/\^{}.*stan=\textbackslash{}s*\textbackslash{}(\textbackslash{}S*\textbackslash{}).*$/\textbackslash{}1/p\textquotesingle{}}
\ControlFlowTok{done}
\end{Highlighting}
\end{Shaded}

Pobrane wyniki zostały zapisane do pliku a następnie wczytane i
przetwarzone przez skrypt w języku \texttt{python}:

    \begin{tcolorbox}[breakable, size=fbox, boxrule=1pt, pad at break*=1mm,colback=cellbackground, colframe=cellborder]
\prompt{In}{incolor}{2}{\boxspacing}
\begin{Verbatim}[commandchars=\\\{\}]
\PY{k+kn}{import} \PY{n+nn}{matplotlib}\PY{n+nn}{.}\PY{n+nn}{pyplot} \PY{k}{as} \PY{n+nn}{plt}
\PY{k+kn}{import} \PY{n+nn}{pandas} \PY{k}{as} \PY{n+nn}{pd}

\PY{n}{df} \PY{o}{=} \PY{n}{pd}\PY{o}{.}\PY{n}{read\PYZus{}csv}\PY{p}{(}\PY{l+s+s2}{\PYZdq{}}\PY{l+s+s2}{output.txt}\PY{l+s+s2}{\PYZdq{}}\PY{p}{,} \PY{n}{header}\PY{o}{=}\PY{k+kc}{None}\PY{p}{)} 

\PY{n+nb}{print}\PY{p}{(}\PY{n}{df}\PY{p}{)} 
\end{Verbatim}
\end{tcolorbox}

    \begin{Verbatim}[commandchars=\\\{\}]
       0
0  -1562
1    708
2     89
3   1055
4      7
..   {\ldots}
95 -2866
96  -233
97   417
98 -1410
99 -1203

[100 rows x 1 columns]
    \end{Verbatim}

    \begin{tcolorbox}[breakable, size=fbox, boxrule=1pt, pad at break*=1mm,colback=cellbackground, colframe=cellborder]
\prompt{In}{incolor}{4}{\boxspacing}
\begin{Verbatim}[commandchars=\\\{\}]
\PY{n}{df}\PY{o}{.}\PY{n}{hist}\PY{p}{(}\PY{n}{bins}\PY{o}{=}\PY{l+m+mi}{20}\PY{p}{)}
\end{Verbatim}
\end{tcolorbox}

            \begin{tcolorbox}[breakable, size=fbox, boxrule=.5pt, pad at break*=1mm, opacityfill=0]
\prompt{Out}{outcolor}{4}{\boxspacing}
\begin{Verbatim}[commandchars=\\\{\}]
array([[<Axes: title=\{'center': '0'\}>]], dtype=object)
\end{Verbatim}
\end{tcolorbox}
        
    \begin{center}
    \adjustimage{max size={0.9\linewidth}{0.9\paperheight}}{output_3_1.png}
    \end{center}
    { \hspace*{\fill} \\}
    
    Jak widać z tego histogramu, wyniki zagęszczają się bliżej wartości
\texttt{0}, ale są również wyniki położone dość daleko wartości
\texttt{0}, np.:

    \begin{tcolorbox}[breakable, size=fbox, boxrule=1pt, pad at break*=1mm,colback=cellbackground, colframe=cellborder]
\prompt{In}{incolor}{5}{\boxspacing}
\begin{Verbatim}[commandchars=\\\{\}]
\PY{n}{df}\PY{o}{.}\PY{n}{abs}\PY{p}{(}\PY{p}{)}\PY{o}{.}\PY{n}{max}\PY{p}{(}\PY{p}{)}
\end{Verbatim}
\end{tcolorbox}

            \begin{tcolorbox}[breakable, size=fbox, boxrule=.5pt, pad at break*=1mm, opacityfill=0]
\prompt{Out}{outcolor}{5}{\boxspacing}
\begin{Verbatim}[commandchars=\\\{\}]
0    4535
dtype: int64
\end{Verbatim}
\end{tcolorbox}
        
    \hypertarget{synchronizacja-wux105tkuxf3w-p.-3}{%
\subsection{Synchronizacja wątków
(p.~3)}\label{synchronizacja-wux105tkuxf3w-p.-3}}

W przypadku zsynchronizowanych wątków, zgodnie z oczekiwaniami na
wyjściu za każdym razem dostajemy \texttt{0}:

    \begin{tcolorbox}[breakable, size=fbox, boxrule=1pt, pad at break*=1mm,colback=cellbackground, colframe=cellborder]
\prompt{In}{incolor}{6}{\boxspacing}
\begin{Verbatim}[commandchars=\\\{\}]
\PY{n}{df\PYZus{}sync} \PY{o}{=} \PY{n}{pd}\PY{o}{.}\PY{n}{read\PYZus{}csv}\PY{p}{(}\PY{l+s+s2}{\PYZdq{}}\PY{l+s+s2}{output\PYZus{}sync.txt}\PY{l+s+s2}{\PYZdq{}}\PY{p}{,} \PY{n}{header}\PY{o}{=}\PY{k+kc}{None}\PY{p}{)}
\PY{n+nb}{print}\PY{p}{(}\PY{n}{df\PYZus{}sync}\PY{p}{)}
\end{Verbatim}
\end{tcolorbox}

    \begin{Verbatim}[commandchars=\\\{\}]
    0
0   0
1   0
2   0
3   0
4   0
.. ..
95  0
96  0
97  0
98  0
99  0

[100 rows x 1 columns]
    \end{Verbatim}

    \begin{tcolorbox}[breakable, size=fbox, boxrule=1pt, pad at break*=1mm,colback=cellbackground, colframe=cellborder]
\prompt{In}{incolor}{12}{\boxspacing}
\begin{Verbatim}[commandchars=\\\{\}]
\PY{n}{df\PYZus{}sync}\PY{p}{[}\PY{n}{df\PYZus{}sync} \PY{o}{==} \PY{l+m+mi}{0}\PY{p}{]}
\end{Verbatim}
\end{tcolorbox}

            \begin{tcolorbox}[breakable, size=fbox, boxrule=.5pt, pad at break*=1mm, opacityfill=0]
\prompt{Out}{outcolor}{12}{\boxspacing}
\begin{Verbatim}[commandchars=\\\{\}]
    0
0   0
1   0
2   0
3   0
4   0
.. ..
95  0
96  0
97  0
98  0
99  0

[100 rows x 1 columns]
\end{Verbatim}
\end{tcolorbox}
        
    Jak widać z powyższego wyniku, wszystkie \texttt{100} wartości są równe
\texttt{0}.

    \hypertarget{wnioski}{%
\section{Wnioski}\label{wnioski}}

\begin{itemize}
\item
  \textbf{Race condition} czyli \textbf{Wyścig}, charakteryzuje się
  niedeterministycznym zachowaniem programu, może powodować błędy trudne
  do wykrycia i pojawia się wtedy, gdy więcej niż jeden wątek korzysta
  jednocześnie z zasobu dzielonego, przy czym co najmniej jeden próbuje
  go zmienić;
\item
  Język \texttt{Java} posiada własne mechanizmy synchronizacji, między
  innymi: semafory (\texttt{Semaphore}), lockery (np.
  \texttt{ReentrantLock}), zmienne atomowe (atomics, np.
  \texttt{AtomicInteger}), słowo kluczowe \texttt{synchronized};
\item
  Stosując modyfikator \texttt{volatile}, wszystkie zmiany dokonane na
  tych zmiennych będą przekazywane do innych wątków;
\item
  Niezsynchronizowany dostęp do licznika powoduje wyścigi, w związku z
  czym nie otrzymujemy na wyjściu oczekiwanej wartości, tzn. \texttt{0};
\item
  Zsynchronizowany dostęp do licznika nie powoduje żadnych race
  condition i na wyjściu za każdym razem wypisuje się wartość
  oczekiwana, czyli \texttt{0}.
\end{itemize}

    \hypertarget{bibliografia}{%
\section{Bibliografia}\label{bibliografia}}

\begin{enumerate}
\def\labelenumi{\arabic{enumi}.}
\tightlist
\item
  \href{https://home.agh.edu.pl/~funika/tw/lab1/}{Materiały do
  laboratorium}
\item
  \href{https://docs.oracle.com/javase/tutorial/essential/concurrency/sync.html}{The
  Java™ Tutorials - Synchronization}
\item
  \href{https://docs.oracle.com/javase/8/docs/api/java/util/concurrent/Semaphore.html}{Java
  Docs - Semaphore}
\item
  \href{https://docs.oracle.com/javase/8/docs/api/java/util/concurrent/locks/ReentrantLock.html}{Java
  Docs - ReentrantLock}
\item
  \href{https://docs.oracle.com/javase/8/docs/api/java/util/concurrent/atomic/AtomicInteger.html}{Java
  Docs - AtomicInteger}
\item
  \href{https://docs.oracle.com/javase/8/docs/api/java/lang/Thread.html}{Java
  Docs - Thread}
\item
  \href{https://www.baeldung.com/java-volatile}{Baeldung - Guide to the
  Volatile Keyword in Java}
\item
  \href{https://en.wikipedia.org/wiki/Race_condition}{Wikipedia - Race
  condition}
\item
  \href{https://en.wikipedia.org/wiki/Busy_waiting}{Wikipedia - Busy
  waiting}
\end{enumerate}


    % Add a bibliography block to the postdoc
    
    
    
\end{document}
